% 主文档:凸优化(中文教学用)—— 分章单文件结构,建议使用 XeLaTeX 编译
\documentclass[zihao=5,a4paper,oneside]{ctexbook}
\usepackage[margin=2.5cm]{geometry}
\usepackage{microtype}
\usepackage{xcolor}
\usepackage{amsmath,amssymb,amsthm}
\usepackage{mdframed}
\usepackage{booktabs}
\usepackage{float}      % 强制图片/表格位置 [H]
\usepackage{array}      % 扩展表格列格式
\usepackage{multirow}   % 表格多行合并
\usepackage{tabularx}   % 自动调整列宽
\usepackage{longtable}  % 跨页长表格

% 带框定理样式
\mdfdefinestyle{theoremstyle}{
  linecolor=blue!60!black,
  linewidth=2pt,
  frametitlerule=true,
  frametitlebackgroundcolor=blue!10,
  innertopmargin=\topskip,
  roundcorner=5pt
}

% 定理环境
\newtheorem{theoreminner}{定理}[chapter]
\newenvironment{theorem}
  {\begin{mdframed}[style=theoremstyle]\begin{theoreminner}}
  {\end{theoreminner}\end{mdframed}}

% 让更多“概念类”环境也使用带框样式(与定理一致)
\newtheorem{lemmainner}{引理}[chapter]
\newenvironment{lemma}
  {\begin{mdframed}[style=theoremstyle]\begin{lemmainner}}
  {\end{lemmainner}\end{mdframed}}

\newtheorem{propositioninner}{命题}[chapter]
\newenvironment{proposition}
  {\begin{mdframed}[style=theoremstyle]\begin{propositioninner}}
  {\end{propositioninner}\end{mdframed}}

\newtheorem{corollaryinner}{推论}[chapter]
\newenvironment{corollary}
  {\begin{mdframed}[style=theoremstyle]\begin{corollaryinner}}
  {\end{corollaryinner}\end{mdframed}}

% 定义/例/注 也使用带框样式(概念类统一样式)
\newtheorem{definitioninner}{定义}[chapter]
\newenvironment{definition}
  {\begin{mdframed}[style=theoremstyle]\begin{definitioninner}}
  {\end{definitioninner}\end{mdframed}}

\newtheorem{exampleinner}{例}[chapter]
\newenvironment{example}
  {\begin{mdframed}[style=theoremstyle]\begin{exampleinner}}
  {\end{exampleinner}\end{mdframed}}

\newtheorem{remarkinner}{注}[chapter]
\newenvironment{remark}
  {\begin{mdframed}[style=theoremstyle]\begin{remarkinner}}
  {\end{remarkinner}\end{mdframed}}

% 算法环境 (带框)
\newtheorem{algorithminner}{算法}[chapter]
\newenvironment{algorithm}
  {\begin{mdframed}[style=theoremstyle]\begin{algorithminner}}
  {\end{algorithminner}\end{mdframed}}

% 重定义证明环境(使用小字号斜体)
\renewenvironment{proof}[1][证明]
  {\par\noindent{\small\textit{#1.}}\par\nopagebreak\small\itshape}
  {\par\vspace{0.5em}}

\usepackage{bm}
\usepackage{graphicx}
\usepackage{pdfpages} % 引入 pdfpages 宏包
\usepackage{hyperref}
\hypersetup{colorlinks=true,linkcolor=blue,citecolor=blue}
\usepackage{fancyhdr}
\usepackage{amsmath}
\pagestyle{fancy}
\fancyhf{}
% 避免 fancyhdr 的 headheight 警告
\setlength{\headheight}{13pt}
\fancyhead[LE,RO]{\thepage}
\fancyhead[RE]{\leftmark}
\fancyhead[LO]{\rightmark}

\title{凸优化笔记(复习用)}
\author{作者:Harry}
\date{\today}

\begin{document}

% 插入封面 pdf (确保 cover_page.pdf 在同级目录下)
\includepdf{cover_page.pdf}

\tableofcontents
\cleardoublepage

% 包含章节(每章单独文件),文件位于 note_latex/chapters/
\chapter{凸优化简介}

% 将公式按小节编号,格式形如 1.1.1
\numberwithin{equation}{subsection}

\section{数学优化}
数学优化问题可以写为如下形式:
\begin{equation}
\begin{aligned}
\text{minimize} \quad & f_0(x) \\
\text{subject to} \quad & f_i(x) \leq b_i,\quad i=1,2,\dots,m
\end{aligned}
\end{equation}

向量$x = (x_1,x_2,\dots,x_n)$称为问题的\textbf{优化变量},函数$f_0$为目标函数,函数$f_i$为约束函数,常数$b_i$为约束上限或者约束边界。

最优解 $x^*$ 是满足所有约束条件并使目标函数达到最小值的变量取值,即:
\begin{equation}
x^* = \arg\min_{x \in \mathbb{R}^n} f_0(x) \quad \text{满足} \quad f_i(x) \leq b_i,\quad i=1,2,\dots,m
\end{equation}

更具体地说,最优解 $x^*$ 满足以下两个条件:

\begin{enumerate}
  \item \textbf{可行性(Feasibility)}:
  \begin{equation}
  f_i(x^*) \leq b_i,\quad \forall i=1,2,\dots,m
  \end{equation}
  即最优解必须满足所有的约束条件。
  
  \item \textbf{最优性(Optimality)}:
  \begin{equation}
  f_0(x^*) \leq f_0(x),\quad \forall x \in \mathbb{R}^n \text{ 且满足 } f_i(x) \leq b_i,\quad i=1,2,\dots,m
  \end{equation}
  即最优解在所有可行解中使目标函数达到最小值。
\end{enumerate}

如果存在多个满足上述条件的解,则它们都称为最优解,且此时目标函数的最优值是唯一的。

\section{线性规划与非线性规划}
线性规划的核心特征为:\textbf{目标函数和约束函数均为线性函数}。

\begin{definition}[线性函数]
线性函数的定义为:对于任意 $x, y \in \mathbb{R}^n$ 和 $\alpha, \beta \in \mathbb{R}$,均满足:
\begin{equation}
f_i(\alpha x + \beta y) = \alpha f_i(x) + \beta f_i(y)
\end{equation}
\end{definition}


\section{凸优化}
凸优化的核心特征为:\textbf{目标函数和约束函数均为凸函数}。

\begin{definition}[凸函数]
凸函数的定义为:对于任意 $x, y \in \mathbb{R}^n$ 和 $\alpha, \beta \in \mathbb{R}$,且满足 $\alpha + \beta = 1$、$\alpha \geq 0$、$\beta \geq 0$,均有:
\begin{equation}
f_i(\alpha x + \beta y) \leq \alpha f_i(x) + \beta f_i(y)
\end{equation}
\end{definition}

\section{最小二乘和线性规划}
广为人知而且应用广泛的两类凸优化问题:最小二乘和线性规划

\subsection{线性规划(Linear Programming, LP)}
线性规划是一类目标函数与约束函数均为线性的优化问题,是最经典的确定性优化模型之一。

一个函数 $f_i: \mathbb{R}^n \to \mathbb{R}$ 为线性函数,当且仅当对任意 $x, y \in \mathbb{R}^n$ 和任意标量 $\alpha, \beta \in \mathbb{R}$,满足\textbf{线性性}(齐次性+叠加性):
\begin{equation}
f_i(\alpha x + \beta y) = \alpha f_i(x) + \beta f_i(y)
\end{equation}

其具体形式可表示为 $f_i(x) = c_i^T x$,其中 $c_i \in \mathbb{R}^n$ 为常数向量,$x \in \mathbb{R}^n$ 为决策变量。

\begin{definition}[线性规划]
线性规划的目标是在 linear 约束下优化 linear 目标函数,标准形式(以最小化为例)为:
\begin{align}
\min_{x \in \mathbb{R}^n} &\quad f(x) = c^T x \nonumber\\
\text{s.t.} &\quad A x = b \\
&\quad x \geq 0 \nonumber
\end{align}
\end{definition}
其中:
\begin{itemize}
  \item $c \in \mathbb{R}^n$:目标函数系数向量;
  \item $A \in \mathbb{R}^{m \times n}$:约束系数矩阵($m < n$);
  \item $b \in \mathbb{R}^m$:约束右端项向量;
  \item $x \geq 0$:决策变量非负约束。
\end{itemize}


\subsection{最小二乘法(Least Squares Method, LSM)}

对于给定的观测数据 $\{(x_1, y_1), (x_2, y_2), \dots, (x_m, y_m)\}$,假设数据满足某种函数关系 $y = f(x; \theta)$($\theta$ 为待估参数),最小二乘法的目标是寻找参数 $\theta^*$,使得\textbf{误差平方和最小}:
\begin{equation}
\theta^* = \arg\min_{\theta} \sum_{i=1}^m [y_i - f(x_i; \theta)]^2
\end{equation}


当待估参数 $\theta$ 与函数 $f(x; \theta)$ 呈线性关系时,称为线性最小二乘。其典型形式为:
已知线性模型 $y = A \theta + \epsilon$($\epsilon$ 为误差项),目标函数为:
\begin{equation}
\min_{\theta \in \mathbb{R}^n} \quad f(\theta) = \|A \theta - y\|_2^2 = (A \theta - y)^T (A \theta - y)
\end{equation}

其中:
\begin{itemize}
  \item $A \in \mathbb{R}^{m \times n}$:设计矩阵($m > n$,保证超定系统);
  \item $y \in \mathbb{R}^m$:观测值向量;
  \item $\theta \in \mathbb{R}^n$:待估参数向量。
\end{itemize}

该目标函数是关于 $\theta$ 的二次凸函数,无约束条件(或仅含线性约束)。

当然最小二乘还有一些拓展,如加权最小二乘和正则化最小二乘,在此不展开。


\section{仿射集合(Affine Sets)}
仿射集合是线性空间中“平移后的线性子空间”,其核心特征是对\textbf{仿射组合}的封闭性。


\subsection{仿射组合与仿射集合}
\subsubsection{(1)仿射组合(Affine Combination)}
\begin{definition}
设 $x_1, x_2, \dots, x_k \in \mathbb{R}^n$,若标量 $\theta_1, \theta_2, \dots, \theta_k \in \mathbb{R}$ 满足 $\sum_{i=1}^k \theta_i = 1$,则称向量:
\begin{equation}
\theta_1 x_1 + \theta_2 x_2 + \dots + \theta_k x_k
\end{equation}
为 $x_1, x_2, \dots, x_k$ 的\textbf{仿射组合}。
\end{definition}

特别地,当 $k=2$ 时,仿射组合为 $\theta x + (1-\theta) y$($\theta \in \mathbb{R}$),其几何意义是过点 $x$ 和 $y$ 的\textbf{整条直线}(区别于后续凸组合对应的“线段”)。


\subsubsection{(2)仿射集合的定义}
\begin{definition}
一个集合 $A \subseteq \mathbb{R}^n$ 被称为\textbf{仿射集合},当且仅当对任意 $x, y \in A$ 及任意 $\theta \in \mathbb{R}$,$x$ 与 $y$ 的仿射组合仍属于 $A$,即:
\begin{equation}
\theta x + (1-\theta) y \in A
\end{equation}
\end{definition}

推广到 $k$ 个点:若 $A$ 是仿射集合,则对任意 $x_1, \dots, x_k \in A$ 及任意满足 $\sum_{i=1}^k \theta_i = 1$ 的 $\theta_1, \dots, \theta_k \in \mathbb{R}$,有 $\sum_{i=1}^k \theta_i x_i \in A$(可通过数学归纳法证明)。


\subsection{仿射集合与线性子空间的关系}
仿射集合可通过“线性子空间的平移”来等价描述,这是理解仿射集合的关键视角。

\subsubsection{(1)平移与线性子空间}
设 $A \subseteq \mathbb{R}^n$ 是仿射集合,任取 $x_0 \in A$,定义集合:
\begin{equation}
L = A - x_0 = \{ x - x_0 \mid x \in A \}
\end{equation}
则 $L$ 是 $\mathbb{R}^n$ 的\textbf{线性子空间}(满足对线性组合封闭:$\forall u, v \in L, \alpha, \beta \in \mathbb{R}$,$\alpha u + \beta v \in L$)。
证明略


\subsection{仿射包(Affine Hull)}
\begin{definition}
对任意集合 $S \subseteq \mathbb{R}^n$,包含 $S$ 的\textbf{最小仿射集合}称为 $S$ 的仿射包,记为 $\text{aff}(S)$。其数学表达式为:
\begin{equation}
	\text{aff}(S) = \left\{ \sum_{i=1}^k \theta_i x_i \mid x_1, \dots, x_k \in S, \theta_1, \dots, \theta_k \in \mathbb{R}, \sum_{i=1}^k \theta_i = 1 \right\}
\end{equation}
\end{definition}

直观理解:仿射包是“由 $S$ 中所有点的仿射组合构成的集合”,例如 $\mathbb{R}^2$ 中两个点的仿射包是过这两点的直线,三个不共线点的仿射包是整个 $\mathbb{R}^2$。


\subsection{典型实例}

\begin{itemize}
  \item 线性方程组的解空间:设 $A \in \mathbb{R}^{m \times n}$,$b \in \mathbb{R}^m$,则 $A x = b$ 的解集合 $X = \{ x \in \mathbb{R}^n \mid A x = b \}$ 是仿射集合(若 $X$ 非空)。
  
  \begin{proof}
  任取 $x, y \in X$,则 $A(\theta x + (1-\theta) y) = \theta A x + (1-\theta) A y = \theta b + (1-\theta) b = b$,故 $\theta x + (1-\theta) y \in X$。
  \end{proof}
  
  \item 单点集 $\{x_0\}$:是仿射集合(仅含自身,仿射组合仍为自身)。
  
  \item 整个空间 $\mathbb{R}^n$:是仿射集合(线性子空间本身,平移量为0)。
\end{itemize}


\section{凸集(Convex Sets)}
凸集是凸优化的核心结构,其特征是对\textbf{凸组合}的封闭性,这直接保证了“局部最优即全局最优”的关键性质。


\subsection{凸组合与凸集}
\subsubsection{(1)凸组合(Convex Combination)}
\begin{definition}
设 $x_1, x_2, \dots, x_k \in \mathbb{R}^n$,若标量 $\theta_1, \theta_2, \dots, \theta_k \in \mathbb{R}$ 满足 $\sum_{i=1}^k \theta_i = 1$ 且 $\theta_i \geq 0$($i=1, \dots, k$),则称向量:
\begin{equation}
\theta_1 x_1 + \theta_2 x_2 + \dots + \theta_k x_k
\end{equation}
为 $x_1, x_2, \dots, x_k$ 的\textbf{凸组合}。
\end{definition}

特别地,当 $k=2$ 时,凸组合为 $\theta x + (1-\theta) y$($\theta \in [0,1]$),其几何意义是连接 $x$ 和 $y$ 的\textbf{线段}(区别于仿射组合的“直线”)。


\subsubsection{(2)凸集的定义}
\begin{definition}
一个集合 $C \subseteq \mathbb{R}^n$ 被称为\textbf{凸集},当且仅当对任意 $x, y \in C$ 及任意 $\theta \in [0,1]$,$x$ 与 $y$ 的凸组合仍属于 $C$,即:
\begin{equation}
\theta x + (1-\theta) y \in C
\end{equation}
\end{definition}

推广到 $k$ 个点:若 $C$ 是凸集,则对任意 $x_1, \dots, x_k \in C$ 及任意满足 $\sum_{i=1}^k \theta_i = 1$ 且 $\theta_i \geq 0$ 的 $\theta_1, \dots, \theta_k \in \mathbb{R}$,有 $\sum_{i=1}^k \theta_i x_i \in C$(数学归纳法可证)。


\subsection{与仿射集合的关系}
仿射集合是特殊的凸集,但凸集不一定是仿射集合,二者的核心差异在于\textbf{组合系数的约束范围}:

\begin{itemize}
  \item 仿射组合:系数仅要求和为1(可正可负、可大于1);
  \item 凸组合:系数要求和为1且非负(仅在 $[0,1]$ 内取值)。
\end{itemize}

因此,仿射集合对更宽松的组合封闭,自然也对凸组合封闭,即:\textbf{若 $A$ 是仿射集合,则 $A$ 必是凸集}。反之,凸集(如线段、球体)不一定是仿射集合(线段对 $\theta > 1$ 的仿射组合不封闭)。


\subsection{重要概念:凸包(Convex Hull)}
\begin{definition}
对任意集合 $S \subseteq \mathbb{R}^n$,包含 $S$ 的\textbf{最小凸集}称为 $S$ 的凸包,记为 $\text{conv}(S)$。其数学表达式为:
\begin{equation}
	\text{conv}(S) = \left\{ \sum_{i=1}^k \theta_i x_i \mid x_1, \dots, x_k \in S, \theta_i \geq 0, \sum_{i=1}^k \theta_i = 1 \right\}
\end{equation}
\end{definition}

直观理解:凸包是“由 $S$ 中所有点的凸组合构成的集合”,例如 $\mathbb{R}^2$ 中三个不共线点的凸包是三角形,圆上所有点的凸包是闭圆盘。


\subsection{典型例子}

\begin{itemize}
  \item 仿射集合的特例:线性子空间、线性方程组解空间、单点集、$\mathbb{R}^n$ 均为凸集;
  
  \item 标准凸集:
  \begin{itemize}
    \item 闭区间 $[a, b] \subseteq \mathbb{R}$;
    \item 球体 $\{ x \in \mathbb{R}^n \mid \|x - x_0\| \leq r \}$($\|\cdot\|$ 为任意范数);
    \item 正象限 $\mathbb{R}_+^n = \{ x \in \mathbb{R}^n \mid x_i \geq 0, i=1, \dots, n \}$;
    \item 半空间 $\{ x \in \mathbb{R}^n \mid a^T x \leq b \}$($a \neq 0$,本质是线性不等式约束的解空间)。
  \end{itemize}
\end{itemize}


\section{锥(Cones)与凸锥(Convex Cones)}
锥是一类对“正齐次性”封闭的集合,凸锥则进一步结合了凸性,是锥优化(如半定规划、二次锥规划)的核心结构。


\subsection{核心定义:锥与凸锥}
\subsubsection{(1)锥的定义}
\begin{definition}
一个非空集合 $K \subseteq \mathbb{R}^n$ 被称为\textbf{锥},当且仅当对任意 $x \in K$ 及任意 $\alpha \geq 0$(非负标量),有 $\alpha x \in K$,即:
\begin{equation}
x \in K, \alpha \geq 0 \implies \alpha x \in K
\end{equation}
\end{definition}

直观理解:锥是“从原点出发的射线族”,若某条射线包含于 $K$,则射线的所有非负伸缩段也包含于 $K$。


\subsubsection{(2)凸锥的定义}
\begin{definition}
一个非空集合 $K \subseteq \mathbb{R}^n$ 被称为\textbf{凸锥},当且仅当它既是锥,又是凸集。其等价刻画有两种:

$K$ 是凸锥,当且仅当对任意 $x_1, \dots, x_k \in K$ 及任意 $\alpha_1, \dots, \alpha_k \geq 0$,有 $\sum_{i=1}^k \alpha_i x_i \in K$(称为对\textbf{非负线性组合}封闭)。
\end{definition}


\subsection{重要概念:凸锥包(Convex Cone Hull)}
\begin{definition}
对任意集合 $S \subseteq \mathbb{R}^n$,包含 $S$ 的\textbf{最小凸锥}称为 $S$ 的凸锥包,记为 $\text{cone}(S)$。其数学表达式为:
\begin{equation}
	\text{cone}(S) = \left\{ \sum_{i=1}^k \alpha_i x_i \mid x_1, \dots, x_k \in S, \alpha_i \geq 0 \right\}
\end{equation}
\end{definition}

直观理解:凸锥包是“由 $S$ 中所有点的非负线性组合构成的集合”,例如 $\mathbb{R}^2$ 中两个不共线向量的凸锥包是它们张成的“角形区域”。


\subsection{典型实例}

\begin{itemize}
  \item 非凸锥:$\mathbb{R}^2$ 中 $\{ (x_1, x_2) \mid x_1 x_2 \geq 0 \}$(第一、三象限的并集,对凸组合不封闭,如 $(1,0)$ 和 $(0,1)$ 的凸组合 $(1/2, 1/2)$ 不属于该集合);
  
  \item 凸锥:
  \begin{itemize}
    \item 原点 $\{0\}$(平凡凸锥);
    \item 正象限 $\mathbb{R}_+^n = \{ x \mid x_i \geq 0 \}$(非负线性组合仍非负);
    \item 线性子空间 $L \subseteq \mathbb{R}^n$(对任意 $\alpha \in \mathbb{R}$ 封闭,自然对 $\alpha \geq 0$ 封闭,且是凸集);
    \item 二次锥(冰淇淋锥):$Q = \{ (x, t) \in \mathbb{R}^n \times \mathbb{R} \mid \|x\| \leq t \}$(验证:若 $(x_1, t_1), (x_2, t_2) \in Q$,$\alpha_1, \alpha_2 \geq 0$,则 $\|\alpha_1 x_1 + \alpha_2 x_2\| \leq \alpha_1 \|x_1\| + \alpha_2 \|x_2\| \leq \alpha_1 t_1 + \alpha_2 t_2$,故 $\alpha_1(x_1, t_1) + \alpha_2(x_2, t_2) \in Q$)。
  \end{itemize}
\end{itemize}


\section{三类集合的核心对比}
为清晰梳理仿射集合、凸集与凸锥的差异与关联,下表从核心定义、组合类型、关键性质三个维度进行总结:

\begin{table}[h]
    \centering
    % 改为4列,调整列宽适配内容
    \begin{tabular}{p{2.5cm}p{3.5cm}p{3.5cm}p{4cm}}
    \toprule
    维度 & 仿射集合(Affine Set) & 凸集(Convex Set) & 凸锥(Convex Cone) \\
    \midrule
    核心定义 & 对仿射组合封闭 & 对凸组合封闭 & 对非负线性组合封闭 \\
    典型组合形式 & $\theta x + (1-\theta) y$($\theta \in \mathbb{R}$) & $\theta x + (1-\theta) y$($\theta \in [0,1]$) & $\alpha x + \beta y$($\alpha, \beta \geq 0$) \\
    与线性子空间关系 & 平移后的线性子空间 & 包含线性子空间的子集(可非平移) & 线性子空间是特殊凸锥(对任意 $\alpha \in \mathbb{R}$ 封闭) \\
    “最小包含集” & 仿射包(aff(S)) & 凸包(conv(S)) & 凸锥包(cone(S)) \\
    \bottomrule
    \end{tabular}
\end{table}

\section{重要的例子}
\subsection{超平面(Hyperplane)与半空间(Halfspace)}
超平面和半空间是由线性函数定义的基本集合,分别对应“线性等式约束”和“线性不等式约束”的解空间,是构建复杂凸集(如多面体)的基石。

\subsubsection{(1)超平面(Hyperplane)}
\paragraph{定义}
\begin{definition}[超平面]
设 $a \in \mathbb{R}^n$ 且 $a \neq 0$(非零法向量),$b \in \mathbb{R}$(常数),则 $\mathbb{R}^n$ 中的\textbf{超平面}定义为:
\begin{equation}
H = \{ x \in \mathbb{R}^n \mid a^T x = b \}
\end{equation}
\end{definition}

\paragraph{几何意义}
超平面是 $\mathbb{R}^n$ 中“维度为 $n-1$ 的仿射集合”,可理解为“与法向量 $a$ 垂直且到原点的‘距离’为 $|b|/\|a\|$ 的平面”。例如:

\begin{itemize}
  \item 当 $n=2$ 时,$H$ 是直线($a_1 x_1 + a_2 x_2 = b$);
  \item 当 $n=3$ 时,$H$ 是平面($a_1 x_1 + a_2 x_2 + a_3 x_3 = b$)。
\end{itemize}

\paragraph{性质:超平面是仿射集合}


\subsubsection{(2)半空间(Halfspace)}
\paragraph{定义}
\begin{definition}[闭半空间]
设 $a \in \mathbb{R}^n$ 且 $a \neq 0$,$b \in \mathbb{R}$,则 $\mathbb{R}^n$ 中的\textbf{闭半空间}(Closed Halfspace)定义为:
\begin{equation}
H_+ = \{ x \in \mathbb{R}^n \mid a^T x \geq b \}, \quad H_- = \{ x \in \mathbb{R}^n \mid a^T x \leq b \}
\end{equation}
若将不等式改为严格不等号($>$ 或 $<$),则称为\textbf{开半空间}(Open Halfspace)。
\end{definition}

\paragraph{几何意义}
半空间是超平面将 $\mathbb{R}^n$ 分割成的两个“半无限区域”,其中 $H_+$ 是法向量 $a$ 指向的一侧,$H_-$ 是相反侧。

\paragraph{性质:半空间是凸集(非仿射集)}

\subsubsection{(3)超平面与半空间的关系}
超平面是两个闭半空间的交集:$H = H_+ \cap H_-$;反之,每个闭半空间都是超平面的“一侧区域”,二者共同构成线性约束的几何表达。


\subsection{球(Ball)与椭球(Ellipsoid)}
球和椭球是基于“距离”定义的凸集,广泛用于建模“变量取值范围的约束”(如稳健优化中的不确定性集合)。

\subsubsection{(1)球(Euclidean Ball)}
\paragraph{定义}
\begin{definition}[欧氏球]
设 $x_0 \in \mathbb{R}^n$(中心),$r > 0$(半径),基于\textbf{欧氏范数}($\|x\|_2 = \sqrt{x_1^2 + \dots + x_n^2}$)的\textbf{闭球}定义为:
\begin{equation}
B(x_0, r) = \{ x \in \mathbb{R}^n \mid \|x - x_0\|_2 \leq r \}
\end{equation}
若将不等号改为 $<$,则称为\textbf{开球}。
\end{definition}

\paragraph{性质:球是凸集}

\subsubsection{(2)椭球(Ellipsoid)}
椭球是球的“仿射变换”,可描述更一般的“椭圆型区域”,在工程优化中常用于拟合数据分布或刻画变量波动范围。

\paragraph{定义}
\begin{definition}[椭球]
设 $x_0 \in \mathbb{R}^n$(中心),$P \in \mathbb{S}_{++}^n$(正定对称矩阵,控制椭球的形状与方向),$r > 0$(缩放因子),则\textbf{椭球}定义为:
\begin{equation}
\mathcal{E} = \{ x \in \mathbb{R}^n \mid (x - x_0)^T P^{-1} (x - x_0) \leq r^2 \}
\end{equation}

等价表达:通过仿射变换 $x = x_0 + r P^{1/2} z$(其中 $P^{1/2}$ 是 $P$ 的正定平方根,$z \in \mathbb{R}^n$),椭球可表示为球的像:
\begin{equation}
\mathcal{E} = \{ x_0 + r P^{1/2} z \mid \|z\|_2 \leq 1 \}
\end{equation}
当 $P = I$(单位矩阵)时,椭球退化为球 $B(x_0, r)$。
\end{definition}

\paragraph{性质:椭球是凸集}


\subsection{范数球(Norm Ball)与范数锥(Norm Cone)}
范数球和范数锥是基于“一般范数”的扩展集合,将球的“距离约束”与锥的“正齐次性”结合,是范数优化、锥优化的核心结构。

\subsubsection{(1)范数的回顾}
首先明确\textbf{范数}的定义:
\begin{definition}[范数]
函数 $\|\cdot\|: \mathbb{R}^n \to \mathbb{R}$ 称为范数,若对任意 $x, y \in \mathbb{R}^n$ 和 $\alpha \in \mathbb{R}$,满足:

\begin{enumerate}
  \item 非负性:$\|x\| \geq 0$,且 $\|x\| = 0 \iff x = 0$;
  \item 齐次性:$\|\alpha x\| = |\alpha| \|x\|$;
  \item 三角不等式:$\|x + y\| \leq \|x\| + \|y\|$。
\end{enumerate}

\end{definition}
常见范数包括欧氏范数($\|\cdot\|_2$)、1-范数($\|x\|_1 = \sum_{i=1}^n |x_i|$)、无穷范数($\|x\|_\infty = \max_{i=1,\dots,n} |x_i|$)等。


\subsubsection{(2)范数球(Norm Ball)}
\paragraph{定义}
\begin{definition}[范数球]
设 $\|\cdot\|$ 是 $\mathbb{R}^n$ 上的范数,$x_0 \in \mathbb{R}^n$,$r > 0$,则\textbf{范数球}定义为:
\begin{equation}
B_{\|\cdot\|}(x_0, r) = \{ x \in \mathbb{R}^n \mid \|x - x_0\| \leq r \}
\end{equation}

当 $x_0 = 0$ 且 $r = 1$ 时,称为\textbf{单位范数球}(Unit Norm Ball)。
\end{definition}
\paragraph{性质:范数球是凸集}

\paragraph{实例}

\begin{itemize}
  \item 1-范数球($\|x\|_1 \leq 1$):在 $\mathbb{R}^2$ 中是菱形,$\mathbb{R}^3$ 中是正八面体;
  \item 无穷范数球($\|x\|_\infty \leq 1$):在 $\mathbb{R}^2$ 中是正方形,$\mathbb{R}^3$ 中是正方体。
\end{itemize}


\subsubsection{(3)范数锥(Norm Cone)}
\paragraph{定义}
\begin{definition}[范数锥]
设 $\|\cdot\|$ 是 $\mathbb{R}^n$ 上的范数,定义\textbf{范数锥}(又称“冰淇淋锥”的推广)为:
\begin{equation}
K_{\|\cdot\|} = \{ (x, t) \in \mathbb{R}^n \times \mathbb{R} \mid \|x\| \leq t \}
\end{equation}

其中 $(x, t)$ 是 $\mathbb{R}^{n+1}$ 中的向量,$t$ 可理解为“范数的上界”。
\end{definition}
\paragraph{性质:范数锥是凸锥}


\paragraph{实例}

\begin{itemize}
  \item 二次锥(Quadratic Cone):当 $\|\cdot\| = \|\cdot\|_2$ 时,范数锥即为二次锥(又称Lorentz锥):
  \begin{equation}
  Q = \{ (x, t) \in \mathbb{R}^n \times \mathbb{R} \mid \|x\|_2 \leq t \}
  \end{equation}
  是锥优化中最常用的凸锥之一。
  
  \item 1-范数锥:$K_{\|\cdot\|_1} = \{ (x, t) \mid \sum_{i=1}^n |x_i| \leq t \}$,在 $\mathbb{R}^2 \times \mathbb{R}$ 中呈“四棱锥”形状。
\end{itemize}


\subsection{多面体(Polyhedron)与单纯形(Simplex)}
多面体是有限个线性等式与不等式约束的交集,是线性规划可行域的抽象;单纯形则是多面体的特殊情况,是“最低维度”的多面体,在数值优化中常用于构建搜索区域。

\subsubsection{(1)多面体(Polyhedron)}
\paragraph{定义}
\begin{definition}{多面体}
  多面体是有限个线性等式与不等式约束的交集,其数学表达式为: 
\begin{equation}
  \mathcal{P} = \{ x \in \mathbb{R}^n \mid A x \leq b, C x = d \}
\end{equation}
\end{definition}

其中:

\begin{itemize}
  \item $A \in \mathbb{R}^{m \times n},\; b \in \mathbb{R}^m$:对应 $m$ 个半空间约束($A x \leq b$ 即 $a_i^T x \leq b_i,\; i=1,\dots,m$)。
  \item $C \in \mathbb{R}^{p \times n},\; d \in \mathbb{R}^p$:对应 $p$ 个超平面约束($C x = d$ 即 $c_j^T x = d_j,\; j=1,\dots,p$)。
\end{itemize}

\paragraph{性质:多面体是凸集}

\paragraph{实例}

\begin{itemize}
  \item 正象限 $\mathbb{R}_+^n = \{ x \mid x_i \geq 0, i=1,\dots,n \}$:是多面体($A = -I$,$b = 0$);
  \item 线性规划的可行域:$X = \{ x \mid A x = b, x \geq 0 \}$,是多面体。
\end{itemize}


\subsubsection{(2)单纯形(Simplex)}
单纯形是“由 $n+1$ 个仿射无关点生成的凸包”,是维度为 $n$ 的“最简单”多面体(顶点数量最少的多面体)。

\begin{definition}{单纯形}
\paragraph{定义1(基于仿射无关点)}
设 $v_0, v_1, \dots, v_n \in \mathbb{R}^n$ 是\textbf{仿射无关}的点(即向量 $v_1 - v_0, \dots, v_n - v_0$ 线性无关),则由这些点生成的\textbf{单纯形}定义为:
\begin{equation}
\Delta = \text{conv}(v_0, v_1, \dots, v_n) = \left\{ \sum_{i=0}^n \theta_i v_i \mid \theta_i \geq 0, \sum_{i=0}^n \theta_i = 1 \right\}
\end{equation}

\paragraph{定义2(标准单纯形)}
最常用的是\textbf{标准单纯形}(以单位向量为顶点),定义为:
\begin{equation}
\Delta_n = \{ x \in \mathbb{R}^n \mid x_1 + x_2 + \dots + x_n = 1, x_i \geq 0, i=1,\dots,n \}
\end{equation}
\end{definition}
其顶点为 $e_1 = (1,0,\dots,0)^T, e_2 = (0,1,\dots,0)^T, \dots, e_n = (0,\dots,1)^T$($n$ 个标准单位向量),但注意:此处 $n$ 维标准单纯形的顶点数量为 $n$,与定义1中“$n$ 维单纯形需 $n+1$ 个顶点”的差异源于“是否包含原点”——若定义为 $\Delta_n' = \{ (x, 1 - \sum x_i) \mid x \in \Delta_n \}$,则顶点为 $e_1,\dots,e_n, 0$,共 $n+1$ 个,符合定义1。

\paragraph{关键性质}

\begin{enumerate}
  \item \textbf{维度}:由 $n+1$ 个仿射无关点生成的单纯形是 $n$ 维的(与空间维度一致)。
  
  \item \textbf{凸性}:单纯形是有限个点的凸包,而凸包是凸集(由凸包定义:所有凸组合的集合,自然对凸组合封闭),故单纯形是凸集。
  
  \item \textbf{多面体属性}:单纯形可表示为有限个线性等式与不等式的交集(如标准单纯形的约束 $x_1+\dots+x_n=1$ 和 $x_i\geq0$),因此是多面体。
\end{enumerate}

\paragraph{实例}

\begin{itemize}
  \item 1维单纯形:$\Delta_1 = \{ x \in \mathbb{R} \mid x = 1, x \geq 0 \}$ 即点 $\{1\}$;或扩展为 $\Delta_1' = \{ (x, 1-x) \mid x \geq 0 \}$ 即线段 $[0,1]$;
  
  \item 2维单纯形(标准):$\Delta_2 = \{ (x_1, x_2) \mid x_1 + x_2 = 1, x_1, x_2 \geq 0 \}$ 即连接 $(1,0)$ 和 $(0,1)$ 的线段;或扩展为 $\Delta_2' = \{ (x_1, x_2, x_3) \mid x_1+x_2+x_3=1, x_i\geq0 \}$ 即三角形(3个顶点);
  
  \item 3维单纯形:扩展形式为四面体(4个顶点)。
\end{itemize}
% 凸优化3(章节已存在,无需重复)

\chapter{凸优化的基本性质}
\section{凸优化问题的形式}

\subsection{一般优化模型}

一个一般优化问题的数学表达式为:
\begin{equation}
\min _{x \in \mathbb{R}^{n}} f_{0}(x) \quad \text{s.t.} \quad f_{i}(x) \leq 0, \, i=1, \ldots, m ; \quad h_{j}(x)=0, \, j=1, \ldots, p
\end{equation}
其中各部分含义如下:

\begin{itemize}
\item \textbf{目标函数}:$f_{0}(x)$,即需要最小化的函数;
\item \textbf{约束函数}:
  \begin{itemize}
  \item 不等式约束:$f_{i}(x) \leq 0$(共$m$个,$i$取值为1到$m$);
  \item 等式约束:$h_{j}(x) = 0$(共$p$个,$j$取值为1到$p$);
  \end{itemize}
\item \textbf{可行域}:满足所有约束条件的变量集合,定义为 $X = \{x \mid f_{i}(x) \leq 0, \, h_{j}(x) = 0\}$。
\end{itemize}

\subsection{凸优化的定义}

\begin{definition}[凸优化问题]
当且仅当满足以下三个条件时,一般优化问题是凸优化问题:

\begin{itemize}
\item 目标函数 $f_{0}(x)$ 为\textbf{凸函数};
\item 不等式约束函数 $f_{i}(x)$($i=1,\ldots,m$)均为\textbf{凸函数};
\item 等式约束函数 $h_{j}(x)$($j=1,\ldots,p$)均为\textbf{仿射函数},即满足形式 $h_{j}(x) = a_{j}^{T} x + b_{j}$(其中$a_{j}$为向量,$b_{j}$为常数)。
\end{itemize}
\end{definition}

此时,凸优化问题可简化描述为:\textbf{minimize a convex function over a convex set}(在凸集上最小化凸函数)。


\section{凸函数的定义以及性质}

\subsection{定义与几何意义}

\subsubsection{数学定义}
\begin{definition}
函数 $f: \mathbb{R}^n \to \mathbb{R}$ 是\textbf{凸函数},当且仅当对任意 $x_1, x_2 \in \text{dom}(f)$($\text{dom}(f)$表示函数$f$的定义域)和任意 $\theta \in [0,1]$,满足以下不等式:
\begin{equation}
f\left(\theta x_1 + (1-\theta) x_2\right) \leq \theta f\left(x_1\right) + (1-\theta) f\left(x_2\right)
\end{equation}
\end{definition}

\subsubsection{几何意义}

凸函数的曲线始终位于连接两点 $(x_1, f(x_1))$和 $(x_2, f(x_2))$的割线之下。

\begin{quote}
\textit{直观理解:在函数定义域内任取两点,两点间的割线不会低于函数曲线本身。}
\end{quote}

\subsection{Jensen 不等式}
Jensen 不等式是凸函数最重要的性质之一,具体表述如下:
\begin{theorem}[Jensen 不等式]
若 $f$为凸函数,$\{x_i\}$为 $\text{dom}(f)$内的任意点集,$\{\theta_i\}$为非负权重且满足 $\sum_{i} \theta_i = 1$,则:
\begin{equation}
f\left(\sum_{i} \theta_i x_i\right) \leq \sum_{i} \theta_i f\left(x_i\right)
\end{equation}
\end{theorem}

\subsubsection{应用场景}

Jensen 不等式不仅是判定函数凸性的重要依据,还广泛应用于:

\begin{itemize}
\item 概率论(如期望相关不等式推导);
\item 信息论(如 KL 散度、熵函数的性质分析);
\item 期望下界的证明。
\end{itemize}

\subsubsection{典型示例}

若 $f(x) = x^2$(已知为凸函数),对随机变量 $X$应用 Jensen 不等式,可得:
\[
(\mathbb{E}[X])^2 \leq \mathbb{E}[X^2]
\]
(其中 $\mathbb{E}[X]$表示 $X$的期望,$\mathbb{E}[X^2]$表示 $X^2$的期望)。

\subsection{一阶条件(First-order condition)}

\paragraph{适用前提}

函数 $f$可微(即梯度 $\nabla f(x)$在定义域内存在)。

\paragraph{判定准则}
\begin{proposition}
函数 $f$是凸函数,当且仅当对任意 $x, y \in \text{dom}(f)$,满足:
\begin{equation}
f(y) \geq f(x) + \nabla f(x)^T (y - x)
\end{equation}
\end{proposition}

\paragraph{几何与直观意义}

\begin{itemize}
\item 数学层面:函数在任意点 $x$处的切平面(或切线,当 $n=1$时)是函数的\textbf{全局下界};
\item 梯度意义:梯度 $\nabla f(x)$始终指向函数的上升方向。
\end{itemize}

\subsection{二阶条件(Hessian 判定)}

\paragraph{适用前提}

函数 $f$二阶可微(即 Hessian 矩阵 $\nabla^2 f(x)$在定义域内存在)。

\paragraph{判定准则}
\begin{proposition}
函数 $f$是凸函数,当且仅当对任意 $x \in \text{dom}(f)$,其 Hessian 矩阵满足\textbf{半正定}:
\begin{equation}
f \text{ 是凸的} \quad \Leftrightarrow \quad \nabla^2 f(x) \succeq 0, \, \forall x
\end{equation}
\end{proposition}

\paragraph{强凸函数的延伸判定}
\begin{definition}
当且仅当对任意 $x \in \text{dom}(f)$,Hessian 矩阵满足\textbf{正定}($\nabla^2 f(x) \succ 0$)时,函数 $f$为\textbf{强凸函数}(强凸是凸函数的更强形式)。
\end{definition}

\paragraph{强凸性定义}

\begin{definition}[强凸函数(等价定义)]
函数 $f$称为强凸函数,若存在常数 $\mu>0$,对所有 $x, y$:
\begin{equation}
f(y) \geq f(x)+\nabla f(x)^{T}(y-x)+\frac{\mu}{2}\| y-x\| ^{2}
\end{equation}

其中 $\mu$称为强凸系数。
\end{definition}

这意味着:
函数不仅“向上弯”,而且弯曲程度有下界;
强凸函数具有唯一最优解。

\begin{itemize}
\item 性质1:强凸函数有且仅有一个极小点;
\item 性质2:梯度法在强凸问题上以线性速率收敛
\end{itemize}

\paragraph{证明}

\begin{proof}[证明:强凸函数梯度法线性收敛]

\subparagraph{证明前提}

\begin{definition}[Lipschitz连续]
梯度Lipschitz连续(工程中常用假设):存在$L>0$,对所有$x,y$有$\|\nabla f(x) - \nabla f(y)\| \leq L\|x - y\|$。
\end{definition}

\begin{itemize}
\item 强凸函数$f$满足\textbf{梯度Lipschitz连续};
\item 梯度法迭代公式:$x_{k+1} = x_k - \alpha \nabla f(x_k)$,其中步长$\alpha$取$\alpha = \frac{1}{L}$(最优步长选择);
\item $x^*$为强凸函数的唯一极小点,故$\nabla f(x^*) = 0$。
\end{itemize}

\subparagraph{证明过程(核心推导)}

1. \textbf{展开迭代误差范数}:  
对$x_{k+1} = x_k - \alpha \nabla f(x_k)$,两边减去$x^*$得:  
\[x_{k+1} - x^* = (x_k - x^*) - \alpha \nabla f(x_k)\]  
取平方范数(利用$\|a - b\|^2 = \|a\|^2 - 2a^T b + \|b\|^2$):  
\begin{equation}
\|x_{k+1} - x^*\|^2 = \|x_k - x^*\|^2 - 2\alpha \nabla f(x_k)^T(x_k - x^*) + \alpha^2\|\nabla f(x_k)\|^2 \tag{1}
\end{equation}

2. \textbf{利用强凸性放缩梯度项}:  
对$x=x_k, y=x^*$应用强凸定义,代入$\nabla f(x^*) = 0$:  
\[f(x^*) \geq f(x_k) + \nabla f(x_k)^T(x^* - x_k) + \frac{\mu}{2}\|x^* - x_k\|^2\]  
由于$f(x^*) \leq f(x_k)$($x^*$是最优解),整理得:  
\begin{equation}
\nabla f(x_k)^T(x_k - x^*) \geq \frac{\mu}{2}\|x_k - x^*\|^2 \tag{2}
\end{equation}

3. \textbf{利用Lipschitz连续放缩梯度范数}:  
对$x=x_k, y=x^*$应用梯度Lipschitz连续,代入$\nabla f(x^*) = 0$:  
\[
\|\nabla f(x_k)\| \leq L\|x_k - x^*\| \implies \|\nabla f(x_k)\|^2 \leq L^2\|x_k - x^*\|^2 \tag{3}
\]

4. \textbf{代入迭代公式证线性收敛}:  
将(2)(3)代入(1),并取$\alpha = \frac{1}{L}$:  
\begin{align}
\|x_{k+1} - x^*\|^2 &\leq \|x_k - x^*\|^2 - 2\alpha \cdot \frac{\mu}{2}\|x_k - x^*\|^2 + \alpha^2 L^2\|x_k - x^*\|^2 \nonumber \\
&= \|x_k - x^*\|^2 \left(1 - \alpha \mu + \alpha^2 L^2\right) \nonumber \\
&= \|x_k - x^*\|^2 \left(1 - \frac{\mu}{L} + \frac{L^2}{L^2}\right) \nonumber \\
&= \|x_k - x^*\|^2 \left(1 - \frac{\mu}{L}\right)
\end{align}  
其中$0 < 1 - \frac{\mu}{L} < 1$(因$\mu < L$,强凸系数小于Lipschitz常数)。

\subparagraph{结论}

梯度法迭代误差满足$\|x_{k+1} - x^*\|^2 \leq \gamma \|x_k - x^*\|^2$($\gamma = 1 - \frac{\mu}{L}$为收敛因子),即\textbf{以线性速率收敛}。
\end{proof}


\section{全局与局部最优}

\subsection{定义回顾}

对于一般优化问题,局部最优解与全局最优解的定义如下:

\begin{definition}[局部与全局最优解]
\begin{itemize}
\item \textbf{局部最优解}:存在$\epsilon>0$,使得对所有满足$\|x - x^{*}\| < \epsilon$的可行点$x$,均有$f(x) \geq f(x^{*})$(即$x^{*}$在自身邻域内是最优的)。
\item \textbf{全局最优解}:对所有可行域内的点$x$,均有$f(x) \geq f(x^{*})$(即$x^{*}$在整个可行域内是最优的)。
\end{itemize}
\end{definition}

\subsection{凸优化的核心定理}

\begin{theorem}[凸优化局部最优即全局最优]
若目标函数$f$是凸函数,且优化问题的可行域$\mathcal{X}$是凸集,则\textbf{任何局部最优解都是全局最优解}。
\end{theorem}

\begin{proof}[证明(反证法)]
\begin{enumerate}
\item \textbf{假设前提}:假设$x^{*}$是局部最优解,但非全局最优解。根据全局最优解的定义,此时存在可行点$x' \in \mathcal{X}$,使得$f(x') < f(x^{*})$。

\item \textbf{构造凸组合}:定义凸组合$x_{\theta} = (1-\theta)x^{*} + \theta x'$,其中$\theta \in (0,1)$(即$x_{\theta}$是$x^{*}$与$x'$连线上的点)。

\item  \textbf{利用凸集性质}:由于可行域$\mathcal{X}$是凸集,根据凸集的定义,$x_{\theta} \in \mathcal{X}$(即$x_{\theta}$是可行点)。

\item \textbf{利用凸函数性质}:由于$f$是凸函数,根据凸函数的定义:  
\[
f(x_{\theta}) = f\left((1-\theta)x^{*} + \theta x'\right) \leq (1-\theta)f(x^{*}) + \theta f(x')
\]

\item  \textbf{推出矛盾}:结合假设$f(x') < f(x^{*})$,代入上式得:  
\[
f(x_{\theta}) \leq (1-\theta)f(x^{*}) + \theta f(x') < (1-\theta)f(x^{*}) + \theta f(x^{*}) = f(x^{*})
\]  
即$f(x_{\theta}) < f(x^{*})$。又因为当$\theta$足够小时,$x_{\theta}$满足$\|x_{\theta} - x^{*}\| = \theta\|x' - x^{*}\| < \epsilon$(符合局部最优解的邻域条件),这与$x^{*}$是局部最优解的定义矛盾。
\end{enumerate}
\end{proof}

\subsection{强凸函数的唯一最优性}

若$f$为$\mu$-强凸函数($\mu>0$为强凸系数),则:

\begin{itemize}
\item \textbf{最优解唯一性}:强凸函数的最优解有且仅有一个(不存在多个最优解);
\item \textbf{函数值差的二次下界}:距离最优点的函数值差满足二次下界关系(具体表现为$f(x) - f(x^{*}) \geq \frac{\mu}{2}\|x - x^{*}\|^2$)。
\end{itemize}

该性质为算法收敛性分析(如梯度下降法、牛顿法等)提供了重要的理论基础。


\section{凸优化的几何意义}

\subsection{可行域与等高线}

\begin{definition}[等高线与凸优化几何特征]
凸优化的几何特征具有明确的直观性:\textbf{目标函数的等高线(level set)与凸形可行域(convex feasible region)相切的点,即为全局最优点}。
\end{definition}

几何补充:等高线是目标函数值等于某一常数的点的集合(如二次函数的椭圆等高线);由于可行域是凸集,其边界呈“凸向外侧”的形态,二者相切时仅存在唯一接触点,该点即为整个可行域内使目标函数最小的点。

\subsection{法向条件(支撑超平面)}

\begin{definition}[支撑超平面]
从几何角度分析,凸优化问题的最优点 $x^*$处必然存在一个\textbf{支撑超平面(supporting hyperplane)},其数学表达式为:
\begin{equation}
\mathbf{a}^T \left( x - x^* \right) \geq 0, \quad \forall x \in \mathcal{X}
\end{equation}
其中:

\begin{itemize}
\item $\mathbf{a} = \nabla f(x^*)$,即目标函数 $f$在最优点 $x^*$处的梯度;
\item $\mathcal{X}$为优化问题的可行域。
\end{itemize}
\end{definition}

\subsubsection{支撑超平面的核心意义:}

\begin{itemize}
\item 几何层面:该超平面与凸集 $\mathcal{X}$在 $x^*$处“相切”,且凸集 $\mathcal{X}$完全位于超平面的一侧;
\item 优化层面:超平面的法向量(即梯度 $\nabla f(x^*)$)指向目标函数的上升方向,因此不存在从 $x^*$出发、指向可行域内部且能使目标函数值下降的方向;
\item 理论关联:该条件与KKT条件(Karush-Kuhn-Tucker条件)完全对应,本质是“梯度与约束法向一致”的几何体现。
\end{itemize}

\begin{quote}
\textit{KKT条件后面会讲到}
\end{quote}

\subsection{凸组合与最优性路径}

在凸优化问题中,全局最优点 $x^*$常可解释为\textbf{多个可行极值点的凸组合(convex combination)},具体表现为:  

以资源分配问题为例:若存在 $k$种可行的资源配置方案,每种方案对应可行域 $\mathcal{X}$内的一个点 $x_i$($i=1,2,\dots,k$),则最终的全局最优解 $x^*$一定位于这些点的\textbf{凸包(convex hull)}内。  

凸包:
\begin{equation}
\text{conv}\{x_1, x_2, \dots, x_k\} = \left\{ \sum_{i=1}^k \theta_i x_i \mid \theta_i \geq 0, \sum_{i=1}^k \theta_i = 1 \right\}
\end{equation}

\subsubsection{关键结论:}

凸包是包含所有点 $x_1, x_2, \dots, x_k$的最小凸集,全局最优解 $x^*$位于凸包内,体现了凸优化最优解的“折中性质”——最终解是多个局部可行方案的合理权衡,而非极端方案。

\begin{quote}
\textit{几何示例:若两种可行方案对应平面上的两个点,其凸包为两点间的线段,全局最优解即为线段上使目标函数最小的点。}
\end{quote}


\section{凸优化的转化以及示例}

\subsection{优化问题的等价变换以及实例}

\begin{definition}[等价变换条件]
等价变换需满足三个条件:\textbf{最优值相同、最优解可相互恢复、可行域一一对应}。
\end{definition}

\subsubsection{1. SVM 软间隔原问题 → Hinge 损失形式(最典型等价转化)}

\begin{itemize}
\item \textbf{原问题(含松弛变量)}:软间隔 SVM 原问题为  
\begin{equation}
\min _{w, b, \xi} \frac{1}{2}\| w\| ^{2}+C \sum_{i} \xi_{i}, \quad \text{s.t.} \quad y_{i}\left(w^{T} x_{i}+b\right) \geq 1-\xi_{i}, \xi_{i} \geq 0
\end{equation}  
该问题含变量 $w, b, \xi$,约束涉及松弛变量 $\xi_i$,求解需同时处理三类变量。

\item \textbf{等价转化:Hinge 损失形式}:通过 “松弛变量消去”的等价思路,将原问题转化为仅含 $w$ 的无约束问题:  
\begin{equation}
\min _{w} \frac{1}{2}\| w\| ^{2}+C \sum_{i} \max \left(0,1-y_{i} w^{T} x_{i}\right)
\end{equation}  
其中 $\max(0,1-y_i w^T x_i)$ 为 Hinge 损失,本质是用“损失项”隐式替代松弛变量 $\xi_i$($\xi_i \geq \max(0,1-y_i w^T x_i)$,最小化目标时二者取等)。

\item \textbf{等价性验证}:  

  \begin{itemize}
  \item 最优值相同:原问题通过 $\xi_i$ 控制“间隔违反程度”,Hinge 损失直接量化该程度,最小化目标的核心逻辑一致;  
  \item 最优解可恢复:从 Hinge 损失的最优解 $w^*$,可反向计算 $b^*$(如通过支持向量满足 $y_i(w^{*T}x_i + b^*) = 1$),松弛变量 $\xi_i^* = \max(0,1-y_i w^{*T}x_i)$;  
  \item 可行域一一对应:原问题的可行域($\xi_i \geq 0$ 且 $y_i(w^T x_i + b) \geq 1-\xi_i$)与 Hinge 损失的“隐含可行域”(无显式约束,但损失项确保等价约束)完全对应 。
  \end{itemize}

\item \textbf{实例价值}:转化后无需处理松弛变量 $\xi$,将带约束问题简化为无约束凸优化,可直接用梯度法求解,同时保持 SVM“最大间隔分类”的核心逻辑。
\end{itemize}
\subsubsection{2. 逻辑回归:似然最大化 → 负对数似然最小化(单调变换等价)}

\begin{itemize}
\item \textbf{原问题(非凸形式)}:逻辑回归的核心是“最大化样本似然概率”,似然函数为 $\prod_{i=1}^m \sigma(y_i w^T x_i)$($\sigma(\cdot)$ 为 Sigmoid 函数),该函数是乘积形式,非凸且求解困难。

\item \textbf{等价转化:凸形式}:利用对数函数的单调性($\log(\cdot)$ 单调递增,最大化 $f$ 等价于最大化 $\log f$),再通过“取负”将最大化问题转化为最小化问题,最终目标函数为:  
\end{itemize}
\begin{equation}
\min_w -\sum_{i=1}^m \log\left(\sigma(y_i w^T x_i)\right)
\end{equation}  
该“负对数似然函数为凸函数”。

\begin{itemize}
\item \textbf{等价性验证}:由于 $\log(\cdot)$ 和“取负”均为单调变换,原问题的最优解 $w^*$ 与转化后问题的最优解完全一致,最优值仅相差常数倍(对数与负号的影响),满足等价变换的三个条件 。

\item \textbf{实例价值}:将非凸的乘积似然转化为凸的加法函数,可通过梯度法、牛顿法高效求解,且保障解为全局最优。
\end{itemize}

\subsubsection{3. 最小二乘回归:隐含约束 → 无约束凸问题(可行域等价)}

- \textbf{原问题(隐含约束)}:最小二乘的目标是 
\begin{equation}
\min_w \|Xw - y\|^2
\end{equation} 
其“可行域”本质是“所有使误差有定义的 $w$”(即全空间 $\mathbb{R}^n$),无显式约束。
\begin{itemize}
\item \textbf{等价转化:无约束凸问题}:无需引入额外变量,直接利用凸函数判定条件——目标函数的 Hessian 矩阵为 $2X^T X \succeq 0$(半正定),因此是凸函数。此时问题等价于“在凸集(全空间)上最小化凸函数”,完全符合凸优化的定义。

\item \textbf{实例价值}:通过“可行域等价”避免复杂约束处理,直接用闭式解($w^* = (X^T X)^{-1} X^T y$)或梯度法求解,是机器学习中最易实现的凸优化实例之一。
\end{itemize}
\subsection{非凸到凸的重构思路在实例中的延伸(解决“非凸建模难题”)}

四类非凸到凸的重构技术:\textbf{函数松弛、对偶化、变量替换、线性化},这些思路在实例中被灵活应用,让原本非凸的模型具备凸优化的“全局最优性”。

\subsubsection{1. 函数松弛:Hinge 损失替代 0-1 损失(SVM 中的非凸近似)}

SVM 的 Hinge 损失形式,本质是“函数松弛”技术的落地——用凸上界替代非凸项:

\begin{itemize}
\item \textbf{非凸痛点}:分类任务的理想损失是“0-1 损失”(正确分类损失为 0,错误分类损失为 1),但 0-1 损失是阶梯函数,非凸且不可微,无法用于凸优化建模。
\item \textbf{凸松弛方案}:根据“函数松弛”思路,用 Hinge 损失($\max(0,1-y_i w^T x_i)$)作为 0-1 损失的\textbf{凸上界}——对所有 $w$,均有 $\max(0,1-y_i w^T x_i) \geq \mathbb{I}(y_i w^T x_i < 1)$($\mathbb{I}$ 为指示函数,即 0-1 损失) 。
\item \textbf{实例应用}:SVM 选择 Hinge 损失作为目标项,既保留“惩罚间隔违反样本”的核心逻辑,又通过“凸松弛”让目标函数成为凸函数 。此时问题转化为凸优化,保障全局最优,避免 0-1 损失导致的“局部最优陷阱”。
\end{itemize}

\subsubsection{2. 对偶化:Lasso 回归的对偶问题(处理不可微凸项)}

Lasso 回归的求解,依赖 “对偶化”技术——通过拉格朗日对偶将原问题转化为更易求解的凸问题:

\begin{itemize}
\item \textbf{原问题痛点}:Lasso 回归的目标函数为 $\min_w \|Xw - y\|^2 + \lambda\|w\|_1$,其中 $\|w\|_1$ 是凸函数但不可微(在 $w_i=0$ 处无梯度),直接用梯度法求解困难 。

\item \textbf{对偶化方案}:根据 “对偶化”思路,构造原问题的拉格朗日对偶问题,给出 Lasso 对偶问题为:  


\begin{equation}
\max _{\alpha} \sum_{i} \alpha_{i}-\frac{1}{2} \sum_{i, j} \alpha_{i} \alpha_{j} y_{i} y_{j}\left(x_{i}^{T} x_{j}\right), \quad \text{s.t.} \quad 0 \leq \alpha_{i} \leq C, \sum_{i} \alpha_{i} y_{i}=0
\end{equation}  
该对偶问题是\textbf{凸二次规划}(目标函数为二次函数,Hessian 半正定;约束均为线性,可行域为凸集) 。


\item \textbf{实例价值}:对偶问题虽与原问题变量不同(从 $w$ 变为 $\alpha$),但满足 Slater 条件(对偶间隙为 0),最优值与原问题相同,且对偶问题可通过成熟的二次规划算法求解(无需处理不可微性)。求解后再通过 $w^* = \sum_{i} \alpha_i^* y_i x_i$ 恢复原问题最优解,同时保留 Lasso“稀疏性”的核心特性 。
\end{itemize}

\subsubsection{3. 线性化:投资组合优化的非线性约束处理(Markowitz 模型延伸)}

\begin{itemize}
\item \textbf{基础模型(凸二次规划)}:标准 Markowitz 模型为 
\begin{equation}
\min_w \frac{1}{2}w^T \Sigma w - \mu^T w, \quad \text{s.t.} \quad 1^T w = 1, \, w_i \geq 0
\end{equation} 
由于 $\Sigma$(协方差矩阵)半正定,目标函数凸,可行域凸,属于凸二次规划 。
\item \textbf{非凸扩展痛点}:实际投资中常存在非线性约束(如交易成本为 $w_i^2$,或最小持仓比例为非线性函数),这些约束会破坏可行域的凸性,导致模型非凸。
\item \textbf{线性化方案}:根据“线性化”思路,对非线性约束进行“局部线性近似”或“分段线性逼近(PWL)”——例如将交易成本 $w_i^2$ 在可行域内分段用线性函数近似,每个分段内约束为线性,整体可行域仍为凸集 。
\item \textbf{实例价值}:线性化后模型仍保持“目标函数凸+可行域凸”的特性,保障全局最优,解决了“实际投资约束下模型不可解”的问题,同时让最优解仍落在“有效前沿”上 。
\end{itemize}
\subsection{转化逻辑的核心价值}

\textbf{所有实例均通过“等价变换”或“非凸重构”,最终满足凸优化的定义(目标函数凸+可行域凸),进而利用 “局部最优即全局最优”的定理,实现“高效求解”与“结果可靠”的双重目标}。具体可总结为三类转化路径:

\begin{itemize}
\item \textbf{复杂凸 → 简洁凸}:如 SVM 原问题→Hinge 损失(消去松弛变量)、最小二乘→无约束问题(简化可行域);
\item \textbf{非凸 → 凸近似}:如 0-1 损失→Hinge 损失(函数松弛)、非线性约束→分段线性约束(线性化);
\item \textbf{难求解凸 → 易求解凸}:如 Lasso 原问题→对偶问题(处理不可微性)。
\end{itemize}
\chapter{无约束优化问题}

\numberwithin{equation}{subsection}

\section{无约束优化问题}

\subsection{问题基本形式}
\begin{definition}[无约束优化问题]
针对\textbf{凸且二阶可微}的目标函数,无约束优化问题的数学形式定义为:
\begin{equation}
\min f(x)
\end{equation}
其中,$f(x)$满足“凸性”与“二阶可微性”这两个核心前提假设。
\end{definition}

\subsection{最优性条件}
\begin{proposition}[最优性条件]
无约束优化问题达到最优解$x^*$的核心必要条件为:
\begin{equation}
\nabla f(x^*) = 0
\end{equation}
即最优解处函数的梯度(一阶导数)等于零向量;若该等式无法直接求解,则需通过迭代方法逼近最优解。
\end{proposition}

\subsection{核心求解框架(下山法迭代格式)}
下山法通过迭代构造严格递减的函数值序列以逼近最优解,其数学化迭代流程如下:
\begin{enumerate}
    \item \textbf{初始设定}:给定初始迭代点$x^{(0)}$,初始化迭代次数$k=0$;
    \item \textbf{方向选取}:确定第$k$次迭代的搜索方向$\Delta x^{(k)}$(后续需进一步优化方向选取规则);
    \item \textbf{步长选取}:确定第$k$次迭代的步长$\alpha > 0$(后续需进一步优化步长选取规则);
    \item \textbf{迭代更新}:按以下公式更新迭代点:
    \begin{equation}
    x^{(k+1)} = x^{(k)} + \alpha \Delta x^{(k)}
    \end{equation}
    同时更新迭代次数$k \leftarrow k+1$,且需满足函数值严格递减条件:
    \begin{equation}
    f(x^{(0)}) > f(x^{(1)}) > f(x^{(2)}) > \dots
    \end{equation}
\end{enumerate}

综上,无约束优化问题的核心数学转化目标为:通过数学规则确定“搜索方向$\Delta x^{(k)}$”与“步长$\alpha$”,使上述迭代流程满足严格递减性并收敛至最优解。

\section{搜索方向的确定}

\subsection{视角1:线性化与下降条件}

\begin{definition}[梯度L-Lipschitz条件]
若函数$f$的梯度满足L-Lipschitz连续性,等价于:
\begin{equation}
\| \nabla f(x)-\nabla f(y)\| _{2} \leq L\| x-y\| _{2}
\end{equation}
其中$L$为Lipschitz常数,$\|\cdot\|_2$表示欧氏范数。
\end{definition}

\begin{proposition}[函数线性化不等式]
利用基本定理与Cauchy–Schwarz公式,可推导出函数在点$x$处的线性化上界:
\begin{equation}
f(x+\Delta) \leq f(x)+\nabla f(x)^{\top} \Delta+\frac{L}{2}\| \Delta\| _{2}^{2}
\end{equation}
\end{proposition}

\begin{proof}[推导:函数线性化不等式]
\textbf{梯度L-Lipschitz连续性}(推导的基础,保证梯度变化有界):
\begin{equation}
\|\nabla f(x) - \nabla f(x + t\Delta)\|_2 \leq L \cdot \|t\Delta\|_2 = L t \|\Delta\|_2 \quad (t \in [0,1])
\end{equation}

\textbf{多元微积分基本定理}:
\begin{equation}
f(x+\Delta) - f(x) = \int_{0}^{1} \nabla f\left(x + t\Delta\right)^{\top} \Delta \, dt
\end{equation}

\textbf{积分拆分与线性项提取}:
\begin{equation}
\int_{0}^{1} \nabla f(x + t\Delta)^{\top}\Delta dt = \nabla f(x)^{\top}\Delta \cdot \int_{0}^{1} dt + \int_{0}^{1} \left[\nabla f(x + t\Delta) - \nabla f(x)\right]^{\top}\Delta dt
\end{equation}
其中第一部分积分结果直接为线性项:
\begin{equation}
\nabla f(x)^{\top}\Delta \cdot \int_{0}^{1} dt = \nabla f(x)^{\top}\Delta
\end{equation}

\textbf{Cauchy-Schwarz不等式+L-Lipschitz条件}(控制第二部分积分,得到“二次项$\frac{L}{2}\|\Delta\|_2^2$”):
由Cauchy-Schwarz不等式:
\begin{equation}
\left[\nabla f(x + t\Delta) - \nabla f(x)\right]^{\top}\Delta \leq \|\nabla f(x + t\Delta) - \nabla f(x)\|_2 \cdot \|\Delta\|_2
\end{equation}
代入L-Lipschitz条件并积分:
\begin{equation}
\int_{0}^{1} \|\nabla f(x + t\Delta) - \nabla f(x)\|_2 \cdot \|\Delta\|_2 dt \leq \int_{0}^{1} L t \|\Delta\|_2^2 dt = \frac{L}{2}\|\Delta\|_2^2
\end{equation}

\textbf{合并得到最终不等式}:
\begin{equation}
f(x+\Delta) - f(x) \leq \nabla f(x)^{\top}\Delta + \frac{L}{2}\|\Delta\|_2^2 \implies f(x+\Delta) \leq f(x) + \nabla f(x)^{\top}\Delta + \frac{L}{2}\|\Delta\|_2^2
\end{equation}
\end{proof}

\begin{definition}[下降方向]
令$\Delta=\alpha p$($\alpha>0$为步长,$p$为搜索方向),只要满足特定条件,就存在足够小的$\alpha>0$使函数值降低——这是下降方向的充要条件。其中,最自然的搜索方向选择为\textbf{负梯度方向}:
\begin{equation}
p_{gd}=-\nabla f(x)
\end{equation}
\end{definition}

\subsection{视角2:一般范数下的“最速下降”}

\begin{definition}[最速下降方向]
对任意范数$\|\cdot\|$及其对应的对偶范数$\|\cdot\|_*$,“最速下降方向”需通过求解以下优化问题得到:
\begin{equation}
p^{*}=\arg \min _{\| d\| _{*} \leq 1} \nabla f(x)^{\top} d
\end{equation}
上述优化问题的解为:
\begin{equation}
p^{*}=-\frac{\nabla f(x)}{\| \nabla f(x)\| }
\end{equation}
\end{definition}

\begin{example}[特殊范数案例]
\begin{itemize}
    \item 当使用\textbf{欧氏范数}时,对偶范数与原范数一致,此时$p^*$即为负梯度方向(与前文中结论一致);
    \item 若使用\textbf{Hessian度量}(定义为$\|d\|_{H(x)}=\sqrt{d^{\top} H(x) d}$,$H(x)$为函数$f$在$x$处的Hessian矩阵),则“最速”方向等价于Newton步(详见后续Newton法相关内容)。
\end{itemize}
\end{example}

\begin{quote}
\textit{解释:}
函数$f$在点$x$沿方向$d$的\textbf{局部下降速度},由梯度与方向的内积$\nabla f(x)^\top d$决定:
\begin{itemize}
    \item 内积越小(越负),函数沿$d$下降越快;
    \item 因此“找最速下降方向”,等价于在约束$\|d\|_* \leq 1$(对偶范数单位球)下,求解:
    \begin{equation}
    p^* = \arg\min_{\|d\|_* \leq 1} \nabla f(x)^\top d
    \end{equation}
\end{itemize}
设$a = \nabla f(x)$,需先确定$a^\top d$(即$\nabla f(x)^\top d$)的最小可能值。
根据\textbf{对偶范数的核心性质}:对任意满足$\|d\|_* \leq 1$的$d$,有:
\begin{equation}
a^\top d \geq -\|a\|
\end{equation}
方向$p^* = -\frac{\nabla f(x)}{\|\nabla f(x)\|}$,既满足对偶范数单位球约束,又能让内积达到最小(下降最快),因此它就是单位球上的最速下降方向。
\end{quote}

\subsection{预条件化的作用}

\textbf{最速下降方向是 “当前范数下,使内积$\nabla f(x)^\top d$最小(下降最快)的方向”}。预条件化的本质是\textbf{改变 “范数度量标准”}:不再用欧氏范数,而是用 “预条件范数”。

\begin{definition}[预条件化]
考虑二次目标函数 $f(x)=\frac{1}{2} x^{\top} A x-b^{\top} x \quad (A \succ 0)$,其等高线为椭圆。
通过坐标变换$y=A^{1/2} x$,可将原椭圆等高线“拉成”圆形(即消除椭圆的“细长”特性),这一过程称为\textbf{预条件化}。
\end{definition}

\textbf{直观意义}:预条件化相当于将优化问题中的“细长谷”地形转化为“圆形洼地”,从而缓解负梯度下降时容易出现的“楼梯形”迂回路径,提升迭代效率。

\section{如何确定步长}

设线搜索的核心目标函数为$\phi(\alpha) = f(x + \alpha p)$,其中$x$为当前迭代点,$p$为已确定的搜索方向,$\alpha \geq 0$为待求解的步长(需满足函数值下降条件$f(x + \alpha p) < f(x)$)。以下分别介绍三种主流线搜索方法:

\subsection{精确线搜索(Exact Line Search)}

\begin{definition}[精确线搜索]
精确线搜索直接求解“使$\phi(\alpha)$最小化”的步长$\alpha^*$,数学表达为:
\begin{equation}
\alpha^{*} = \arg \min _{\alpha \geq 0} \phi(\alpha)
\end{equation}
其目标是找到“当前方向下最优的步长”,理论上能让单次迭代的函数值下降幅度最大。
\end{definition}

\begin{example}[二次函数的封闭解]
若目标函数为二次函数$f(x) = \frac{1}{2} x^\top A x - b^\top x$(其中$A \succ 0$,即$A$为正定矩阵),且搜索方向$p = -\nabla f(x) = -g$($g = \nabla f(x)$为当前梯度),则精确线搜索的步长有\textbf{封闭解}:
\begin{equation}
\alpha^{*} = \frac{g^\top g}{g^\top A g}
\end{equation}
该解的本质是:在二次函数的近似下,使“更新后的梯度$\nabla f(x + \alpha p)$与搜索方向$p$垂直”(即一次更新在二次近似意义上最有效),等价于求解$\arg \min _{\alpha} \|g - \alpha A g\|_2^2$。
\end{example}

\textbf{与固定步长的比较}:在函数满足 \textbf{L-光滑性}(梯度L-Lipschitz连续)的前提下,精确线搜索得到的$\alpha^*$至少不劣于固定步长$\alpha = 1/L$(固定步长仅能保证“函数值下降”,但无法保证下降幅度最优)。

\subsection{黄金分割(Golden Section Search)}

\textbf{适用场景}:当目标函数$\phi(\alpha)$是\textbf{单峰函数}(即区间内仅有一个最小值点),且计算梯度(或导数$\phi'(\alpha)$)代价高、难度大时,采用黄金分割法通过“区间收缩”逼近最优步长$\alpha^*$。

\textbf{核心流程}:
\begin{enumerate}
    \item \textbf{初始区间设定}:确定初始搜索区间$[a, b]$,满足$\phi(0) < \phi(b)$(保证最小值点在区间内,因$\alpha=0$对应当前点,函数值最大),初始令$a=0$;
    \item \textbf{区间收缩规则}:
    \begin{itemize}
        \item 在区间$[a, b]$内选取两个对称点$t_1$和$t_2$($a < t_1 < t_2 < b$),两点间距与区间总长的比例为“黄金分割系数”$c = \frac{1}{2}(\sqrt{5} - 1) \approx 0.618$,即:
        \begin{equation}
        t_1 = a + (1 - c)(b - a), \quad t_2 = a + c(b - a)
        \end{equation}
        \item 比较函数值:
        \begin{itemize}
            \item 若$\phi(t_2) > \phi(t_1)$:说明最小值点$\alpha^* \in [a, t_2]$,令新区间为$[a, t_2]$;
            \item 若$\phi(t_1) > \phi(t_2)$:说明最小值点$\alpha^* \in [t_1, b]$,令新区间为$[t_1, b]$;
        \end{itemize}
    \end{itemize}
    \item \textbf{迭代收敛}:重复步骤2,不断收缩区间,直到区间长度小于预设精度。最终区间内的任意点均可作为$\alpha^*$的近似值,收敛误差上界与$0.618^k$成正比($k$为迭代次数)。
\end{enumerate}

\textbf{特点}:优点是无需计算梯度/导数,仅通过函数值比较即可迭代;缺点是收敛速度较慢(线性收敛),仅适用于单峰函数。

\subsection{回溯搜索(Backtracking Line Search)与 Armijo–Wolfe 准则}

回溯搜索通过“先试后调”的方式确定步长,需满足两个核心条件(保证步长既“足够大”以加速收敛,又“足够小”以保证函数值下降):

\begin{definition}[Armijo 条件(充分下降条件)]
确保步长能使函数值显著下降,数学表达为:
\begin{equation}
f(x + \alpha p) \leq f(x) + c_1 \alpha \nabla f(x)^\top p
\end{equation}
其中$0 < c_1 < 1$(通常取$c_1 = 10^{-4}$),$\nabla f(x)^\top p < 0$(因$p$为下降方向),右边项为函数值的“预期下降下限”。
\end{definition}

\begin{definition}[Wolfe 条件(曲率条件)]
确保步长不会过小(避免收敛过慢),数学表达为:
\begin{equation}
\nabla f(x + \alpha p)^\top p \geq c_2 \nabla f(x)^\top p
\end{equation}
其中$c_1 < c_2 < 1$(通常取$c_2 = 0.9$),该条件要求“更新后的梯度与搜索方向的内积”不小于“初始梯度与搜索方向内积”的$c_2$倍,避免步长停留在“函数值下降缓慢的区域”。
\end{definition}

\textbf{回溯搜索算法流程}:
给定后退因子$\beta \in (0, 1)$(通常取$\beta = 0.5$或$0.8$),步骤如下:
\begin{enumerate}
    \item \textbf{初始步长尝试}:令初始步长$\alpha \leftarrow 1$(默认从“单位步长”开始,适配 Newton 法等需要大步长的场景);
    \item \textbf{条件判断与步长调整}:若当前$\alpha$不满足 Armijo 条件( Armijo–Wolfe 联合条件),则按比例缩小步长:$\alpha \leftarrow \beta \alpha$;
    \item \textbf{终止}:重复步骤2,直到$\alpha$满足预设条件,输出最终步长$\alpha$。
\end{enumerate}

\textbf{关键性质与应用}:
\begin{itemize}
    \item \textbf{终止性}:由 Descent Lemma 可证明:当$\alpha$足够小时,Armijo 条件必成立,因此回溯搜索一定能终止;
    \item \textbf{与 Newton 法的结合(两阶段收敛)}:
    \begin{itemize}
        \item 阶段 I(远离最优解时):$\alpha < 1$,通过回溯调整步长进入“可接受域”(满足 Armijo–Wolfe 条件);
        \item 阶段 II(靠近最优解时):步长会触发$\alpha = 1$(单位步长),此时 Newton 法可实现二次收敛(收敛速度远快于梯度下降)。
    \end{itemize}
\end{itemize}

\section{收敛率:强凸 / PL 条件与“楼梯现象”}

\subsection{强凸 + L-光滑:线性收敛}

\begin{definition}[强凸与L-光滑]
目标函数$f(x)$需同时满足两大性质:
\begin{enumerate}
    \item \textbf{强凸性}:存在常数$\mu > 0$,对任意迭代点$x$,其Hessian矩阵(二阶导数矩阵)满足下界约束:
    \begin{equation}
    \mu I \preceq \nabla^2 f(x)
    \end{equation}
    (“强凸”保证函数有唯一最小值点,且函数形态“下凸程度”可控);
    \item \textbf{L-光滑性}:存在常数$L > 0$,对任意迭代点$x$,其Hessian矩阵满足上界约束:
    \begin{equation}
    \nabla^2 f(x) \preceq L I
    \end{equation}
    (“L-光滑”保证函数曲率不超过阈值,梯度变化平缓,避免局部剧烈波动)。
\end{enumerate}
\end{definition}

\begin{theorem}[线性收敛性]
当步长取$\alpha = 1/L$时,梯度下降迭代满足严格的线性收敛性质:
\begin{enumerate}
    \item 函数值下降界(每次迭代函数值必递减且幅度可控):
    \begin{equation}
    f(x_{k+1}) \leq f(x_k) - \frac{1}{2L} \|\nabla f(x_k)\|_2^2
    \end{equation}
    \item 迭代点误差界($x^*$为函数最优解,与最优解的距离按固定比例缩小):
    \begin{equation}
    \|x_{k+1} - x^*\|_2^2 \leq \left(1 - \frac{\mu}{L}\right) \|x_k - x^*\|_2^2
    \end{equation}
\end{enumerate}
\end{theorem}

\begin{proof}[证明:线性收敛性]
\textbf{一、明确核心前提与已有结论}

在证明前,需明确2个关键性质(强凸+L-光滑)的推论,及1个已证结论:

1. \textbf{强凸性的核心推论}:
若函数$f$强凸(存在$\mu>0$,使$\mu I \preceq \nabla^2 f(x)$),则对任意迭代点$x$与最优解$x^*$(满足$\nabla f(x^*)=0$),有:
\begin{equation}
f(x) - f(x^*) \geq \frac{\mu}{2}\|x - x^*\|_2^2 \tag{1}
\end{equation}
(强凸性保证“函数值与最优值的差距”不小于“迭代点与最优解距离平方”的固定倍数,建立误差与函数值差的关联)

同时,强凸性还可推出“梯度范数与函数值差的关系”:
因为$\nabla f(x^*)=0$,
\begin{equation}
f(x^*)\ge f(x)+\nabla f(x)^{\top}(x^*-x)+\frac{\mu}{2}|x^*-x|^2.
\end{equation}
移项:
\begin{equation}
f(x)-f(x^*)\le\nabla f(x)^{\top}(x-x^*)-\frac{\mu}{2}|x-x^*|^2. \tag{B}
\end{equation}
由 Cauchy–Schwarz:
\begin{equation}
\nabla f(x)^{\top}(x-x^*) \le |\nabla f(x)|\cdot|x-x^*|.
\end{equation}
令右侧关于$|x-x^*|$的表达最小化,可视为二次函数
\begin{equation}
|\nabla f(x)|\cdot|x-x^*|-\frac{\mu}{2}|x-x^*|^2.
\end{equation}
其最大值出现在$|x-x^*|=\frac{|\nabla f(x)|}{\mu}$,代入得
\begin{equation}
f(x)-f(x^*)\le \frac{1}{2\mu}|\nabla f(x)|^2.
\end{equation}
即:
\begin{equation}
\|\nabla f(x)\|_2^2 \geq 2\mu(f(x) - f(x^*)) \tag{2}
\end{equation}
(梯度大小能反映函数值与最优值的差距,为后续替换梯度项做准备)

2. \textbf{已证的函数值下降界}:
由L-光滑性(梯度L-Lipschitz连续)及步长$\alpha=1/L$,已证明函数值满足:
\begin{equation}
f(x_{k+1}) \leq f(x_k) - \frac{1}{2L}\|\nabla f(x_k)\|_2^2 \tag{3}
\end{equation}

\textbf{二、步骤1:推导函数值差的线性衰减关系}

将式(3)变形为“相邻迭代的函数值差下界”:
\begin{equation}
f(x_k) - f(x_{k+1}) \geq \frac{1}{2L}\|\nabla f(x_k)\|_2^2
\end{equation}
将强凸性推论式(2)($\|\nabla f(x_k)\|_2^2 \geq 2\mu(f(x_k) - f(x^*))$)代入上式,替换梯度范数项:
\begin{equation}
f(x_k) - f(x_{k+1}) \geq \frac{1}{2L} \cdot 2\mu(f(x_k) - f(x^*))
\end{equation}
化简后得到“函数值差的衰减关系”:
\begin{equation}
f(x_k) - f(x_{k+1}) \geq \frac{\mu}{L}(f(x_k) - f(x^*))
\end{equation}
进一步整理,将函数值差聚焦到“与最优值的差距”:
\begin{equation}
f(x_{k+1}) - f(x^*) \leq f(x_k) - f(x^*) - \frac{\mu}{L}(f(x_k) - f(x^*))
\end{equation}
\begin{equation}
f(x_{k+1}) - f(x^*) \leq \left(1 - \frac{\mu}{L}\right)\left(f(x_k) - f(x^*)\right) \tag{4}
\end{equation}
式(4)表明:迭代中“函数值与最优值的差距”按$(1 - \mu/L)$的比例线性衰减。

\textbf{三、步骤2:将函数值差衰减转化为迭代误差衰减}

利用强凸性推论式(1),分别对$x_{k+1}$和$x_k$建立“迭代误差与函数值差的关联”:

- 对$x_{k+1}$:
\begin{equation}
\|x_{k+1} - x^*\|_2^2 \leq \frac{2}{\mu}\left(f(x_{k+1}) - f(x^*)\right) \tag{5}
\end{equation}
(由式(1)变形:两边同乘$2/\mu$,不等号方向不变)

- 对$x_k$:
\begin{equation}
f(x_k) - f(x^*) \geq \frac{\mu}{2}\|x_k - x^*\|_2^2 \implies \frac{2}{\mu}\left(f(x_k) - f(x^*)\right) \geq \|x_k - x^*\|_2^2 \tag{6}
\end{equation}

\textbf{四、步骤3:合并推导得到迭代误差界}

将式(4)(函数值差衰减)代入式(5),再结合式(6)(函数值差与$x_k$误差的关联):
\begin{equation}
\|x_{k+1} - x^*\|_2^2 \leq \frac{2}{\mu} \cdot \left(1 - \frac{\mu}{L}\right)\left(f(x_k) - f(x^*)\right)
\end{equation}
由式(6)可知$\frac{2}{\mu}\left(f(x_k) - f(x^*)\right) \geq \|x_k - x^*\|_2^2$,因此:
\begin{equation}
\|x_{k+1} - x^*\|_2^2 \leq \left(1 - \frac{\mu}{L}\right) \cdot \frac{2}{\mu}\left(f(x_k) - f(x^*)\right) \leq \left(1 - \frac{\mu}{L}\right)\|x_k - x^*\|_2^2
\end{equation}

\textbf{五、结论}

最终证得迭代误差界公式:
\begin{equation}
\| x_{k+1}-x^{*}\| _{2}^{2} \leq\left(1-\frac{\mu}{L}\right)\left\| x_{k}-x^{*}\right\| _{2}^{2}
\end{equation}
该公式表明:强凸+L-光滑条件下,梯度下降的迭代误差按$(1 - \mu/L)$的线性因子衰减,收敛速度由条件数$\kappa=L/\mu$决定——$\kappa$越大,$(1 - 1/\kappa)$越接近1,误差衰减越慢,且易因等高线“细长”出现“楼梯形”迂回路径。
\end{proof}

\begin{definition}[条件数]
定义条件数$\kappa = L/\mu$(L与μ的比值),线性收敛的“衰减速度”由$\kappa$决定:
\begin{itemize}
    \item $\kappa$越小(L与μ接近):衰减因子$(1 - 1/\kappa)$越接近0,收敛越快;
    \item $\kappa$越大(L远大于μ):衰减因子越接近1,收敛越慢,且极易出现“楼梯现象”。
\end{itemize}
\end{definition}

\subsection{PL(Polyak–Lojasiewicz)不等式}

\begin{definition}[PL条件]
若存在常数$\mu > 0$,对任意迭代点$x$,函数值差与梯度范数满足以下关系:
\begin{equation}
f(x) - f(x^*) \leq \frac{1}{2\mu} \|\nabla f(x)\|_2^2
\end{equation}
($f(x^*)$为函数最小值,PL条件是强凸性的“弱化版本”——无需函数严格强凸,仅通过“函数值差距”与“梯度大小”的关联约束函数形态)。
\end{definition}

\begin{theorem}[PL条件下的收敛性]
即使$f(x)$非强凸,只要满足PL条件,梯度下降仍能实现\textbf{线性型收敛},函数值差的衰减公式为:
\begin{equation}
f(x_k) - f(x^*) \leq \left(1 - \frac{\mu}{L}\right)^k \left[f(x_0) - f(x^*)\right]
\end{equation}
($x_0$为初始迭代点,$k$为迭代次数)。
\end{theorem}

\textbf{实际意义}:解释了深度学习训练中的常见现象——深度网络的损失函数通常非强凸,但可能满足PL条件,因此训练时损失曲线会呈现“近似线性下降”的稳定趋势。

\subsection{“楼梯现象”:成因与缓解}
\begin{itemize}
    \item \textbf{现象描述}:当函数条件数$\kappa = L/\mu$过大时,负梯度下降的迭代路径会呈现“楼梯形”:沿细长的等高线(如二次函数的椭圆等高线)迂回前进,每次仅沿等高线短轴方向小幅下降,无法直接逼近最小值点,迭代效率极低。
    \item \textbf{核心成因}:条件数$\kappa$过大导致函数等高线“细长扁平”,负梯度方向(沿等高线法向)与“最优下降方向”(沿等高线长轴方向)偏差极大,梯度下降陷入“来回震荡、缓慢逼近”的困境。
    \item \textbf{缓解方法}:\textbf{预条件化}(或输入/参数归一化)——通过线性变换(如文档中二次函数的坐标变换$y = A^{1/2}x$)将“细长谷”地形转化为“圆形洼地”,本质是减小条件数$\kappa$,使负梯度方向更接近最优下降方向,从而消除楼梯现象。
\end{itemize}

\subsection{实践提示}
\begin{enumerate}
    \item \textbf{预条件化的核心价值}:通过调整优化空间的度量规则(如引入预条件矩阵),显著降低条件数$\kappa$,从根本上改善收敛速度;
    \item \textbf{迭代停止准则}:无需迭代至完全收敛,满足以下任一条件即可终止:
    \begin{itemize}
        \item 梯度范数足够小(函数接近平稳):$\|\nabla f(x_k)\|_2 \leq \varepsilon$($\varepsilon$为预设精度,如$10^{-6}$);
        \item 函数值相对下降不足(继续迭代收益极低):$\frac{f(x_{k-1}) - f(x_k)}{\max(1, f(x_{k-1}))} \leq \varepsilon$。
    \end{itemize}
\end{enumerate}

\section{Newton法:局部二次近似与两阶段收敛}
Newton法是比梯度下降更高效的优化方法,核心是通过\textbf{函数的局部二次近似}确定搜索方向,兼具“局部快速收敛”与“全局有效下降”的特性,其核心逻辑围绕“二阶展开→牛顿步→收敛性”展开。

\subsection{核心思路:函数的局部二次近似}
梯度下降仅用“一阶信息(梯度)”将函数局部近似为线性函数,而Newton法引入“二阶信息(Hessian矩阵)”,将函数局部近似为\textbf{二次函数}(更贴合非凸函数的局部曲率)。

\begin{definition}[局部二次近似]
对迭代点$x$,将目标函数$f(x+\Delta)$在$x$处做二阶泰勒展开($\Delta$为搜索方向向量):
\begin{equation}
f(x+\Delta) \approx f(x) + \nabla f(x)^\top \Delta + \frac{1}{2}\Delta^\top H(x) \Delta
\end{equation}
其中:
\begin{itemize}
    \item $\nabla f(x)$是$f(x)$的梯度(一阶导数);
    \item $H(x) = \nabla^2 f(x)$是$f(x)$的Hessian矩阵(二阶导数矩阵),反映函数在$x$处的局部曲率。
\end{itemize}
Newton法的核心是:\textbf{最小化上述二次近似函数},直接求解使近似函数最小的搜索方向$\Delta$。
\end{definition}

\subsection{牛顿步(Newton Step)的推导}
对二阶近似函数关于$\Delta$求导,并令导数为0(二次函数的极值点条件):
\begin{equation}
\frac{\partial}{\partial \Delta}\left[ f(x) + \nabla f(x)^\top \Delta + \frac{1}{2}\Delta^\top H(x) \Delta \right] = \nabla f(x) + H(x) \Delta = 0
\end{equation}

\begin{definition}[牛顿步]
若Hessian矩阵\textbf{正定}($H(x) \succ 0$,保证二次近似函数是凸函数,极值点为最小值点),则可解出唯一的搜索方向——\textbf{牛顿步}:
\begin{equation}
\Delta_{nt} = -H(x)^{-1} \nabla f(x)
\end{equation}
\end{definition}

\subsection{牛顿步的下降性}
牛顿步能保证是“下降方向”的前提是$H(x) \succ 0$,证明如下:
计算梯度与牛顿步的内积(判断方向是否下降的核心指标,内积<0则为下降方向):
\begin{equation}
\nabla f(x)^\top \Delta_{nt} = \nabla f(x)^\top \left( -H(x)^{-1} \nabla f(x) \right)
\end{equation}
因$H(x) \succ 0$,其逆矩阵$H(x)^{-1}$也正定,故对任意非零向量$\nabla f(x)$,有$\nabla f(x)^\top H(x)^{-1} \nabla f(x) > 0$,因此:
\begin{equation}
\nabla f(x)^\top \Delta_{nt} < 0
\end{equation}
即牛顿步满足“下降方向”的核心条件。

\subsection{局部二次收敛:牛顿法的核心优势}
当迭代点足够靠近最优解$x^*$时,Newton法会呈现\textbf{二次收敛}(收敛速度远快于梯度下降的线性收敛)。

\begin{theorem}[牛顿法局部二次收敛]
若满足以下两个前提:
\begin{enumerate}
    \item Hessian矩阵$H(x)$在$x^*$的邻域内\textbf{Lipschitz连续}(曲率变化平缓);
    \item 初始迭代点$x^{(0)}$足够靠近$x^*$(进入“局部收敛域”)。
\end{enumerate}
此时存在常数$C > 0$,使得迭代误差满足:
\begin{equation}
\| x_{k+1} - x^* \| \leq C \cdot \| x_k - x^* \|^2
\end{equation}
\end{theorem}

\begin{proof}[证明:局部二次收敛]
\textbf{结论:}
在$H$在某邻域内满足 Lipschitz(存在常数$M$使得$|H(x)-H(y)| \le M|x-y|$)且$H(x^*)$非奇异的情形,从足够近的初值出发,牛顿迭代局部二次收敛,即存在常数$C>0$和半径$r>0$,当$|x_k-x^*| \le r$时
\begin{equation}
|x_{k+1}-x^*| \le C|x_k-x^*|^2.
\end{equation}

\textbf{证明:}
设误差$e_k := x_k - x^*$。由$\nabla f(x^*) = 0$和一维积分形式的泰勒公式:
\begin{equation}
\nabla f(x_k) = \int_0^1 H(x^* + t e_k) e_k \, dt.
\end{equation}
牛顿更新写作:
\begin{equation}
e_{k+1} = x_k - x^* - H(x_k)^{-1} \nabla f(x_k) = H(x_k)^{-1} \left( H(x_k) - \int_0^1 H(x^* + t e_k) dt \right) e_k.
\end{equation}
取范数并用三角不等式得:
\begin{equation}
|e_{k+1}| \le |H(x_k)^{-1}| \int_0^1 |H(x_k) - H(x^* + t e_k)| dt \cdot |e_k|.
\end{equation}
利用 Hessian 的 Lipschitz 性质(常数记为$M$):
\begin{equation}
|H(x_k) - H(x^* + t e_k)| \le M |x_k - (x^* + t e_k)| = M(1 - t)|e_k|.
\end{equation}
代入并对$t$积分:
\begin{equation}
|e_{k+1}| \le |H(x_k)^{-1}| \cdot M \left( \int_0^1 (1 - t) dt \right) |e_k|^2 = \frac{M}{2} |H(x_k)^{-1}| |e_k|^2.
\end{equation}
由于$H$连续且$H(x^*)$非奇异,存在半径$r>0$,使得对所有$|x - x^*| \le r$,$H(x)$可逆,且$|H(x)^{-1}| \le B$($B$为该闭球上逆的上界)。因此当$|e_k| \le r$时:
\begin{equation}
|e_{k+1}| \le \frac{M B}{2} |e_k|^2.
\end{equation}
令常数$C := \frac{M B}{2}$,即得局部二次收敛估计。
\end{proof}

\textbf{直观意义}:二次收敛意味着“每次迭代后,误差的有效位数会翻倍”——例如:若第$k$步误差为$10^{-2}$,第$k+1$步误差可降至$10^{-4}$,第$k+2$步可降至$10^{-8}$,接近最优解时收敛极快。

\chapter{随机梯度下降(Stochastic Gradient Descent, SGD)}

\section{随机梯度下降基础}

当面对大规模数据集(数据量记为$N$,单个数据为$x_{i}$,$i=1, \dots, N$),需要优化目标函数 $\min _{x} \sum_{i=1}^{N} f_{i}(x)$ 时,若无法一次性获取所有数据 $x_{i}$ 或对应函数 $f_{i}$,则可通过随机梯度下降(SGD)实现优化。

\subsection{核心思路:用“部分数据”估算梯度}
由于无法计算全部数据的完整梯度 $\nabla f$,SGD通过\textbf{随机选取部分数据(称为“小批量”,记为 $\mathcal{B}^{(k)}$,其数据量记为 $|\mathcal{B}^{(k)}|$)},用这部分数据的梯度平均值近似整体梯度,即:
\begin{equation}
\nabla f \approx \frac{1}{|\mathcal{B}^{(k)}|} \sum_{i \in \mathcal{B}^{(k)}} \nabla \ell_{i}\left(x^{(k)}\right)
\end{equation}
其中 $\nabla \ell_{i}\left(x^{(k)}\right)$ 是单个数据 $i$ 在当前参数 $x^{(k)}$ 下的梯度,近似得到的整体梯度记为 $g^{(k)}$,即 $g^{(k)}=\frac{1}{|\mathcal{B}^{(k)}|} \sum_{i \in \mathcal{B}^{(k)}} \nabla \ell_{i}\left(x^{(k)}\right)$。

\subsection{完整更新流程}
SGD的优化流程是在传统梯度下降(GD)基础上,修改“梯度计算方式”,具体步骤如下:
\begin{enumerate}
    \item \textbf{初始值设定}:从初始参数 $x^{(0)}$ 开始,迭代次数 $k=0$;
    \item \textbf{确定下降方向}:基于随机选取的小批量数据 $\mathcal{B}^{(k)}$,计算近似梯度 $g^{(k)}$,下降方向为 $\Delta x^{(k)}=-g^{(k)}$(负梯度方向,保证函数值下降);
    \item \textbf{选择步长(学习率)}:步长 $\alpha^{(k)}$ 可设为常数,也可随迭代次数动态调整(如后期逐步减小,避免参数震荡);
    \item \textbf{参数更新}:按以下公式更新参数,使新参数对应的函数值更小(即 $f(x^{(k+1)}) < f(x^{(k)})$),之后迭代次数 $k=k+1$,重复步骤2-4:
    \begin{equation}
    x^{(k+1)}=x^{(k)}-\alpha^{(k)} g^{(k)}
    \end{equation}
\end{enumerate}

\subsection{关键超参数}
SGD的效果依赖两个核心超参数的设置,需根据数据和任务调整:
\begin{itemize}
    \item \textbf{批量大小(Batch Size)}:即小批量数据 $\mathcal{B}^{(k)}$ 包含的数据量 $|\mathcal{B}^{(k)}|$。批量越大,梯度估算越精准(噪声越小),但计算速度越慢;批量越小,计算越快,但梯度噪声越大,参数易震荡。
    \item \textbf{学习率(Learning Rate)}:即步长 $\alpha^{(k)}$。学习率过大可能导致参数“越过”最优解,函数值不下降反而上升;学习率过小则参数更新缓慢,需更多迭代次数才能收敛。
\end{itemize}

\section{一个随机估计问题}

先来看一个随机估计问题。

\subsection{有限样本下的均值计算}
\begin{itemize}
    \item 当我们采样得到 $n$ 个样本时,均值可表示为:
    \begin{equation}
    f_{n}(x)=\frac{1}{n} \sum_{i \in[1, n]} f\left(x_{i}\right)
    \end{equation}
    \item 当继续采样到第 $n+1$ 个样本时,新的均值为:
    \begin{equation}
    f_{n+1}(x)=\frac{1}{n+1} \sum_{i \in[1, n+1]} f\left(x_{i}\right)
    \end{equation}
\end{itemize}

\subsection{前后均值的递推关系}
通过数学变形,可建立 $f_{n}(x)$ 与 $f_{n+1}(x)$ 的关联,避免重复计算所有样本:
\begin{equation}
\begin{aligned}
f_{n+1}(x) & =\frac{1}{n+1}\left(f\left(x_{n+1}\right)+\sum_{i \in[1, n]} f\left(x_{i}\right)\right) \\
& =\frac{1}{n+1}\left(f\left(x_{n+1}\right)+n f_{n}(x)\right) \\
& =\left(1-\frac{1}{n+1}\right) f_{n}(x)+\frac{1}{n+1} f\left(x_{n+1}\right)
\end{aligned}
\end{equation}
若令步长 $\alpha=\frac{1}{n+1}$,则递推式可简化为更通用的形式:
\begin{equation}
f_{n+1}(x)=f_{n}(x)+\alpha\left(f\left(x_{n+1}\right)-f_{n}(x)\right)
\end{equation}
这意味着新均值=旧均值+步长$\times$(新样本值-旧均值),无需存储所有历史样本,仅需保留旧均值即可更新。

\subsection{均值收敛的条件}
要保证当样本数量 $n \to \infty$ 时,均值 $f_{n}(x)$ 能稳定收敛到真实期望,需满足 Robbins-Monro(1951)提出的步长条件:
\begin{equation}
\sum_{n=1}^{\infty} \alpha_{n}=\infty, \quad \sum_{n=1}^{\infty} \alpha_{n}^{2}<\infty
\end{equation}
\begin{itemize}
    \item 第一个条件 $\sum_{n=1}^{\infty} \alpha_{n}=\infty$:保证步长累积足够大,均值能持续向真实期望靠近,避免“半途停滞”;
    \item 第二个条件 $\sum_{n=1}^{\infty} \alpha_{n}^{2}<\infty$:保证步长衰减足够快,避免后期新样本对均值的干扰过大,导致结果震荡。
\end{itemize}

\section{Robbins-Monro(RM)算法}

首先回顾 Robbins-Monro(RM)算法的基础,它是推导 SGD 的起点。

\subsection{RM算法的目标}
RM算法用于求解\textbf{黑箱函数的根},即找到 $w^*$ 满足:
\begin{equation}
g(w^*) = 0
\end{equation}
其中 $g: \mathbb{R}^d \to \mathbb{R}^d$ 是未知函数(黑箱),仅能通过带噪声的观测获取信息:
\begin{equation}
\tilde{g}(w, \eta) = g(w) + \eta
\end{equation}
$\eta$ 是观测噪声,满足 $\mathbb{E}[\eta \mid H_k] = 0$($H_k = \{w_k, w_{k-1}, \dots\}$ 为历史信息),且方差有界 $\mathbb{E}[\eta^2 \mid H_k] < \infty$。

\subsection{RM算法的迭代公式}
为求解 $g(w) = 0$,RM算法的迭代更新规则为:
\begin{equation}
w_{k+1} = w_k - a_k \cdot \tilde{g}(w_k, \eta_k)
\end{equation}
其中 $a_k > 0$ 是步长序列,$w_k$ 是第 $k$ 次迭代的估计值。

\subsection{RM算法的收敛条件}
要保证 $w_k \to w^*$(几乎必然收敛),需满足3个核心条件:
\begin{enumerate}
    \item \textbf{函数单调性}:$g(w)$ 单调递增,且梯度有界 $0 < c_1 \leq \nabla_w g(w) \leq c_2$(确保根唯一);
    \item \textbf{步长条件}:$\sum_{k=1}^\infty a_k = \infty$(步长不收敛太快,保证能逼近根)且 $\sum_{k=1}^\infty a_k^2 < \infty$(步长趋于0,避免震荡);
    \item \textbf{噪声条件}:$\mathbb{E}[\eta_k \mid H_k] = 0$ 且 $\mathbb{E}[\eta_k^2 \mid H_k] < \infty$(噪声无偏且方差有界)。
\end{enumerate}

\section{SGD之问:为何能够收敛?}

在此之前,我们先来看一个引理。

\begin{lemma}[Robbins–Siegmund 超鞅收敛引理]
给定非负可测序列 $(X_k)$,满足条件:
\begin{equation}
\mathbb{E}[X_{k+1}\mid \mathcal F_k] \le (1 - a_k) X_k + b_k,
\end{equation}
其中:
\begin{itemize}
    \item $0 \le a_k \le 1$,控制“衰减比例”;
    \item $b_k \ge 0$,表示小的扰动或噪声;
    \item $\sum a_k = \infty$,保证长期衰减足够;
    \item $\sum b_k < \infty$,保证扰动总量有限。
\end{itemize}

\textbf{结论:}
\begin{enumerate}
    \item $(X_k)$ 几乎处处收敛;
    \item $\sum a_k X_k < \infty$ 几乎处处成立。
\end{enumerate}

\textbf{直观理解:}
\begin{itemize}
    \item $(X_k)$ 类似“衰减量 + 小扰动”的随机过程;
    \item $(a_k)$ 保证每步都有“收敛拉力”,而 $(b_k)$ 干扰有限;
    \item 因此 $(X_k)$ 不会发散,最终收敛,并且累计衰减量 $(\sum a_k X_k)$ 有限。
\end{itemize}
\end{lemma}

Robbins–Siegmund 引理提供了\textbf{在随机衰减 + 有限扰动下的序列收敛保证},是随机优化与在线算法理论分析的核心工具。

\subsection{设定与记号}
\begin{itemize}
    \item \textbf{数据与参数}:数据(或小批量数据)为 $x^{(k)}$,模型参数为 $\theta \in \mathbb{R}^{d}$($d$ 为参数维度)。
    \item \textbf{目标函数}:目标函数定义为期望损失,即 $f(\theta) \triangleq \mathbb{E}_{x}[L(x, \theta)]$,其中 $L(x, \theta)$ 是单个数据(或小批量数据)的损失函数。
    \item \textbf{SGD更新公式}:参数更新遵循 $\pmb{\theta}^{(k+1)}=\pmb{\theta}^{(k)}-\eta_{k} g^{(k)}$,其中 $\eta_k$ 是第 $k$ 步的学习率,$g^{(k)} \equiv \nabla_{\theta} L\left(x^{(k)}, \theta^{(k)}\right)$ 是第 $k$ 步的随机梯度(基于小批量数据计算)。
    \item \textbf{噪声分解}:将随机梯度拆分为“真实梯度”与“噪声”两部分,即 $g^{(k)}=\nabla f(\theta^{(k)})+\xi^{(k)}$。其中 $\nabla f(\theta^{(k)})$ 是目标函数在 $\theta^{(k)}$ 处的真实梯度,$\xi^{(k)}$ 是随机噪声,且满足条件 $\mathbb{E}[\xi^{(k)} | \mathcal{F}_{k}]=0$($\mathcal{F}_{k}$ 表示到第 $k$ 步的所有观测信息集合,即“自然滤子”)。
\end{itemize}
这一分解恰好契合 \textbf{Robbins–Monro框架}:该框架旨在寻找方程 $h(\theta)=0$ 的根(即目标函数极小值点,此时 $\nabla f(\theta^*)=0$),但仅能获得带噪声的观测 $H(\theta, x)$(对应此处的随机梯度 $g^{(k)}$),且观测的期望等于真实函数(即 $\mathbb{E}[g^{(k)} | \mathcal{F}_k] = \nabla f(\theta^{(k)})$)。令 $h(\theta)=\nabla f(\theta)$,即可将 SGD 纳入该框架分析收敛性。

\subsection{收敛性证明的核心假设}
要证明 SGD 收敛,需满足以下5个关键假设(记 $\theta^*$ 为目标函数极小值点,即 $\nabla f(\theta^*)=0$):
\begin{itemize}
    \item \textbf{(A1) 无偏噪声}:随机梯度的条件期望等于真实梯度,即 $\mathbb{E}[g^{(k)} | \mathcal{F}_{k}]=\nabla f(\theta^{(k)})$。
    \item \textbf{(A2) 有界二阶矩}:噪声的条件二阶矩有上限,即 $\mathbb{E}[\left\|\xi^{(k)}\right\|^{2} | \mathcal{F}_{k}] \leq\sigma^{2}+c\left\|\nabla f(\theta^{(k)})\right\|^{2}$。其中 $\sigma^2$ 是常数,$c$ 是系数,该假设限制了噪声的“强度”,避免噪声过大导致参数震荡不收敛。常用特例为“常数方差”,即 $\mathbb{E}[\left\|\xi^{(k)}\right\|^{2} | \mathcal{F}_{k}] \leq\sigma^{2}$。
    \item \textbf{(A3) L平滑}:目标函数的梯度满足 Lipschitz 连续条件,即 $\|\nabla f(\theta)-\nabla f(\phi)\| \leq L\|\theta-\phi\|$($L$ 为 Lipschitz 常数)。
    \item \textbf{(A4) $\mu$-强凸}:目标函数是 $\mu$-强凸的,即 $(\nabla f(\theta)-\nabla f(\phi))^{\top}(\theta-\phi) \geq \mu\|\theta-\phi\|^{2}$($\mu>0$ 为强凸系数)。强凸性保证目标函数有唯一极小值点 $\theta^*$,且参数会“持续向极小值点靠近”,不会在多个局部极小值间徘徊。
    \item \textbf{(A5) Robbins–Monro步长条件}:学习率序列 $\{\eta_k\}$ 需满足两个条件:
    \begin{enumerate}
        \item $\sum_{k=1}^{\infty} \eta_{k}=\infty$(学习率累积和为无穷大):保证参数有足够的“推进力”,能持续向极小值点靠近,避免因步长过小而“半途停滞”;
        \item $\sum_{k=1}^{\infty} \eta_{k}^{2}<\infty$(学习率平方的累积和有限):保证后期步长足够小,避免参数在极小值点附近“来回震荡”。
    \end{enumerate}
    典型的满足该条件的学习率形式为 $\eta_{k}=\frac{\alpha}{k+\beta}$($\alpha>0$,$\beta \geq 0$)。
\end{itemize}

\subsection{收敛性结论}

\subsubsection{结论一:几乎处处收敛(基于Robbins–Siegmund引理)}
\begin{theorem}[a.s. 收敛]
在假设(A1)–(A5)成立的前提下,SGD生成的参数序列满足:
\begin{equation}
\theta^{(k)} \underset{k \to \infty}{\stackrel{a.s.}{\to}} \theta^{*}, \quad \sum_{k=1}^{\infty} \eta_{k}\left\|\nabla f\left(\theta^{(k)}\right)\right\|^{2}<\infty \quad a.s.
\end{equation}
其中“a.s.”表示“几乎必然”(即除了概率为0的特殊情况外,参数序列一定收敛到 $\theta^*$)。
\end{theorem}

\textbf{证明核心思路(关键不等式与引理应用):}
\begin{enumerate}
    \item \textbf{定义距离变量}:令 $\Delta^{(k)} \triangleq \theta^{(k)}-\theta^*$(即当前参数与极小值点的距离向量),需证明 $\|\Delta^{(k)}\| \to 0$(距离趋近于0)。
    \item \textbf{展开距离平方的递推关系}:根据 SGD 更新公式,展开 $\|\Delta^{(k+1)}\|^2$(第 $k+1$ 步的距离平方):
    \begin{equation}
    \begin{aligned}
    \left\|\Delta^{(k+1)}\right\|^{2} &= \left\|\theta^{(k+1)}-\theta^*\right\|^{2} \\
    &= \left\|\theta^{(k)} - \eta_k g^{(k)} - \theta^*\right\|^{2} \\
    &= \left\|\Delta^{(k)} - \eta_k g^{(k)}\right\|^{2} \\
    &= \left\|\Delta^{(k)}\right\|^{2} - 2\eta_k \Delta^{(k)\top} g^{(k)} + \eta_k^2 \left\|g^{(k)}\right\|^{2}
    \end{aligned}
    \end{equation}
    \item \textbf{取条件期望并代入假设}:对等式两侧关于 $\mathcal{F}_k$ 取条件期望,结合(A1)(无偏噪声)和(A2)(有界二阶矩),将 $g^{(k)}=\nabla f(\theta^{(k)})+\xi^{(k)}$ 代入,可化简得到:
    \begin{equation}
    \begin{aligned}
    \mathbb{E}\left[\left\|\Delta^{(k+1)}\right\|^{2} | \mathcal{F}_k\right] \leq \mathbb{E}\left[\left\|\Delta^{(k)}\right\|^{2} | \mathcal{F}_k\right] - 2\eta_k \Delta^{(k)\top} \nabla f(\theta^{(k)}) \\
    + \eta_k^2 \left( \left\|\nabla f(\theta^{(k)})\right\|^2 + \sigma^2 + c\left\|\nabla f(\theta^{(k)})\right\|^2 \right)
    \end{aligned}
    \end{equation}
    \item \textbf{利用强凸与L平滑简化}:由(A4)($\mu$-强凸)可得 $\Delta^{(k)\top} \nabla f(\theta^{(k)}) \geq \mu \left\|\Delta^{(k)}\right\|^2$;由(A3)(L平滑)可得 $\left\|\nabla f(\theta^{(k)})\right\| \leq L \left\|\Delta^{(k)}\right\|$(因 $\nabla f(\theta^*)=0$)。代入上式后,可整理得到:
    \begin{equation}
    \mathbb{E}\left[\left\|\Delta^{(k+1)}\right\|^{2} | \mathcal{F}_k\right] \leq \left(1 - 2\mu \eta_k + C \eta_k^2\right) \left\|\Delta^{(k)}\right\|^2 + \sigma^2 \eta_k^2
    \end{equation}
    其中 $C \triangleq (1+c)L^2$(常数)。当 $k$ 足够大时,$\eta_k$ 足够小,可满足 $1 - 2\mu \eta_k + C \eta_k^2 \leq 1 - \mu \eta_k$。
    \item \textbf{应用Robbins–Siegmund引理}:令 $X_k = \left\|\Delta^{(k)}\right\|^2$(待分析的非负序列),$a_k = \mu \eta_k$,$b_k = \sigma^2 \eta_k^2$,则上述不等式可化为引理要求的形式:
    \begin{equation}
    \mathbb{E}\left[X_{k+1} | \mathcal{F}_k\right] \leq (1 - a_k)X_k + b_k
    \end{equation}
    结合(A5),$\sum a_k = \mu \sum \eta_k = \infty$,$\sum b_k = \sigma^2 \sum \eta_k^2 < \infty$,满足引理条件。根据引理可得出:$X_k$ 几乎必然收敛,且 $\sum a_k X_k < \infty$。再结合 $\sum a_k = \infty$,可推出 $\liminf_{k \to \infty} X_k = 0$;又因目标函数强凸((A4)),最终可得 $X_k \to 0$(即 $\theta^{(k)} \to \theta^*$)几乎必然成立。
\end{enumerate}

\subsubsection{结论二:强凸下的收敛速率与Polyak–Ruppert平均}
\begin{theorem}[期望二次误差 $O(1/k)$]
假设(A1)–(A4)成立,若取学习率 $\eta_k = \frac{\alpha}{k+\beta}$(其中 $\alpha > \frac{1}{\mu}$,$\beta \geq 1$),则存在常数 $K$,使得:
\begin{equation}
\mathbb{E}\left[\left\|\theta^{(k)} - \theta^*\right\|^2\right] \leq \frac{K}{k+\beta}
\end{equation}
即参数与极小值点的“期望平方距离”随迭代次数 $k$ 增长,以 $O(1/k)$ 的速率衰减。
\end{theorem}

\begin{proof}
将结论一证明中的“距离平方的条件期望递推式”取全期望,令 $u_k = \mathbb{E}\left[\left\|\Delta^{(k)}\right\|^2\right]$(期望平方距离),代入 $\eta_k = \frac{\alpha}{k+\beta}$ 后可得到:
\begin{equation}
u_{k+1} \leq \left(1 - \frac{2\mu \alpha}{k+\beta} + \frac{C \alpha^2}{(k+\beta)^2}\right) u_k + \frac{\sigma^2 \alpha^2}{(k+\beta)^2}
\end{equation}
通过定义辅助变量 $v_k = (k+\beta) u_k$,利用“差分比较法”可证明 $u_k = O(1/k)$,进而得到上述期望误差界。
\end{proof}

\textbf{随着迭代次数 $k$ 增加,参数与最优解的 “平均距离平方” 会以 $1/k$ 的速度变小}。

\begin{theorem}[Polyak–Ruppert迭代平均的最优渐近方差]
定义参数的迭代平均为:
\begin{equation}
\overline{\theta}^{(T)} \triangleq \frac{1}{T} \sum_{k=1}^{T} \theta^{(k)}
\end{equation}
在假设(A1)–(A4)与(A5)(步长 $\eta_k = \frac{\alpha}{k+\beta}$)成立的前提下,有:
\begin{equation}
\sqrt{T}\left(\overline{\theta}^{(T)} - \theta^*\right) \Rightarrow \mathcal{N}\left(0, A^{-1} S A^{-\top}\right)
\end{equation}
其中 $A \triangleq \nabla^2 f(\theta^*)$(目标函数在极小值点处的 Hessian 矩阵),$S$ 是噪声协方差的极限值。
\end{theorem}

该结论表明:通过对参数序列做“迭代平均”,可使 SGD 达到随机逼近(SA)框架下的“最优渐近效率”——即平均后的参数估计量,其渐近方差是最小的,在实践中能显著减小噪声导致的参数波动,提升收敛稳定性。

普通 SGD 是“每步更新一个参数,最后用最后一步的参数”;而 Polyak–Ruppert 方法是“先迭代 $T$ 步,得到 $T$ 个参数 $\theta^{(1)},\theta^{(2)},\dots,\theta^{(T)}$,再求它们的平均值 $\overline{\theta}^{(T)} = \frac{1}{T}\sum_{k=1}^T \theta^{(k)}$”。

用“迭代平均”后的参数 $\overline{\theta}^{(T)}$,\textbf{随着迭代次数 $T$ 增加,它与最优解的差距会服从“正态分布”,且这个差距的“波动范围(方差)是最小的”}(即“最优渐近方差”)。

SGD 的收敛性本质上源于 \textbf{Robbins–Monro 随机逼近框架}与 \textbf{Robbins–Siegmund 超鞅引理}的支撑:
\begin{enumerate}
    \item 强凸目标函数+满足 Robbins–Monro 条件的学习率(如 $\eta_k \propto 1/k$),可保证参数“几乎处处收敛”到极小值点,且期望二次误差以 $O(1/k)$ 速率衰减;
    \item 对参数做 Polyak–Ruppert 迭代平均,能进一步优化渐近方差,提升收敛精度与稳定性。
\end{enumerate}

\subsection{非强凸场景下的SGD收敛性(仅凸/一般非凸)}

在之前的分析中,我们默认目标函数满足“$\mu$-强凸”条件(假设(A4)),但实际场景中很多目标函数不具备强凸性(如仅凸函数、非凸函数),因此需要单独分析这类场景下 SGD 的收敛表现。

\subsubsection{1. 仅凸场景(无强凸性,仅满足凸性)}
\textbf{核心设定}:此时目标函数可能存在“平坦区域”或“多个最优解(构成凸集)”,无法保证参数收敛到唯一极小值点,但可保证“函数值收敛到最优值”。

\textbf{收敛性结论}:若调整学习率为 $\eta_k = \frac{1}{k^\alpha}$(其中 $\alpha \in (1/2, 1]$,满足 Robbins–Monro 条件 $\sum \eta_k = \infty$ 且 $\sum \eta_k^2 < \infty$),则对参数的迭代平均值 $\overline{\theta}^{(T)} = \frac{1}{T}\sum_{k=1}^T \theta^{(k)}$,有:
\begin{equation}
f(\overline{\theta}^{(T)}) - f(\theta^*) = o(1)
\end{equation}
即随着迭代次数 $T$ 增大,平均参数对应的函数值会“逐步逼近最优函数值 $f(\theta^*)$”,最终趋近于 0。

若进一步量化收敛速率,通常为 $O\left(\frac{1}{T^{1-\alpha}}\right)$ 量级(如 $\alpha=0.8$ 时,速率为 $O\left(\frac{1}{T^{0.2}}\right)$)——相比强凸场景下的 $O\left(\frac{1}{T}\right)$,仅凸场景的收敛更慢,这是因为缺少强凸性带来的“强制向最优解靠近”的约束。

\subsubsection{2. 一般非凸场景(无凸性,仅满足L-平滑)}
\textbf{核心设定}:“一般非凸”指目标函数既不满足强凸性,也不满足凸性,仅满足 L-平滑条件(假设(A3):梯度变化平缓,$\|\nabla f(\theta)-\nabla f(\phi)\| \leq L\|\theta-\phi\|$)。
这类场景在深度学习中最常见(如神经网络的损失函数),目标函数可能存在大量局部极小值、鞍点,无法保证参数收敛到全局最优解,只能退而求其次——保证参数收敛到“一阶驻点”(即梯度趋近于 0 的点,$\nabla f(\theta) \approx 0$,此时参数再更新也难以显著降低函数值)。

\textbf{收敛性结论}:在无偏噪声(假设(A1))和噪声方差有界(假设(A2))的前提下,若采用“分段常数步长”或“$\eta_k = \frac{1}{\sqrt{k}}$ 步长”(注:$\frac{1}{\sqrt{k}}$ 不满足 Robbins–Monro 的 $\sum \eta_k^2 < \infty$,因此不具备“几乎处处收敛”性质,仅能保证“梯度的期望有界”),则有:
\begin{equation}
\min_{1 \leq k \leq T} \mathbb{E}\left[\left\|\nabla f(\theta^{(k)})\right\|^2\right] = \mathcal{O}\left(\frac{1}{\sqrt{T}}\right)
\end{equation}
该结论的含义是:在 $T$ 次迭代中,\textbf{至少存在某一步的参数 $\theta^{(k)}$,其梯度的期望平方值不超过 $\frac{C}{\sqrt{T}}$($C$ 为常数)},且随着 $T$ 增大,这个“最小梯度期望”会以 $\frac{1}{\sqrt{T}}$ 的速率减小,逐步趋近于 0。

需要特别注意:
\begin{enumerate}
    \item 该速率是“到一阶驻点”的速率,而非“到全局最优解”的速率——最终参数可能停在局部极小值或鞍点,但这些点的梯度已足够小,函数值难以继续下降;
    \item 与强凸/仅凸场景不同,非凸场景的收敛结论不涉及“参数是否收敛”或“函数值是否收敛到最优”,仅保证“梯度足够小”,这是因为非凸函数的全局最优解难以通过 SGD 的随机搜索触及,“找到驻点”已是实际能达到的目标。
\end{enumerate}

\subsubsection{3. 非强凸场景与强凸场景的核心差异}
为了更清晰理解不同场景的收敛特性,可通过下表对比:

\begin{table}[htbp]
\centering
\small
\begin{tabular}{@{}llllll@{}}
\toprule
\textbf{场景} & \textbf{目标函数性质} & \textbf{收敛目标} & \textbf{学习率要求} & \textbf{收敛速率(期望)} & \textbf{关键限制} \\ \midrule
强凸 & 强凸+L-平滑 & 全局最优解 $\theta^*$ & $\eta_k \propto \frac{1}{k}$ (满足RM) & $O(1/T)$ & 需强凸性,适用场景有限 \\
仅凸 & 凸+L-平滑 & 最优函数值 $f(\theta^*)$ & $\eta_k = \frac{1}{k^\alpha}$ ($\alpha \in (0.5,1]$) & $O(1/T^{1-\alpha})$ & 收敛慢,无唯一最优参数 \\
一般非凸 & L-平滑(无凸性) & 一阶驻点 ($\nabla f \approx 0$) & 分段常数/$\eta_k \propto \frac{1}{\sqrt{k}}$ & $O(1/\sqrt{T})$ (梯度期望) & 仅能到驻点,可能是局部最优 \\ \bottomrule
\end{tabular}
\caption{不同场景下的收敛特性对比}
\end{table}

\subsection{两种不同目标下的步长设计及收敛策略差异}

\subsubsection{1. OGD/Regret(在线学习/长期平均性能)}
\begin{itemize}
    \item \textbf{目标}:保证长期平均损失接近最优,即 \textbf{后悔(regret)界} 小。
    \item \textbf{对应公式常见形式}:
    \begin{equation}
    \text{Regret}(T) = \sum_{t=1}^T f_t(x_t) - \min_x \sum_{t=1}^T f_t(x) \le O(\sqrt{T}) \text{ 或 } O(\log T)
    \end{equation}
    \item \textbf{步长选择}:通常用 \textbf{非递减或者 $1/\sqrt{t}$ 形式},保证平均损失下降快。
\end{itemize}

\subsubsection{2. RM/SA(Robbins–Monro / Stochastic Approximation)}
\begin{itemize}
    \item \textbf{目标}:保证\textbf{参数序列几乎处处收敛到最优点}(a.s. convergence),属于点估计/统计意义。
    \item \textbf{收敛条件}:
    \begin{equation}
    \sum_{k=1}^{\infty} a_k = \infty,\quad \sum_{k=1}^{\infty} a_k^2 < \infty
    \end{equation}
    \item \textbf{常用步长}:$a_k = 1/k$ 或 $1/k^\gamma$ ($0.5<\gamma\le1$)。
    \item \textbf{意义}:每步衰减足够慢以保证探索,但衰减快以抑制噪声,满足 Robbins–Siegmund 引理条件。
\end{itemize}

\begin{itemize}
    \item 如果使用 \textbf{OGD/Regret 的步长策略} 来保证 \textbf{几乎处处收敛},可能违反 RM/SA 的平方可积条件($\sum < \infty$),因此不能保证 a.s. 收敛。
    \item 反之,如果严格使用 RM/SA 的条件($\sum < \infty$)来优化在线 regret,可能收敛太慢,导致平均损失下降慢。
\end{itemize}
\textbf{关键点}:两种方法的目标不同,不能直接互换步长策略。

\subsubsection{3. 折中 / 统一策略}
\textbf{选用 $a_k = 1/k^\gamma$ ($0.5< \gamma < 1$)}
\begin{itemize}
    \item 满足 RM/SA 条件:$\sum a_k = \infty$ 且 $\sum a_k^2 < \infty$,保证 a.s. 收敛。
    \item 同时保持较慢衰减,平均性能也不错(在线学习效果可接受)。
\end{itemize}

\textbf{阶段常数步长 + Polyak–Ruppert 平均 + Doubling Trick}
\begin{itemize}
    \item \textbf{阶段常数步长}:将迭代分阶段,每阶段使用\textbf{近似最优常数步长},提升该阶段的平均损失性能(降低 regret)。
    \item \textbf{Doubling Trick}:阶段长度每次加倍(doubling trick),保证整体步长衰减满足 RM/SA 条件,确保 a.s. 收敛。
    \item \textbf{Polyak–Ruppert 平均}:阶段内取参数平均,进一步稳定收敛。
\end{itemize}

\begin{remark}
\textbf{OGD/Regret 关注长期平均损失;RM/SA 关注参数几乎处处收敛。两者步长策略冲突,但可以通过衰减指数、阶段常数步长和 Polyak–Ruppert 平均实现折中,兼顾在线性能和几乎处处收敛。}
\end{remark}

\section{从随机估计到动力学}

从动力学视角分析 SGD,核心是将“离散的参数更新过程”与“连续的物理运动方程”建立关联——通过极限近似,把梯度下降(GD)对应到确定性的常微分方程(ODE),把随机梯度下降(SGD)对应到含噪声的随机微分方程(SDE),从而用物理运动规律解释 SGD 的收敛行为、噪声影响及参数调整逻辑。

\subsection{从GD到ODE:离散更新是梯度流的“显式欧拉积分”}
梯度下降(GD)的参数更新是离散步骤,而通过“时间标度转换”和“连续极限”,可将其转化为描述“确定性下降运动”的常微分方程(ODE),即“梯度流”。

\subsubsection{1.1 离散更新与时间标度定义}
GD 的离散更新公式为:
\begin{equation}
\theta^{(k+1)} = \theta^{(k)} - \eta \nabla L(\theta^{(k)})
\end{equation}
其中:
\begin{itemize}
    \item $\theta^{(k)}$:第 $k$ 步的参数;
    \item $\eta$:步长(学习率);
    \item $\nabla L(\theta^{(k)})$:目标函数 $L$ 在 $\theta^{(k)}$ 处的梯度(确定性,无噪声)。
\end{itemize}
为建立连续关联,定义“连续时间” $t_k = k \cdot \eta$——即把每一步更新的“步长 $\eta$”视为“时间增量”,迭代次数 $k$ 越多,对应的连续时间 $t_k$ 越大。

\subsubsection{1.2 连续极限:从离散更新到梯度流ODE}
当步长 $\eta \to 0$(时间增量无限小)、且 $t_k \to t$(连续时间趋近于某个值)时,对 GD 的离散更新公式做“差分近似”:
左边参数增量除以时间增量,近似为连续时间下的参数变化率(导数):
\begin{equation}
\frac{\theta^{(k+1)} - \theta^{(k)}}{\eta} \Rightarrow \dot{\theta}(t)
\end{equation}
右边代入 GD 的更新规则,可得连续时间下的“梯度流方程”(ODE):
\begin{equation}
\dot{\theta}(t) = -\nabla L(\theta(t))
\end{equation}

\textbf{物理意义}:GD 的离散更新,本质是对“梯度流 ODE”的“显式欧拉数值积分”——每一步按当前梯度方向“迈一小步”,步长越小,离散的参数轨迹越贴近 ODE 描述的“连续下降路径”(类似下山时“小步慢走”更贴近顺滑的山坡轨迹)。

\subsubsection{1.3 数值稳定性与曲率的关系}
GD 的收敛稳定性(是否会“震荡不收敛”),与目标函数的“曲率”直接相关,可通过二次函数案例直观理解:
\begin{itemize}
    \item 若目标函数为二次形式 $L(\theta) = \frac{1}{2}\theta^\top H \theta$($H \succeq 0$ 为 Hessian 矩阵,代表函数曲率),则 GD 的更新公式可改写为:
    \begin{equation}
    \theta^{(k+1)} = (I - \eta H) \theta^{(k)}
    \end{equation}
    其中 $I$ 为单位矩阵。
    \item 收敛条件:该线性迭代收敛的充要条件是“矩阵 $I - \eta H$ 的谱半径 $\rho(I - \eta H) < 1$”,等价于步长需满足:
    \begin{equation}
    0 < \eta < \frac{2}{\lambda_{\text{max}}(H)}
    \end{equation}
    ($\lambda_{\text{max}}(H)$ 是 Hessian 矩阵的最大特征值,代表函数的“最大曲率”)。
    \item 一般 L-平滑场景:若目标函数的梯度满足 L-Lipschitz 连续(L 为平滑常数,可理解为“梯度变化的最大速率”),则取 $0 < \eta < \frac{2}{L}$ 可保证每步更新后函数值下降;若同时满足强凸性,取 $0 < \eta \leq \frac{1}{L}$ 还能获得“线性收敛速率”(参数快速靠近最优解)。
\end{itemize}

\textbf{核心启发(A)}:可将学习率 $\eta$ 视为“时间步长”——函数曲率越大($\lambda_{\text{max}}(H)$ 或 $L$ 越大),“显式欧拉积分”的稳定范围越窄,GD 需要更小的学习率才能避免震荡;实际中“分段调整学习率”“周期衰减学习率”,本质是通过“细化时间网格”提升数值稳定性,让参数更新更贴合梯度流的顺滑路径。

\subsection{从SGD到SDE:扩散极限与朗之万动力学}
SGD 的核心是“用随机小批量梯度近似真实梯度”,存在噪声干扰。通过类似的连续极限,可将其转化为含噪声的随机微分方程(SDE),即“朗之万动力学”,从而用“扩散运动”解释 SGD 的噪声探索与收敛平衡。

\subsubsection{2.1 噪声分解:随机梯度的构成}
SGD 的小批量梯度包含“真实梯度”和“噪声”两部分,分解公式为:
\begin{equation}
\nabla L_{\mathcal{B}}(\theta^{(k)}) = \nabla L(\theta^{(k)}) + \xi^{(k)}
\end{equation}
其中:
\begin{itemize}
    \item $\nabla L_{\mathcal{B}}(\theta^{(k)})$:基于小批量 $\mathcal{B}$ 计算的随机梯度;
    \item $\nabla L(\theta^{(k)})$:目标函数的真实梯度(确定性部分);
    \item $\xi^{(k)}$:小批量采样引入的噪声,满足 $\mathbb{E}[\xi^{(k)} | \theta^{(k)}] = 0$(无偏噪声),其协方差 $\text{Cov}[\xi^{(k)}] \approx \Sigma(\theta^{(k)})$(随参数变化的噪声强度)。
\end{itemize}
基于此,SGD 的离散更新公式可改写为:
\begin{equation}
\theta^{(k+1)} = \theta^{(k)} - \eta \left( \nabla L(\theta^{(k)}) + \xi^{(k)} \right)
\end{equation}

\subsubsection{2.2 扩散极限:从离散SGD到SDE(欧拉–丸山连续化)}
同样定义连续时间 $t_k = k \cdot \eta$,当步长 $\eta \to 0$(时间增量无限小)、且小批量噪声近似高斯分布时,可将 SGD 的离散更新转化为“随机微分方程(SDE)”:
\begin{equation}
d\theta_t = -\nabla L(\theta_t) dt + G(\theta_t) dW_t
\end{equation}
其中:
\begin{itemize}
    \item $\theta_t$:连续时间 $t$ 下的参数;
    \item $dt$:连续时间增量;
    \item $dW_t$:多维布朗运动(Wiener 过程),代表连续时间下的随机噪声(均值为 0,方差为 $dt$);
    \item $G(\theta_t)$:噪声强度矩阵,满足 $G(\theta_t) G(\theta_t)^\top \approx \eta \cdot \Sigma(\theta_t)$(将离散噪声的协方差与连续时间的噪声强度关联)。
\end{itemize}

\textbf{规范朗之万形式}:
若令噪声强度为“各向同性常数”(即不同参数方向的噪声强度相同),设 $G = \sqrt{2T}$($T$ 为“温度”参数,控制噪声整体强度),则 SDE 可简化为标准的“朗之万动力学方程”:
\begin{equation}
d\theta_t = -\nabla L(\theta_t) dt + \sqrt{2T} dW_t
\end{equation}
其核心性质是:若存在平稳分布(参数长期运动的稳定概率分布),则该分布与目标函数 $L$ 的 Gibbs 权重成正比,即 $\propto \exp\left(-\frac{L}{T}\right)$——“温度” $T$ 越高,噪声越强,参数探索范围越广(更易跳出局部极小值);$T$ 越低,噪声越弱,参数越容易收敛到目标函数的低价值区域(极小值附近)。

\subsubsection{2.3 两类极限:消噪极限与扩散极限}
SGD 的连续极限存在两种典型场景,对应不同的训练阶段目标:
\begin{itemize}
    \item \textbf{消噪极限}:若步长 $\eta \to 0$,同时小批量大小 $B$ 增大(使噪声协方差 $\Sigma \to 0$),则 SDE 中的扩散项(噪声部分)$G(\theta_t) dW_t$ 会逐渐消失,SDE 退化为 GD 对应的“梯度流 ODE”——这对应训练后期“增大批量、减小学习率”的策略,目的是“消除噪声,精准收敛到最优解”。
    \item \textbf{扩散极限}:若按比例调整步长 $\eta$ 和批量大小 $B$(如保持 $\frac{\eta}{B}$ 为常数),使“有效噪声强度”($\eta \cdot \Sigma$)保持不变,则 SDE 的扩散项非平凡(噪声持续存在)——这对应训练前期“小批量、稍大学习率”的策略,目的是“保留噪声,通过随机探索找到更优的参数区域”。
\end{itemize}

\subsubsection{2.4 Fokker–Planck视角:参数分布的演化}
SGD 的参数在连续时间下的概率密度 $p_t(\theta)$(即参数在时刻 $t$ 处于某个值的概率),满足“Fokker–Planck 方程”:
\begin{equation}
\partial_t p_t = \nabla \cdot \left( p_t \nabla L \right) + \frac{1}{2} \sum_{i,j} \partial_i \partial_j \left( [D(\theta)]_{ij} p_t \right)
\end{equation}
其中 $D(\theta) = G(\theta) G(\theta)^\top$ 是扩散系数矩阵(代表噪声在不同参数方向的强度)。

该方程的意义是:参数密度的变化由两部分驱动——
\begin{enumerate}
    \item 确定性漂移项($\nabla \cdot (p_t \nabla L)$):由目标函数梯度主导,使参数密度向 $L$ 的低价值区域聚集(类似水流向低处);
    \item 随机扩散项(二阶导数项):由噪声主导,使参数密度向周围扩散(类似墨水在水中扩散)。
\end{enumerate}
若 $D(\theta)$ 为常数且各向同性(噪声在所有参数方向强度相同),则平稳密度为 Gibbs 分布;若 $D(\theta)$ 随参数变化或各向异性(不同方向噪声强度不同),则平稳密度会偏离简单的 Gibbs 分布——这解释了实际 SGD 中“噪声具有方向性”的现象:某些参数方向的噪声更强,参数在这些方向的探索更活跃,最终收敛位置也会偏向噪声影响更小的“平坦区域”(与“平坦极小值泛化更好”的经验观察一致)。

\subsection{局部二次近似与OU过程:常步长下的方差-曲率权衡}
在目标函数的极小值点 $\theta^*$ 附近,可将函数近似为二次形式(局部二次近似),此时 SGD 的连续极限(SDE)可简化为“Ornstein–Uhlenbeck(OU)过程”——通过分析 OU 过程的平稳分布,能清晰理解“参数曲率”与“噪声方差”的平衡关系。

\subsubsection{3.1 局部二次近似}
在 $\theta^*$ 附近,对目标函数 $L(\theta)$ 做泰勒展开并忽略高阶项,得到二次近似:
\begin{equation}
L(\theta) \approx L(\theta^*) + \frac{1}{2} (\theta - \theta^*)^\top H (\theta - \theta^*)
\end{equation}
其中 $H = \nabla^2 L(\theta^*)$ 是目标函数在 $\theta^*$ 处的 Hessian 矩阵($H \succ 0$,因 $\theta^*$ 是极小值点),代表函数在极小值附近的“局部曲率”——$H$ 的特征值越大,对应参数方向的曲率越大(函数在该方向越“陡峭”)。

\subsubsection{3.2 OU过程与平稳协方差}
令 $\vartheta_t = \theta_t - \theta^*$(参数与极小值点的偏差),代入朗之万 SDE,结合局部二次近似($\nabla L(\theta_t) \approx H \vartheta_t$),可得偏差 $\vartheta_t$ 满足的 OU 过程:
\begin{equation}
d\vartheta_t = -H \vartheta_t dt + \sqrt{2T} dW_t
\end{equation}
OU 过程是“带阻尼的线性随机过程”,其核心性质是存在\textbf{平稳分布}(当时间 $t \to \infty$ 时,$\vartheta_t$ 的分布不再变化):
\begin{itemize}
    \item 平稳分布为高斯分布 $\mathcal{N}(0, P)$,其中 $P$ 是协方差矩阵,满足“Lyapunov 方程”:
    \begin{equation}
    H P + P H = 2T I
    \end{equation}
    ($I$ 为单位矩阵,$T$ 为温度参数)。
    \item 若噪声为各向异性常数扩散($D = G G^\top$,非单位矩阵),则 Lyapunov 方程推广为:
    \begin{equation}
    H P + P H = D
    \end{equation}
\end{itemize}

\subsubsection{3.3 核心启发(B):“宽谷偏好”的物理解释}
从 Lyapunov 方程可直接推导“曲率”与“平稳方差”的关系:对 Hessian 矩阵 $H$ 的某个特征值 $\lambda_i$(对应第 $i$ 个参数方向的曲率),其对应的平稳方差 $P_{ii}$(参数在该方向的波动范围)满足:
\begin{equation}
P_{ii} = \frac{T}{\lambda_i}
\end{equation}
这意味着:在相同噪声强度(温度 $T$)下,\textbf{曲率越小($\lambda_i$ 越小)的参数方向,平稳方差越大}——即目标函数的“宽谷区域”(曲率小)对参数的“吸引概率”更高,参数更易在宽谷中稳定下来。

这一结论完美解释了深度学习中的经验观察:“更平坦的极小值泛化性能更好”——因为 SGD 的噪声会使参数自然偏向宽谷区域,而宽谷区域的参数对数据扰动更不敏感,泛化能力更强。

\subsection{学习率、批量与“温度”的定量关系}
通过动力学分析,可建立 SGD 中“学习率($\eta$)”“批量大小($B$)”与“温度($T$,噪声强度)”的明确关联,为超参数调整提供理论依据。

\subsubsection{4.1 小批量梯度的方差尺度}
设数据集总大小为 $N$,小批量大小为 $B$,在“样本独立同分布(IID)”的近似下,小批量梯度的协方差满足:
\begin{equation}
\text{Cov}\left[ \nabla L_{\mathcal{B}}(\theta) \right] \approx \left( \frac{1}{B} - \frac{1}{N} \right) C(\theta)
\end{equation}
其中 $C(\theta)$ 是单个样本梯度的协方差(与参数 $\theta$ 相关,代表数据本身的梯度波动)。
当 $B \ll N$(小批量远小于总数据量)时,$\frac{1}{N}$ 可忽略,协方差近似为 $\frac{1}{B} C(\theta)$——即批量越大,随机梯度的噪声越小(方差与批量大小成反比)。

\subsubsection{4.2 有效温度与“噪声刻度”}
结合 SDE 的扩散系数定义($D \approx \eta \cdot \text{Cov}[\nabla L_{\mathcal{B}}(\theta)]$)和 OU 过程的平稳协方差($P \propto \frac{T}{\lambda}$),可推导出“有效温度” $T$ 与学习率 $\eta$、批量大小 $B$ 的关系:
\begin{equation}
T \propto \eta \cdot \left( \frac{1}{B} - \frac{1}{N} \right)
\end{equation}
(比例系数由问题本身的尺度决定,如 $C(\theta)$ 的大小)。

这一关系揭示了“保持温度不变(噪声强度不变)”的两种等效策略:
\begin{enumerate}
    \item 减小学习率 $\eta$(降温):若批量 $B$ 不变,减小 $\eta$ 会降低有效温度,使参数更稳定收敛;
    \item 增大批量 $B$(降温):若学习率 $\eta$ 不变,增大 $B$ 会减小 $\frac{1}{B}$,同样降低有效温度,且增大批量更利于并行计算(比减小学习率更高效)。
\end{enumerate}

\textbf{核心启发(C):等效退火策略}
训练后期需要“降低噪声,精准收敛”,可采用“逐步增大批量”的“等效退火”策略——相比传统的“学习率衰减”,增大批量能在不降低更新速度的前提下减小噪声,同时利用并行计算提升训练效率,是更优的超参数调整方案。

\subsection{训练策略:将动力学结论落地到实践}
基于上述动力学分析,可总结出5条切实可行的 SGD 训练策略,直接指导实际调参与优化:

\begin{enumerate}
    \item \textbf{两阶段训练日程}:
    \begin{itemize}
        \item \textbf{探索阶段(前期)}:采用“小批量+稍大学习率”——对应扩散极限,保持较高的有效温度(噪声强度),让参数通过随机探索跳出局部极小值,找到更优的参数区域;
        \item \textbf{精调阶段(后期)}:采用“逐步增大批量或衰减学习率”——对应消噪极限,降低有效温度,使参数在优质区域内精准收敛到极小值点。
    \end{itemize}
    \item \textbf{学习率的经验上界}:若目标函数满足 L-平滑条件,优先保证学习率 $\eta < \frac{2}{L}$(避免 GD 的显式欧拉积分不稳定);若在极小值附近(局部二次区域),可通过 Hessian 矩阵的最大特征值 $\lambda_{\text{max}}$ 估算学习率上界($\eta < \frac{2}{\lambda_{\text{max}}}$),进一步提升稳定性。
    \item \textbf{常步长+迭代平均}:常步长 SGD 在局部二次近似下易形成稳定的平稳分布(OU 过程的平稳态),但参数会因噪声存在波动;对参数序列做“迭代平均”(如 Polyak–Ruppert 平均),可显著降低噪声导致的抖动,同时保留平稳分布的“宽谷偏好”,提升收敛精度与泛化能力。
    \item \textbf{预条件缓解各向异性}:目标函数的各向异性(不同参数方向曲率差异大)会导致 SGD 在大曲率方向震荡、小曲率方向推进缓慢;通过“预条件”(如对不同参数方向设置不同的有效步长,或使用 Adam 等自适应优化器),可平衡不同方向的曲率与噪声强度,缓解“快慢维”问题,加快整体收敛速度。
    \item \textbf{早停并非悖论}:虽然朗之万动力学的平稳分布需要“长时间混合”(参数充分探索),但扩散近似表明:SGD 的“先探索后收束”是宏观规律——训练前期参数快速向优质区域移动,后期若继续训练,参数可能因噪声在平稳区域内波动,反而导致泛化性能下降。因此,结合验证集监控的“早停”策略,本质是在“探索充分”与“收敛稳定”之间找最优平衡点,并非与动力学规律矛盾。
\end{enumerate}

\subsection{动力学近似的失效场景}
需注意,上述 ODE/SDE 近似并非万能,在以下4种场景中会失效,需结合实际情况调整策略:
\begin{enumerate}
    \item \textbf{大步长或强非线性区域}:步长过大时,显式欧拉积分的误差增大,ODE/SDE 的连续近似失真;目标函数强非线性(如激活函数导致的非光滑区域)会破坏局部二次近似,OU 过程的假设不成立;
    \item \textbf{重尾或异方差噪声}:若小批量梯度的噪声不满足高斯分布(重尾分布),或噪声方差随参数剧烈变化(异方差),扩散近似的偏差会显著增大;
    \item \textbf{非IID/强自相关采样}:若小批量采样非独立同分布(如时序数据的连续采样),会破坏噪声的无偏性与方差稳定性,噪声项存在“记忆效应”,SDE 的布朗运动假设(无记忆性)失效;
    \item \textbf{强各向异性}:若目标函数的曲率与噪声强度在不同方向差异极大(强各向异性),单一温度参数无法刻画噪声的方向性,需更精细的“随机平均场(SME)”或“Fokker–Planck 方程”分析,或通过预条件技术针对性优化。
\end{enumerate}

\subsection{核心关系总结}
\begin{table}[htbp]
\centering
\small
\begin{tabular}{@{}llll@{}}
\toprule
\textbf{离散算法} & \textbf{连续动力学模型} & \textbf{核心方程} & \textbf{关键参数/概念} \\ \midrule
梯度下降(GD) & 梯度流(ODE) & $\dot{\theta}(t) = -\nabla L(\theta(t))$ & 学习率 $\eta$(时间步长), L-平滑常数 $L$ \\
随机梯度下降(SGD) & 朗之万动力学(SDE) & $d\theta_t = -\nabla L dt + \sqrt{2T}dW_t$ & 有效温度 $T$(噪声强度), 批量 $B$, 扩散系数 $D$ \\
极小值附近SGD & OU过程 & $d\vartheta_t = -H\vartheta_t dt + \sqrt{2T}dW_t$ & Hessian $H$(曲率), 平稳协方差 $P$, Lyapunov方程 \\ \bottomrule
\end{tabular}
\caption{离散算法与连续动力学模型的对应关系}
\end{table}

\begin{remark}[一些简单的总结]
\textbf{1. GD $\to$ ODE}
离散更新:$\theta_{k+1} = \theta_k - \eta \nabla L(\theta_k)$ 是连续方程 $\dot{\theta}(t) = -\nabla L(\theta(t))$ 的显式欧拉近似。学习率 $\eta$ 对应时间步长。$\eta$ 过大会导致数值不稳定。稳定区间与 Hessian 最大特征值 $\lambda_{\max}$ 相关:$0<\eta<2/\lambda_{\max}(H)$。

\textbf{2. SGD $\to$ SDE}
随机梯度 $\nabla L_B = \nabla L + \xi$ 在 $\eta \to 0$ 时可连续化为 $d\theta_t = -\nabla L(\theta_t) dt + G(\theta_t) dW_t$,即朗之万动力学。噪声强度矩阵 $G G^\top \approx \eta \Sigma$。

\textbf{3. 两类极限}
\begin{itemize}
    \item \textbf{消噪极限($\eta \to 0$, $B$ 大)}:SDE $\to$ ODE,收敛。
    \item \textbf{扩散极限($\eta, B$ 按比例缩放)}:保持噪声强度,探索。
\end{itemize}

\textbf{4. Fokker–Planck 方程}
描述参数分布演化,平衡确定性漂移与随机扩散。噪声方向性解释了“平坦极小值泛化更好”。

\textbf{5. 局部二次近似 $\to$ OU 过程}
在极小值附近:$d\vartheta_t = -H\vartheta_t dt + \sqrt{2T} dW_t$。平稳协方差 $P$ 满足 $HP + PH = 2T I$。各方向方差 $P_{ii} = T/\lambda_i$,曲率越小波动越大 $\to$ SGD 自然偏向宽谷。

\textbf{6. 温度刻度}
小批量方差 $\propto 1/B$。有效温度 $T \propto \eta \left( \frac{1}{B} - \frac{1}{N} \right)$。增大 $B$ 或减小 $\eta$ 均可“降温”,即退火。

\textbf{7. 实践策略}
\begin{itemize}
    \item \textbf{前期}:小批量 + 大学习率(高温扩散探索)。
    \item \textbf{后期}:增大批量或衰减学习率(降温收敛)。
    \item \textbf{常步长 + 迭代平均} 降低噪声。
    \item \textbf{预条件处理} 各向异性。
    \item \textbf{早停} 平衡探索与收敛。
\end{itemize}

\textbf{8. 失效条件}
步长过大、非高斯噪声、非 IID 采样、强各向异性。此时需改用随机平均场或 Fokker–Planck 分析。

\textbf{核心洞见}:\textbf{SGD 是朗之万扩散在能量地形 $L(\theta)$ 上的近似积分}。学习率控制时间步,批量控制温度。广义目标是利用噪声探索宽谷、再逐步降温精调。
\end{remark}

\section{SGD之问:为什么需要动量?}

\subsection{从SGD出发:我们到底缺什么?}
SGD 的核心更新公式为:
\begin{equation}
\theta^{(k+1)}=\theta^{(k)}-\eta g^{(k)}, \quad g^{(k)} \equiv \nabla_{\theta} L\left(x^{(k)}, \theta^{(k)}\right)
\end{equation}
其中 $\theta^{(k)}$ 是第 $k$ 步参数,$\eta$ 是学习率,$g^{(k)}$ 是基于小批量数据计算的随机梯度。

在实际训练中,纯 SGD 会暴露三个典型“痛点”,这正是动量(如 Heavy-Ball、NAG)要解决的核心问题:
\begin{enumerate}
    \item \textbf{痛点1:收敛慢}
    在 L-平滑凸目标函数上,纯 SGD 即使用最优常数步长,函数值收敛速率也只能达到 $O(1/k)$;若目标函数是强凸的,收敛速率还会受“条件数 $\kappa=L/\mu$”($L$ 为平滑常数,$\mu$ 为强凸系数)控制——条件数越大(如高维模型),收敛越慢,甚至出现“硬问题”(迭代数千步仍无明显下降)。
    \item \textbf{痛点2:“峡谷之字形”震荡}
    当目标函数存在“各向异性曲率”(Hessian 矩阵特征值跨度大,即“峡谷地形”)时,纯 SGD 会在峡谷两侧来回震荡:大曲率方向(峡谷壁)的梯度大,迫使步长被“钳制”得很小;而小曲率方向(峡谷底)的梯度小,小步长导致推进缓慢,整体呈现“之字形”路径,严重浪费迭代次数。
    \item \textbf{痛点3:噪声底难以突破}
    小批量数据带来的梯度噪声,会使纯 SGD 在训练后期陷入“噪声主导”状态——参数围绕极小值点反复波动,无法继续降低函数值,形成“噪声底”,难以收敛到更优解。
\end{enumerate}

\subsection{Heavy-Ball(HB):用“惯性”优化SGD的核心痛点}
Heavy-Ball 是最经典的动量方法,核心是给 SGD 加入“惯性记忆”,通过累积历史更新方向,实现“抑制震荡、加快收敛、抵抗噪声”的效果。

\subsubsection{两种等价实现形式}
\begin{itemize}
    \item \textbf{位移形式(经典 Polyak 公式)}:直接通过历史参数差引入惯性
    \begin{equation}
    \theta^{(k+1)}=\theta^{(k)}-\eta g^{(k)}+\beta\left(\theta^{(k)}-\theta^{(k-1)}\right)
    \end{equation}
    其中 $\beta \in [0,1)$ 是动量系数,$\theta^{(k)}-\theta^{(k-1)}$ 是上一步的参数更新量(历史方向),$\beta$ 越大,惯性越强。
    \item \textbf{速度-EMA形式(深度学习常用)}:通过“指数移动平均(EMA)”维护一个“速度”变量,间接引入惯性
    \begin{equation}
    v^{(k+1)}=\beta v^{(k)}+(1-\beta) g^{(k)}, \quad \theta^{(k+1)}=\theta^{(k)}-\eta v^{(k+1)}
    \end{equation}
    其中 $v^{(k)}$ 是速度变量(可理解为“加权平均后的梯度”),$(1-\beta)$ 是当前梯度的权重,$\beta$ 是历史速度的权重——本质是对梯度做平滑,降低噪声影响。
\end{itemize}

\subsubsection{Heavy-Ball为什么有效?(基于一维/特征方向的直觉)}
以“二次目标函数”(最易理解的凸函数)为例,设 $L(\theta)=\frac{1}{2}\lambda(\theta-\theta^*)^2$($\theta^*$ 是最优解,$\lambda$ 是曲率),此时 HB 的误差($e^{(k)}=\theta^{(k)}-\theta^*$)满足二阶递推关系:
\begin{equation}
e^{(k+1)}=(1-\eta \lambda+\beta) e^{(k)}-\beta e^{(k-1)}
\end{equation}
通过选择合适的 $(\eta, \beta)$(如 $\beta=\frac{\sqrt{\kappa}-1}{\sqrt{\kappa}+1}$,$\kappa=L/\mu$ 为条件数),可实现三大优化:
\begin{itemize}
    \item \textbf{加速收敛}:将强凸二次函数上的收敛复杂度从纯 SGD 的 $O(\kappa \log \frac{1}{\varepsilon})$($\varepsilon$ 为精度要求)降低到 $O(\sqrt{\kappa} \log \frac{1}{\varepsilon})$——条件数越大,加速效果越明显;
    \item \textbf{抑制“之字形”震荡}:动量对“高频反向梯度”(如峡谷壁的来回震荡方向)提供阻尼(历史方向与当前方向相反时,惯性会抵消部分更新),对“低频一致梯度”(如峡谷底的前进方向)做累积放大,使参数沿谷底顺滑推进;
    \item \textbf{抵抗噪声}:EMA 形式的速度变量会将梯度噪声的方差按系数 $\frac{1-\beta}{1+\beta}$ 压低——例如 $\beta=0.9$ 时,噪声方差仅为纯 SGD 的约 5\%,轨迹更平滑,后期更易突破噪声底。
\end{itemize}

\subsubsection{局限性}
HB 在“二次函数”或“局部强凸目标”上效果显著,但对“一般凸目标”(无强凸性)的全局加速速率,缺乏像 NAG 那样的普适理论保证——在非强凸场景下,HB 的加速效果可能不稳定。

\subsection{Nesterov(NAG):“前瞻-校正”实现更稳健的加速}
Nesterov 动量(简称 NAG)是对 HB 的改进,核心是加入“前瞻步骤”:先根据历史惯性“预判”下一步的参数位置,再用该位置的梯度做校正,避免 HB 可能出现的“过冲”问题,实现更稳健的全局加速。

\subsubsection{3.1 两种常见实现形式}
\begin{itemize}
    \item \textbf{原始 Nesterov 加速梯度(FISTA/FGM 形态)}:先计算前瞻点,再用前瞻点的梯度更新
    \begin{equation}
    y^{(k)}=\theta^{(k)}+\beta\left(\theta^{(k)}-\theta^{(k-1)}\right), \quad \theta^{(k+1)}=y^{(k)}-\eta \nabla L\left(y^{(k)}\right)
    \end{equation}
    其中 $y^{(k)}$ 是“前瞻点”(基于历史惯性预判的下一步参数),$\nabla L(y^{(k)})$ 是前瞻点的梯度——相比 HB 直接用当前点梯度,NAG 用前瞻点梯度能更精准地捕捉“下一步的真实坡度”。
    \item \textbf{深度学习 NAG(look-ahead 梯度形态)}:结合速度变量的前瞻更新
    \begin{equation}
    v^{(k+1)}=\beta v^{(k)}+g\left(\theta^{(k)}-\eta \beta v^{(k)}\right), \quad \theta^{(k+1)}=\theta^{(k)}-\eta v^{(k+1)}
    \end{equation}
    其中 $\theta^{(k)}-\eta \beta v^{(k)}$ 是前瞻点(用历史速度预判的位置),$g(\cdot)$ 是前瞻点的梯度——本质与原始形态一致,只是用速度变量简化了计算。
\end{itemize}

\subsubsection{Nesterov为什么更优?(核心是“前瞻-校正”)}
\begin{itemize}
    \item \textbf{凸问题的全局加速保证}:在 L-平滑凸目标函数上,NAG 能通过“前瞻-校正”实现 $O(1/k^2)$ 的函数值收敛速率(纯 SGD 是 $O(1/k)$);在强凸目标上,能达到与 HB 相同的线性收敛率(最坏收敛因子 $\propto 1-\frac{1}{\sqrt{\kappa}}$),但全局理论保证更普适。
    \item \textbf{减少“过冲”与误判}:在强各向异性的“峡谷地形”中,HB 的惯性可能导致参数“冲过”谷底(因用当前点梯度判断方向,未考虑下一步的坡度变化);而 NAG 的前瞻点梯度更贴近“下一步真正会到的位置”的曲率,能提前校正惯性方向,减少反复横跳,稳定性更强。
\end{itemize}

\subsection{HB与NAG的噪声鲁棒性对比}
在相同步幅(有效更新量)下,HB 和 NAG 都能抵抗梯度噪声,但机制与适用场景略有差异:
\begin{itemize}
    \item \textbf{HB 的优势}:EMA 形式的速度平滑对“低噪声、强曲率”场景更友好——例如小批量较大(噪声小)的强凸任务,HB 的惯性能快速累积前进方向,且实现简单、调参成本低;
    \item \textbf{NAG 的优势}:前瞻-校正对“高噪声、强非线性”场景更稳健——例如小批量较小(噪声大)的深度学习任务,NAG 能减少“陈旧梯度”(当前点梯度与下一步实际梯度的偏差)导致的误判,在弯曲剧烈的噪声场中轨迹更稳定。
\end{itemize}
两者的共同局限是:若训练后期不“降温”(减学习率或增批量),都会因噪声存在“噪声底”——因此需配合“迭代平均”或“等效退火”策略,进一步提升收敛精度。

\subsection{实践选择:何时用HB,何时用NAG?}
基于目标函数特性与工程需求,可按以下原则选择:

\begin{table}[htbp]
\centering
\small
\begin{tabular}{@{}lll@{}}
\toprule
\textbf{场景特征} & \textbf{优先选择} & \textbf{理由} \\ \midrule
有理论保证需求(凸/强凸任务) & NAG & 能提供 $O(1/k^2)$ 或线性收敛的全局理论保证,结果更可控 \\
明显“峡谷地形”、过冲严重 & NAG & 前瞻-校正能减少惯性过冲,抑制横跳 \\
调参简洁、兼容现有SGD流程 & HB & EMA形式易集成到现有代码,仅需新增一个动量系数 $\beta$,调参成本低 \\
大批量、低噪声、强凸任务 & HB & 噪声小,无需复杂的前瞻校正,HB的惯性加速更直接 \\
极大条件数、低噪声(确定性任务) & 两者均可 & 都能实现 $\sqrt{\kappa}$ 级加速,NAG理论保证更优,HB实现更简单 \\ \bottomrule
\end{tabular}
\caption{HB与NAG的选择指南}
\end{table}

\subsection{核心总结}
动量的本质是给 SGD 加入“历史方向记忆”,解决纯 SGD“慢、晃、抖”的痛点:
\begin{itemize}
    \item HB 通过“惯性累积”实现加速与震荡抑制,适合简单场景与低噪声任务;
    \item NAG 通过“前瞻-校正”实现更稳健的全局加速,适合复杂场景与高噪声任务;
    \item 无论选择哪种动量,训练后期都需配合“降温”(减学习率/增批量)或“迭代平均”,才能突破噪声底,实现精准收敛。
\end{itemize}

\chapter{无约束优化之动量}

\section{符号说明}

\begin{table}[H]
\centering
\begin{tabular}{|l|l|}
\hline
\textbf{符号} & \textbf{含义} \\ \hline
$x_t$ & 第$t$步的参数向量 \\ \hline
$\nabla f(x_t)$ & 目标函数在$x_t$处的梯度 \\ \hline
$\eta$ & 学习率(步长) \\ \hline
$\beta, \beta_1, \beta_2$ & 动量/滑动平均系数(通常取 0.9, 0.999 等) \\ \hline
$v_t$ & 第$t$步的动量项(速度) \\ \hline
$G_{t,i}$ & 第$i$个参数到第$t$步的历史梯度平方和(AdaGrad) \\ \hline
$E[g^2]_t$ & 梯度平方的指数滑动平均(RMSProp) \\ \hline
$m_t$ & 一阶动量(梯度的指数滑动平均,Adam) \\ \hline
$v_t$ & 二阶动量(梯度平方的指数滑动平均,Adam) \\ \hline
$\hat{m}_t, \hat{v}_t$ & 偏置校正后的一阶/二阶动量(Adam) \\ \hline
$\varepsilon$ & 数值稳定项(防止除以零,通常取$10^{-8}$) \\ \hline
$\lambda$ & 权重衰减系数(正则化强度) \\ \hline
$\nabla_{i} f_t(x_t)$ & 目标函数在$x_t$处关于第$i$个参数的偏导数 \\ \hline
$\text{diag}(G_t)$ & 以$G_t$为对角线元素的对角矩阵 \\ \hline
$I$ & 单位矩阵 \\ \hline
\end{tabular}
\caption{符号说明}
\end{table}

\section{SGD 的缺点及动量方法改进}

\subsection{问题出发点:SGD 的局限}

标准随机梯度下降(SGD)更新:
\[
x_{t+1} = x_t - \eta \nabla f(x_t)
\]

\textbf{问题:}
\begin{itemize}
    \item 在狭长峡谷形损失面(即特征方向尺度差异大)中,梯度方向不断剧烈摆动;
    \item 沿陡峭方向振荡,沿平缓方向进展缓慢;
    \item 收敛速率接近 $O(1/t)$,远慢于二阶方法。
\end{itemize}

\subsection{引入动量的核心思想:惯性}

\textbf{想法:} 像物理中有质量的粒子那样,让优化“带惯性”。动量项记为速度 $v_t$,模拟动能积累。
\[
v_{t+1} = \beta v_t + (1-\beta)(-\nabla f(x_t))
\]
\[
x_{t+1} = x_t + \eta v_{t+1}
\]

这就是 \textbf{Polyak’s Heavy Ball (1964)}。

\subsection{Heavy-Ball (Polyak Momentum)}

\begin{algorithm}[Heavy-Ball 更新规则]
\textbf{形式:}
\[
x_{t+1} = x_t - \eta \nabla f(x_t) + \beta (x_t - x_{t-1})
\]
\end{algorithm}

\textbf{解释:}
\begin{itemize}
    \item 当前步沿梯度下降;
    \item 再加上前一步的“惯性”;
    \item 像滚动的重球在势场中前进,惯性帮助越过小坑和振荡区。
\end{itemize}

\textbf{优点:}
\begin{itemize}
    \item 加速收敛;
    \item 缓解振荡。
\end{itemize}

\textbf{缺点:}
\begin{itemize}
    \item 对非凸问题容易过冲;
    \item 需要精心调节 $\beta$ 与 $\eta$。
\end{itemize}

\subsection{Nesterov 加速梯度 (NAG, 1983)}

Nesterov 注意到 Heavy-Ball 更新\textbf{滞后}:你先算完梯度,再加惯性,但惯性早就改变了位置。他提出:\textbf{提前感知未来位置}。

\begin{algorithm}[Nesterov 加速梯度 (NAG)]
\[
v_{t+1} = \beta v_t - \eta \nabla f(x_t + \beta v_t)
\]
\[
x_{t+1} = x_t + v_{t+1}
\]
\end{algorithm}

\textbf{解释:}
\begin{itemize}
    \item 先“预测”下一个位置;
    \item 在预测点计算梯度;
    \item 因此能提前修正方向。
\end{itemize}

\textbf{直觉:}
Heavy-Ball 是“被动加速”,Nesterov 是“前瞻修正”。

\subsection{对比总结}

\begin{table}[H]
\centering
\begin{tabular}{|l|l|l|}
\hline
\textbf{特性} & \textbf{Heavy-Ball} & \textbf{Nesterov} \\ \hline
物理意义 & 惯性滚动 & 预判修正 \\ \hline
梯度计算点 & 当前点 & 预测点 \\ \hline
稳定性 & 易过冲 & 更平稳 \\ \hline
收敛速度 & $O(1/k)$ & $O(1/k^2)$在凸情形 \\ \hline
\end{tabular}
\caption{Heavy-Ball 与 Nesterov 对比}
\end{table}

\subsection{从 SGD 到动量方法的逻辑链}

SGD $\to$ 抖动严重 \\
$\to$ 引入“惯性”平滑更新 (Heavy-Ball) \\
$\to$ 进一步在“预测点”计算梯度 (Nesterov) \\
$\to$ 演化出现代动量优化器(如 Adam, RMSProp, AdaBelief)中融合动量思想的分支。

\section{AdaGrad(Duchi et al., 2011)}

\subsection{动机:SGD 学习率“一刀切”的问题}

SGD 使用固定学习率:
\[
x_{t+1} = x_t - \eta \nabla f_t(x_t)
\]

\textbf{缺陷:}
\begin{enumerate}
    \item 各参数维度梯度尺度不同,统一学习率不合理。
    \item 稀疏特征学习慢(小梯度参数被忽略)。
    \item 学习率难以手动调整。
\end{enumerate}

\textbf{核心想法:} \\
让每个参数拥有\textbf{独立的、自适应的学习率}。 \\
梯度大的维度 $\to$ 下降步长变小; \\
梯度小的维度 $\to$ 下降步长变大。

\subsection{算法公式}

\begin{algorithm}[AdaGrad 算法]
设第 $i$ 个参数的历史梯度平方和为:
\[
G_{t,i} = \sum_{\tau=1}^{t} (\nabla_{i} f_\tau(x_\tau))^2
\]

更新规则为:
\[
x_{t+1,i} = x_{t,i} - \frac{\eta}{\sqrt{G_{t,i}} + \varepsilon} \nabla_{i} f_t(x_t)
\]

或向量形式:
\[
x_{t+1} = x_t - \eta \cdot D_t^{-1/2} \nabla f_t(x_t)
\]
其中:
\[
D_t = \mathrm{diag}(G_t) + \varepsilon I
\]
\end{algorithm}

\subsection{性质与效果}

\begin{enumerate}
    \item \textbf{方向自适应}:梯度大(噪声多)的维度衰减快,梯度小的维度保持较大学习率。
    \item \textbf{天然适用于稀疏数据}(如 NLP 中的 embedding 训练),罕见词梯度少 $\to$ 步长较大。
    \item \textbf{单调递减学习率}: \\
    因为 $G_{t,i}$ 累积增长,$\sqrt{G_{t,i}}$ 也持续增大。 \\
    结果是学习率不断衰减,最终趋近于 0。
\end{enumerate}

\subsection{优缺点}

\textbf{优点:}
\begin{itemize}
    \item 不需要手动调节学习率;
    \item 稀疏特征训练效果突出;
    \item 理论上可证明收敛率 $O(1/\sqrt{t})$。
\end{itemize}

\textbf{缺点:}
\begin{itemize}
    \item 学习率衰减过快(在非凸问题上几乎停止更新);
    \item 对密集梯度任务表现差(如深度网络)。
\end{itemize}

\subsection{本质理解:累积“几何尺度”的归一化}

从几何角度看,AdaGrad 在梯度空间中进行\textbf{各向异性缩放}(anisotropic scaling):
\[
\Delta x_t = -\eta (G_t)^{-1/2} \nabla f_t(x_t)
\]
即在每个方向上使用“与历史梯度能量成反比”的缩放。 \\
可以理解为在一个逐步扭曲的\textbf{黎曼度量(Riemannian metric)}下优化。

\begin{quote}
AdaGrad 相当于在每一步都重新定义“距离”的概念,让常被更新的方向走得更谨慎,少被更新的方向走得更大胆。
\end{quote}

\section{RMSProp}

\subsection{动机:修正 AdaGrad 的“学习率枯竭”}

AdaGrad 的问题核心在于
\[ G_{t,i} = \sum_{\tau=1}^{t} (\nabla_{i} f_\tau)^2 \]
不断累加,使得分母
\[ \sqrt{G_{t,i}} \]
持续增大 $\to$ 学习率单调下降 $\to$ 在训练后期几乎不动。

\textbf{RMSProp 的核心改进:}
不再无限累积,而是使用\textbf{指数滑动平均(EMA)}仅保留“近期”梯度信息。

\subsection{算法公式}

\begin{algorithm}[RMSProp 算法]
定义平方梯度的滑动平均:
\[
 E[g^2]_t = \rho E[g^2]_{t-1} + (1-\rho) (\nabla f_t(x_t))^2
\]
其中 $\rho \in [0,1)$ 通常取 0.9。

更新规则:
\[
 x_{t+1} = x_t - \frac{\eta}{\sqrt{E[g^2]_t + \varepsilon}} \odot \nabla f_t(x_t)
\]

或分量形式:
\[
 x_{t+1,i} = x_{t,i} - \frac{\eta}{\sqrt{E[g_i^2]_t + \varepsilon}} \cdot \nabla_{i} f_t(x_t)
\]
\end{algorithm}

\subsection{性质与直觉}

\begin{enumerate}
    \item \textbf{滑动窗口的能量归一化:}
    不再记住所有历史,而只记住“近几步”的梯度能量。
    这样学习率不会无限衰减。
    \item \textbf{自适应但平稳:}
    对于方差大的维度(梯度震荡),分母变大 $\to$ 学习率下降;
    对于稳定的维度,学习率保持相对较大。
    \item \textbf{鲁棒性好}:
    对非平稳损失(如深度网络早期阶段的抖动)有缓冲作用。
\end{enumerate}

\subsection{与 AdaGrad 的对比}

\begin{table}[H]
\centering
\begin{tabular}{|l|l|l|}
\hline
\textbf{特性} & \textbf{AdaGrad} & \textbf{RMSProp} \\ \hline
梯度记忆 & 全历史累积 & 指数滑动平均 \\ \hline
学习率 & 单调递减至零 & 稳定在某范围 \\ \hline
稀疏特征适配 & 强 & 一般 \\ \hline
深度网络表现 & 弱 & 强 \\ \hline
\end{tabular}
\caption{AdaGrad 与 RMSProp 对比}
\end{table}

\subsection{本质理解}

RMSProp 相当于让优化器在一个“动态调整的、局部平滑”的度量空间中前进。
在几何意义上,它不是单调放大的尺度,而是根据当前梯度方差\textbf{实时自适应地拉伸或压缩}参数空间。

换句话说:

\begin{quote}
AdaGrad 是“记仇”的学生(过去的错误全记着); \\
RMSProp 是“善忘”的学生(只记得最近的错误)。
\end{quote}

\section{Adam(Adaptive Moment Estimation)}

\subsection{动机:融合动量与自适应学习率}

前两者的优劣:
\begin{itemize}
    \item \textbf{动量法}(Heavy-Ball / Nesterov)平滑方向,加速收敛。
    \item \textbf{RMSProp} 通过平方梯度的滑动平均调节学习率,抑制震荡。
\end{itemize}

Adam 结合两者:

\begin{quote}
“动量提供方向一致性,RMSProp提供步长自适应。”
\end{quote}

同时引入\textbf{偏置校正}(bias correction)解决初始化期估计偏小的问题。

\subsection{算法核心公式}

\begin{algorithm}[Adam 算法]
定义梯度:
\[
 g_t = \nabla f_t(x_t)
\]

\begin{enumerate}
    \item \textbf{一阶动量(梯度均值)}
    \[
    m_t = \beta_1 m_{t-1} + (1 - \beta_1) g_t
    \]

    \item \textbf{二阶动量(平方梯度均值)}
    \[
    v_t = \beta_2 v_{t-1} + (1 - \beta_2) g_t^2
    \]

    \item \textbf{偏置校正(bias correction)}
    在初期 $m_t, v_t$ 向 0 偏移,故修正为:
    \[
    \hat{m}_t = \frac{m_t}{1 - \beta_1^t}, \quad
    \hat{v}_t = \frac{v_t}{1 - \beta_2^t}
    \]

    \item \textbf{更新规则:}
    \[
    x_{t+1} = x_t - \eta \frac{\hat{m}_t}{\sqrt{\hat{v}_t} + \varepsilon}
    \]
\end{enumerate}
\end{algorithm}

\subsection{性质与优点}

\begin{enumerate}
    \item \textbf{方向平滑 + 步长自适应:}
    \begin{itemize}
        \item $m_t$ 平滑梯度方向,避免震荡;
        \item $v_t$ 控制各维度学习率,防止陡峭方向过冲。
    \end{itemize}
    \item \textbf{偏置校正保证早期稳定性:}
    使早期梯度统计不再被低估。
    \item \textbf{无需手动调参:}
    在大部分任务中默认参数即可工作良好。
    \item \textbf{适用于非平稳目标、稀疏特征与深度网络。}
\end{enumerate}

\subsection{缺点与改进方向}

\begin{enumerate}
    \item \textbf{可能欠收敛}:
    在部分凸问题中,Adam 不一定收敛到最优点(Reddi et al., 2018 指出)。
    \item \textbf{过度自适应导致步长不稳定}:
    解决方案有 \textbf{AMSGrad}、\textbf{AdamW}、\textbf{AdaBelief} 等。
\end{enumerate}

\subsection{本质理解}

Adam 在几何意义上是\textbf{时间加权的各向异性梯度法}:

\begin{itemize}
    \item $m_t$ 代表\textbf{一阶动量场},提供“惯性”;
    \item $v_t$ 则定义一个\textbf{时变的度量张量(metric tensor)},自适应调整每个方向的步长;
    \item 整体行为等价于在动态曲率修正的黎曼空间中作平滑梯度下降。
\end{itemize}

直观比喻:

\begin{quote}
SGD 是盲目走路的人, \\
RMSProp 是谨慎地根据地形调整步伐的人, \\
Adam 是既看惯性又看地形的“自动巡航”行者。
\end{quote}

\section{AdamW(Adam with Decoupled Weight)}

\subsection{动机:修正 Adam 正则化的逻辑错误}

Adam 原版通常通过 \textbf{L2 正则项} 实现权重衰减:
\[
 \min_x f(x) + \frac{\lambda}{2}|x|^2
\]
SGD 中这等价于\textbf{在梯度中加项} $\lambda x$:
\[
 x_{t+1} = x_t - \eta (\nabla f_t(x_t) + \lambda x_t)
\]
但 Adam 的更新包含自适应缩放:
\[
x_{t+1} = x_t - \eta \frac{\hat{m}_t}{\sqrt{\hat{v}_t} + \varepsilon}
\]
此时把 $+\lambda x_t$ 加进梯度会被 $(\sqrt{\hat{v}_t}+\varepsilon)^{-1}$ 缩放,
导致正则化强度与梯度统计耦合,\textbf{不再是纯粹的权重衰减}。

\subsection{核心思想:权重衰减与梯度更新解耦}

AdamW 的关键修改:
不再把 $\lambda x_t$ 加入梯度,而是直接对参数施加衰减:
\begin{algorithm}[AdamW 算法]
\[
m_t = \beta_1 m_{t-1} + (1 - \beta_1) g_t
\]
\[
v_t = \beta_2 v_{t-1} + (1 - \beta_2) g_t^2
\]
\[
\hat{m}_t = \frac{m_t}{1 - \beta_1^t}
\]
\[
\hat{v}_t = \frac{v_t}{1 - \beta_2^t}
\]
\[
x_{t+1} = x_t - \eta \left( \frac{\hat{m}_t}{\sqrt{\hat{v}_t} + \varepsilon} + \lambda x_t \right)
\]
\end{algorithm}

注意:
衰减项 $\lambda x_t$ 不再乘以自适应比例因子。

\subsection{性质与效果}

\begin{table}[H]
\centering
\begin{tabular}{|l|l|l|}
\hline
\textbf{特性} & \textbf{Adam} & \textbf{AdamW} \\ \hline
权重衰减 & 通过梯度项实现,受自适应缩放影响 & 与梯度更新解耦,恒定衰减率 \\ \hline
正则化一致性 & 不稳定 & 稳定且可控 \\ \hline
理论收敛性 & 弱 & 改善(更接近 SGD 行为) \\ \hline
实践效果 & 对超参数敏感 & 更鲁棒,普遍优于原版 Adam \\ \hline
\end{tabular}
\caption{Adam 与 AdamW 对比}
\end{table}

\subsection{本质理解}

AdamW 将优化器的两个任务分离:

\begin{enumerate}
    \item \textbf{梯度驱动更新}:由 $\hat{m}_t / \sqrt{\hat{v}_t}$ 决定方向与步长。
    \item \textbf{权重衰减}:独立控制参数幅度,起到正则作用。
\end{enumerate}

几何视角:

\begin{itemize}
    \item Adam 在一个\textbf{动态加权的度量空间}中做下降;
    \item AdamW 额外加入一个\textbf{欧式长度惩罚},保持参数范数稳定;
    \item 两者独立,因此正则化强度与自适应缩放无关。
\end{itemize}

\chapter{阻尼牛顿法}

\section{牛顿法复习}

这是牛顿法区别于梯度下降的核心,用二次函数拟合目标函数局部形态:
\[
f(x+\Delta) \approx f(x) + \nabla f(x)^\top \Delta + \frac{1}{2}\Delta^\top H(x) \Delta
\]

对二次近似函数求极值(导数为0),解出的搜索方向即“牛顿步”:
\[
\Delta_{nt} = -H(x)^{-1} \nabla f(x)
\]

\begin{itemize}
    \item \textbf{作用}:直接给出使局部二次函数最小的方向,无需像梯度下降那样手动调整步长(理论上步长为1时局部最优)。
    \item \textbf{前提}:$H(x) \succ 0$(Hessian正定),保证解唯一且为最小值点。
\end{itemize}

通过梯度与牛顿步的内积,证明牛顿步是“有效下降方向”:
\[
\nabla f(x)^\top \Delta_{nt} = -\nabla f(x)^\top H(x)^{-1} \nabla f(x) < 0
\]

\begin{itemize}
    \item \textbf{作用}:确保沿牛顿步迭代时,目标函数值会减小(内积$<0$,方向与梯度相反且符合曲率)。
    \item \textbf{关键}:因 $H(x) \succ 0$,其逆矩阵也正定,故 $\nabla f(x)^\top H(x)^{-1} \nabla f(x) > 0$,最终内积为负。
\end{itemize}

靠近最优解 $x^*$ 时,迭代误差呈“平方级”减小,收敛速度远快于梯度下降:
\[
\| x_{k+1} - x^* \| \leq C \cdot \| x_k - x^* \|^2 \quad (C>0)
\]

\begin{itemize}
    \item \textbf{作用}:体现牛顿法的核心优势——一旦进入“局部收敛域”,迭代会快速收敛到最优解。
    \item \textbf{前提}:$H(x)$ 在 $x^*$ 邻域 Lipschitz 连续,且初始点 $x^{(0)}$ 足够靠近 $x^*$。
\end{itemize}

\section{阻尼牛顿法}

阻尼牛顿法(Damped Newton Method)是为解决\textbf{纯牛顿法在远离最优解时可能不下降、发散}的问题而提出的改进方法,核心思路是:\textbf{保留牛顿方向的优势,通过引入线搜索(Line Search)确定合适的步长},而非纯牛顿法中默认的步长1,从而保证每次迭代都能使目标函数值下降,兼顾“局部快速收敛”与“全局有效下降”。

\subsection{纯牛顿法的局限性(阻尼牛顿法的必要性)}
纯牛顿法的迭代公式为 $x_{k+1} = x_k + \Delta_{nt}$(步长固定为1),但存在两个关键问题:
\begin{enumerate}
    \item \textbf{Hessian矩阵非正定}:当 $H(x_k)$ 不正定时,牛顿步 $\Delta_{nt} = -H(x_k)^{-1}\nabla f(x_k)$ 可能不是下降方向(甚至是上升方向),导致 $f(x_{k+1}) > f(x_k)$。
    \item \textbf{步长1过大}:即使 $H(x_k)$ 正定(牛顿步是下降方向),但远离最优解时,二次近似的误差较大,步长1可能导致迭代点“迈过”最优解,反而使函数值上升。
\end{enumerate}

\subsection{阻尼牛顿法的核心改进:牛顿方向 + 线搜索}
阻尼牛顿法保留“牛顿步”作为搜索方向(利用二阶信息的高效性),但通过\textbf{线搜索}动态调整步长 $\alpha_k$($\alpha_k > 0$),确保每次迭代满足 $f(x_{k+1}) < f(x_k)$。

\textbf{迭代公式:}
\[
x_{k+1} = x_k + \alpha_k \cdot \Delta_{nt}
\]
其中:
\begin{itemize}
    \item $\Delta_{nt} = -H(x_k)^{-1}\nabla f(x_k)$ 是牛顿方向(与纯牛顿法一致);
    \item $\alpha_k$ 是通过线搜索确定的步长(核心改进点,替代固定步长1)。
\end{itemize}

\subsection{线搜索:如何确定步长 $\alpha_k$?}
线搜索的目标是找到最小的 $\alpha_k$(通常从1开始尝试),使得目标函数值“充分下降”。常用准则为\textbf{Armijo准则}(保证下降性的同时避免步长过小):

\begin{definition}[Armijo准则]
Armijo准则(Armijo Criterion)是\textbf{线搜索(Line Search)中用于确定合适步长}的经典准则。

设当前迭代点为 $x_k$,搜索方向为 $d_k$(需满足“下降方向”,即 $\nabla f(x_k)^\top d_k < 0$),步长为 $\alpha > 0$。Armijo准则要求步长 $\alpha$ 满足:
\[
f(x_k + \alpha \cdot d_k) \leq f(x_k) + c \cdot \alpha \cdot \nabla f(x_k)^\top d_k
\]
其中:
\begin{itemize}
    \item $f(\cdot)$ 是目标函数;
    \item $\nabla f(x_k)$ 是 $f$ 在 $x_k$ 处的梯度;
    \item $c$ 是预设常数,满足 $0 < c < 1$(通常取 $c=10^{-4}$,控制“最小可接受的下降量”)。
\end{itemize}
\end{definition}

\textbf{实际计算步骤(回溯线搜索):}
\begin{enumerate}
    \item 初始尝试步长 $\alpha = 1$(纯牛顿法的默认步长,靠近最优解时通常有效);
    \item 检查是否满足 Armijo 准则:若 $f(x_k + \alpha d_k) \leq f(x_k) + c \cdot \alpha \cdot \nabla f(x_k)^\top d_k$,则接受该 $\alpha$;
    \item 若不满足,缩小步长(如乘以收缩因子 $\beta$,$0 < \beta < 1$,常用 $\beta=0.5$),即 $\alpha = \beta \cdot \alpha$,重复步骤2;
    \item 直到找到满足准则的 $\alpha$(理论上,因 $d_k$ 是下降方向,当 $\alpha \to 0$ 时准则必然满足,故一定能找到)。
\end{enumerate}

对于阻尼牛顿法,步长 $\alpha_k$ 需满足:
\[
f(x_k + \alpha_k \Delta_{nt}) \leq f(x_k) + c \cdot \alpha_k \cdot \nabla f(x_k)^\top \Delta_{nt}
\]

\textbf{说明:}
\begin{itemize}
    \item 不等式右边是函数值的“预期下降量”(基于一阶近似),左边是实际下降量。
    \item 因牛顿步是下降方向(若 $H$ 正定,则 $\nabla f(x_k)^\top \Delta_{nt} < 0$),$c \cdot \alpha_k \cdot (\text{负数})$ 会使右边小于 $f(x_k)$,从而强制左边(实际函数值)必须下降。
    \item 若 $\alpha=1$ 满足 Armijo 准则,则直接使用(接近最优解时通常成立,保持二次收敛);否则按比例缩小 $\alpha$(如乘以0.5),直到满足条件。
\end{itemize}

\subsection{阻尼牛顿法的优势}
\begin{enumerate}
    \item \textbf{全局下降保证}:通过线搜索,无论 $H(x_k)$ 是否正定(即使牛顿步不是下降方向,线搜索会筛选出使函数下降的步长),都能确保 $f(x_{k+1}) < f(x_k)$,避免发散。
    \item \textbf{保留局部快速收敛}:当迭代点靠近最优解 $x^*$ 时,二次近似误差很小,$\alpha_k$ 会趋近于1,此时阻尼牛顿法退化为纯牛顿法,仍保持二次收敛速度。
    \item \textbf{适用性更广}:相比纯牛顿法,对初始点的要求更低(无需“足够靠近最优解”),在非凸问题或远离最优解的场景中更稳定。
\end{enumerate}

\subsection{阻尼牛顿法的步骤总结}
\begin{algorithm}[阻尼牛顿法]
\begin{enumerate}
    \item 初始化迭代点 $x_0$,设置精度阈值 $\epsilon > 0$;
    \item 计算梯度 $\nabla f(x_k)$,若 $\|\nabla f(x_k)\| < \epsilon$,停止迭代(已收敛);
    \item 计算 Hessian 矩阵 $H(x_k)$,求解牛顿步 $\Delta_{nt} = -H(x_k)^{-1}\nabla f(x_k)$;
    \item 通过 Armijo 准则线搜索确定步长 $\alpha_k$;
    \item 更新迭代点:$x_{k+1} = x_k + \alpha_k \Delta_{nt}$,返回步骤2。
\end{enumerate}
\end{algorithm}

简言之,阻尼牛顿法通过“方向用牛顿(高效),步长靠搜索(稳定)”的策略,完美弥补了纯牛顿法的缺陷,是实际中更常用的牛顿类优化方法。

\section{阻尼牛顿法性质}

\subsection{一些假设}

\begin{itemize}
    \item (A1) $ f \in C^2(\mathbb{R}^n) $,且梯度 \textbf{Lipschitz连续}:存在常数 $ L > 0 $ 使
    \[ \|\nabla f(x) - \nabla f(y)\| \leq L\|x - y\|, \quad \forall x, y. \]
    等价于 $ f $ 为 $ L $-光滑($ L $-smooth)。
    \item (A2) $ f $ \textbf{下有界}:$ \inf_x f(x) > -\infty $。
    \item (A3) 方向矩阵 \textbf{一致有界且正定}:存在常数 $ 0 < m \leq M < \infty $,对所有 $ k $ 有
    \[ mI \preceq H_k \preceq MI. \]
    (若直接用 Hessian,则这相当于全局强凸与曲率上界;在一般非凸情形,可通过\textbf{正定化}保证。)
    \item (A4) 步长 $ \alpha_k $ 由\textbf{回溯}满足 Armijo 条件(或更强的 Wolfe 条件)。
\end{itemize}

\subsection{全局收敛到临界点}

要证明阻尼牛顿法的\textbf{全局收敛性}($\|\nabla f(x_k)\| \to 0$),我们基于上述两个引理和假设(A1)-(A4),分步骤推导:

\begin{lemma}[牛顿方向的下降性与有界性]
设 $g_k = \nabla f(x_k)$,牛顿方向 $p_k = -H_k^{-1} g_k$(由 $H_k p_k = -g_k$ 定义)。

\begin{enumerate}
    \item \textbf{下降方向证明}:
    因 $H_k \succ 0$,其逆矩阵 $H_k^{-1} \succ 0$,故
    \[
    -g_k^\top p_k = g_k^\top H_k^{-1} g_k \geq \frac{1}{\|H_k\|} \|g_k\|^2 \geq \frac{1}{M} \|g_k\|^2 > 0
    \]
    这说明 $p_k$ 是\textbf{下降方向}(梯度与方向的内积为正,即方向与负梯度同向)。

    \item \textbf{方向有界性与夹角估计}:
    由 $H_k \succeq mI$,得 $\|H_k^{-1}\| \leq \frac{1}{m}$,因此
    \[
    \|p_k\| = \|H_k^{-1} g_k\| \leq \|H_k^{-1}\| \|g_k\| \leq \frac{1}{m} \|g_k\|
    \]
    方向与负梯度的夹角余弦为:
    \[
    \cos\theta_k = \frac{-g_k^\top p_k}{\|g_k\| \|p_k\|} \geq \frac{1/M}{1/m} = \frac{m}{M} > 0
    \]
    即方向与负梯度夹角远离 $90^\circ$,是“有效下降方向”。
\end{enumerate}
\end{lemma}

\begin{lemma}[回溯线搜索的步长下界]
设(A1)-(A3)成立,定义
\[
\alpha_0 := \frac{2(1 - c_1) m^2}{L M}
\]
则回溯法接受的步长 $\alpha_k$ 满足
\[
\alpha_k \geq \underline{\alpha} := \min\{1, \beta \alpha_0\} > 0
\]
\end{lemma}

\begin{proof}
由 $f$ 是 $L$-光滑(A1),其泰勒展开满足上界:
\[
f(x_k + \alpha p_k) \leq f(x_k) + \alpha g_k^\top p_k + \frac{L}{2} \alpha^2 \|p_k\|^2
\]
Armijo 条件要求:
\[
f(x_k + \alpha p_k) \leq f(x_k) + c_1 \alpha g_k^\top p_k
\]
结合上式,只需:
\[
(1 - c_1) \alpha g_k^\top p_k + \frac{L}{2} \alpha^2 \|p_k\|^2 \leq 0
\]
令 $\delta_k = -g_k^\top p_k > 0$(因 $p_k$ 是下降方向),上式等价于:
\[
-(1 - c_1) \alpha \delta_k + \frac{L}{2} \alpha^2 \|p_k\|^2 \leq 0
\]
当 $\alpha \leq \frac{2(1 - c_1) \delta_k}{L \|p_k\|^2}$ 时成立。

由引理1,$\delta_k \geq \frac{1}{M} \|g_k\|^2$ 且 $\|p_k\| \leq \frac{1}{m} \|g_k\|$,代入得:
\[
\frac{2(1 - c_1) \delta_k}{L \|p_k\|^2} \geq \frac{2(1 - c_1)(1/M)\|g_k\|^2}{L(1/m^2)\|g_k\|^2} = \frac{2(1 - c_1) m^2}{L M} = \alpha_0
\]
因此,当 $\alpha \leq \alpha_0$ 时 Armijo 条件必成立。

回溯法从 $\alpha=1$ 开始按因子 $\beta$($0 < \beta < 1$)递减,首次满足条件的步长不少于 $\beta \alpha_0$,故 $\alpha_k \geq \min\{1, \beta \alpha_0\} = \underline{\alpha} > 0$。
\end{proof}

\begin{theorem}[全局收敛性]
在假设(A1)-(A4)下,阻尼牛顿法生成的序列满足 $\|\nabla f(x_k)\| \to 0$,即全局收敛。
\end{theorem}

\begin{proof}
\begin{enumerate}
    \item \textbf{函数值序列的单调性与有界性}:
    由 Armijo 条件(A4),$f(x_{k+1}) < f(x_k)$,即 $\{f(x_k)\}$ 单调递减。又 $f$ 下有界(A2),故 $\{f(x_k)\}$ 收敛,即
    \[
    \lim_{k\to\infty} \left( f(x_k) - f(x_{k+1}) \right) = 0
    \]

    \item \textbf{结合步长下界与下降量的关系}:
    由引理1,$-g_k^\top p_k \geq \frac{1}{M} \|g_k\|^2$;由引理2,$\alpha_k \geq \underline{\alpha} > 0$。结合 Armijo 条件的下降量:
    \[
    f(x_k) - f(x_{k+1}) \geq c_1 \alpha_k (-g_k^\top p_k) \geq c_1 \underline{\alpha} \cdot \frac{1}{M} \|g_k\|^2
    \]

    \item \textbf{级数收敛推导出梯度范数收敛}:
    由于 $\sum_{k=0}^\infty \left( f(x_k) - f(x_{k+1}) \right)$ 收敛,其通项必须趋于0,即
    \[
    \lim_{k\to\infty} \|g_k\|^2 = \lim_{k\to\infty} \|\nabla f(x_k)\|^2 = 0
    \]
    因此
    \[
    \lim_{k\to\infty} \|\nabla f(x_k)\| = 0
    \]
\end{enumerate}
\end{proof}

\subsection{局部收敛阶段的退化}

这部分内容是在\textbf{分析阻尼牛顿法在“局部收敛阶段”的行为},核心是证明:当迭代点足够靠近最优解时,阻尼牛顿法的步长会恢复为 $\alpha_k = 1$(即“单位步”,退化为纯牛顿法),从而\textbf{恢复局部二次收敛的快速速率}。

\begin{lemma}
目的是给出一个\textbf{充分条件},使得步长 $\alpha_k = 1$ 满足 Armijo 条件(即无需缩小步长,直接用纯牛顿步迭代)。

\begin{itemize}
    \item \textbf{条件}:若牛顿步 $p_k = -H_k^{-1}g_k$($g_k = \nabla f(x_k)$)满足
    \[ \|p_k\| \leq \frac{3(1 - 2c_1)}{L_H} \lambda_{\min}(H_k), \quad c_1 \in (0, \frac{1}{2}) \]
    则 $\alpha_k = 1$ 满足 Armijo 条件。
    \item \textbf{意义}:当该条件成立时,阻尼牛顿法的步长不再需要“回溯缩小”,直接用纯牛顿步迭代,从而继承纯牛顿法的\textbf{局部二次收敛速率}。
\end{itemize}
\end{lemma}

通过\textbf{带三阶余项的泰勒展开}分析函数值变化,验证 $\alpha=1$ 时 Armijo 条件成立:
\begin{itemize}
    \item 对 $f(x_k + p_k)$ 做三阶泰勒展开(余项由 Hessian 的 Lipschitz 常数 $L_H$ 控制):
    \[ f(x_k + p_k) \leq f(x_k) + g_k^\top p_k + \frac{1}{2} p_k^\top H_k p_k + \frac{L_H}{6} \|p_k\|^3 \]
    \item 代入牛顿步的定义 $g_k = -H_k p_k$,化简得:
    \[ f(x_k + p_k) \leq f(x_k) - \frac{1}{2} p_k^\top H_k p_k + \frac{L_H}{6} \|p_k\|^3 \]
    \item 对比 Armijo 条件的要求($f(x_k + p_k) \leq f(x_k) - c_1 p_k^\top H_k p_k$),推导得:当
    \[ \left( \frac{1}{2} - c_1 \right) \lambda_{\min}(H_k) \|p_k\|^2 \geq \frac{L_H}{6} \|p_k\|^3 \]
    时,Armijo 条件成立。整理后即得到引理中的条件 $\|p_k\| \leq \frac{3(1 - 2c_1)}{L_H} \lambda_{\min}(H_k)$。
\end{itemize}

当迭代点足够靠近最优解 $x^*$ 时:
\begin{itemize}
    \item 牛顿步的长度 $\|p_k\| = O(\|e_k\|)$($e_k = x_k - x^*$ 是迭代误差),且 Hessian 的最小特征值 $\lambda_{\min}(H_k) \to \lambda_{\min}(H^*)$(因 Hessian 在邻域 Lipschitz 连续)。
    \item 此时引理的条件会被满足,回溯线搜索将以 $\alpha_k = 1$ 终止,迭代退化为\textbf{纯牛顿步}:$x_{k+1} = x_k + p_k$。
    \item 进而恢复纯牛顿法的\textbf{局部二次收敛速率},误差满足:
    \[ \|e_{k+1}\| \leq C \|e_k\|^2, \quad C = \frac{L_H}{2m} \]
\end{itemize}

这部分内容的核心是\textbf{桥接阻尼牛顿法的“全局收敛稳定性”与纯牛顿法的“局部二次收敛快速性”}:通过证明“局部邻域内步长恢复为1”,说明阻尼牛顿法在全局收敛后,会快速进入二次收敛阶段,从而兼具“全局稳定下降”和“局部快速收敛”的优势。

\section{非显式求逆方法}

\subsection{Cholesky 分解(Cholesky Decomposition)}

\textbf{定义}:
若矩阵 $ A \in \mathbb{R}^{n \times n} $ 是\textbf{对称正定矩阵(symmetric positive definite)},则存在唯一的下三角矩阵 $ L $ 满足:
\[
A = L L^{\mathrm{T}}
\]
其中 $ L $ 的对角元素全为正。

\textbf{含义}:
\begin{itemize}
    \item 把正定矩阵看作“平方”出来的结果。
    \item 等价于高斯消元的稳定版本。
    \item 数值优化中,常用于求解 $ A\mathbf{x} = \mathbf{b} $ 时避免直接求逆:
    \[
    L L^{\mathrm{T}} \mathbf{x} = \mathbf{b} \Rightarrow L \mathbf{y} = \mathbf{b},\ L^{\mathrm{T}} \mathbf{x} = \mathbf{y}
    \]
\end{itemize}

\textbf{计算方式}:
对每个元素递推:
\[
L_{ii} = \sqrt{A_{ii} - \sum_{k=1}^{i-1} L_{ik}^2}, \quad
L_{ij} = \frac{1}{L_{jj}}\Big(A_{ij} - \sum_{k=1}^{j-1} L_{ik}L_{jk}\Big)\ (i>j)
\]

\textbf{条件}:
必须是正定矩阵,否则根号项出现负数,分解失败。

\subsection{LDLᵀ 分解(LDL Decomposition)}

\textbf{定义}:
若矩阵 $ A \in \mathbb{R}^{n \times n} $ 是\textbf{对称矩阵}(不要求正定),则可以分解为:
\[
A = L D L^{\mathrm{T}}
\]
其中:
\begin{itemize}
    \item $ L $ 是单位下三角矩阵(对角元为 1),
    \item $ D $ 是对角矩阵(可以含正或负元素)。
\end{itemize}

\textbf{含义}:
\begin{itemize}
    \item 是对称矩阵的广义 Cholesky 分解。
    \item 适用于\textbf{半正定或不定矩阵},不会出现在 Cholesky 分解中那样的平方根问题。
    \item 若 $ A $ 正定,则 $ D $ 全为正,退化为标准 Cholesky:
    \[
    A = (L \sqrt{D})(L \sqrt{D})^{\mathrm{T}}
    \]
\end{itemize}

\textbf{算法形式}:
递推式:
\[
D_{jj} = A_{jj} - \sum_{k=1}^{j-1} L_{jk}^2 D_{kk}, \quad
L_{ij} = \frac{1}{D_{jj}}\Big(A_{ij} - \sum_{k=1}^{j-1} L_{ik} D_{kk} L_{jk}\Big)
\]

\textbf{优点}:
\begin{itemize}
    \item 不需要取平方根,数值更稳定。
    \item 可处理不定矩阵。
\end{itemize}

\chapter{牛顿法和拟牛顿法}

\section{符号说明}

\begin{table}[H]
\centering
\begin{tabular}{|l|l|p{8cm}|}
\hline
\textbf{符号} & \textbf{类型} & \textbf{定义与用途} \\ \hline
$f(\mathbf{x})$ & 标量函数 & 无约束优化问题的目标函数,输入为$n$维优化变量$\mathbf{x}$,输出为实数值。 \\ \hline
$\nabla f(\mathbf{x})$ & 向量函数 & 目标函数$f(\mathbf{x})$的梯度,输入为$\mathbf{x}$,输出为$n$维梯度向量。 \\ \hline
$\mathbf{x}_k$ & 向量 & 第$k$步迭代点,$\mathbf{x}_k \in \mathbb{R}^n$;$\mathbf{x}_0$为初始迭代点。 \\ \hline
$\mathbf{x}^*$ & 向量 & 目标函数$f(\mathbf{x})$的最优解(近似)。 \\ \hline
$\mathbf{g}_k$ & 向量 & 第$k$步迭代点的梯度,即$\mathbf{g}_k = \nabla f(\mathbf{x}_k)$,$\mathbf{g}_k \in \mathbb{R}^n$。 \\ \hline
$\mathbf{H}$ & 矩阵 & 目标函数$f(\mathbf{x})$的Hessian矩阵(二阶导数矩阵),$\mathbf{H} \in \mathbb{R}^{n \times n}$;$\mathbf{H}_k = \nabla^2 f(\mathbf{x}_k)$为第$k$步迭代点的Hessian矩阵。 \\ \hline
$\mathbf{B}_k$ & 矩阵 & BFGS方法中第$k$步的Hessian矩阵近似,$\mathbf{B}_k \in \mathbb{R}^{n \times n}$,满足拟牛顿条件$\mathbf{B}_{k+1}\mathbf{s}_k = \mathbf{y}_k$。 \\ \hline
$\mathbf{G}_k$ & 矩阵 & $\mathbf{B}_k$的逆矩阵近似(即$\mathbf{G}_k \approx \mathbf{B}_k^{-1}$),$\mathbf{G}_k \in \mathbb{R}^{n \times n}$,用于直接计算搜索方向,避免矩阵求逆。 \\ \hline
$\mathbf{p}_k$ & 向量 & 第$k$步的搜索方向(下降方向),牛顿法中$\mathbf{p}_k = -\mathbf{H}_k^{-1}\mathbf{g}_k$,BFGS中$\mathbf{p}_k = -\mathbf{G}_k \mathbf{g}_k$。 \\ \hline
$\alpha_k$ & 标量 & 第$k$步的迭代步长,通过Wolfe条件(Armijo条件+曲率条件)确定,$\alpha_k > 0$。 \\ \hline
$\mathbf{s}_k$ & 向量 & 第$k$步的变量增量,即$\mathbf{s}_k = \mathbf{x}_{k+1} - \mathbf{x}_k$,反映迭代点的变化量。 \\ \hline
$\mathbf{y}_k$ & 向量 & 第$k$步的梯度增量,即$\mathbf{y}_k = \mathbf{g}_{k+1} - \mathbf{g}_k$,反映梯度的变化量。 \\ \hline
\end{tabular}
\caption{符号说明}
\end{table}

\section{牛顿法的严格数学建模(复习)}

Newton法是比梯度下降更高效的优化方法,核心是通过\textbf{函数的局部二次近似}确定搜索方向,兼具“局部快速收敛”与“全局有效下降”的特性,其核心逻辑围绕“二阶展开$\to$牛顿步$\to$收敛性”展开:

\subsection{核心思路:函数的局部二次近似}

梯度下降仅用“一阶信息(梯度)”将函数局部近似为线性函数,而Newton法引入“二阶信息(Hessian矩阵)”,将函数局部近似为\textbf{二次函数}(更贴合非凸函数的局部曲率),具体如下:

对迭代点$x_k$,将目标函数$f(x_k+p)$在$x_k$处做二阶泰勒展开($p$为搜索方向向量):
\[
f(x_k+p) \approx f(x_k) + g_k^\top p + \frac{1}{2}p^\top H_k p
\]
其中:
\begin{itemize}
    \item $g_k = \nabla f(x_k)$是$f(x)$在$x_k$处的梯度(一阶导数);
    \item $H_k = \nabla^2 f(x_k)$是$f(x)$在$x_k$处的Hessian矩阵(二阶导数矩阵),反映函数在$x_k$处的局部曲率。
\end{itemize}

Newton法的核心是:\textbf{最小化上述二次近似函数},直接求解使近似函数最小的搜索方向$p$。

\subsection{牛顿步(Newton Step)的推导}

对二阶近似函数关于$p$求导,并令导数为0(二次函数的极值点条件):
\[
\frac{\partial}{\partial p}\left[ f(x_k) + g_k^\top p + \frac{1}{2}p^\top H_k p \right] = g_k + H_k p = 0
\]
若Hessian矩阵\textbf{正定}($H_k \succ 0$,保证二次近似函数是凸函数,极值点为最小值点),则可解出唯一的搜索方向——\textbf{牛顿步}:
\[
p_k = -H_k^{-1} g_k
\]

\subsection{牛顿步的下降性}

牛顿步能保证是“下降方向”的前提是$H_k \succ 0$,证明如下:
计算梯度与牛顿步的内积(判断方向是否下降的核心指标,内积$<0$则为下降方向):
\[
g_k^\top p_k = g_k^\top \left( -H_k^{-1} g_k \right)
\]
因$H_k \succ 0$,其逆矩阵$H_k^{-1}$也正定,故对任意非零向量$g_k$,有$g_k^\top H_k^{-1} g_k > 0$,因此:
\[
g_k^\top p_k < 0
\]
即牛顿步满足“下降方向”的核心条件。

\subsection{局部二次收敛:牛顿法的核心优势}

当迭代点足够靠近最优解$x^*$时,Newton法会呈现\textbf{二次收敛}(收敛速度远快于梯度下降的线性收敛),严格定义如下:

\subsubsection{收敛条件}
若满足以下两个前提:
\begin{enumerate}
    \item Hessian矩阵$H(x)$在$x^*$的邻域内\textbf{Lipschitz连续}(曲率变化平缓);
    \item 初始迭代点$x_0$足够靠近$x^*$(进入“局部收敛域”)。
\end{enumerate}

\subsubsection{二次收敛公式}
此时存在常数$C > 0$,使得迭代误差满足:
\[
\| x_{k+1} - x^* \| \leq C \cdot \| x_k - x^* \|^2
\]

\subsubsection{直观意义}
二次收敛意味着“每次迭代后,误差的有效位数会翻倍”——例如:若第$k$步误差为$10^{-2}$,第$k+1$步误差可降至$10^{-4}$,第$k+2$步可降至$10^{-8}$,接近最优解时收敛极快。

牛顿法也有缺陷,数学层面的问题包括:
\begin{enumerate}
    \item \textbf{Hessian构造与求解成本高}:
    \begin{itemize}
        \item 构造Hessian矩阵$H_k$需计算$n(n+1)/2$个二阶偏导数(复杂度$O(n^2)$);
        \item 求解线性方程组$H_k p_k = -g_k$需$O(n^3)$复杂度(如LU分解),当$n$较大(如大规模优化)时计算成本不可承受。
    \end{itemize}
    \item \textbf{Hessian不定导致方向非下降}:
    若$H_k$不定(非正定时),牛顿方向$p_k = -H_k^{-1} g_k$可能不满足“下降方向”条件(即$g_k^\top p_k \geq 0$),导致迭代点$x_{k+1}$处$f(x_{k+1}) > f(x_k)$,算法不稳定。
\end{enumerate}

\section{拟牛顿法}

BFGS(Broyden-Fletcher-Goldfarb-Shanno)是最常用的拟牛顿方法之一,主要用于求解无约束优化问题。拟牛顿法的目标是通过逐步逼近目标函数的Hessian矩阵来优化目标函数,而无需显式计算二阶导数。其核心思想是通过逐步更新一个近似Hessian矩阵,使得每一步的更新尽可能接近真实的Hessian矩阵。

\subsection{拟牛顿法的基本框架}

考虑一个无约束优化问题:
\[
\min_{x} f(x),
\]
其中$f(x)$是一个可微的目标函数,$x \in \mathbb{R}^n$是优化变量。拟牛顿法的基本思路是通过计算一系列梯度来逼近目标函数的Hessian矩阵,而不是直接计算Hessian矩阵。

牛顿法的更新公式为:
\[
x_{k+1} = x_k - H_k^{-1} g_k,
\]
其中$H_k$是$f(x)$在$x_k$处的Hessian矩阵。由于计算Hessian矩阵代价较高,拟牛顿法通过构造一个近似矩阵$B_k$来代替真实的$H_k$,从而避免显式计算它。

\subsection{BFGS方法}

BFGS是一种常用的拟牛顿方法,给出了如何通过一系列梯度信息更新一个Hessian近似矩阵$B_k$的规则。其核心思想是通过引入变量$s_k = x_{k+1} - x_k$和$y_k = g_{k+1} - g_k$来更新近似Hessian矩阵。

\subsubsection{BFGS 更新公式}

BFGS方法通过以下更新公式计算$B_{k+1}$:
\[
B_{k+1} = B_k + \frac{y_k y_k^\top}{y_k^\top s_k} - \frac{B_k s_k s_k^\top B_k}{s_k^\top B_k s_k},
\]
其中:
\begin{itemize}
    \item $B_k$是第$k$步的Hessian近似。
    \item $s_k = x_{k+1} - x_k$ 是变量的变化。
    \item $y_k = g_{k+1} - g_k$ 是梯度的变化。
\end{itemize}

该公式满足\textbf{割线方程}$B_{k+1}s_k = y_k$,这是对Hessian矩阵$H$性质$H s \approx y$的近似。

公式中两个修正项的直观解释:
\begin{itemize}
    \item 第一项$\frac{y_k y_k^\top}{y_k^\top s_k}$:引入了最新的梯度变化信息$y_k$和步长信息$s_k$,是对当前Hessian近似的主要校正。
    \item 第二项$-\frac{B_k s_k s_k^\top B_k}{s_k^\top B_k s_k}$:移除了旧的、与新步长$s_k$方向相关的信息,为新的信息腾出空间。
\end{itemize}

\begin{remark}[从矩阵秩的角度理解这个式子]
\begin{enumerate}
    \item \textbf{秩1矩阵的基本性质}:若 $ a $ 是一个非零向量,则矩阵 $ a a^\top $ 是\textbf{秩1矩阵}(因为其列向量都可由 $ a $ 线性表示,秩最多为1)。除以一个非零标量(内积 $ a^\top b $ 是标量)不会改变其秩,因此形如 $ \frac{a a^\top}{a^\top b} $ 的矩阵仍为\textbf{秩1矩阵}。
    \item \textbf{分解式子中的秩1项}:
    对于更新式 $ B_{k+1} = B_k + \frac{y_k y_k^\top}{y_k^\top s_k} - \frac{B_k s_k s_k^\top B_k}{s_k^\top B_k s_k} $,我们分别分析两个修正项:
    \begin{itemize}
        \item 项1:$ \frac{y_k y_k^\top}{y_k^\top s_k} $ 是\textbf{秩1矩阵}(因 $ y_k $ 是非零向量,且 $ y_k^\top s_k \neq 0 $)。
        \item 项2:令 $ v_k = B_k s_k $($ v_k $ 是向量),则项2可表示为 $ \frac{v_k v_k^\top}{v_k^\top s_k} $,也是\textbf{秩1矩阵}(因 $ v_k $ 非零且 $ v_k^\top s_k \neq 0 $)。
    \end{itemize}
    \item \textbf{秩的变化:“秩2修正”}:整个更新式可看作对 $ B_k $ 进行\textbf{两个秩1矩阵的加减操作},即:
    \[ B_{k+1} = B_k + (\text{秩1矩阵}) - (\text{秩1矩阵}) \]
    根据矩阵秩的不等式:若 $ A, C $ 是秩分别为 $ r_A, r_C $ 的矩阵,则 $ \text{rank}(A + C) \leq r_A + r_C $,且 $ \text{rank}(A - C) \geq |r_A - r_C| $。
    这里两个修正项都是秩1矩阵,且在拟牛顿法的背景下(满足拟牛顿条件 $ B_{k+1} s_k = y_k $),这两个秩1矩阵\textbf{线性无关},因此对 $ B_k $ 的修正属于\textbf{秩2修正}(即 $ B_{k+1} $ 与 $ B_k $ 的秩差最多为2)。
    \item \textbf{拟牛顿法中的秩意义}:在拟牛顿法中,$ B_k $ 是Hessian矩阵的近似。通过这种\textbf{秩2更新},可以在保持矩阵对称性(或正定性,在一定条件下)的同时,逐步调整 $ B_k $ 的秩结构,使其逼近真实的Hessian矩阵(在算法收敛时)。
    初始时 $ B_0 $ 通常取满秩矩阵(如单位矩阵),经过每次秩2修正后,$ B_k $ 的秩会逐渐适应Hessian的秩特性,从而实现高效的梯度近似迭代。
\end{enumerate}
综上,从秩的角度看,该式子是对 $ B_k $ 进行\textbf{秩2修正}(由两个秩1矩阵的加减构成),通过这种修正来近似Hessian矩阵,同时满足拟牛顿条件,保证迭代的有效性。
\end{remark}

\subsubsection{推导过程}

\begin{itemize}
    \item \textbf{目标}:构造一个新的$B_{k+1}$来近似Hessian矩阵,使得在每一步迭代中,拟牛顿方法的更新步长和真实牛顿法的更新步长尽量相似。
    \item \textbf{确保更新的对称性和正定性}:
    \begin{itemize}
        \item 对称性:若$B_k$对称,则$B_{k+1}$也是对称的,因为$y_k y_k^\top$和$B_k s_k s_k^\top B_k$都是对称矩阵。
        \item 正定性:可以证明,如果初始矩阵$B_0$是正定的,并且线搜索满足Wolfe条件,那么后续所有的$B_k$都会保持正定。
    \end{itemize}
\end{itemize}

\begin{remark}[\texorpdfstring{$\mathbf{B}_k \to \mathbf{B}_{k+1}$}{Bk -> Bk+1}构造推导(BFGS框架)]
\textbf{一、前提与核心定义(统一符号)}
首先明确推导所需的基础记号、约束条件与优化目标:
\begin{enumerate}
    \item \textbf{迭代核心变量}:
    \begin{itemize}
        \item 位移向量:$\mathbf{s} = \mathbf{x}_{k+1} - \mathbf{x}_k$(第$k$到$k+1$步的迭代位移);
        \item 梯度差分:$\mathbf{y} = \nabla f(\mathbf{x}_{k+1}) - \nabla f(\mathbf{x}_k) = \mathbf{g}_{k+1} - \mathbf{g}_k$(对应Hessian作用于位移的近似);
    \end{itemize}
    \item \textbf{初始近似矩阵}:
    \begin{itemize}
        \item 已知$\mathbf{B}_k \succ 0$(对称正定的Hessian近似矩阵),其逆矩阵为$\mathbf{G}_k = \mathbf{B}_k^{-1}$(对称正定的逆Hessian近似);
    \end{itemize}
    \item \textbf{加权范数(度量“矩阵接近度”)}:
    为量化$\mathbf{B}_{k+1}$与$\mathbf{B}_k$的“改变量”,定义基于$\mathbf{G}_k$的加权内积与范数(保证更新对\textbf{线性变换(单位缩放、坐标变化)不变},优于仅对正交变换不变的Frobenius范数$\|\cdot\|_F$):
    \[
    \langle \mathbf{X}, \mathbf{Z} \rangle_{\mathbf{G}_k} = \text{tr}(\mathbf{G}_k \mathbf{X} \mathbf{G}_k \mathbf{Z}), \quad \|\mathbf{X}\|_{\mathbf{G}_k}^2 = \langle \mathbf{X}, \mathbf{X} \rangle_{\mathbf{G}_k}
    \]
    其中$\text{tr}(\cdot)$为矩阵迹算子;
    \item \textbf{优化目标}:
    寻找对称矩阵$\mathbf{B}_{k+1}$,满足两大核心约束:
    \begin{itemize}
        \item 割线约束:$\mathbf{B}_{k+1} \mathbf{s} = \mathbf{y}$(拟合“平均曲率”,即拟牛顿条件);
        \item 最小改变量:在满足割线约束的所有对称矩阵中,$\mathbf{B}_{k+1}$与$\mathbf{B}_k$在$\|\cdot\|_{\mathbf{G}_k}$范数下最接近。
    \end{itemize}
\end{enumerate}

\textbf{二、第一步:“先忘”——构造子空间最优矩阵$\mathbf{B}'$}
首先在\textbf{子空间$\mathcal{S}_0 = \{\mathbf{B} \mid \mathbf{B} = \mathbf{B}^\top, \mathbf{B} \mathbf{s} = 0\}$} 中,找到与$\mathbf{B}_k$最接近的矩阵$\mathbf{B}'$(即“擦除”$\mathbf{B}_k$中沿$\mathbf{s}$方向的曲率信息,使其满足$\mathbf{B}' \mathbf{s} = 0$)。

\textbf{2.1 优化问题转化(白化变换)}
直接求解$\mathbf{B}$的优化问题较复杂,通过“白化变换”将其转化为Frobenius范数下的简化问题(利用$\mathbf{G}_k = \mathbf{B}_k^{-1}$的正定性):
\begin{enumerate}
    \item \textbf{白化矩阵定义}:
    令$\mathbf{C} = \mathbf{G}_k^{1/2} \mathbf{B} \mathbf{G}_k^{1/2}$,其中$\mathbf{G}_k^{1/2}$是$\mathbf{G}_k$的对称平方根(因$\mathbf{G}_k \succ 0$,平方根存在且对称正定);
    \item \textbf{初始白化矩阵}:
    代入$\mathbf{G}_k = \mathbf{B}_k^{-1}$,得初始白化矩阵:
    \[
    \mathbf{C}_k = \mathbf{G}_k^{1/2} \mathbf{B}_k \mathbf{G}_k^{1/2} = \mathbf{G}_k^{1/2} \mathbf{G}_k^{-1} \mathbf{G}_k^{1/2} = \mathbf{I}
    \]
    其中$\mathbf{I}$为单位矩阵;
    \item \textbf{位移白化}:
    令$\tilde{\mathbf{s}} = \mathbf{G}_k^{-1/2} \mathbf{s}$(将位移向量转化到白化空间)。
\end{enumerate}

\textbf{(1)范数等价转化}
利用\textbf{迹的循环不变性}($\text{tr}(\mathbf{AB}) = \text{tr}(\mathbf{BA})$),可证明$\mathbf{B}$与$\mathbf{B}_k$的加权距离等价于白化矩阵的Frobenius距离:
\[
\|\mathbf{B} - \mathbf{B}_k\|_{\mathbf{G}_k}^2 = \text{tr}(\mathbf{G}_k (\mathbf{B}-\mathbf{B}_k) \mathbf{G}_k (\mathbf{B}-\mathbf{B}_k))
\]
将$\mathbf{B} - \mathbf{B}_k = \mathbf{G}_k^{-1/2} (\mathbf{C} - \mathbf{C}_k) \mathbf{G}_k^{-1/2}$(由$\mathbf{C} = \mathbf{G}_k^{1/2} \mathbf{B} \mathbf{G}_k^{1/2}$变形得)代入,展开后利用迹的循环不变性抵消$\mathbf{G}_k^{1/2}$与$\mathbf{G}_k^{-1/2}$,最终得:
\[
\|\mathbf{B} - \mathbf{B}_k\|_{\mathbf{G}_k}^2 = \|\mathbf{C} - \mathbf{I}\|_F^2
\]

\textbf{(2)约束等价转化}
割线约束$\mathbf{B} \mathbf{s} = 0$可转化为白化空间的约束:
\[
\mathbf{B} \mathbf{s} = 0 \implies \mathbf{G}_k^{1/2} \mathbf{B} \mathbf{G}_k^{1/2} \cdot \mathbf{G}_k^{-1/2} \mathbf{s} = \mathbf{G}_k^{1/2} \mathbf{B} \mathbf{s} = 0 \implies \mathbf{C} \tilde{\mathbf{s}} = 0
\]

\textbf{(3)转化后的优化问题}
原问题(加权范数下的约束优化)等价为Frobenius范数下的简化问题:
\[
\min_{\mathbf{C} = \mathbf{C}^\top} \frac{1}{2} \|\mathbf{C} - \mathbf{I}\|_F^2 \quad \text{s.t.} \quad \mathbf{C} \tilde{\mathbf{s}} = 0
\]

\textbf{2.2 求解白化空间优化问题(正交相似变换)}
通过构造正交基对齐约束,简化$\mathbf{C}$的结构并求解最小值:
\begin{enumerate}
    \item \textbf{构造正交矩阵$\mathbf{Q}$}:
    取正交矩阵$\mathbf{Q} = [\mathbf{q}_1, \mathbf{q}_2, \dots, \mathbf{q}_n]$(满足$\mathbf{Q}^\top \mathbf{Q} = \mathbf{Q} \mathbf{Q}^\top = \mathbf{I}$),其中:
    \begin{itemize}
        \item 第一列$\mathbf{q}_1 = \frac{\tilde{\mathbf{s}}}{\|\tilde{\mathbf{s}}\|}$($\tilde{\mathbf{s}}$的单位向量,因$\mathbf{s} \neq 0$且$\mathbf{G}_k^{-1/2}$可逆,故$\|\tilde{\mathbf{s}}\| \neq 0$);
        \item 其余列$\mathbf{q}_2, \dots, \mathbf{q}_n$为$\mathbf{q}_1$正交补空间的标准正交基。
    \end{itemize}
    此时$\mathbf{Q}^\top \tilde{\mathbf{s}} = \|\tilde{\mathbf{s}}\| \mathbf{e}_1$($\mathbf{e}_1 = (1,0,\dots,0)^\top$为标准基向量),因$\mathbf{q}_1^\top \tilde{\mathbf{s}} = \|\tilde{\mathbf{s}}\|$,$\mathbf{q}_j^\top \tilde{\mathbf{s}} = 0$($j \geq 2$)。

    \item \textbf{分块对角化$\mathbf{C}$}:
    令$\mathbf{D} = \mathbf{Q}^\top \mathbf{C} \mathbf{Q}$(正交相似变换),则:
    \begin{itemize}
        \item 对称性:因$\mathbf{C} = \mathbf{C}^\top$且$\mathbf{Q}$正交,故$\mathbf{D} = \mathbf{D}^\top$;
        \item 范数不变性:Frobenius范数在正交变换下不变,即$\|\mathbf{C} - \mathbf{I}\|_F^2 = \|\mathbf{D} - \mathbf{I}\|_F^2$;
        \item 约束转化:$\mathbf{C} \tilde{\mathbf{s}} = 0 \implies \mathbf{D} (\mathbf{Q}^\top \tilde{\mathbf{s}}) = 0 \implies \mathbf{D} (\|\tilde{\mathbf{s}}\| \mathbf{e}_1) = 0 \implies \mathbf{D} \mathbf{e}_1 = 0$(因$\|\tilde{\mathbf{s}}\| \neq 0$)。
    \end{itemize}

    \item \textbf{最小化目标函数}:
    由$\mathbf{D} = \mathbf{D}^\top$且$\mathbf{D} \mathbf{e}_1 = 0$,可知$\mathbf{D}$的第一列全为0,对称性导致第一行也全为0,故$\mathbf{D}$可分块为:
    \[
    \mathbf{D} = \begin{bmatrix} 0 & \mathbf{0}^\top \\ \mathbf{0} & \mathbf{M} \end{bmatrix}, \quad \mathbf{M} = \mathbf{M}^\top \in \mathbb{R}^{(n-1) \times (n-1)}
    \]
    代入目标函数:
    \[
    \|\mathbf{D} - \mathbf{I}\|_F^2 = \left\| \begin{bmatrix} -1 & \mathbf{0}^\top \\ \mathbf{0} & \mathbf{M} - \mathbf{I}_{n-1} \end{bmatrix} \right\|_F^2 = 1 + \|\mathbf{M} - \mathbf{I}_{n-1}\|_F^2
    \]
    要最小化该式,需$\|\mathbf{M} - \mathbf{I}_{n-1}\|_F^2 = 0$,即$\mathbf{M} = \mathbf{I}_{n-1}$。因此,白化空间的最优解为:
    \[
    \mathbf{D}^* = \begin{bmatrix} 0 & \mathbf{0}^\top \\ \mathbf{0} & \mathbf{I}_{n-1} \end{bmatrix} = \mathbf{I} - \mathbf{e}_1 \mathbf{e}_1^\top
    \]
\end{enumerate}

\textbf{2.3 反变换回$\mathbf{B}'$(从白化空间到原空间)}
\begin{enumerate}
    \item \textbf{从$\mathbf{D}^*$到$\mathbf{C}^*$}:
    由$\mathbf{D}^* = \mathbf{Q}^\top \mathbf{C}^* \mathbf{Q}$,得$\mathbf{C}^* = \mathbf{Q} \mathbf{D}^* \mathbf{Q}^\top$。代入$\mathbf{D}^* = \mathbf{I} - \mathbf{e}_1 \mathbf{e}_1^\top$,并利用$\mathbf{Q} \mathbf{e}_1 = \mathbf{q}_1 = \frac{\tilde{\mathbf{s}}}{\|\tilde{\mathbf{s}}\|}$,得:
    \[
    \mathbf{C}^* = \mathbf{I} - \mathbf{Q} \mathbf{e}_1 \mathbf{e}_1^\top \mathbf{Q}^\top = \mathbf{I} - \frac{\tilde{\mathbf{s}} \tilde{\mathbf{s}}^\top}{\|\tilde{\mathbf{s}}\|^2}
    \]

    \item \textbf{从$\mathbf{C}^*$到$\mathbf{B}'$}:
    由$\mathbf{C}^* = \mathbf{G}_k^{1/2} \mathbf{B}' \mathbf{G}_k^{1/2}$,得$\mathbf{B}' = \mathbf{G}_k^{-1/2} \mathbf{C}^* \mathbf{G}_k^{-1/2}$。代入$\mathbf{C}^*$并分拆两项展开:
    \begin{itemize}
        \item 第一项:$\mathbf{G}_k^{-1/2} \mathbf{I} \mathbf{G}_k^{-1/2} = \mathbf{G}_k^{-1} = \mathbf{B}_k$(因$\mathbf{G}_k = \mathbf{B}_k^{-1}$);
        \item 第二项:$\mathbf{G}_k^{-1/2} \cdot \frac{\tilde{\mathbf{s}} \tilde{\mathbf{s}}^\top}{\|\tilde{\mathbf{s}}\|^2} \cdot \mathbf{G}_k^{-1/2}$。
    \end{itemize}
    进一步计算第二项的分子与分母:
    \begin{itemize}
        \item 分子:$\mathbf{G}_k^{-1/2} \tilde{\mathbf{s}} = \mathbf{G}_k^{-1/2} \cdot \mathbf{G}_k^{-1/2} \mathbf{s} = \mathbf{G}_k^{-1} \mathbf{s} = \mathbf{B}_k \mathbf{s}$,故分子为$(\mathbf{B}_k \mathbf{s})(\mathbf{B}_k \mathbf{s})^\top = \mathbf{B}_k \mathbf{s} \mathbf{s}^\top \mathbf{B}_k$;
        \item 分母:$\|\tilde{\mathbf{s}}\|^2 = \tilde{\mathbf{s}}^\top \tilde{\mathbf{s}} = (\mathbf{G}_k^{-1/2} \mathbf{s})^\top (\mathbf{G}_k^{-1/2} \mathbf{s}) = \mathbf{s}^\top \mathbf{G}_k^{-1} \mathbf{s} = \mathbf{s}^\top \mathbf{B}_k \mathbf{s}$。
    \end{itemize}
    因此,“先忘”步骤的结果为:
    \[
    \mathbf{B}' = \mathbf{B}_k - \frac{\mathbf{B}_k \mathbf{s} \mathbf{s}^\top \mathbf{B}_k}{\mathbf{s}^\top \mathbf{B}_k \mathbf{s}}
    \]
    验证约束:$\mathbf{B}' \mathbf{s} = \mathbf{B}_k \mathbf{s} - \frac{\mathbf{B}_k \mathbf{s} \mathbf{s}^\top \mathbf{B}_k \mathbf{s}}{\mathbf{s}^\top \mathbf{B}_k \mathbf{s}} = \mathbf{B}_k \mathbf{s} - \mathbf{B}_k \mathbf{s} = 0$,满足子空间约束。
\end{enumerate}

\textbf{三、第二步:“再写”——构造修正项$\Delta^+$}
在$\mathbf{B}'$的基础上,添加最小改动的对称矩阵$\Delta^+$,使$\mathbf{B}_{k+1} = \mathbf{B}' + \Delta^+$满足割线约束$\mathbf{B}_{k+1} \mathbf{s} = \mathbf{y}$。

\textbf{3.1 白化空间的修正项$\Delta_{\mathbf{C}}^+$}
因$\mathbf{B}' \mathbf{s} = 0$,故割线约束等价于$\Delta^+ \mathbf{s} = \mathbf{y}$。转化到白化空间:
\begin{enumerate}
    \item \textbf{梯度差分白化}:令$\tilde{\mathbf{y}} = \mathbf{G}_k^{1/2} \mathbf{y}$(与位移白化对应);
    \item \textbf{约束转化}:$\Delta_{\mathbf{C}}^+ \tilde{\mathbf{s}} = \tilde{\mathbf{y}}$(推导同2.1.2,利用$\Delta^+ = \mathbf{G}_k^{-1/2} \Delta_{\mathbf{C}}^+ \mathbf{G}_k^{-1/2}$)。
\end{enumerate}
为满足约束且最小化改动,选择\textbf{对称秩-1矩阵}作为$\Delta_{\mathbf{C}}^+$:
\[
\Delta_{\mathbf{C}}^+ = \frac{\tilde{\mathbf{y}} \tilde{\mathbf{y}}^\top}{\tilde{\mathbf{y}}^\top \tilde{\mathbf{s}}}
\]
验证约束:$\Delta_{\mathbf{C}}^+ \tilde{\mathbf{s}} = \frac{\tilde{\mathbf{y}} (\tilde{\mathbf{y}}^\top \tilde{\mathbf{s}})}{\tilde{\mathbf{y}}^\top \tilde{\mathbf{s}}} = \tilde{\mathbf{y}}$,完全满足。

\textbf{3.2 反变换回$\Delta^+$(原空间修正项)}
由$\Delta^+ = \mathbf{G}_k^{-1/2} \Delta_{\mathbf{C}}^+ \mathbf{G}_k^{-1/2}$,代入$\Delta_{\mathbf{C}}^+$展开:
\begin{itemize}
    \item 分子:$\mathbf{G}_k^{-1/2} \tilde{\mathbf{y}} = \mathbf{G}_k^{-1/2} \cdot \mathbf{G}_k^{1/2} \mathbf{y} = \mathbf{y}$,故分子为$\mathbf{y} \mathbf{y}^\top$;
    \item 分母:$\tilde{\mathbf{y}}^\top \tilde{\mathbf{s}} = (\mathbf{G}_k^{1/2} \mathbf{y})^\top (\mathbf{G}_k^{-1/2} \mathbf{s}) = \mathbf{y}^\top \mathbf{G}_k^{1/2} \mathbf{G}_k^{-1/2} \mathbf{s} = \mathbf{y}^\top \mathbf{s}$。
\end{itemize}
因此,原空间的修正项为:
\[
\Delta^+ = \frac{\mathbf{y} \mathbf{y}^\top}{\mathbf{y}^\top \mathbf{s}}
\]
验证约束:$\Delta^+ \mathbf{s} = \frac{\mathbf{y} (\mathbf{y}^\top \mathbf{s})}{\mathbf{y}^\top \mathbf{s}} = \mathbf{y}$,满足割线约束要求。

\textbf{四、最终构造:$\mathbf{B}_{k+1}$的表达式与正定性验证}
\textbf{4.1 $\mathbf{B}_{k+1}$的最终公式}
合并“先忘”步骤的$\mathbf{B}'$与“再写”步骤的$\Delta^+$,得到BFGS方法中Hessian近似矩阵的更新公式(B-form):
\[
\boxed{\mathbf{B}_{k+1} = \mathbf{B}_k - \frac{\mathbf{B}_k \mathbf{s} \mathbf{s}^\top \mathbf{B}_k}{\mathbf{s}^\top \mathbf{B}_k \mathbf{s}} + \frac{\mathbf{y} \mathbf{y}^\top}{\mathbf{y}^\top \mathbf{s}}}
\]

\textbf{4.2 正定性验证(保证下降方向)}
若满足\textbf{曲率条件$\mathbf{s}^\top \mathbf{y} > 0$}(强Wolfe线搜索可保证),则$\mathbf{B}_{k+1} \succ 0$(对称正定),证明如下:
对任意非零向量$\mathbf{z}$,代入$\mathbf{B}_{k+1}$的表达式得:
\[
\mathbf{z}^\top \mathbf{B}_{k+1} \mathbf{z} = \mathbf{z}^\top \mathbf{B}_k \mathbf{z} - \frac{(\mathbf{z}^\top \mathbf{B}_k \mathbf{s})^2}{\mathbf{s}^\top \mathbf{B}_k \mathbf{s}} + \frac{(\mathbf{z}^\top \mathbf{y})^2}{\mathbf{y}^\top \mathbf{s}}
\]
由\textbf{Cauchy-Schwarz不等式},$\frac{(\mathbf{z}^\top \mathbf{B}_k \mathbf{s})^2}{\mathbf{s}^\top \mathbf{B}_k \mathbf{s}} \leq \mathbf{z}^\top \mathbf{B}_k \mathbf{z}$(等号仅当$\mathbf{z}$与$\mathbf{s}$线性相关时成立)。结合$\mathbf{s}^\top \mathbf{y} > 0$,第二项$\frac{(\mathbf{z}^\top \mathbf{y})^2}{\mathbf{y}^\top \mathbf{s}} \geq 0$,且仅当$\mathbf{z}^\top \mathbf{y} = 0$时为0。
\begin{itemize}
    \item 若$\mathbf{z}^\top \mathbf{y} \neq 0$:$\mathbf{z}^\top \mathbf{B}_{k+1} \mathbf{z} > 0$;
    \item 若$\mathbf{z}^\top \mathbf{y} = 0$:$\mathbf{z}^\top \mathbf{B}_{k+1} \mathbf{z} = \mathbf{z}^\top \mathbf{B}_k \mathbf{z} - \frac{(\mathbf{z}^\top \mathbf{B}_k \mathbf{s})^2}{\mathbf{s}^\top \mathbf{B}_k \mathbf{s}} \geq 0$,且仅当$\mathbf{z} = 0$时取等号(因$\mathbf{B}_k \succ 0$)。
\end{itemize}
综上,$\mathbf{B}_{k+1} \succ 0$,保证后续搜索方向$\mathbf{p}_k = -\mathbf{G}_{k+1} \mathbf{g}_k$($\mathbf{G}_{k+1} = \mathbf{B}_{k+1}^{-1}$)为下降方向。
\end{remark}

\subsection{准牛顿法的其他变种}

除了BFGS,拟牛顿法还有其他变种,例如DFP(Davidon-Fletcher-Powell)方法。DFP的更新规则与BFGS相似,但它直接更新Hessian的逆矩阵近似$G_k$。DFP方法的更新公式为:
\[
G_{k+1} = G_k + \frac{s_k s_k^\top}{s_k^\top y_k} - \frac{G_k y_k y_k^\top G_k}{y_k^\top G_k y_k}
\]

DFP与BFGS实际上是\textbf{对偶关系}。BFGS更新Hessian的近似$B_k$,而DFP更新其逆的近似$G_k$。

\section{收敛性}

\subsection{BFGS的收敛性}
BFGS的收敛性需分\textbf{目标函数类型}(二次/非二次)和\textbf{线搜索策略}(精确/不精确)讨论,核心依赖"拟牛顿条件"和"近似Hessian矩阵的正定性"。

\subsubsection{1. 二次函数下的收敛性}
若目标函数为\textbf{严格凸二次函数}(即Hessian矩阵$H$正定且恒定),且采用\textbf{精确线搜索}(即每次步长选择使目标函数沿搜索方向最小化),BFGS具有以下收敛性质:
\begin{itemize}
    \item \textbf{有限终止性}:对于$n$维二次函数,BFGS最多迭代$n$步即可收敛到全局最优解。\\
    原因:二次函数的Hessian恒定,BFGS通过迭代更新的$B_k$会逐步逼近真实$H$,当$B_k = H$时,一步即可达到最优,而理论上最多$n$步可完成逼近。
    \item \textbf{正定性保持}:若初始近似矩阵$B_0$正定,则所有迭代过程中的$B_k$均保持正定,确保搜索方向始终为"下降方向"(即$-B_k^{-1} g_k$与梯度反向),避免迭代发散。
\end{itemize}

\subsubsection{2. 非二次函数下的收敛性}
实际优化问题多为非二次函数(如机器学习中的损失函数),此时需假设目标函数满足\textbf{光滑性和凸性条件}(如二阶导数Lipschitz连续、Hessian在最优解附近正定),BFGS的收敛性质为:
\begin{itemize}
    \item \textbf{局部收敛性}:若初始点$x_0$足够接近全局最优解$x^*$,且采用精确/不精确线搜索(如Wolfe条件),BFGS会收敛到$x^*$。\\
    关键前提:最优解$x^*$处的Hessian$H^* = \nabla^2 f(x^*)$正定,确保迭代过程中梯度方向的"有效性"。
    \item \textbf{全局收敛性(凸函数下)}:若目标函数为\textbf{严格凸函数},且采用精确线搜索或满足Wolfe条件的不精确线搜索,BFGS可实现\textbf{全局收敛}(即无论初始点$x_0$如何,最终均收敛到全局最优解)。\\
    非凸函数下:仅能保证\textbf{局部收敛}(可能收敛到局部最优解),这是多数无约束优化方法的共性(除非结合全局优化策略,如随机初始化)。
\end{itemize}

\subsection{BFGS的收敛速度:介于梯度下降与牛顿法之间,逼近牛顿法}
收敛速度衡量迭代序列$\{x_k\}$趋近最优解$x^*$的快慢,常用"收敛阶"定义(如线性收敛、超线性收敛、二次收敛)。BFGS的收敛速度需结合函数类型分析:

\subsubsection{1. 收敛阶的定义(参考基准)}
\begin{itemize}
    \item \textbf{线性收敛}:存在常数$0 < c < 1$,使$\lim_{k \to \infty} \frac{\|x_{k+1} - x^*\|}{\|x_k - x^*\|} = c$(如梯度下降法)。
    \item \textbf{超线性收敛}:$\lim_{k \to \infty} \frac{\|x_{k+1} - x^*\|}{\|x_k - x^*\|} = 0$(比线性快)。
    \item \textbf{二次收敛}:存在常数$c > 0$,使$\lim_{k \to \infty} \frac{\|x_{k+1} - x^*\|}{\|x_k - x^*\|^2} = c$(如牛顿法,最快)。
\end{itemize}

\subsubsection{2. BFGS的收敛速度}
\begin{itemize}
    \item \textbf{二次函数下}:若采用精确线搜索,BFGS对$n$维二次函数是\textbf{有限收敛}(最多$n$步),本质上比二次收敛更快(无需无限迭代)。
    \item \textbf{非二次函数下}:在最优解$x^*$附近(满足Hessian正定且Lipschitz连续),BFGS是\textbf{超线性收敛}。\\
    BFGS的近似Hessian$B_k$满足$\lim_{k \to \infty} \frac{\|(B_k - H^*)p_k\|}{\|p_k\|} = 0$(Dennis-Moré条件),使得迭代步长逐渐接近牛顿法的最优步长,因此收敛速度接近牛顿法,但无需显式计算Hessian。
\end{itemize}

\subsection{BFGS的最优性:理论性质与实际性能的"最优平衡"}
BFGS的"最优性"并非指它在所有场景下都是"最好"的优化方法,而是指它在\textbf{计算成本、数值稳定性、收敛性能}三者间达到了工程应用中的"最优权衡",具体体现在以下方面:

\subsubsection{1. 理论最优性:满足拟牛顿法的核心目标}
拟牛顿法的核心目标是"用梯度信息逼近Hessian,以降低牛顿法的计算成本",BFGS在这一目标下满足:
\begin{itemize}
    \item \textbf{拟牛顿条件的严格满足}:每次更新的$B_{k+1}$均严格满足$B_{k+1} s_k = y_k$(这是逼近Hessian的核心条件),确保$B_k$对真实Hessian的逼近是"有效"的。
    \item \textbf{正定性的严格保持}:若$B_0$正定且线搜索满足"充分下降条件"(如Wolfe条件),则所有$B_k$均正定,避免搜索方向变为"上升方向"(这是DFP等方法有时难以保证的,BFGS的数值稳定性更优)。
\end{itemize}

\subsubsection{2. 实际应用中的最优性:兼顾效率与稳定性}
\begin{itemize}
    \item \textbf{计算成本最优}:
    \begin{itemize}
        \item 每次迭代仅需计算1次梯度(成本$O(n)$),更新$B_k$的成本为$O(n^2)$(无需计算Hessian的$O(n^3)$成本)。
        \item 存储成本为$O(n^2)$(仅需存储$B_k$或其逆矩阵$G_k$),适用于中大规模问题($n$从几百到几万)。
    \end{itemize}
    \item \textbf{数值稳定性最优}:\\
    相比DFP、SR1等其他拟牛顿法,BFGS对"线搜索误差"和"梯度噪声"的容忍度更高,即使采用不精确线搜索(实际应用中常用),也不易出现$B_k$奇异或迭代发散的情况。
    \item \textbf{收敛性能最优}:\\
    在相同计算成本下,BFGS的收敛速度远快于梯度下降(线性收敛),且接近牛顿法(二次收敛),同时避免了牛顿法计算Hessian和求解线性方程组的高昂成本,因此在机器学习、工程优化等领域成为"首选方法"之一。
\end{itemize}

\begin{definition}[Wolfe条件]
Wolfe条件是\textbf{不精确线搜索}的核心准则。设目标函数为$f(x)$,梯度为$g(x)$,搜索方向为$p_k$,则步长$\alpha_k$需满足:
\begin{enumerate}
    \item \textbf{Armijo条件(充分下降条件)}:
    \[ f(x_k + \alpha_k p_k) \leq f(x_k) + c_1 \alpha_k g_k^\top p_k \]
    \item \textbf{曲率条件(步长不太小条件)}:
    \[ g(x_k + \alpha_k p_k)^\top p_k \geq c_2 g_k^\top p_k \]
\end{enumerate}
其中 $0 < c_1 < c_2 < 1$(典型值 $c_1 = 10^{-4}, c_2 = 0.9$)。
\end{definition}

\section{伪代码}

\begin{algorithm}[BFGS拟牛顿优化算法]
\textbf{输入}:目标函数$f(x)$,梯度函数$\nabla f(x)$,初始点$x_0 \in \mathbb{R}^n$,收敛阈值$\epsilon > 0$,最大迭代次数$T$ \\
\textbf{输出}:最优解近似$x^*$

\begin{enumerate}
    \item \textbf{初始化}:
    \begin{itemize}
        \item 令$k = 0$,$x_k = x_0$
        \item 计算梯度$g_k = \nabla f(x_k)$
        \item 初始化Hessian逆矩阵近似$G_k = I_n$($n \times n$单位矩阵)
    \end{itemize}

    \item \textbf{收敛判断}:
    若$\|g_k\| < \epsilon$,则输出$x^* = x_k$并终止

    \item \textbf{迭代主循环}(当$k < T$时):
    \begin{enumerate}
        \item \textbf{计算搜索方向}:$p_k = -G_k g_k$(下降方向)
        \item \textbf{线搜索}:寻找步长$\alpha_k > 0$,使其满足Wolfe条件:
        \[
        \begin{cases}
        f(x_k + \alpha_k p_k) \leq f(x_k) + c_1 \alpha_k g_k^\top p_k \\
        \nabla f(x_k + \alpha_k p_k)^\top p_k \geq c_2 g_k^\top p_k
        \end{cases}
        \]
        \item \textbf{更新迭代点}:$x_{k+1} = x_k + \alpha_k p_k$
        \item \textbf{更新梯度}:$g_{k+1} = \nabla f(x_{k+1})$
        \item \textbf{计算增量}:
        \begin{itemize}
            \item 变量增量:$s_k = x_{k+1} - x_k$
            \item 梯度增量:$y_k = g_{k+1} - g_k$
        \end{itemize}
        \item \textbf{检查正定性条件}:
        若$y_k^\top s_k \leq 0$(不满足曲率条件),则令$G_{k+1} = G_k$(跳过更新);
        否则,按BFGS公式更新Hessian逆近似:
        \[
        G_{k+1} = \left(I - \frac{s_k y_k^\top}{y_k^\top s_k}\right) G_k \left(I - \frac{y_k s_k^\top}{y_k^\top s_k}\right) + \frac{s_k s_k^\top}{y_k^\top s_k}
        \]
        \item \textbf{迭代更新}:$k = k + 1$,返回步骤2
    \end{enumerate}

    \item \textbf{终止}:若达到最大迭代次数,输出$x^* = x_k$
\end{enumerate}
\end{algorithm}

\begin{remark}[逆Hessian近似形式(G-form)的推导与使用原因]
在BFGS拟牛顿法中,\textbf{逆Hessian近似形式(G-form)} 指直接对逆Hessian近似矩阵$\mathbf{G}_k = \mathbf{B}_k^{-1}$($\mathbf{B}_k$为Hessian近似矩阵)进行更新,而非对$\mathbf{B}_k$(B-form)更新。以下先推导G-form的数学表达式,再详细说明为何优先使用逆BFGS形式。

\textbf{一、G-form(逆Hessian近似)的数学推导}
G-form的更新公式需从B-form($\mathbf{B}_{k+1}$的更新)出发,利用\textbf{Woodbury矩阵求逆公式}(适用于秩修正矩阵的逆计算)推导$\mathbf{G}_{k+1} = \mathbf{B}_{k+1}^{-1}$的表达式,核心步骤如下:

\textbf{1. 已知前提与工具}
\begin{itemize}
    \item \textbf{B-form更新公式}(已推导的Hessian近似更新):
    \[
    \mathbf{B}_{k+1} = \mathbf{B}_k - \frac{\mathbf{B}_k \mathbf{s}_k \mathbf{s}_k^\top \mathbf{B}_k}{\mathbf{s}_k^\top \mathbf{B}_k \mathbf{s}_k} + \frac{\mathbf{y}_k \mathbf{y}_k^\top}{\mathbf{y}_k^\top \mathbf{s}_k}
    \]
    其中$\mathbf{s}_k = \mathbf{x}_{k+1} - \mathbf{x}_k$(位移),$\mathbf{y}_k = \mathbf{g}_{k+1} - \mathbf{g}_k$(梯度差分),且$\mathbf{B}_k \succ 0$(对称正定,故$\mathbf{G}_k = \mathbf{B}_k^{-1}$存在)。

    \item \textbf{Woodbury公式}(秩-$r$修正矩阵的逆):
    若矩阵$\mathbf{A}$可逆,$\mathbf{U},\mathbf{V}$为$n \times r$矩阵,$\mathbf{C}$为$r \times r$可逆矩阵,则:
    \[
    (\mathbf{A} + \mathbf{U} \mathbf{C} \mathbf{V}^\top)^{-1} = \mathbf{A}^{-1} - \mathbf{A}^{-1} \mathbf{U} (\mathbf{C}^{-1} + \mathbf{V}^\top \mathbf{A}^{-1} \mathbf{U})^{-1} \mathbf{V}^\top \mathbf{A}^{-1}
    \]
    本推导中$\mathbf{B}_{k+1} = \mathbf{B}_k + \mathbf{U} \mathbf{M} \mathbf{U}^\top$(秩-2修正,$\mathbf{U}$为$n \times 2$矩阵,$\mathbf{M}$为$2 \times 2$对角矩阵),符合Woodbury公式适用场景。
\end{itemize}

\textbf{2. 步骤1:将B-form改写为“原矩阵+秩修正”形式}
令:
\begin{itemize}
    \item 秩修正矩阵的列向量:$\mathbf{U} = [\mathbf{y}_k, \mathbf{B}_k \mathbf{s}_k]$($n \times 2$矩阵);
    \item 对角系数矩阵:$\mathbf{M} = \text{diag}\left( \frac{1}{\mathbf{y}_k^\top \mathbf{s}_k}, -\frac{1}{\mathbf{s}_k^\top \mathbf{B}_k \mathbf{s}_k} \right)$($2 \times 2$矩阵)。
\end{itemize}
则B-form可改写为:
\[
\mathbf{B}_{k+1} = \mathbf{B}_k + \mathbf{U} \mathbf{M} \mathbf{U}^\top
\]
(验证:$\mathbf{U} \mathbf{M} \mathbf{U}^\top = \frac{\mathbf{y}_k \mathbf{y}_k^\top}{\mathbf{y}_k^\top \mathbf{s}_k} - \frac{\mathbf{B}_k \mathbf{s}_k \mathbf{s}_k^\top \mathbf{B}_k}{\mathbf{s}_k^\top \mathbf{B}_k \mathbf{s}_k}$,与B-form一致)。

\textbf{3. 步骤2:应用Woodbury公式求$\mathbf{G}_{k+1} = \mathbf{B}_{k+1}^{-1}$}
将$\mathbf{A} = \mathbf{B}_k$、$\mathbf{V} = \mathbf{U}$、$\mathbf{C} = \mathbf{M}$代入Woodbury公式,且$\mathbf{G}_k = \mathbf{B}_k^{-1}$,展开计算:
\begin{enumerate}
    \item 计算$\mathbf{C}^{-1} = \mathbf{M}^{-1}$(对角矩阵逆为对角元素倒数):
    \[
    \mathbf{M}^{-1} = \text{diag}\left( \mathbf{y}_k^\top \mathbf{s}_k, -\mathbf{s}_k^\top \mathbf{B}_k \mathbf{s}_k \right)
    \]
    \item 计算$\mathbf{V}^\top \mathbf{A}^{-1} \mathbf{U} = \mathbf{U}^\top \mathbf{G}_k \mathbf{U}$($2 \times 2$矩阵):
    \[
    \mathbf{U}^\top \mathbf{G}_k \mathbf{U} = \begin{bmatrix} \mathbf{y}_k^\top \mathbf{G}_k \mathbf{y}_k & \mathbf{y}_k^\top \mathbf{s}_k \\ \mathbf{s}_k^\top \mathbf{y}_k & \mathbf{s}_k^\top \mathbf{B}_k \mathbf{s}_k \end{bmatrix}
    \]
    (因$\mathbf{G}_k \mathbf{B}_k = \mathbf{I}$,故$\mathbf{G}_k \mathbf{B}_k \mathbf{s}_k = \mathbf{s}_k$)。
    \item 计算$\mathbf{C}^{-1} + \mathbf{U}^\top \mathbf{G}_k \mathbf{U}$(记为$\mathbf{S}$,$2 \times 2$矩阵):
    令$\rho_k = \frac{1}{\mathbf{y}_k^\top \mathbf{s}_k}$(简化符号),则$\mathbf{y}_k^\top \mathbf{s}_k = \frac{1}{\rho_k}$,代入得:
    \[
    \mathbf{S} = \begin{bmatrix} \mathbf{y}_k^\top \mathbf{G}_k \mathbf{y}_k + \frac{1}{\rho_k} & \frac{1}{\rho_k} \\ \frac{1}{\rho_k} & 0 \end{bmatrix}
    \]
    求$\mathbf{S}^{-1}$(2阶矩阵逆公式),并代入Woodbury公式展开、化简后,最终得到\textbf{G-form的紧凑更新公式}:
    \[
    \boxed{\mathbf{G}_{k+1} = \left( \mathbf{I} - \rho_k \mathbf{s}_k \mathbf{y}_k^\top \right) \mathbf{G}_k \left( \mathbf{I} - \rho_k \mathbf{y}_k \mathbf{s}_k^\top \right) + \rho_k \mathbf{s}_k \mathbf{s}_k^\top}
    \]
\end{enumerate}

\textbf{4. G-form的关键性质验证}
\begin{itemize}
    \item \textbf{割线条件}:$\mathbf{G}_{k+1} \mathbf{y}_k = \mathbf{s}_k$(与B-form的$\mathbf{B}_{k+1} \mathbf{s}_k = \mathbf{y}_k$等价,保证曲率拟合);
    \item \textbf{对称正定}:若$\mathbf{G}_k \succ 0$且$\mathbf{s}_k^\top \mathbf{y}_k > 0$(曲率条件),则$\mathbf{G}_{k+1} \succ 0$(保证搜索方向为下降方向)。
\end{itemize}

\textbf{二、总结}
逆BFGS形式(G-form)的核心价值在于\textbf{“以更低的计算/存储成本,保留牛顿方向的高质量性”}:
\begin{enumerate}
    \item 推导上,通过Woodbury公式从B-form转化而来,严格满足拟牛顿的割线条件与正定性质;
    \item 实践上,其搜索方向计算仅需矩阵-向量乘法,且L-BFGS基于G-form实现了“有限记忆+低复杂度”,成为大规模无约束优化(如深度学习、科学计算)的默认基线方法。
\end{enumerate}
相比之下,B-form因需解线性方程组、存储成本高,仅在小规模问题($n \ll 10^3$)中偶尔使用,工程价值远低于G-form。
\end{remark}

\section{L-BFGS(有限记忆二阶近似)}

L-BFGS(Limited-Memory BFGS)的核心是\textbf{放弃显式存储$n \times n$的逆Hessian近似$\mathbf{G}_k$},仅保留最近$m$对曲率信息$(\mathbf{s}_i, \mathbf{y}_i)$($m \in [5,20]$,远小于变量维度$n$),通过“两环递推”在线构造$\mathbf{G}_k \mathbf{v}$($\mathbf{v}$为任意向量,通常取梯度$\mathbf{g}_k$)的结果,实现“低内存+低复杂度”的二阶优化。

\subsection{推导起点:BFGS的G-form递推关系}
L-BFGS源于BFGS的逆Hessian更新公式(G-form)。回顾已推导的G-form:
\[
\mathbf{G}_{k+1} = \underbrace{(\mathbf{I} - \rho_k \mathbf{s}_k \mathbf{y}_k^\top)}_{V_k} \mathbf{G}_k \underbrace{(\mathbf{I} - \rho_k \mathbf{y}_k \mathbf{s}_k^\top)}_{W_k} + \underbrace{\rho_k \mathbf{s}_k \mathbf{s}_k^\top}_{R_k} \tag{1}
\]
其中:
\begin{itemize}
    \item $V_k, W_k$为“投影矩阵”(秩-$n-1$,用于消除旧曲率中与$\mathbf{s}_k$相关的冗余信息);
    \item $R_k$为“秩-1修正项”(用于注入新曲率$(\mathbf{s}_k, \mathbf{y}_k)$)。
\end{itemize}

\textbf{关键观察:$\mathbf{G}_k$的“乘向量算子”属性}
L-BFGS不直接存储$\mathbf{G}_k$,而是关注$\mathbf{G}_k$对任意向量$\mathbf{v}$的作用(记为$\mathbf{G}_k \mathbf{v}$)。对式(1)两边同时右乘$\mathbf{v}$,展开得:
\[
\mathbf{G}_{k+1} \mathbf{v} = V_k \cdot \left( \mathbf{G}_k \cdot (W_k^\top \mathbf{v}) \right) + R_k \mathbf{v} \tag{2}
\]
上式揭示:$\mathbf{G}_{k+1} \mathbf{v}$可由$\mathbf{G}_k$对“预处理后的$\mathbf{v}$”(即$W_k^\top \mathbf{v}$)的作用,再经$V_k$投影和$R_k$修正得到。\textbf{递推展开该式},即可用历史$\{(\mathbf{s}_i, \mathbf{y}_i)\}$和初始$\mathbf{G}_0$表示$\mathbf{G}_k \mathbf{v}$,无需显式$\mathbf{G}_k$。

\subsection{Step 1:递推展开\texorpdfstring{$\mathbf{G}_k \mathbf{v}$}{G_k v}}
假设保留最近$m$对曲率信息$(\mathbf{s}_{k-m}, \mathbf{y}_{k-m}), \dots, (\mathbf{s}_{k-1}, \mathbf{y}_{k-1})$,对式(2)从$\mathbf{G}_k$反向递推至$\mathbf{G}_{k-m}$(初始逆Hessian近似,通常取缩放单位矩阵$\mathbf{G}_0 = \gamma_k \mathbf{I}$):
\begin{enumerate}
    \item 对$\mathbf{G}_k$:$\mathbf{G}_k \mathbf{v} = V_{k-1} \cdot \left( \mathbf{G}_{k-1} \cdot (W_{k-1}^\top \mathbf{v}) \right) + R_{k-1} \mathbf{v}$
    \item 对$\mathbf{G}_{k-1}$:$\mathbf{G}_{k-1} \cdot (W_{k-1}^\top \mathbf{v}) = V_{k-2} \cdot \left( \mathbf{G}_{k-2} \cdot (W_{k-2}^\top \cdot W_{k-1}^\top \mathbf{v}) \right) + R_{k-2} \cdot (W_{k-1}^\top \mathbf{v})$
    \item 以此类推,直到$\mathbf{G}_{k-m}$:
    \[
    \mathbf{G}_k \mathbf{v} = V_{k-1} V_{k-2} \dots V_{k-m} \cdot \left( \mathbf{G}_{k-m} \cdot (W_{k-m}^\top \dots W_{k-2}^\top W_{k-1}^\top \mathbf{v}) \right) + \text{秩-1修正项总和} \tag{3}
    \]
\end{enumerate}
式(3)可拆分为两部分:
\begin{itemize}
    \item \textbf{右乘链}:$W_{k-m}^\top \dots W_{k-1}^\top \mathbf{v}$(对应“反向环”,处理所有$W_i^\top$对$\mathbf{v}$的预处理);
    \item \textbf{左乘链+修正项}:$V_{k-m} \dots V_{k-1} \cdot (\mathbf{G}_0 \cdot \text{右乘结果}) + \text{修正项}$(对应“正向环”,处理$V_i$投影和$R_i$修正)。
\end{itemize}

\subsection{Step 2:反向环(Right Loop)——处理右乘链}
反向环的目标是计算“预处理后的向量”$\mathbf{q}$,即式(3)中$\mathbf{G}_0$的输入:$\mathbf{q} = W_{k-m}^\top \dots W_{k-1}^\top \mathbf{v}$。

\textbf{3.1 单个$W_i^\top$的作用}
由$W_i = \mathbf{I} - \rho_i \mathbf{y}_i \mathbf{s}_i^\top$,其转置为$W_i^\top = \mathbf{I} - \rho_i \mathbf{s}_i \mathbf{y}_i^\top$(因$\mathbf{s}_i \mathbf{y}_i^\top$的转置为$\mathbf{y}_i \mathbf{s}_i^\top$)。对任意向量$\mathbf{z}$,$W_i^\top \mathbf{z}$的计算为:
\[
W_i^\top \mathbf{z} = \mathbf{z} - \rho_i \mathbf{s}_i (\mathbf{y}_i^\top \mathbf{z}) \tag{4}
\]
但结合递推顺序(从$i=k-1$到$i=k-m$),需定义\textbf{中间系数$\alpha_i$} 简化计算:令$\alpha_i = \rho_i (\mathbf{s}_i^\top \mathbf{z})$,则式(4)可改写为:
\[
\mathbf{z} \leftarrow \mathbf{z} - \alpha_i \mathbf{y}_i \tag{5}
\]
(注:$\alpha_i$记录了$\mathbf{s}_i$与当前$\mathbf{z}$的内积信息,后续正向环需复用该系数,避免重复计算。)

\textbf{3.2 反向环完整流程}
初始化$\mathbf{q} = \mathbf{v}$(初始向量,若计算搜索方向则$\mathbf{v} = \mathbf{g}_k$),从最近的曲率对开始,自后向前迭代($i = k-1, k-2, \dots, k-m$):
\[
\begin{cases}
\alpha_i = \rho_i \cdot \mathbf{s}_i^\top \mathbf{q} \\
\mathbf{q} = \mathbf{q} - \alpha_i \cdot \mathbf{y}_i
\end{cases} \tag{6}
\]
\textbf{作用}:通过$m$次迭代,将所有$W_i^\top$的作用“吸收”到$\mathbf{q}$中,得到$\mathbf{q} = W_{k-m}^\top \dots W_{k-1}^\top \mathbf{v}$,为后续$\mathbf{G}_0$作用做准备。

\subsection{Step 3:初始缩放(Initial Scaling)——\texorpdfstring{$\mathbf{G}_0$}{G_0}的作用}
L-BFGS的初始逆Hessian近似$\mathbf{G}_0$不直接取$\mathbf{I}$(单位矩阵),而是取\textbf{缩放单位矩阵},目的是让$\mathbf{G}_0$的尺度接近真实逆Hessian$\nabla^2 f(\mathbf{x}_k)^{-1}$的尺度,提升方向质量。

\textbf{4.1 缩放系数$\gamma_k$的选择}
缩放系数$\gamma_k$由最近一次的曲率对$(\mathbf{s}_{k-1}, \mathbf{y}_{k-1})$计算,满足“模拟Hessian的对角尺度”:
\[
\gamma_k = \frac{\mathbf{s}_{k-1}^\top \mathbf{y}_{k-1}}{\mathbf{y}_{k-1}^\top \mathbf{y}_{k-1}} \tag{7}
\]
\textbf{物理意义}:$\mathbf{s}_{k-1}^\top \mathbf{y}_{k-1}$是“平均曲率”的近似($\mathbf{y}_{k-1} \approx \nabla^2 f(\bar{\mathbf{x}}) \mathbf{s}_{k-1}$,故$\mathbf{s}_{k-1}^\top \mathbf{y}_{k-1} \approx \mathbf{s}_{k-1}^\top \nabla^2 f(\bar{\mathbf{x}}) \mathbf{s}_{k-1}$),$\gamma_k$相当于$\nabla^2 f(\bar{\mathbf{x}})^{-1}$的“对角平均”,确保$\mathbf{G}_0 = \gamma_k \mathbf{I}$的尺度合理。

\textbf{4.2 初始缩放计算}
对反向环得到的$\mathbf{q}$,施加$\mathbf{G}_0$的作用:
\[
\mathbf{r} = \gamma_k \cdot \mathbf{q} \tag{8}
\]
此时$\mathbf{r} = \mathbf{G}_0 \cdot \mathbf{q} = \gamma_k W_{k-m}^\top \dots W_{k-1}^\top \mathbf{v}$,对应式(3)中$\mathbf{G}_{k-m} \cdot (\text{右乘结果})$。

\subsection{Step 4:正向环(Left Loop)——处理左乘链与修正项}
正向环的目标是将式(3)中的“左乘链$V_{k-m} \dots V_{k-1}$”和“秩-1修正项总和”融入$\mathbf{r}$,最终得到$\mathbf{G}_k \mathbf{v}$。

\textbf{5.1 单个$V_i$与$R_i$的作用}
由$V_i = \mathbf{I} - \rho_i \mathbf{s}_i \mathbf{y}_i^\top$和$R_i = \rho_i \mathbf{s}_i \mathbf{s}_i^\top$,结合式(2)的递推逻辑,对任意向量$\mathbf{z}$,$V_i \mathbf{z} + R_i \mathbf{v}$的计算为:
\[
V_i \mathbf{z} + R_i \mathbf{v} = \mathbf{z} - \rho_i \mathbf{s}_i (\mathbf{y}_i^\top \mathbf{z}) + \rho_i \mathbf{s}_i (\mathbf{s}_i^\top \mathbf{v}) \tag{9}
\]
利用反向环中已存储的$\alpha_i = \rho_i (\mathbf{s}_i^\top \mathbf{v})$(式(6)),定义\textbf{中间系数$\beta_i = \rho_i (\mathbf{y}_i^\top \mathbf{z})$},则式(9)可简化为:
\[
\mathbf{z} \leftarrow \mathbf{z} + \mathbf{s}_i (\alpha_i - \beta_i) \tag{10}
\]
\textbf{推导验证}:
\[
\mathbf{z} - \beta_i \mathbf{s}_i + \alpha_i \mathbf{s}_i = \mathbf{z} + \mathbf{s}_i (\alpha_i - \beta_i)
\]
完全匹配式(9),且复用了反向环的$\alpha_i$,避免重复计算$\mathbf{s}_i^\top \mathbf{v}$。

\textbf{5.2 正向环完整流程}
从最早保留的曲率对开始,自前向后迭代($i = k-m, k-m+1, \dots, k-1$):
\[
\begin{cases}
\beta_i = \rho_i \cdot \mathbf{y}_i^\top \mathbf{r} \\
\mathbf{r} = \mathbf{r} + \mathbf{s}_i \cdot (\alpha_i - \beta_i)
\end{cases} \tag{11}
\]
\textbf{作用}:通过$m$次迭代,将所有$V_i$的左乘作用和$R_i$的秩-1修正融入$\mathbf{r}$,最终$\mathbf{r}$即为$\mathbf{G}_k \mathbf{v}$的结果:
\[
\mathbf{r} = \mathbf{G}_k \mathbf{v} \tag{12}
\]

\subsection{Step 5:L-BFGS搜索方向与迭代流程}
当$\mathbf{v} = \mathbf{g}_k$(当前梯度)时,由式(12)得$\mathbf{r} = \mathbf{G}_k \mathbf{g}_k$,因此\textbf{L-BFGS的搜索方向}为:
\[
\mathbf{p}_k = -\mathbf{r} = -\mathbf{G}_k \mathbf{g}_k \tag{13}
\]
(与BFGS方向一致,保证是下降方向,因$\mathbf{G}_k \succ 0$且$\mathbf{g}_k^\top \mathbf{p}_k = -\mathbf{g}_k^\top \mathbf{G}_k \mathbf{g}_k < 0$)。

\begin{algorithm}[L-BFGS完整迭代流程]
\begin{enumerate}
    \item \textbf{初始化}:
    \begin{itemize}
        \item 初始点$\mathbf{x}_0$,最大记忆数$m$,线搜索参数(强Wolfe),容忍度$\text{tol}$;
        \item 清空存储队列$\mathcal{S} = []$(存$\mathbf{s}_i$)、$\mathcal{Y} = []$(存$\mathbf{y}_i$),$k=0$;
        \item 计算$\mathbf{g}_0 = \nabla f(\mathbf{x}_0)$,若$\|\mathbf{g}_0\| \leq \text{tol}$,停止。
    \end{itemize}

    \item \textbf{方向计算}:
    \begin{itemize}
        \item 若$k=0$(无历史曲率):取$\mathbf{p}_0 = -\gamma_0 \mathbf{g}_0$($\gamma_0$取1或经验值);
        \item 若$k \geq 1$:执行“反向环(式6)$\to$ 初始缩放(式8)$\to$ 正向环(式11)”,得$\mathbf{p}_k = -\mathbf{r}$。
    \end{itemize}

    \item \textbf{线搜索}:
    用强Wolfe条件求步长$\alpha_k > 0$,满足:
    \[
    f(\mathbf{x}_k + \alpha_k \mathbf{p}_k) \leq f(\mathbf{x}_k) + c_1 \alpha_k \mathbf{g}_k^\top \mathbf{p}_k, \quad |\mathbf{g}(\mathbf{x}_k + \alpha_k \mathbf{p}_k)^\top \mathbf{p}_k| \leq c_2 |\mathbf{g}_k^\top \mathbf{p}_k|
    \]
    ($c_1 \in (0,1), c_2 \in (c_1,1)$,强Wolfe保证$\mathbf{s}_k^\top \mathbf{y}_k > 0$,即曲率条件成立)。

    \item \textbf{更新与存储管理}:
    \begin{itemize}
        \item 计算$\mathbf{x}_{k+1} = \mathbf{x}_k + \alpha_k \mathbf{p}_k$,$\mathbf{s}_k = \alpha_k \mathbf{p}_k$,$\mathbf{g}_{k+1} = \nabla f(\mathbf{x}_{k+1})$,$\mathbf{y}_k = \mathbf{g}_{k+1} - \mathbf{g}_k$;
        \item 若$\mathbf{s}_k^\top \mathbf{y}_k > 0$(曲率条件):将$\mathbf{s}_k$加入$\mathcal{S}$,$\mathbf{y}_k$加入$\mathcal{Y}$;若队列长度$>m$,删除最早的$\mathbf{s}_{k-m}$和$\mathbf{y}_{k-m}$;
        \item 计算$\gamma_{k+1} = \mathbf{s}_k^\top \mathbf{y}_k / (\mathbf{y}_k^\top \mathbf{y}_k)$(供下次初始缩放)。
    \end{itemize}

    \item \textbf{终止判断}:
    若$\|\mathbf{g}_{k+1}\| \leq \text{tol}$,停止;否则$k=k+1$,返回步骤2。
\end{enumerate}
\end{algorithm}

\subsection{关键性质验证}
\begin{enumerate}
    \item \textbf{割线条件保持}:L-BFGS的两环递推严格继承BFGS的割线条件$\mathbf{G}_k \mathbf{y}_i = \mathbf{s}_i$($i = k-m, \dots, k-1$),确保曲率拟合的准确性;
    \item \textbf{正定性保证}:若所有保留的$\mathbf{s}_i^\top \mathbf{y}_i > 0$,则$\mathbf{G}_k \succ 0$,搜索方向$\mathbf{p}_k$必为下降方向;
    \item \textbf{复杂度优势}:
    \begin{itemize}
        \item 内存复杂度:仅存储$2m$个$n$维向量($\mathcal{S}$和$\mathcal{Y}$),为$O(nm)$(标准BFGS为$O(n^2)$);
        \item 时间复杂度:每次方向计算需$2m$次向量内积和$2m$次向量加法,为$O(nm)$(标准BFGS为$O(n^2)$),适配大规模问题($n \gtrsim 10^5$)。
    \end{itemize}
\end{enumerate}

L-BFGS的推导核心是\textbf{“将G-form的矩阵递推转化为向量操作”}:通过反向环吸收右乘投影、正向环融合左乘投影与秩-1修正,仅用$m$对历史曲率对在线模拟$\mathbf{G}_k \mathbf{v}$的计算,既保留了BFGS的二阶收敛性,又解决了标准BFGS的内存瓶颈。其本质是“用少量历史信息近似逆Hessian”,是大规模无约束优化(如深度学习、科学计算)的默认基线方法。

\chapter{优化算法的评价}

\section{预备知识与符号}

\begin{itemize}
    \item \textbf{目标函数}:$f: \mathbb{R}^n \to \mathbb{R}$,可微。
    \item \textbf{L-光滑(Lipschitz梯度)}:存在$L > 0$使
    \[
    \|\nabla f(x) - \nabla f(y)\| \leq L\|x - y\|, \quad \forall x, y.
    \]
    \item \textbf{凸/强凸}:$f$凸;若存在$\mu > 0$使
    \[
    f(y) \geq f(x) + \nabla f(x)^\top (y - x) + \frac{\mu}{2}\|y - x\|^2,
    \]
    则$f$为$\mu$-强凸。
    \item \textbf{PL(Polyak-Łojasiewicz)不等式}:存在$\mu > 0$使
    \[
    \frac{1}{2}\|\nabla f(x)\|^2 \geq \mu\left(f(x) - f^*\right), \quad \forall x.
    \]
    \textit{注}:PL $\nRightarrow$ 凸,但常见于过参数模型/深网络的局部区域。
    \item \textbf{GD更新}:$x_{k+1} = x_k - \eta_k \nabla f(x_k)$,其中步长$\eta_k > 0$。
\end{itemize}

\section{基本工具:下降引理(Descent Lemma)}

\begin{lemma}[下降引理]
若函数$f$为$L$-光滑,则对任意$x, d$与$\eta > 0$,有:
\[
f(x + \eta d) \leq f(x) + \eta \nabla f(x)^\top d + \frac{L}{2}\eta^2 \|d\|^2
\]
\end{lemma}

\begin{proof}
由$L$-光滑性的定义,对任意$y$有:
\[
f(y) \leq f(x) + \nabla f(x)^\top (y - x) + \frac{L}{2}\|y - x\|^2
\]
令$y = x + \eta d$(即沿方向$d$步长$\eta$移动),代入后即得下降引理的不等式。
\end{proof}

\begin{corollary}[沿负梯度下降]
取$d = -\nabla f(x)$(沿\textbf{负梯度方向}更新,梯度下降算法的核心思想),代入下降引理得:
\[
f(x - \eta \nabla f(x)) \leq f(x) - \left( \eta - \frac{L}{2}\eta^2 \right) \|\nabla f(x)\|^2
\]

进一步,当$0 < \eta \leq \frac{1}{L}$时,函数值\textbf{单调下降},且有:
\[
f(x_{k+1}) \leq f(x_k) - \frac{\eta}{2} \|\nabla f(x_k)\|^2
\]

\begin{itemize}
    \item 该推论明确了\textbf{梯度下降的下降性保证}:只要步长$\eta$满足$0 < \eta \leq \frac{1}{L}$,每一步迭代后目标函数值必严格减小,确保算法的"下降"特性。
    \item 为\textbf{步长选择}提供了理论依据(步长不超过$\frac{1}{L}$时算法稳定下降),也为后续收敛速率分析奠定了基础。
\end{itemize}
\end{corollary}

\section{基于Descent Lemma的四类典型收敛结果}

\subsection{非凸 + L-光滑}

由推论2.2,当$0 < \eta \leq \frac{1}{L}$时,梯度下降的单步迭代满足:
\[
f(x_{k+1}) \leq f(x_k) - \frac{\eta}{2} \|\nabla f(x_k)\|^2
\]

将$k = 0, 1, \dots, K-1$的不等式依次展开并累加:
\begin{align*}
f(x_1) &\leq f(x_0) - \frac{\eta}{2} \|\nabla f(x_0)\|^2 \\
f(x_2) &\leq f(x_1) - \frac{\eta}{2} \|\nabla f(x_1)\|^2 \\
&\vdots \\
f(x_K) &\leq f(x_{K-1}) - \frac{\eta}{2} \|\nabla f(x_{K-1})\|^2
\end{align*}
将这些不等式左右两边分别相加,\textbf{中间项$f(x_1), f(x_2), \dots, f(x_{K-1})$会相互抵消},最终得到:
\[
f(x_K) \leq f(x_0) - \frac{\eta}{2} \sum_{k=0}^{K-1} \|\nabla f(x_k)\|^2
\]

已知$f$下有界(即$f_{\text{inf}} = \inf_x f(x) > -\infty$),因此对任意$K$,有$f(x_K) \geq f_{\text{inf}}$。将其代入上式:
\[
f_{\text{inf}} \leq f(x_0) - \frac{\eta}{2} \sum_{k=0}^{K-1} \|\nabla f(x_k)\|^2
\]
整理得梯度范数平方和的上界:
\[
\sum_{k=0}^{K-1} \|\nabla f(x_k)\|^2 \leq \frac{2\left(f(x_0) - f_{\text{inf}}\right)}{\eta}
\]

将上式两边同时除以$K$,得到\textbf{平均梯度范数平方}的上界:
\[
\frac{1}{K} \sum_{k=0}^{K-1} \|\nabla f(x_k)\|^2 \leq \frac{2\left(f(x_0) - f_{\text{inf}}\right)}{\eta K}
\]

进一步分析\textbf{最小梯度范数}的量级:
由于平均值不小于"集合中的最小值"(即$\frac{1}{K} \sum_{k=0}^{K-1} \|\nabla f(x_k)\|^2 \geq \min_{0 \leq k < K} \|\nabla f(x_k)\|^2$),因此:
\[
\min_{0 \leq k < K} \|\nabla f(x_k)\|^2 \leq \frac{2\left(f(x_0) - f_{\text{inf}}\right)}{\eta K} = O\left(\frac{1}{K}\right)
\]

该定理在\textbf{非凸深度学习场景}中具有关键价值:
\begin{enumerate}
    \item 即使损失函数非凸,只要满足L-光滑且下有界,梯度下降的\textbf{平均梯度范数会随迭代次数增加而趋于0},说明算法能逐步逼近"驻点"(最优解的必要条件)。
    \item 收敛速率为$O(1/K)$,明确了"迭代次数越多,平均梯度越小"的量化规律,为训练过程的收敛性分析提供了理论依据。
    \item 解释了"为什么深度网络在梯度下降训练中能逐步收敛"——即使损失函数非凸,L-光滑性和下有界性确保了梯度的整体衰减趋势。
\end{enumerate}

\subsection{凸 + L-光滑}

对\textbf{"凸且L-光滑函数"的梯度下降(GD)算法进行收敛性分析},核心是推导"函数值次优量"与"解的平方距离差分"的关联不等式(即\textbf{望远镜技巧}的核心步骤)

\begin{itemize}
    \item \textbf{函数设定}:$f: \mathbb{R}^n \to \mathbb{R}$是\textbf{凸且L-光滑}的(即满足$\|\nabla f(y) - \nabla f(x)\| \leq L\|y - x\|$),最优解为$x^* \in \arg\min f$。
    \item \textbf{算法设定}:梯度下降取\textbf{固定步长}$\eta = \frac{1}{L}$,迭代规则为$x_{k+1} = x_k - \frac{1}{L}\nabla f(x_k)$。
    \item \textbf{基础工具}:
    \begin{itemize}
        \item 凸函数的一阶性质(3.2.a):$f(y) \geq f(x) + \langle \nabla f(x), y - x \rangle$(凸函数的定义式,用于将"梯度内积"转化为"函数值差距")。
        \item 向量范数展开(3.2.c):$\|a - c\|^2 = \|b - c\|^2 + 2\langle a - b, b - c \rangle + \|a - b\|^2$(用于对"解的距离"$\|x_{k+1} - x^*\|^2$做代数展开)。
    \end{itemize}
\end{itemize}

\subsubsection{步骤1:解的距离展开(3.2.d)}
令$a = x_{k+1}$、$b = x_k$、$c = x^*$,结合梯度下降的更新式$x_{k+1} - x_k = -\frac{1}{L}\nabla f(x_k)$,代入范数展开式(3.2.c)得:
\[
\|x_{k+1} - x^*\|^2 = \|x_k - x^*\|^2 - \frac{2}{L}\langle \nabla f(x_k), x_k - x^* \rangle + \frac{1}{L^2}\|\nabla f(x_k)\|^2
\]

\subsubsection{步骤2:结合凸性,转化梯度内积为函数值差距}
由凸函数的性质(3.2.b):$f(x) - f^* \leq \langle \nabla f(x), x - x^* \rangle$,代入(3.2.d)得:
\[
\|x_{k+1} - x^*\|^2 \leq \|x_k - x^*\|^2 - \frac{2}{L}(f(x_k) - f^*) + \frac{1}{L^2}\|\nabla f(x_k)\|^2
\]

\subsubsection{步骤3:消去梯度范数,关联函数值下降量}
利用L-光滑函数的"下降性质",推导得:
\[
\frac{1}{L^2}\|\nabla f(x_k)\|^2 \leq \frac{2}{L}(f(x_k) - f(x_{k+1}))
\]

\subsubsection{步骤4:关键不等式(望远镜差分,3.2.g)}
将(3.2.f)代回(3.2.e)并整理,最终得到:
\[
\frac{2}{L}(f(x_{k+1}) - f^*) \leq \|x_k - x^*\|^2 - \|x_{k+1} - x^*\|^2
\]
\textbf{意义}:这是"望远镜技巧"的核心——它将"函数值到最优的差距"与"解的距离的差分"直接关联。后续对$k$累加该不等式,可消去中间项,从而推导出\textbf{收敛速率}(如函数值次优量的$O(1/K)$速率)。

\begin{theorem}[$O(1/k)$次优率]
若$f$凸且$L$-光滑,取$\eta = \frac{1}{L}$,则
\[
f(x_k) - f^* \leq \frac{L}{2k}\|x_0 - x^*\|^2, \quad k \geq 1.
\]
\end{theorem}

\begin{proof}[证明(望远镜技巧)]
对$k = 0, \dots, T-1$求和,右侧望远镜展开:
\[
\frac{2}{L}\sum_{k=0}^{T-1} \left(f(x_{k+1}) - f^*\right) \leq \|x_0 - x^*\|^2 - \|x_T - x^*\|^2 \leq \|x_0 - x^*\|^2.
\]

因此
\[
\frac{1}{T}\sum_{k=1}^{T} \left(f(x_k) - f^*\right) \leq \frac{L}{2T}\|x_0 - x^*\|^2. \tag{3.2.h}
\]

又知$\{f(x_k)\}$单调不增,故
\[
f(x_T) - f^* \leq \frac{1}{T}\sum_{k=1}^{T} \left(f(x_k) - f^*\right) \leq \frac{L}{2T}\|x_0 - x^*\|^2.
\]

令$T = k$即得结论。

\textbf{一般步长$\eta \in (0, 1/L]$:}

完全同样的推导(把上文中的$1/L$换为一般$\eta$)给出
\[
f(x_k) - f^* \leq \frac{\|x_0 - x^*\|^2}{2\eta k}, \quad \text{取} \ \eta = \frac{1}{L} \text{恢复定理常数。} \tag{3.2.i}
\]
\end{proof}

\subsection{$\mu$-强凸 + L-光滑:线性(几何)收敛}

强凸与光滑结合下的梯度下降"线性收敛"

\begin{theorem}
若$f$为$\mu$-强凸且$L$-光滑,取$0 < \eta \leq \frac{2}{\mu + L}$,则
\[
\|x_{k+1} - x^*\|^2 \leq \rho^2 \|x_k - x^*\|^2, \quad \rho := \max\{1 - \eta\mu, |1 - \eta L|\} < 1,
\]
进而
\[
f(x_k) - f^* \leq \frac{L}{2} \rho^{2k} \|x_0 - x^*\|^2.
\]
\end{theorem}

\begin{proof}[证明:从"梯度与Hessian"到"几何收缩"]
我们先引入误差项$e_k := x_k - x^*$,梯度下降的更新式为$x_{k+1} = x_k - \eta \nabla f(x_k)$。

\paragraph{第一步:Hessian的"平均积分表示"}
由一维积分形式的平均Hessian,梯度的差可表示为:
\[
\nabla f(x_k) - \nabla f(x^*) = \left( \int_0^1 \nabla^2 f(x^* + t(x_k - x^*)) dt \right) (x_k - x^*) =: H_k e_k,
\]
其中$H_k$是对称矩阵,且满足$\mu I \preceq H_k \preceq L I$(强凸和L-光滑的核心体现:Hessian的特征值被$\mu$和$L$"夹逼")。

\begin{remark}[Hessian积分表示的推导]
令$g(t) = \nabla f\left(x^* + t(x_k - x^*)\right)$,其中$t \in [0,1]$。
\begin{itemize}
    \item 当$t=0$时,$g(0) = \nabla f(x^*)$;
    \item 当$t=1$时,$g(1) = \nabla f(x_k)$。
\end{itemize}

对$g(t)$关于$t$求导(利用\textbf{Hessian的定义}:$\nabla f$的导数是Hessian矩阵$\nabla^2 f$):
\[
g'(t) = \nabla^2 f\left(x^* + t(x_k - x^*)\right) \cdot (x_k - x^*)
\]
根据\textbf{牛顿-莱布尼茨公式}(微积分基本定理),有:
\[
g(1) - g(0) = \int_0^1 g'(t) \, dt
\]
将$g(1) = \nabla f(x_k)$、$g(0) = \nabla f(x^*)$和$g'(t)$的表达式代入,即可得到:
\[
\nabla f(x_k) - \nabla f(x^*) = \left( \int_0^1 \nabla^2 f\left(x^* + t(x_k - x^*)\right) dt \right) (x_k - x^*)
\]
这一步的价值在于\textbf{将"梯度差"转化为"Hessian的积分平均形式"},从而可以利用"$\mu$-强凸(Hessian下界$\mu I$)"和"L-光滑(Hessian上界$L I$)"的条件,分析误差项$x_k - x^*$的收缩性(即后续的几何收敛)。

简单来说,它是"强凸+L-光滑"场景下\textbf{梯度下降线性收敛证明的"第一块基石"}——通过把梯度差和Hessian联系起来,我们才能量化误差的几何收缩速率。
\end{remark}

\paragraph{第二步:误差项的迭代收缩}
将更新式代入误差项$e_{k+1} = x_{k+1} - x^*$,得:
\[
e_{k+1} = e_k - \eta (\nabla f(x_k) - \nabla f(x^*)) = (I - \eta H_k) e_k.
\]

由于$H_k$可正交对角化,矩阵$I - \eta H_k$的\textbf{谱范数}(即最大特征值的绝对值)由$H_k$的特征值范围决定。结合$\mu \leq \lambda(H_k) \leq L$,得:
\[
\|e_{k+1}\| \leq \max_{\lambda \in [\mu, L]} |1 - \eta \lambda| \cdot \|e_k\|.
\]

记$\rho := \max\{1 - \eta\mu, |1 - \eta L|\}$,当$0 < \eta \leq \frac{2}{\mu + L}$时,可验证$1 - \eta\mu < 1$且$|1 - \eta L| < 1$,故$\rho < 1$。因此误差项的平方满足\textbf{几何收缩}:
\[
\|e_{k+1}\|^2 \leq \rho^2 \|e_k\|^2.
\]

\paragraph{第三步:函数值次优量的收敛}
迭代展开误差项得$\|e_k\|^2 \leq \rho^{2k} \|e_0\|^2$。再结合"上二次界"$f(x) - f^* \leq \frac{L}{2} \|x - x^*\|^2$,即可推出函数值次优量的几何收敛:
\[
f(x_k) - f^* \leq \frac{L}{2} \rho^{2k} \|x_0 - x^*\|^2.
\]
\end{proof}

\subsubsection{最优步长的"黄金选择"}
当取$\eta^* = \frac{2}{\mu + L}$时,两端点的绝对值相等:$|1 - \eta^* \mu| = |1 - \eta^* L| = \frac{L - \mu}{L + \mu}$,此时最优收缩因子$\rho^* = \frac{L - \mu}{L + \mu} = \frac{\kappa - 1}{\kappa + 1}$(其中$\kappa = \frac{L}{\mu}$是"条件数",刻画强凸与光滑的平衡)。

之前凸光滑场景是$O(1/k)$的\textbf{次线性收敛},而这里强凸+光滑下是$\rho^{2k}$的\textbf{几何收敛}(也叫线性收敛)——前者是"慢慢悠悠逼近",后者是"指数级收缩",效率天差地别。这解释了为什么"强凸假设"能让优化算法"跑起来"。

强凸($\mu > 0$)让函数"长得更陡",L-光滑($L < \infty$)让函数"长得不突兀"。两者结合时,Hessian的特征值被"夹在$\mu$和$L$之间",这才使得误差项能通过矩阵谱范数实现"几何收缩"。这种"刚柔并济"的结构,是线性收敛的核心密码。

虽然深度学习中损失函数大多非强凸,但这个结论是\textbf{正则化、fine-tuning阶段}的理论参照——当模型接近收敛时,局部可能近似满足强凸性,此时梯度下降的收敛会呈现"加速特性"。同时,"最优步长$\eta^* = \frac{2}{\mu + L}$"也为自适应步长设计提供了灵感。

\subsection{PL条件 + L-光滑:线性收敛(无需凸)}

这部分聚焦\textbf{非凸场景下的线性收敛分析},核心是通过"PL条件(替代凸性)+ L-光滑(保证梯度平滑)"的组合,证明梯度下降在非凸函数上也能实现线性收敛

\begin{theorem}
若$f$满足\textbf{PL条件}且\textbf{L-光滑},取步长$\eta = \frac{1}{L}$,则
\[
f(x_{k+1}) - f^* \leq \left(1 - \frac{\mu}{L}\right)\left(f(x_k) - f^*\right).
\]
\end{theorem}

PL(Polyak-Łojasiewicz)不等式定义:存在$\mu > 0$,使得对所有$x$,
\[
\frac{1}{2}\|\nabla f(x)\|^2 \geq \mu \left(f(x) - f^*\right), \quad \forall x.
\]
\begin{remark}[PL条件的意义]
它把"梯度的大小"和"函数离最优值的差距(次优量$f(x)-f^*$)"直接关联,是"非凸场景下替代凸性"的关键——无需函数全局凸,只要局部满足该不等式,就能量化梯度与优化程度的关系。
\end{remark}

由L-光滑的"沿负梯度下降"推论,当$0 < \eta \leq \frac{1}{L}$时,梯度下降单步迭代满足:
\[
f(x - \eta \nabla f(x)) \leq f(x) - \frac{\eta}{2}\|\nabla f(x)\|^2.
\]
将步长$\eta = \frac{1}{L}$代入不等式,得:
\[
f(x_{k+1}) \leq f(x_k) - \frac{1}{2L}\|\nabla f(x_k)\|^2.
\]

再结合PL不等式$\frac{1}{2}\|\nabla f(x_k)\|^2 \geq \mu \left(f(x_k) - f^*\right)$,两边同乘$\frac{1}{L}$得:
\[
\frac{1}{2L}\|\nabla f(x_k)\|^2 \geq \frac{\mu}{L}\left(f(x_k) - f^*\right).
\]

将其代入函数值下降的不等式,整理后得到:
\[
f(x_{k+1}) - f^* \leq \left(1 - \frac{\mu}{L}\right)\left(f(x_k) - f^*\right).
\]

\begin{itemize}
    \item \textbf{非凸场景的线性收敛突破}:无需凸性假设,仅靠"PL(梯度与次优量挂钩)+ L-光滑(梯度变化平滑)",就能让函数次优量以\textbf{几何级数(线性速率)}收缩——这解释了"为什么非凸的深度模型训练能收敛到较好效果"(过参数化网络局部常满足PL条件)。
    \item \textbf{条件的普适性}:PL条件比凸性弱,更贴合深度学习中损失函数的非凸特性,是分析非凸优化收敛性的核心工具之一。
\end{itemize}

\chapter{从无约束到等式约束:拉格朗日乘子}

\section{拉格朗日乘子数学建模}

\subsection{1. 无约束优化的数学模型}
无约束优化的基本模型为:
\[
\min_{x \in \mathbb{R}^n} f(x)
\]
其中$f: \mathbb{R}^n \to \mathbb{R}$是可微函数。其\textbf{一阶最优性条件}(极值点的必要条件)为:
\[
\nabla f(x^*) = 0
\]
即目标函数在最优解$x^*$处的梯度为零(可微函数的无约束极值点必是梯度为零的点)。

\subsection{2. 等式约束优化的数学模型}
当引入\textbf{等式约束}后,模型变为:
\[
\begin{cases} \min_{x \in \mathbb{R}^n} & f(x) \\ \text{s.t.} & h_i(x) = 0,\ i=1,\dots,m \end{cases}
\]
其中$h_i: \mathbb{R}^n \to \mathbb{R}$是可微的约束函数,$m$是约束个数(通常$m < n$)。

此时,变量$x$被限制在由$m$个等式约束定义的\textbf{可行域}上,无法直接应用无约束的"梯度为零"条件——因为可行域是约束的"限制空间",需考虑约束对优化的"限制作用"。

\subsection{3. 拉格朗日函数的建模思想}
为了将\textbf{等式约束"融入"目标函数},我们引入\textbf{拉格朗日乘子}$\lambda = (\lambda_1, \dots, \lambda_m)^\top \in \mathbb{R}^m$,构造\textbf{拉格朗日函数}:
\[
\mathcal{L}(x, \lambda) = f(x) + \lambda^\top h(x)
\]
其中$h(x) = (h_1(x), \dots, h_m(x))^\top$是约束函数的向量形式,$\lambda^\top h(x) = \sum_{i=1}^m \lambda_i h_i(x)$是"约束项与乘子的耦合项"。

\subsection{4. 拉格朗日函数的最优性条件(一阶KKT条件)}
对拉格朗日函数$\mathcal{L}(x, \lambda)$分别关于$x$和$\lambda$求偏导,并令其为零,得到\textbf{一阶最优性条件}(等式约束下的KKT条件核心):

\begin{itemize}
    \item \textbf{对$x$求梯度并令其为零}:
    \[
    \nabla_x \mathcal{L}(x^*, \lambda^*) = \nabla f(x^*) + \sum_{i=1}^p \lambda_i^* \nabla h_i(x^*) = 0
    \]
    即目标函数的梯度可表示为约束函数梯度的\textbf{线性组合},组合系数就是拉格朗日乘子$\lambda_i^*$。这一条件体现了"目标函数与约束的梯度平衡"——在最优解处,目标函数的梯度被约束的梯度"抵消",使得在可行域的切空间内无下降方向。

    \item \textbf{对$\lambda$求梯度并令其为零}:
    \[
    \nabla_\lambda \mathcal{L}(x^*, \lambda^*) = h(x^*) = 0
    \]
    这就是原问题的\textbf{等式约束条件},保证最优解满足约束。
\end{itemize}

\subsection{5. 几何意义(直观理解)}
等式约束$h_i(x)=0$定义了一个\textbf{可行域(流形)},其在最优解$x^*$处的\textbf{切空间}由"与所有$\nabla h_i(x^*)$正交的方向"构成。

目标函数的梯度$\nabla f(x^*)$若要使$x^*$是极值点,必须"无法在切空间内找到下降方向",即$\nabla f(x^*)$必须位于可行域的\textbf{法空间}中(法空间由$\nabla h_1(x^*), \dots, \nabla h_m(x^*)$张成)。因此,$\nabla f(x^*)$可表示为这些法向量的线性组合,这正是$\nabla_x \mathcal{L} = 0$所表达的"梯度线性组合"关系。

综上,拉格朗日函数的建模是通过\textbf{引入乘子变量$\lambda$},将\textbf{等式约束转化为无约束优化的梯度条件},从而把"带约束的优化"转化为"对$(x, \lambda)$的无约束优化(在一阶条件意义下)",实现了从无约束到等式约束优化的数学衔接。

\section{拉格朗日乘子法的必要条件}

\subsection{一阶必要条件及其证明}

\begin{theorem}[一阶必要条件]
设$x^*$是等式约束优化问题的局部极小点,且约束梯度$\nabla g_1(x^*), \dots, \nabla g_m(x^*)$线性无关(即\textbf{约束规格满足})。则存在唯一的拉格朗日乘子$\lambda^* = (\lambda_1^*, \dots, \lambda_m^*)^T$使得:
\[
\begin{cases}
\nabla_x \mathcal{L}(x^*, \lambda^*) = 0 \\
\nabla_\lambda \mathcal{L}(x^*, \lambda^*) = 0
\end{cases}
\]
\end{theorem}

\subsubsection{证明}

\begin{proof}
考虑等式约束优化问题:
\[
\begin{cases} \min_{x \in \mathbb{R}^n} & f(x) \\ \text{s.t.} & h_i(x) = 0,\ i=1,\dots,m \end{cases}
\]
其中$f, h_i$连续可微,$x^*$是局部极小点,且\textbf{约束梯度线性无关}(即约束规格满足):$\nabla h_1(x^*), \dots, \nabla h_m(x^*)$线性无关。

记约束函数的向量形式为$h(x) = (h_1(x), \dots, h_m(x))^\top$,其雅可比矩阵为$J_h(x) \in \mathbb{R}^{m \times n}$,第$i$行为$\nabla h_i(x)^\top$。由约束梯度线性无关,$J_h(x^*)$的\textbf{秩为$m$}(列满秩)。

将变量$x$分块为$x = (y, z)$,其中$y \in \mathbb{R}^m$,$z \in \mathbb{R}^{n-m}$(通过变量重排,可假设$J_h(x^*)$的前$m$列构成的子矩阵$\nabla_y h(x^*)$是\textbf{可逆的$m \times m$矩阵})。

根据\textbf{隐函数定理},存在$x^*$的邻域和可微函数$g: \mathbb{R}^{n-m} \to \mathbb{R}^m$,使得在该邻域内,约束$h(y, z) = 0$可唯一表示为$y = g(z)$,且$g(z^*) = y^*$(即$x^* = (y^*, z^*)$)。

原约束问题可转化为关于$z$的\textbf{无约束优化问题}:
\[
\min_{z \in \mathbb{R}^{n-m}} \ f(g(z), z)
\]

由于$x^*$是原问题的局部极小点,$z^*$是上述无约束问题的局部极小点。根据\textbf{无约束优化的一阶必要条件},对$z$的梯度为0:
\[
\nabla_z f(x^*) + \nabla_y f(x^*) \cdot \nabla_z g(z^*) = 0 \tag{1}
\]

对$h(g(z), z) = 0$关于$z$求导,由链式法则得:
\[
\nabla_y h(x^*) \cdot \nabla_z g(z^*) + \nabla_z h(x^*) = 0
\]

由于$\nabla_y h(x^*)$可逆,解得:
\[
\nabla_z g(z^*) = - \left( \nabla_y h(x^*) \right)^{-1} \nabla_z h(x^*) \tag{2}
\]

将式(2)代入式(1):
\[
\nabla_z f(x^*) - \nabla_y f(x^*) \left( \nabla_y h(x^*) \right)^{-1} \nabla_z h(x^*) = 0
\]

定义\textbf{拉格朗日乘子}$\lambda^* = \left( \nabla_y h(x^*) \right)^{-T} \nabla_y f(x^*)^\top$(转置是因为矩阵逆的转置等于转置的逆)。

拉格朗日函数为$\mathcal{L}(x, \lambda) = f(x) + \lambda^\top h(x)$,其对$x$的梯度为:
\[
\nabla_x \mathcal{L}(x, \lambda) = \nabla f(x) + J_h(x)^\top \lambda
\]

将$\lambda^*$代入,分块验证:
\begin{itemize}
    \item \textbf{$y$分量}:$\nabla_y f(x^*) + \nabla_y h(x^*)^\top \lambda^*$。代入$\lambda^* = \left( \nabla_y h(x^*) \right)^{-T} \nabla_y f(x^*)^\top$,得$\nabla_y f(x^*) + \nabla_y f(x^*) = 0$(转置后等式成立)。
    \item \textbf{$z$分量}:结合式(1)(2)的推导,可验证$\nabla_z f(x^*) + \nabla_z h(x^*)^\top \lambda^* = 0$。
\end{itemize}

因此,$\nabla_x \mathcal{L}(x^*, \lambda^*) = 0$;同时,$\nabla_\lambda \mathcal{L}(x^*, \lambda^*) = h(x^*) = 0$(因$x^*$是可行点)。

\textbf{唯一性:}

假设存在两个乘子$\lambda^*$和$\lambda'^*$,满足:
\[
\nabla f(x^*) + J_h(x^*)^\top \lambda^* = 0, \quad \nabla f(x^*) + J_h(x^*)^\top \lambda'^* = 0
\]

两式相减得$J_h(x^*)^\top (\lambda^* - \lambda'^*) = 0$。由于$J_h(x^*)$列满秩,$J_h(x^*)^\top$的零空间仅含零向量,故$\lambda^* = \lambda'^*$,唯一性得证。
\end{proof}

\subsection{二阶最优性条件及其证明}

\begin{theorem}[二阶必要条件]
设$x^*$是等式约束优化问题的局部极小点,$f$和$g_i$在$x^*$处二阶连续可微,且约束梯度$\nabla g_1(x^*), \dots, \nabla g_m(x^*)$线性无关。则存在$\lambda^*$使得一阶条件成立,且对任意满足$J_g(x^*)d = 0$的$d \in \mathbb{R}^n$,有:
\[
d^T \nabla^2_{xx} \mathcal{L}(x^*, \lambda^*) d \geq 0
\]
其中$\nabla^2_{xx} \mathcal{L}(x^*, \lambda^*)$是拉格朗日函数关于$x$的Hessian矩阵。
\end{theorem}

\begin{theorem}[二阶充分条件]
设$f$和$g_i$在$x^*$处二阶连续可微,存在$\lambda^*$满足一阶条件,且对任意非零向量$d$满足$J_g(x^*)d = 0$,有:
\[
d^T \nabla^2_{xx} \mathcal{L}(x^*, \lambda^*) d > 0
\]
则$x^*$是严格局部极小点。
\end{theorem}

\begin{proof}[二阶必要条件证明]
要证明\textbf{等式约束优化的二阶必要条件},我们通过\textbf{泰勒展开}结合\textbf{一阶必要条件}推导,步骤如下:

考虑等式约束优化问题:
\[
\begin{cases} \min_{x \in \mathbb{R}^n} & f(x) \\ \text{s.t.} & h_i(x) = 0,\ i=1,\dots,m \end{cases}
\]
其中$x^*$是局部极小点,$f, h_i$二阶连续可微,且约束梯度$\nabla h_1(x^*), \dots, \nabla h_m(x^*)$线性无关(约束规格满足)。

根据\textbf{一阶必要条件(拉格朗日条件)},存在拉格朗日乘子$\lambda^* = (\lambda_1^*, \dots, \lambda_m^*)^\top$,使得:
\[
\begin{cases} \nabla_x \mathcal{L}(x^*, \lambda^*) = \nabla f(x^*) + \sum_{i=1}^m \lambda_i^* \nabla h_i(x^*) = 0 \\ h(x^*) = 0 \end{cases}
\]
其中拉格朗日函数$\mathcal{L}(x, \lambda) = f(x) + \lambda^\top h(x)$。

考虑\textbf{切空间中的方向}$d \in \mathbb{R}^n$,即满足约束雅可比矩阵$J_h(x^*)$零空间的方向:
\[
J_h(x^*) d = 0 \implies \nabla h_i(x^*)^\top d = 0,\ \forall i=1,\dots,m
\]
(该方向$d$是"不违反约束的微小移动方向",即当$t$充分小时,$x^* + t d$是可行点)。

对$f(x^* + t d)$做\textbf{二阶泰勒展开}:
\[
f(x^* + t d) = f(x^*) + t \nabla f(x^*)^\top d + \frac{t^2}{2} d^\top \nabla^2 f(x^*) d + o(t^2)
\]

对每个约束$h_i(x^* + t d)$做\textbf{一阶泰勒展开}(因$h_i(x^*) = 0$且$\nabla h_i(x^*)^\top d = 0$,一阶项为0):
\[
h_i(x^* + t d) = 0 + t \nabla h_i(x^*)^\top d + \frac{t^2}{2} d^\top \nabla^2 h_i(x^*) d + o(t^2) = 0 \quad (\text{可行点})
\]

由一阶条件$\nabla f(x^*) = - \sum_{i=1}^m \lambda_i^* \nabla h_i(x^*)$,代入$\nabla f(x^*)^\top d$得:
\[
\nabla f(x^*)^\top d = - \sum_{i=1}^m \lambda_i^* \nabla h_i(x^*)^\top d = 0 \quad (\text{因 } \nabla h_i(x^*)^\top d = 0)
\]

因此,$f(x^* + t d) - f(x^*)$的展开式可简化为:
\[
f(x^* + t d) - f(x^*) = \frac{t^2}{2} d^\top \left( \nabla^2 f(x^*) + \sum_{i=1}^m \lambda_i^* \nabla^2 h_i(x^*) \right) d + o(t^2)
\]

注意到拉格朗日函数关于$x$的\textbf{二阶Hessian矩阵}为:
\[
\nabla^2_{xx} \mathcal{L}(x^*, \lambda^*) = \nabla^2 f(x^*) + \sum_{i=1}^m \lambda_i^* \nabla^2 h_i(x^*)
\]

因此,上式可写为:
\[
f(x^* + t d) - f(x^*) = \frac{t^2}{2} d^\top \nabla^2_{xx} \mathcal{L}(x^*, \lambda^*) d + o(t^2)
\]

由于$x^*$是\textbf{局部极小点},存在$\delta > 0$,使得对所有$t \in (0, \delta)$,若$x^* + t d$可行,则$f(x^* + t d) \geq f(x^*)$。

因此,对充分小的$t > 0$,有:
\[
\frac{t^2}{2} d^\top \nabla^2_{xx} \mathcal{L}(x^*, \lambda^*) d + o(t^2) \geq 0
\]

两边除以$\frac{t^2}{2}$($t > 0$,故$\frac{t^2}{2} > 0$),并令$t \to 0$,得:
\[
d^\top \nabla^2_{xx} \mathcal{L}(x^*, \lambda^*) d \geq 0
\]

对任意满足$J_h(x^*) d = 0$的$d \in \mathbb{R}^n$,有$d^\top \nabla^2_{xx} \mathcal{L}(x^*, \lambda^*) d \geq 0$,即二阶必要条件成立。
\end{proof}

\begin{proof}[二阶充分条件证明]
证明二阶充分条件,通过泰勒展开和严格局部极小的定义推导,步骤如下:

考虑等式约束优化问题:
\[
\begin{cases} \min_{x \in \mathbb{R}^n} & f(x) \\ \text{s.t.} & h_i(x) = 0,\ i=1,\dots,m \end{cases}
\]

已知:
\begin{enumerate}
    \item $f, h_i$在$x^*$处\textbf{二阶连续可微};
    \item 存在拉格朗日乘子$\lambda^*$,满足\textbf{一阶拉格朗日条件}:$\nabla_x \mathcal{L}(x^*, \lambda^*) = 0$且$h(x^*) = 0$(其中$\mathcal{L}(x, \lambda) = f(x) + \lambda^\top h(x)$是拉格朗日函数);
    \item 对任意\textbf{非零}向量$d \in \mathbb{R}^n$满足$J_h(x^*) d = 0$(即$\nabla h_i(x^*)^\top d = 0,\ \forall i=1,\dots,m$),有$d^\top \nabla^2_{xx} \mathcal{L}(x^*, \lambda^*) d > 0$(拉格朗日函数关于$x$的Hessian矩阵在切空间方向上正定)。
\end{enumerate}

\paragraph{证明目标}
需证明$x^*$是\textbf{严格局部极小点},即存在邻域$\mathcal{N}(x^*)$,使得对所有可行点$x \in \mathcal{N}(x^*)$且$x \neq x^*$,有$f(x) > f(x^*)$。

\paragraph{证明步骤}

\textbf{步骤1:可行点的局部参数化(切空间方向)}

设$d \in \mathbb{R}^n$满足$J_h(x^*) d = 0$(称此类$d$为\textbf{切空间方向},沿该方向移动不违反约束)。对充分小的$t$,定义$x(t) = x^* + t d$。

对约束$h_i(x(t))$做\textbf{二阶泰勒展开}:
\[
h_i(x(t)) = h_i(x^*) + t \nabla h_i(x^*)^\top d + \frac{t^2}{2} d^\top \nabla^2 h_i(x^*) d + o(t^2)
\]

由$h_i(x^*) = 0$且$\nabla h_i(x^*)^\top d = 0$,得:
\[
h_i(x(t)) = \frac{t^2}{2} d^\top \nabla^2 h_i(x^*) d + o(t^2)
\]

当$t$充分小时,$h_i(x(t)) = 0$(高阶小量可忽略),故$x(t)$是\textbf{可行点}。

\textbf{步骤2:拉格朗日函数的二阶泰勒展开}

拉格朗日函数$\mathcal{L}(x, \lambda^*) = f(x) + (\lambda^*)^\top h(x)$,对$x(t)$做\textbf{二阶泰勒展开}:
\[
\mathcal{L}(x(t), \lambda^*) = \mathcal{L}(x^*, \lambda^*) + t \nabla_x \mathcal{L}(x^*, \lambda^*)^\top d + \frac{t^2}{2} d^\top \nabla^2_{xx} \mathcal{L}(x^*, \lambda^*) d + o(t^2)
\]

由\textbf{一阶拉格朗日条件}$\nabla_x \mathcal{L}(x^*, \lambda^*) = 0$,上式简化为:
\[
\mathcal{L}(x(t), \lambda^*) = \mathcal{L}(x^*, \lambda^*) + \frac{t^2}{2} d^\top \nabla^2_{xx} \mathcal{L}(x^*, \lambda^*) d + o(t^2) \tag{1}
\]

\textbf{步骤3:结合可行点的目标函数}

因$x(t)$是可行点($h(x(t)) = 0$),故$\mathcal{L}(x(t), \lambda^*) = f(x(t))$。同时,$\mathcal{L}(x^*, \lambda^*) = f(x^*)$(因$h(x^*) = 0$)。

将式(1)改写为:
\[
f(x(t)) = f(x^*) + \frac{t^2}{2} d^\top \nabla^2_{xx} \mathcal{L}(x^*, \lambda^*) d + o(t^2)
\]

\textbf{步骤4:利用二阶正定性推导严格不等式}

由已知条件,对\textbf{非零}$d$,有$d^\top \nabla^2_{xx} \mathcal{L}(x^*, \lambda^*) d > 0$。因此,存在$t_0 > 0$,当$0 < |t| < t_0$时:
\[
\frac{t^2}{2} d^\top \nabla^2_{xx} \mathcal{L}(x^*, \lambda^*) d + o(t^2) > 0
\]
即$f(x(t)) > f(x^*)$。

\textbf{步骤5:验证严格局部极小}

由隐函数定理,可行域在$x^*$附近的局部结构可由所有切空间方向$d$生成的$x(t)$覆盖。因此,存在$x^*$的邻域$\mathcal{N}(x^*)$,使得对所有可行点$x \in \mathcal{N}(x^*)$且$x \neq x^*$,必有$f(x) > f(x^*)$,即$x^*$是\textbf{严格局部极小点}。
\end{proof}

\chapter{从等式约束到不等式约束:KKT}

\section{为什么需要KKT}

等式约束优化(拉格朗日乘子法)仅能处理光滑边界、无需区分约束是否"起作用"的场景;而不等式约束的可行集有"内部/边界(活跃约束)",可行方向呈锥形,等式方法无法区分活跃约束、也没法处理约束梯度需非负权重的需求。KKT正是补上这一缺口,将"目标在可行方向不下降"的几何要求,转化为含"活跃约束区分、非负乘子、互补松弛"的代数条件,从而能解不等式约束的优化问题。

\section{不等式约束建模}

\subsection{等式约束优化问题(基础模型)}
设优化目标为最小化\textbf{目标函数}$f(\boldsymbol{x})$,仅受\textbf{等式约束}限制,数学建模如下:
\[
\begin{cases}
\min_{\boldsymbol{x} \in \mathbb{R}^n} \, f(\boldsymbol{x}) \\
\text{subject to} \quad g_i(\boldsymbol{x}) = 0 \quad (i = 1, 2, \dots, m)
\end{cases}
\]
其中:

\begin{itemize}
    \item $\boldsymbol{x} = (x_1, x_2, \dots, x_n)^T$是$n$维决策变量($\mathbb{R}^n$表示$n$维实数空间);
    \item $f: \mathbb{R}^n \to \mathbb{R}$是连续可微的目标函数(映射到实数域);
    \item $g_i: \mathbb{R}^n \to \mathbb{R}$($i=1,\dots,m$)是连续可微的等式约束函数,$m < n$(约束数量少于变量维度,保证可行集非空)。
\end{itemize}

\subsection{含不等式约束的优化问题(扩展模型)}
在等式约束基础上引入\textbf{不等式约束},形成更通用的优化模型:
\[
\begin{cases}
\min_{\boldsymbol{x} \in \mathbb{R}^n} \, f(\boldsymbol{x}) \\
\text{subject to} \quad 
\begin{aligned}
g_i(\boldsymbol{x}) &= 0 \quad (i = 1, 2, \dots, m) \\
h_j(\boldsymbol{x}) &\leq 0 \quad (j = 1, 2, \dots, p)
\end{aligned}
\end{cases}
\]
其中:

\begin{itemize}
    \item 新增$p$个连续可微的\textbf{不等式约束函数}$h_j: \mathbb{R}^n \to \mathbb{R}$($j=1,\dots,p$),约束满足$h_j(\boldsymbol{x}) \leq 0$;
    \item 可行集定义为$\mathcal{X} = \{\boldsymbol{x} \in \mathbb{R}^n \mid g_i(\boldsymbol{x})=0, h_j(\boldsymbol{x}) \leq 0, \forall i,j\}$,最优解需在$\mathcal{X}$内使$f(\boldsymbol{x})$最小;
    \item 需通过\textbf{积极约束集}$\mathcal{A}(\boldsymbol{x}^*) = \{ j \in \{1,\dots,p\} \mid h_j(\boldsymbol{x}^*) = 0 \}$区分"起作用"(边界)与"不起作用"(内部)的不等式约束,为后续KKT条件奠基。
\end{itemize}

\section{KKT条件的推导}

\subsection{一阶必要条件(KKT条件)的严格数学建模}

\subsubsection{前提假设}
设约束优化问题为:
\[
\mathcal{P}: \begin{cases}
\min_{\boldsymbol{x} \in \mathbb{R}^n} \, f(\boldsymbol{x}) \\
\text{s.t.} \quad g_i(\boldsymbol{x}) = 0 \quad (i=1,2,\dots,m) \\
\quad \quad h_j(\boldsymbol{x}) \leq 0 \quad (j=1,2,\dots,p)
\end{cases}
\]
其中:

\begin{itemize}
    \item $f, g_i, h_j \in C^1(\mathbb{R}^n)$(均为一阶连续可微函数);
    \item $\boldsymbol{x}^* \in \mathcal{X}$($\mathcal{X}$为可行集,即$\mathcal{X} = \{\boldsymbol{x} \mid g_i(\boldsymbol{x})=0, h_j(\boldsymbol{x})\leq0\}$),且$\boldsymbol{x}^*$是$\mathcal{P}$的\textbf{局部极小点};
    \item 满足\textbf{约束规格(Constraint Qualification, CQ)}(后续定义)。
\end{itemize}

\subsubsection{数学结论}
存在拉格朗日乘子$\lambda^* = (\lambda_1^*, \lambda_2^*, \dots, \lambda_m^*)^T \in \mathbb{R}^m$和$\mu^* = (\mu_1^*, \mu_2^*, \dots, \mu_p^*)^T \in \mathbb{R}^p$,使得以下4个条件同时成立:

\begin{enumerate}
    \item \textbf{平稳性条件}(目标梯度与约束梯度平衡):
    \[
    \nabla f(\boldsymbol{x}^*) + \sum_{i=1}^m \lambda_i^* \nabla g_i(\boldsymbol{x}^*) + \sum_{j=1}^p \mu_j^* \nabla h_j(\boldsymbol{x}^*) = \boldsymbol{0} \in \mathbb{R}^n
    \]
    
    \item \textbf{原始可行性条件}(解满足所有约束):
    \[
    h_j(\boldsymbol{x}^*) \leq 0 \quad \forall j \in \{1,2,\dots,p\}
    \]
    (注:等式约束$g_i(\boldsymbol{x}^*) = 0$已隐含在$\boldsymbol{x}^* \in \mathcal{X}$中,此处补充不等式约束的显式条件)
    
    \item \textbf{对偶可行性条件}(不等式约束乘子非负):
    \[
    \mu_j^* \geq 0 \quad \forall j \in \{1,2,\dots,p\}
    \]
    
    \item \textbf{互补松弛条件}(非活跃约束乘子为0):
    \[
    \mu_j^* h_j(\boldsymbol{x}^*) = 0 \quad \forall j \in \{1,2,\dots,p\}
    \]
\end{enumerate}

\subsection{约束规格(CQ)的严格数学建模}
约束规格是保证"局部极小点满足KKT条件"的关键假设,以下为两类核心CQ的严格定义:

\subsubsection{线性无关约束规格(Linear Independence Constraint Qualification, LICQ)}
设$\boldsymbol{x}^* \in \mathcal{X}$,记\textbf{积极约束集}$\mathcal{A}(\boldsymbol{x}^*) = \{ j \in \{1,\dots,p\} \mid h_j(\boldsymbol{x}^*) = 0 \}$(即不等式约束中"起作用"的集合)。

若向量集合:
\[
\mathcal{G}(\boldsymbol{x}^*) = \left\{ \nabla g_i(\boldsymbol{x}^*) \mid i=1,\dots,m \right\} \cup \left\{ \nabla h_j(\boldsymbol{x}^*) \mid j \in \mathcal{A}(\boldsymbol{x}^*) \right\}
\]
满足\textbf{线性无关}(即不存在不全为零的常数$\alpha_1,\dots,\alpha_m, \beta_j (j\in\mathcal{A}(\boldsymbol{x}^*))$,使得$\sum_{i=1}^m \alpha_i \nabla g_i(\boldsymbol{x}^*) + \sum_{j\in\mathcal{A}(\boldsymbol{x}^*)} \beta_j \nabla h_j(\boldsymbol{x}^*) = \boldsymbol{0}$),则称在$\boldsymbol{x}^*$处满足LICQ。

\subsubsection{Mangasarian-Fromovitz约束规格(Mangasarian-Fromovitz Constraint Qualification, MFCQ)}
设$\boldsymbol{x}^* \in \mathcal{X}$,若满足以下两个条件:

\begin{enumerate}
    \item 等式约束梯度集$\{ \nabla g_i(\boldsymbol{x}^*) \mid i=1,\dots,m \}$线性无关;
    \item 存在可行方向$\boldsymbol{d} \in \mathbb{R}^n \setminus \{ \boldsymbol{0} \}$(非零方向),使得:
    \[
    \nabla g_i(\boldsymbol{x}^*)^T \boldsymbol{d} = 0 \quad \forall i \in \{1,\dots,m\}
    \]
    \[
    \nabla h_j(\boldsymbol{x}^*)^T \boldsymbol{d} < 0 \quad \forall j \in \mathcal{A}(\boldsymbol{x}^*)
    \]
\end{enumerate}

则称在$\boldsymbol{x}^*$处满足MFCQ。

\begin{remark}
MFCQ是弱于LICQ的约束规格,即LICQ成立可推出MFCQ成立,但反之不成立。
\end{remark}

\subsection{LICQ下KKT条件的严格证明建模}

\subsubsection{第一步:定义线性化可行方向锥}
设$\boldsymbol{x}^*$是$\mathcal{P}$的局部极小点,且在$\boldsymbol{x}^*$处满足LICQ。定义\textbf{线性化可行方向锥}:
\[
\mathcal{F}(\boldsymbol{x}^*) = \left\{ \boldsymbol{d} \in \mathbb{R}^n \mid \nabla g_i(\boldsymbol{x}^*)^T \boldsymbol{d} = 0 \ (\forall i=1,\dots,m), \ \nabla h_j(\boldsymbol{x}^*)^T \boldsymbol{d} \leq 0 \ (\forall j \in \mathcal{A}(\boldsymbol{x}^*)) \right\}
\]
几何意义:$\mathcal{F}(\boldsymbol{x}^*)$是在$\boldsymbol{x}^*$处"沿该方向移动,线性近似下仍可行"的所有方向集合。

\subsubsection{第二步:证明核心引理}

\begin{lemma}
对所有$\boldsymbol{d} \in \mathcal{F}(\boldsymbol{x}^*)$,有$\nabla f(\boldsymbol{x}^*)^T \boldsymbol{d} \geq 0$。
\end{lemma}

\begin{proof}[证明(反证法)]
假设存在$\boldsymbol{d}_0 \in \mathcal{F}(\boldsymbol{x}^*)$使得$\nabla f(\boldsymbol{x}^*)^T \boldsymbol{d}_0 < 0$。

由LICQ成立,根据隐函数定理,可构造\textbf{可行曲线}:
\[
\boldsymbol{x}(t) = \boldsymbol{x}^* + t \boldsymbol{d}_0 + o(t) \quad (t \geq 0)
\]
其中$o(t)$是高阶无穷小(满足$\lim_{t \to 0^+} \frac{\|o(t)\|}{t} = 0$),且曲线满足:

\begin{enumerate}
    \item 对等式约束:$g_i(\boldsymbol{x}(t)) = 0 + t \nabla g_i(\boldsymbol{x}^*)^T \boldsymbol{d}_0 + o(t) = o(t)$(因$\boldsymbol{d}_0 \in \mathcal{F}(\boldsymbol{x}^*)$,故$\nabla g_i(\boldsymbol{x}^*)^T \boldsymbol{d}_0 = 0$);
    \item 对积极不等式约束($j \in \mathcal{A}(\boldsymbol{x}^*)$):$h_j(\boldsymbol{x}(t)) = 0 + t \nabla h_j(\boldsymbol{x}^*)^T \boldsymbol{d}_0 + o(t) \leq 0 + o(t)$(因$\boldsymbol{d}_0 \in \mathcal{F}(\boldsymbol{x}^*)$,故$\nabla h_j(\boldsymbol{x}^*)^T \boldsymbol{d}_0 \leq 0$)。
\end{enumerate}

取充分小的$t_0 > 0$,当$t \in (0, t_0)$时:
\begin{itemize}
    \item $|o(t)| < t \cdot \min\left\{ 1, \left| \nabla h_j(\boldsymbol{x}^*)^T \boldsymbol{d}_0 \right| \ (\forall j \in \mathcal{A}(\boldsymbol{x}^*)) \right\}$,故$g_i(\boldsymbol{x}(t)) \approx 0$、$h_j(\boldsymbol{x}(t)) \leq 0$,即$\boldsymbol{x}(t) \in \mathcal{X}$(可行);
    \item 目标函数泰勒展开:$f(\boldsymbol{x}(t)) = f(\boldsymbol{x}^*) + t \nabla f(\boldsymbol{x}^*)^T \boldsymbol{d}_0 + o(t) < f(\boldsymbol{x}^*)$(因$\nabla f(\boldsymbol{x}^*)^T \boldsymbol{d}_0 < 0$,且$t$充分小)。
\end{itemize}

这与$\boldsymbol{x}^*$是局部极小点矛盾,故假设不成立,引理得证。
\end{proof}

\subsubsection{第三步:应用Farkas引理推导乘子存在性}

\begin{lemma}[Farkas引理]
设$\boldsymbol{A} \in \mathbb{R}^{k \times n}$,$\boldsymbol{b} \in \mathbb{R}^n$,则以下两个系统\textbf{有且仅有一个}有解:
\begin{itemize}
    \item 系统1:$\boldsymbol{A} \boldsymbol{d} \leq \boldsymbol{0}$,$\boldsymbol{b}^T \boldsymbol{d} > 0$($\boldsymbol{d} \in \mathbb{R}^n$);
    \item 系统2:$\boldsymbol{A}^T \boldsymbol{y} = \boldsymbol{b}$,$\boldsymbol{y} \geq \boldsymbol{0}$($\boldsymbol{y} \in \mathbb{R}^k$)。
\end{itemize}
\end{lemma}

将第二步引理转化为Farkas引理的"系统1无解"场景:

构造矩阵$\boldsymbol{A}$和向量$\boldsymbol{b}$如下:
\begin{itemize}
    \item $\boldsymbol{A}$的行由$\nabla g_i(\boldsymbol{x}^*)^T$、$-\nabla g_i(\boldsymbol{x}^*)^T$(对应$\nabla g_i(\boldsymbol{x}^*)^T \boldsymbol{d} = 0$拆分为$\leq 0$和$\geq 0$)、$\nabla h_j(\boldsymbol{x}^*)^T$($j \in \mathcal{A}(\boldsymbol{x}^*)$)组成;
    \item $\boldsymbol{b} = -\nabla f(\boldsymbol{x}^*)$。
\end{itemize}

由第二步引理,"系统1:$\boldsymbol{A} \boldsymbol{d} \leq \boldsymbol{0}$,$\boldsymbol{b}^T \boldsymbol{d} > 0$"无解,故由Farkas引理,"系统2:$\boldsymbol{A}^T \boldsymbol{y} = \boldsymbol{b}$,$\boldsymbol{y} \geq \boldsymbol{0}$"有解。

整理系统2的解,可得:
\begin{itemize}
    \item 存在$\lambda_i^* \in \mathbb{R}$(对应等式约束的乘子,由$\nabla g_i(\boldsymbol{x}^*)^T$和$-\nabla g_i(\boldsymbol{x}^*)^T$的系数合成);
    \item 存在$\mu_j^* \geq 0$(对应积极不等式约束的乘子,由$\nabla h_j(\boldsymbol{x}^*)^T$的系数给出);
    \item 对非积极约束($j \notin \mathcal{A}(\boldsymbol{x}^*)$),令$\mu_j^* = 0$,则平稳性条件成立。
\end{itemize}

同时,互补松弛条件$\mu_j^* h_j(\boldsymbol{x}^*) = 0$自然满足(非积极约束$\mu_j^* = 0$,积极约束$h_j(\boldsymbol{x}^*) = 0$),对偶可行性条件$\mu_j^* \geq 0$由Farkas引理的$\boldsymbol{y} \geq \boldsymbol{0}$保证。

综上,KKT条件在LICQ下得证。

\subsection{Farkas引理证明}

\begin{theorem}[Farkas引理]
设$A \in \mathbb{R}^{m \times n}$,$b \in \mathbb{R}^{n}$,则以下两个系统\textbf{有且仅有一个}有解:
\begin{itemize}
    \item 系统1:$A x \leq 0$,$b^T x > 0$($x \in \mathbb{R}^{n}$);
    \item 系统2:$A^T y = b$,$y \geq 0$($y \in \mathbb{R}^{m}$)。
\end{itemize}
\end{theorem}

\begin{proof}[证明(分两步核心逻辑)]

\paragraph{第一步:证明"两系统不能同时有解"(矛盾法)}

假设系统1、系统2同时存在解,即存在$x \in \mathbb{R}^n$满足$A x \leq 0$且$b^T x > 0$,同时存在$y \in \mathbb{R}^m$满足$A^T y = b$且$y \geq 0$。

对$b^T x$做代数变形:

由$A^T y = b$,两边转置得$b^T = y^T A$,因此$b^T x = y^T (A x)$。

结合已知条件分析:
\begin{itemize}
    \item 因$A x \leq 0$(系统1)且$y \geq 0$(系统2),向量内积$y^T (A x) \leq 0$,即$b^T x \leq 0$;
    \item 但系统1要求$b^T x > 0$,二者矛盾。故两系统不能同时有解。
\end{itemize}

\paragraph{第二步:证明"若系统1无解,则系统2必有解"(凸集分离定理+矛盾法)}

\begin{enumerate}
    \item \textbf{定义闭凸锥}:设集合$C = \{ A^T y \mid y \geq 0 \}$,易证$C$是$\mathbb{R}^n$中的\textbf{闭凸锥}(闭性由线性映射连续性+非负锥闭性保证,凸性由线性映射凸性+非负锥凸性保证)。
    
    \item \textbf{反证假设与凸集分离}:
    
    假设系统1无解,且系统2也无解(即$b \notin C$,若$b \in C$则存在$y \geq 0$使$A^T y = b$,系统2有解)。
    
    因$C$是闭凸集且$b \notin C$,由\textbf{凸集分离定理},存在非零向量$x \in \mathbb{R}^n$和实数$\alpha \in \mathbb{R}$,使得:
    \[
    b^T x > \alpha \quad \text{且} \quad (A^T y)^T x \leq \alpha \quad \forall y \geq 0
    \]
    
    \item \textbf{推导系统1有解(矛盾)}:
    \begin{itemize}
        \item 令$y = 0$(因$0 \in \{ y \mid y \geq 0 \}$,故$A^T 0 = 0 \in C$),代入右边不等式得$0^T x \leq \alpha$,即$\alpha \geq 0$,因此$b^T x > \alpha \geq 0$,即$b^T x > 0$。
        \item 若存在某个分量$(A x)_k > 0$,取$y = t e_k$($e_k$为第$k$个单位向量,$t > 0$),则$(A^T y)^T x = t (A x)_k$。当$t \to +\infty$时,$t (A x)_k \to +\infty$,与$(A^T y)^T x \leq \alpha$($\alpha$是固定实数)矛盾。故必须$A x \leq 0$。
    \end{itemize}
    
    此时$x$满足$A x \leq 0$且$b^T x > 0$,即系统1有解——与"系统1无解"的初始假设矛盾。故系统2必有解。
\end{enumerate}
\end{proof}

\section{二阶最优性条件}

在学习优化问题时,\textbf{一阶KKT条件}告诉我们"最优解处的梯度要平衡约束"(相当于"坡度为零"),但这只够判断"可能是极值点"——就像走到平地上,分不清是山顶、山谷还是半山腰的平台。而\textbf{二阶最优性条件}是在一阶条件基础上,通过判断"曲率"(相当于地面的弯曲方向),明确这个"平地"到底是不是真正的极小点。

要判断"曲率",首先得明确:在KKT点$x^*$处,还有可能让目标函数下降的方向有哪些?这个方向集合就是"临界锥"。

\subsection{临界锥}

\begin{definition}[临界锥]
在KKT点$x^{*}$处,临界锥定义为:
\[
\mathcal{C}\left(x^{*}, \mu^{*}\right)=\left\{d \in \mathbb{R}^{n}: \begin{aligned}
& \nabla g_{i} \left(x^{*}\right)^{T} d=0, \quad i=1, \dots, m \\
& \nabla h_{j}\left(x^{*}\right)^{T} d=0, \quad j \in \mathcal{A}\left(x^{*}\right) \text{ 且 } \mu_{j}^{*}>0 \\
& \nabla h_{j}\left(x^{*}\right)^{T} d \leq 0, \quad j \in \mathcal{A}\left(x^{*}\right) \text{ 且 } \mu_{j}^{*}=0
\end{aligned}\right\}
\]
\end{definition}

临界锥包含了在$x^{*}$处所有可能的"候选下降方向",即"在不违反任何约束的前提下,可能让目标函数下降的所有方向"。

\subsection{二阶必要条件}

\begin{theorem}[二阶必要条件]
设$x^*$是优化问题的\textbf{局部极小点},且满足线性无关约束规格(LICQ),$(\lambda^*,\mu^*)$是对应的KKT乘子,则对所有方向$d \in \mathcal{C}(x^*,\mu^*)$,有:
\[
d^T \nabla_{xx}^2 \mathcal{L}(x^*,\lambda^*,\mu^*) d \geq 0
\]
\end{theorem}

其中:
\begin{itemize}
    \item \textbf{拉格朗日函数}:$\mathcal{L}(x,\lambda,\mu) = f(x) + \sum_{i=1}^m \lambda_i g_i(x) + \sum_{j=1}^p \mu_j h_j(x)$,其中$\lambda \in \mathbb{R}^m$、$\mu \in \mathbb{R}^p$为拉格朗日乘子;
    \item \textbf{拉格朗日Hessian矩阵}:$\nabla_{xx}^2 \mathcal{L}(x^*,\lambda^*,\mu^*)$是拉格朗日函数在$(x^*,\lambda^*,\mu^*)$处关于$x$的二阶偏导数矩阵(Hessian矩阵),严格定义为:
    \[
    \nabla_{xx}^2 \mathcal{L}(x^*,\lambda^*,\mu^*) = \nabla^2 f(x^*) + \sum_{i=1}^m \lambda_i^* \nabla^2 g_i(x^*) + \sum_{j=1}^p \mu_j^* \nabla^2 h_j(x^*)
    \]
    其中$\nabla^2 f(x^*)$、$\nabla^2 g_i(x^*)$、$\nabla^2 h_j(x^*)$分别表示$f$、$g_i$、$h_j$在$x^*$处的Hessian矩阵($n \times n$对称矩阵)。
\end{itemize}

\begin{remark}
"必要条件"的意思是:\textbf{如果$x^*$是局部极小点,那么在所有"可能下降的方向"(临界锥内),综合曲率必须≥0}。
\end{remark}

\subsection{二阶充分条件}

\begin{theorem}[二阶充分条件]
设$x^*$是优化问题的\textbf{可行点}(满足$g_i(x^*)=0$、$h_j(x^*)\leq0$),存在乘子$(\lambda^*,\mu^*)$使得$(x^*,\lambda^*,\mu^*)$满足KKT条件,且对所有\textbf{非零方向}$d \in \mathcal{C}(x^*,\mu^*)$,有:
\[
d^T \nabla_{xx}^2 \mathcal{L}(x^*,\lambda^*,\mu^*) d > 0
\]
则$x^*$是优化问题的\textbf{严格局部极小点}。
\end{theorem}

\begin{remark}
"充分条件"的意思是:\textbf{只要在所有"可能下降的方向"(临界锥内的非零方向),综合曲率都>0,那么$x^*$一定是局部极小点}。
\end{remark}

\chapter{从等式约束到不等式约束:对偶}

\section{从KKT条件到对偶问题的严格数学建模}

\subsection{原始问题(Primal Problem)建模}

\subsubsection{定义(原始约束优化问题)}
给定变量空间 $\mathbb{R}^n$,目标函数与约束函数满足:
\begin{itemize}
    \item 目标函数:$f: \mathbb{R}^n \to \mathbb{R}$(从n维欧氏空间到实数域的映射)
    \item 等式约束函数:$g_i: \mathbb{R}^n \to \mathbb{R}$($i=1,2,\dots,m$),共$m$个等式约束
    \item 不等式约束函数:$h_j: \mathbb{R}^n \to \mathbb{R}$($j=1,2,\dots,p$),共$p$个不等式约束
\end{itemize}

原始问题的数学表达式为:
\[
\begin{cases}
\min\limits_{x \in \mathbb{R}^n} & f(x) \\
\text{subject to} & g_i(x) = 0, \quad i=1,2,\dots,m \\
& h_j(x) \leq 0, \quad j=1,2,\dots,p
\end{cases}
\]
其中,$x = (x_1,x_2,\dots,x_n)^T$ 为原始问题的优化变量,原始问题的最优值记为 $p^* = \inf\{f(x) \mid x \text{ 满足所有约束}\}$。


\subsection{拉格朗日函数建模}

\subsubsection{定义(拉格朗日函数)}
引入拉格朗日乘子:
\begin{itemize}
    \item 等式约束乘子:$\lambda = (\lambda_1,\lambda_2,\dots,\lambda_m)^T \in \mathbb{R}^m$(无符号限制)
    \item 不等式约束乘子:$\mu = (\mu_1,\mu_2,\dots,\mu_p)^T \in \mathbb{R}^p$(后续将限定非负)
\end{itemize}

拉格朗日函数定义为:
\[
\mathcal{L}(x, \lambda, \mu) = f(x) + \sum_{i=1}^m \lambda_i g_i(x) + \sum_{j=1}^p \mu_j h_j(x)
\]
其中,$\mathcal{L}: \mathbb{R}^n \times \mathbb{R}^m \times \mathbb{R}^p \to \mathbb{R}$(从$n+m+p$维乘积空间到实数域的映射)。


\subsection{拉格朗日对偶函数建模}

\begin{definition}[拉格朗日对偶函数]
拉格朗日对偶函数是拉格朗日函数关于原始变量$x$的\textbf{逐点下确界},定义为:
\[
d(\lambda, \mu) = \inf_{x \in \mathbb{R}^n} \mathcal{L}(x, \lambda, \mu)
\]
其中,$d: \mathbb{R}^m \times \mathbb{R}^p \to \mathbb{R} \cup \{-\infty\}$(对偶函数的值域可包含负无穷,当下确界不存在时取$-\infty$)。
\end{definition}


\subsection{对偶函数的核心性质建模}

\begin{theorem}[对偶函数的凹性]
对偶函数 $d(\lambda, \mu)$ 是关于 $(\lambda, \mu)$ 的凹函数。
\end{theorem}

\begin{proof}
\begin{enumerate}
    \item 对任意固定的 $x \in \mathbb{R}^n$,构造函数 $\phi_{x}(\lambda, \mu) = \mathcal{L}(x, \lambda, \mu)$。由拉格朗日函数的定义可知:
    \[
    \phi_{x}(\lambda, \mu) = f(x) + \sum_{i=1}^m \lambda_i g_i(x) + \sum_{j=1}^p \mu_j h_j(x)
    \]
    其中 $f(x), g_i(x), h_j(x)$ 均与 $(\lambda, \mu)$ 无关,因此 $\phi_{x}(\lambda, \mu)$ 是关于 $(\lambda, \mu)$ 的\textbf{仿射函数}(线性函数加常数项)。
    
    \item 对偶函数 $d(\lambda, \mu) = \inf_{x \in \mathbb{R}^n} \phi_{x}(\lambda, \mu)$,即对偶函数是一族仿射函数 $\{\phi_{x}(\lambda, \mu) \mid x \in \mathbb{R}^n\}$ 的逐点下确界。
    
    \item 由凸分析基本性质:\textbf{一族仿射函数的逐点下确界是凹函数},因此 $d(\lambda, \mu)$ 是凹函数。
\end{enumerate}
\end{proof}

\begin{theorem}[对偶函数的下界性质]
对任意 $\lambda \in \mathbb{R}^m$ 和 $\mu \geq 0$(即 $\mu_j \geq 0$ 对所有 $j=1,2,\dots,p$ 成立),有:
\[
d(\lambda, \mu) \leq p^*
\]
其中 $p^*$ 是原始问题的最优值。
\end{theorem}

\begin{proof}
\begin{enumerate}
    \item 设 $x$ 是原始问题的任意可行点,即满足:
    \[
    g_i(x) = 0 \quad (i=1,\dots,m), \quad h_j(x) \leq 0 \quad (j=1,\dots,p)
    \]
    
    \item 由于 $\mu \geq 0$ 且 $h_j(x) \leq 0$,可得:
    \[
    \sum_{j=1}^p \mu_j h_j(x) \leq 0
    \]
    又因为 $g_i(x) = 0$,故:
    \[
    \sum_{i=1}^m \lambda_i g_i(x) + \sum_{j=1}^p \mu_j h_j(x) = 0 + \sum_{j=1}^p \mu_j h_j(x) \leq 0
    \]
    
    \item 代入拉格朗日函数得:
    \[
    \mathcal{L}(x, \lambda, \mu) = f(x) + \sum_{i=1}^m \lambda_i g_i(x) + \sum_{j=1}^p \mu_j h_j(x) \leq f(x)
    \]
    
    \item 由对偶函数的定义(下确界性质),对任意 $x \in \mathbb{R}^n$ 有:
    \[
    d(\lambda, \mu) = \inf_{y \in \mathbb{R}^n} \mathcal{L}(y, \lambda, \mu) \leq \mathcal{L}(x, \lambda, \mu)
    \]
    
    \item 结合步骤3和步骤4,得:
    \[
    d(\lambda, \mu) \leq \mathcal{L}(x, \lambda, \mu) \leq f(x)
    \]
    
    \item 由于该不等式对所有原始可行点 $x$ 成立,而 \textbf{$p^*$ 是所有可行点对应的 $f(x)$ 的下确界},因此:
    \[
    d(\lambda, \mu) \leq p^*
    \]
\end{enumerate}
\end{proof}


\subsection{拉格朗日对偶问题建模}

\subsubsection{定义(对偶问题)}
基于对偶函数的下界性质,对偶问题的目标是\textbf{最大化对偶函数的下界},同时满足乘子约束 $\mu \geq 0$。其数学表达式为:
\[
\begin{cases}
\max\limits_{\lambda \in \mathbb{R}^m, \mu \in \mathbb{R}^p} & d(\lambda, \mu) \\
\text{subject to} & \mu_j \geq 0, \quad j=1,2,\dots,p
\end{cases}
\]

\subsubsection{关键定义补充}
\begin{enumerate}
    \item 对偶问题的最优值:$d^* = \sup\{d(\lambda, \mu) \mid \mu \geq 0\}$(上确界,因对偶函数是凹函数,最大值若存在则上确界等于最大值)。
    \item 对偶间隙:原始问题最优值与对偶问题最优值的差值,即 $\Delta = p^* - d^*$。
    \item 弱对偶性:由定理4.2直接可得 $d^* \leq p^*$,即对偶间隙非负($\Delta \geq 0$),该性质对所有原始-对偶问题对恒成立。
\end{enumerate}


\section{强对偶}

当对偶间隙消失(即原始问题与对偶问题的最优值相等)时,称\textbf{强对偶成立},这是对偶理论中"原始-对偶等价"的关键条件。

\begin{definition}[强对偶]
若对偶间隙$\Delta = 0$,即:
\[
p^* = d^*
\]
则称原始问题与对偶问题满足强对偶性。
\end{definition}

需特别说明:强对偶并非对所有约束优化问题恒成立,仅在满足特定条件(如Slater条件)时可被保证。


\subsection{Slater条件}

\begin{theorem}[Slater条件]
若原始问题是凸问题,且存在\textbf{严格可行点}$x_{\text{strict}} \in \mathbb{R}^n$,满足:
\[
\begin{cases}
g_i(x_{\text{strict}}) = 0, & i=1,\dots,m \quad (\text{等式约束仍严格满足}) \\
h_j(x_{\text{strict}}) < 0, & j=1,\dots,p \quad (\text{不等式约束严格满足,无"紧约束"})
\end{cases}
\]
则原始问题与对偶问题满足强对偶性($p^* = d^*$)。
\end{theorem}

\begin{remark}
Slater条件是强对偶成立的\textbf{充分条件,而非必要条件}——即满足Slater条件一定有强对偶,但强对偶成立时未必满足Slater条件。
\end{remark}


\subsection{几何解释}

构造三维乘积空间$\mathbb{R}^m \times \mathbb{R}^p \times \mathbb{R}$中的集合$\mathcal{G}$,其元素为满足"约束与目标函数不等式"的三元组$(u, v, t)$,数学定义为:
\[
\mathcal{G} = \left\{ (u, v, t) \in \mathbb{R}^m \times \mathbb{R}^p \times \mathbb{R} \mid \exists x \in \mathbb{R}^n, \ g_i(x)=u_i\ (i=1,\dots,m),\ h_j(x)\leq v_j\ (j=1,\dots,p),\ f(x)\leq t \right\}
\]
其中:
\begin{itemize}
    \item $u = (u_1,\dots,u_m)^T$(等式约束$g_i(x)$的取值);
    \item $v = (v_1,\dots,v_p)^T$(不等式约束$h_j(x)$的上界);
    \item $t$(目标函数$f(x)$的上界)。
\end{itemize}

原始问题的最优值$p^*$是"使$(0,0,t) \in \mathcal{G}$的最小$t$"——即当等式约束取$u=0$、不等式约束取$v=0$(满足原始约束)时,目标函数上界$t$的下确界,数学表达式为:
\[
p^* = \inf \left\{ t \mid (0, 0, t) \in \mathcal{G} \right\}
\]
几何意义:$\mathcal{G}$中所有"第一分量为0、第二分量为0"的点,其第三分量的最小值即为$p^*$。

对偶问题的最优值$d^*$是"对$\mu \geq 0$,在$\mathcal{G}$上最小化$t+\lambda^T u + \mu^T v$的最大值",数学表达式为:
\[
d^* = \sup_{\mu \geq 0} \inf_{(u, v, t) \in \mathcal{G}} \left\{ t + \lambda^T u + \mu^T v \right\}
\]
其中,内层下确界对应对偶函数$d(\lambda,\mu) = \inf_{(u,v,t)\in\mathcal{G}} \{t+\lambda^T u + \mu^T v\}$,外层上界对应对偶问题的最大化目标。

强对偶成立($p^* = d^*$)的几何意义是:原始最优值的"下确界"与对偶最优值的"上确界-下确界"相等,即$\mathcal{G}$的极值特性满足"对偶无间隙"。


\subsection{强对偶定理证明}

\begin{theorem}[强对偶定理]
若原始问题是凸问题($f$、$h_j$凸,$g_i$仿射),且满足Slater条件(存在严格可行点$x_{\text{strict}}$),则强对偶成立,即$p^* = d^*$。
\end{theorem}

\begin{proof}
定义集合$A = \mathcal{G}$(即前述几何建模中的凸集,因原始问题是凸问题,$A$是凸集);定义集合$B = \left\{ (0, 0, s) \in \mathbb{R}^m \times \mathbb{R}^p \times \mathbb{R} \mid s < p^* \right\}$(所有"前两分量为0、第三分量小于$p^*$"的点,$B$是凸集)。

\paragraph{关键引理:}$A$与$B$不相交。

若存在$(0,0,s) \in A \cap B$,则$s < p^*$且$(0,0,s) \in \mathcal{G}$——由$\mathcal{G}$的定义,存在$x$满足$g_i(x)=0$、$h_j(x)\leq0$、$f(x)\leq s < p^*$,这与$p^*$是原始问题最优值(最小$f(x)$)矛盾,故$A \cap B = \emptyset$。

由于$A$、$B$是$\mathbb{R}^{m+p+1}$中的不相交凸集,根据\textbf{凸集分离定理},存在非零向量$(\tilde{\lambda}, \tilde{\mu}, \nu) \in \mathbb{R}^m \times \mathbb{R}^p \times \mathbb{R}$和实数$\alpha \in \mathbb{R}$,使得对所有$(u, v, t) \in A$、$(0,0,s) \in B$,有:
\[
\begin{cases}
\tilde{\lambda}^T u + \tilde{\mu}^T v + \nu t \geq \alpha \quad (\text{超平面上方包含} \ A) \\
\tilde{\lambda}^T \cdot 0 + \tilde{\mu}^T \cdot 0 + \nu s \leq \alpha \quad (\text{超平面下方包含} \ B)
\end{cases}
\]

\paragraph{第一步:证明$\nu \geq 0$}

假设$\nu < 0$,对任意$(u, v, t) \in A$,当$t \to +\infty$时,$\tilde{\lambda}^T u + \tilde{\mu}^T v + \nu t \to -\infty$,与"$\geq \alpha$"矛盾,故$\nu \geq 0$。

\paragraph{第二步:证明$\nu \neq 0$}

假设$\nu = 0$,则分离不等式变为$\tilde{\lambda}^T u + \tilde{\mu}^T v \geq \alpha$(对所有$(u, v, t) \in A$)。由Slater条件,存在严格可行点$x_{\text{strict}}$,使得$g_i(x_{\text{strict}})=0$、$h_j(x_{\text{strict}})=v_j < 0$,即$(0, v, t) \in A$。代入得$\tilde{\mu}^T v \geq \alpha$。若令$v_j \to -\infty$(通过调整$x$),则$\tilde{\mu}^T v \to -\infty$,与"$\geq \alpha$"矛盾,故$\nu \neq 0$。

由于$\nu > 0$,可对分离向量$(\tilde{\lambda}, \tilde{\mu}, \nu)$进行缩放(不影响分离性质),令$\nu = 1$。此时:

\begin{enumerate}
    \item 分离不等式变为$\tilde{\lambda}^T u + \tilde{\mu}^T v + t \geq \alpha$(对所有$(u, v, t) \in A$)。对原始问题的任意可行点$x$,取$u_i = g_i(x)=0$、$v_j = h_j(x)\leq0$、$t = f(x)$,代入得:
    \[\tilde{\lambda}^T \cdot 0 + \tilde{\mu}^T h(x) + f(x) \geq \alpha\]
    
    \item 对$B$的不等式($\nu s \leq \alpha$,$s < p^*$),令$s \to p^*$,得$p^* \leq \alpha$(因$\nu=1$)。
    
    \item 结合1和2,得$f(x) + \tilde{\mu}^T h(x) \geq \alpha \geq p^*$。对所有可行$x$取下确界(即对偶函数定义):
    \[
    d(\tilde{\lambda}, \tilde{\mu}) = \inf_x \left\{ f(x) + \tilde{\lambda}^T g(x) + \tilde{\mu}^T h(x) \right\} \geq p^*
    \]
    
    \item 由弱对偶性($d(\lambda,\mu) \leq p^*$),得$d(\tilde{\lambda}, \tilde{\mu}) = p^*$。
\end{enumerate}

假设存在$j$使得$\tilde{\mu}_j < 0$,则选择$x$使$h_j(x)$足够大(正值),可令$\tilde{\lambda}^T g(x) + \tilde{\mu}^T h(x) + f(x) \to -\infty$,与"$\geq p^*$"矛盾,故$\tilde{\mu} \geq 0$。

$(\tilde{\lambda}, \tilde{\mu})$是对偶问题的可行解($\tilde{\mu} \geq 0$),且$d(\tilde{\lambda}, \tilde{\mu}) = p^*$。由对偶最优值$d^* = \sup\{d(\lambda,\mu) \mid \mu \geq 0\}$,得$d^* \geq p^*$;结合弱对偶性$d^* \leq p^*$,故$d^* = p^*$,强对偶成立。
\end{proof}


\section{互补松弛条件}

设原始问题与对偶问题满足:
\begin{enumerate}
    \item 原始问题最优解$x^*$存在,对偶问题最优解$(\lambda^*, \mu^*)$存在;
    \item 强对偶成立($p^* = d^*$)。
\end{enumerate}

则必有以下两个等价结论:
\begin{enumerate}
    \item \textbf{拉格朗日函数极值等式}:$\mathcal{L}(x^*, \lambda^*, \mu^*) = f(x^*)$;
    \item \textbf{乘子-约束乘积为零}:对所有不等式约束的下标$j=1,2,\dots,p$,有
    \[\mu_j^* \cdot h_j(x^*) = 0\]
\end{enumerate}

(注:结论2是互补松弛条件的"核心量化形式",也是实际应用中最常用的表述。)

\begin{proof}
由强对偶成立($p^* = d^*$),结合原始最优值与对偶函数的定义,可得:
\[
f(x^*) = p^* = d^* = d(\lambda^*, \mu^*) \tag{1}
\]

根据对偶函数的定义($d(\lambda,\mu) = \inf_{x \in \mathbb{R}^n} \mathcal{L}(x,\lambda,\mu)$),对任意$x \in \mathbb{R}^n$,对偶函数是拉格朗日函数的下确界,因此:
\[
d(\lambda^*, \mu^*) \leq \mathcal{L}(x^*, \lambda^*, \mu^*) \tag{2}
\]

结合式(1)与式(2),得:
\[
f(x^*) \leq \mathcal{L}(x^*, \lambda^*, \mu^*) \tag{3}
\]

将拉格朗日函数在$(x^*, \lambda^*, \mu^*)$处展开:
\[
\mathcal{L}(x^*, \lambda^*, \mu^*) = f(x^*) + \sum_{i=1}^m \lambda_i^* g_i(x^*) + \sum_{j=1}^p \mu_j^* h_j(x^*) \tag{4}
\]

由\textbf{原始可行性}($x^*$满足等式约束),对所有$i=1,\dots,m$,有$g_i(x^*) = 0$,因此式(4)中的等式约束乘子项消失:
\[
\mathcal{L}(x^*, \lambda^*, \mu^*) = f(x^*) + \sum_{j=1}^p \mu_j^* h_j(x^*) \tag{5}
\]

由\textbf{对偶可行性}($\mu^* \geq 0$),对所有$j=1,\dots,p$,有$\mu_j^* \geq 0$;由\textbf{原始可行性}($x^*$满足不等式约束),对所有$j=1,\dots,p$,有$h_j(x^*) \leq 0$。

因此,对每个$j$,$\mu_j^* \cdot h_j(x^*) \leq 0$(非正数),求和后仍为非正数:
\[
\sum_{j=1}^p \mu_j^* h_j(x^*) \leq 0 \tag{6}
\]

将式(6)代入式(5),得:
\[
\mathcal{L}(x^*, \lambda^*, \mu^*) \leq f(x^*) \tag{7}
\]

结合式(3)($f(x^*) \leq \mathcal{L}(x^*, \lambda^*, \mu^*)$)与式(7)($\mathcal{L}(x^*, \lambda^*, \mu^*) \leq f(x^*)$),所有不等式变为等式:
\[
\mathcal{L}(x^*, \lambda^*, \mu^*) = f(x^*) \tag{8}
\]

将式(8)代入式(5),得:
\[
f(x^*) = f(x^*) + \sum_{j=1}^p \mu_j^* h_j(x^*) \implies \sum_{j=1}^p \mu_j^* h_j(x^*) = 0 \tag{9}
\]

由步骤3可知,每个$\mu_j^* h_j(x^*) \leq 0$(非正数),而\textbf{非正数的和为零,当且仅当每个非正数均为零}。因此:
\[
\mu_j^* h_j(x^*) = 0 \quad \forall j=1,2,\dots,p \tag{10}
\]

式(8)与式(10)共同构成互补松弛条件。
\end{proof}

\begin{remark}
互补松弛条件$\mu_j^* h_j(x^*) = 0$的本质是\textbf{判断不等式约束对最优解的"活性"},可分为两种互斥情况,直观反映约束是否影响最优解:

\begin{itemize}
    \item \textbf{情况1}:$\mu_j^* > 0$(对偶乘子为正)。对应约束$h_j(x) \leq 0$是\textbf{紧约束}(起作用的约束)。最优解$x^*$恰好落在约束边界上($h_j(x^*) = 0$),该约束限制了目标函数的进一步优化。
    \item \textbf{情况2}:$\mu_j^* = 0$(对偶乘子为零)。对应约束$h_j(x) \leq 0$是\textbf{非紧约束}(不起作用的约束)。最优解$x^*$落在约束内部($h_j(x^*) < 0$),即使移除该约束,最优解也不会改变。
\end{itemize}

(注:等式约束$g_i(x) = 0$始终为"紧约束",无互补松弛判断,因其对偶乘子$\lambda_i^*$无符号限制,无需通过乘积为零判断活性。)
\end{remark}

互补松弛条件是KKT(Karush-Kuhn-Tucker)最优性条件的重要组成部分。在\textbf{凸问题+强对偶成立+函数可微}的前提下,KKT条件是原始-对偶最优解的充要条件,其结构如下:

\subsection{KKT条件的完整构成(含互补松弛)}
设原始问题为凸问题($f$、$h_j$凸,$g_i$仿射),$f$、$g_i$、$h_j$可微,$x^*$为原始最优解,$(\lambda^*, \mu^*)$为对偶最优解,则:
\begin{enumerate}
    \item \textbf{平稳性}:$\nabla_x \mathcal{L}(x^*, \lambda^*, \mu^*) = 0$(拉格朗日函数在$x^*$处梯度为零,即无改进方向);
    \item \textbf{原始可行性}:$g_i(x^*) = 0\ (i=1,\dots,m)$,$h_j(x^*) \leq 0\ (j=1,\dots,p)$;
    \item \textbf{对偶可行性}:$\mu_j^* \geq 0\ (j=1,\dots,p)$;
    \item \textbf{互补松弛}:$\mu_j^* h_j(x^*) = 0\ (j=1,\dots,p)$。
\end{enumerate}

可见,互补松弛条件是KKT条件的"闭环环节"——它连接了原始约束的可行性($h_j(x^*) \leq 0$)与对偶乘子的可行性($\mu_j^* \geq 0$),确保原始-对偶最优解的一致性。


\section{对偶理论的应用实例}

\subsection{线性规划的对偶}

考虑线性规划问题(原始问题):
\[
\begin{cases}
\min\limits_{x} & c^{T} x \\
\text{subject to} & A x = b \\
& x \geq 0
\end{cases}
\]
其中,$x \in \mathbb{R}^n$ 为原始优化变量,$c \in \mathbb{R}^n$、$A \in \mathbb{R}^{m \times n}$、$b \in \mathbb{R}^m$ 为已知参数。

引入拉格朗日乘子:
\begin{itemize}
    \item 等式约束 $A x = b$ 对应的乘子:$\lambda \in \mathbb{R}^m$(无符号限制);
    \item 不等式约束 $x \geq 0$ 对应的乘子:$\mu \in \mathbb{R}^n$(满足 $\mu \geq 0$)。
\end{itemize}

拉格朗日函数定义为:
\[
\mathcal{L}(x, \lambda, \mu) = c^{T} x + \lambda^{T}(b - A x) - \mu^{T} x
\]

对偶函数 $d(\lambda, \mu)$ 是拉格朗日函数关于 $x$ 的下确界,即:
\[
d(\lambda, \mu) = \inf_{x} \left[ c^{T} x + \lambda^{T}(b - A x) - \mu^{T} x \right]
\]
将函数整理为关于 $x$ 的线性形式:
\[
d(\lambda, \mu) = \inf_{x} \left[ (c - A^{T} \lambda - \mu)^{T} x + \lambda^{T} b \right]
\]

为使下确界有限(避免取值为 $-\infty$),需满足线性项系数为零:
\[
c - A^{T} \lambda - \mu = 0 \implies \mu = c - A^{T} \lambda
\]
结合 $\mu \geq 0$ 的约束,可得:
\[
A^{T} \lambda \leq c
\]

将 $\mu = c - A^{T} \lambda$ 代入拉格朗日函数,对偶函数简化为:
\[
d(\lambda) = \lambda^{T} b
\]

对偶问题的目标是最大化对偶函数 $d(\lambda)$,同时满足对偶可行性约束,即:
\[
\begin{cases}
\max\limits_{\lambda} & b^{T} \lambda \\
\text{subject to} & A^{T} \lambda \leq c
\end{cases}
\]
此为线性规划的标准对偶形式。


\subsection{二次规划的对偶}

考虑二次规划问题(原始问题):
\[
\begin{cases}
\min\limits_{x} & \frac{1}{2} x^{T} Q x + c^{T} x \\
\text{subject to} & A x = b \\
& x \geq 0
\end{cases}
\]
其中,$x \in \mathbb{R}^n$ 为原始优化变量,$Q \in \mathbb{R}^{n \times n}$ 为\textbf{半正定矩阵}($Q \succeq 0$,保证目标函数凸),$c \in \mathbb{R}^n$、$A \in \mathbb{R}^{m \times n}$、$b \in \mathbb{R}^m$ 为已知参数。

引入拉格朗日乘子:
\begin{itemize}
    \item 等式约束 $A x = b$ 对应的乘子:$\lambda \in \mathbb{R}^m$(无符号限制);
    \item 不等式约束 $x \geq 0$ 对应的乘子:$\mu \in \mathbb{R}^n$(满足 $\mu \geq 0$)。
\end{itemize}

拉格朗日函数定义为:
\[
\mathcal{L}(x, \lambda, \mu) = \frac{1}{2} x^{T} Q x + c^{T} x + \lambda^{T}(b - A x) - \mu^{T} x
\]

对偶函数 $d(\lambda, \mu)$ 是拉格朗日函数关于 $x$ 的下确界,即:
\[
d(\lambda, \mu) = \inf_{x} \left[ \frac{1}{2} x^{T} Q x + \left( c - A^{T} \lambda - \mu \right)^{T} x + \lambda^{T} b \right]
\]

由于目标函数是关于 $x$ 的二次函数,且 $Q \succeq 0$(凸函数),当 $Q$ \textbf{正定}($Q \succ 0$)时,函数存在唯一极小值。对 $x$ 求导并令梯度为零,得最优 $x$:
\[
\nabla_x \mathcal{L} = Q x + (c - A^{T} \lambda - \mu) = 0 \implies x = -Q^{-1}(c - A^{T} \lambda - \mu)
\]

将最优 $x$ 代入拉格朗日函数,化简得对偶函数:
\[
d(\lambda, \mu) = -\frac{1}{2} \left( c - A^{T} \lambda - \mu \right)^{T} Q^{-1} \left( c - A^{T} \lambda - \mu \right) + \lambda^{T} b
\]

对偶问题的目标是最大化对偶函数 $d(\lambda, \mu)$,同时满足对偶可行性约束,即:
\[
\begin{cases}
\max\limits_{\lambda, \mu} & -\frac{1}{2} \left( c - A^{T} \lambda - \mu \right)^{T} Q^{-1} \left( c - A^{T} \lambda - \mu \right) + \lambda^{T} b \\
\text{subject to} & \mu \geq 0
\end{cases}
\]


\subsection{数值例题(二次规划对偶求解)}

以具体二次规划问题为例,验证对偶理论的应用及强对偶性。

\[
\begin{cases}
\min\limits_{x_1, x_2} & \frac{1}{2}(x_1^2 + x_2^2) \\
\text{subject to} & x_1 + x_2 = 1 \quad (\text{等式约束}) \\
& x_1 \geq 0, \ x_2 \geq 0 \quad (\text{不等式约束})
\end{cases}
\]

\subsubsection{步骤1:构造拉格朗日函数}
引入乘子:
\begin{itemize}
    \item 等式约束 $x_1 + x_2 = 1$ 对应 $\lambda \in \mathbb{R}$;
    \item 不等式约束 $x_1 \geq 0$、$x_2 \geq 0$ 对应 $\mu_1 \geq 0$、$\mu_2 \geq 0$。
\end{itemize}

拉格朗日函数:
\[
\mathcal{L}(x, \lambda, \mu) = \frac{1}{2}(x_1^2 + x_2^2) + \lambda(1 - x_1 - x_2) - \mu_1 x_1 - \mu_2 x_2
\]

\subsubsection{步骤2:列写KKT条件}
\begin{enumerate}
    \item \textbf{平稳性}:$\nabla_x \mathcal{L} = 0$
    \[\frac{\partial \mathcal{L}}{\partial x_1} = x_1 - \lambda - \mu_1 = 0\]
    \[\frac{\partial \mathcal{L}}{\partial x_2} = x_2 - \lambda - \mu_2 = 0\]
    \item \textbf{原始可行性}:$x_1 + x_2 = 1$,$x_1 \geq 0$,$x_2 \geq 0$;
    \item \textbf{对偶可行性}:$\mu_1 \geq 0$,$\mu_2 \geq 0$;
    \item \textbf{互补松弛}:$\mu_1 x_1 = 0$,$\mu_2 x_2 = 0$。
\end{enumerate}

\subsubsection{步骤3:分析最优解}
假设 $x_1 > 0$ 且 $x_2 > 0$,由互补松弛条件得 $\mu_1 = \mu_2 = 0$。代入平稳性条件:
\[x_1 = \lambda, \quad x_2 = \lambda\]
结合等式约束 $x_1 + x_2 = 1$,得 $\lambda = 0.5$,$x_1 = x_2 = 0.5$。

验证所有约束均满足,因此原始最优解为 $x^* = (0.5, 0.5)$,原始最优值 $p^* = \frac{1}{2}(0.5^2 + 0.5^2) = 0.25$。

\subsubsection{步骤4:对偶问题求解}

对偶函数是拉格朗日函数关于 $x_1, x_2$ 的下确界:
\[
d(\lambda, \mu) = \inf_{x_1, x_2} \left[ \frac{1}{2}(x_1^2 + x_2^2) + \lambda(1 - x_1 - x_2) - \mu_1 x_1 - \mu_2 x_2 \right]
\]

对 $x_1, x_2$ 求导并令梯度为零,得:
\[x_1 = \lambda + \mu_1, \quad x_2 = \lambda + \mu_2\]

将 $x_1, x_2$ 代入拉格朗日函数,展开化简:
\begin{align*}
d(\lambda, \mu) &= \frac{1}{2}\left[ (\lambda + \mu_1)^2 + (\lambda + \mu_2)^2 \right] + \lambda\left(1 - 2\lambda - \mu_1 - \mu_2\right) \\
&\quad - \mu_1(\lambda + \mu_1) - \mu_2(\lambda + \mu_2) \\
&= -\lambda^2 - \lambda \mu_1 - \lambda \mu_2 - \frac{1}{2}\mu_1^2 - \frac{1}{2}\mu_2^2 + \lambda
\end{align*}

对偶问题为:
\[
\begin{cases}
\max\limits_{\lambda, \mu_1, \mu_2} & -\lambda^2 - \lambda \mu_1 - \lambda \mu_2 - \frac{1}{2}\mu_1^2 - \frac{1}{2}\mu_2^2 + \lambda \\
\text{subject to} & \mu_1 \geq 0, \ \mu_2 \geq 0
\end{cases}
\]

由原始最优解的互补松弛条件($\mu_1 = \mu_2 = 0$),代入对偶函数:
\[
d(\lambda, 0, 0) = -\lambda^2 + \lambda
\]

对 $\lambda$ 求导并令导数为零,得 $\lambda = 0.5$,此时对偶最优值 $d^* = -(0.5)^2 + 0.5 = 0.25$。

原始最优值 $p^* = 0.25$,对偶最优值 $d^* = 0.25$,满足 $p^* = d^*$,强对偶成立。


\subsection{SVM中的原问题与对偶问题}

支持向量机(SVM)是对偶理论在机器学习中的典型应用,核心是通过对偶问题简化高维特征空间中的优化求解。

\subsubsection{关键结论(原对偶关系)}
\begin{itemize}
    \item 原始问题(Primal):优化变量为模型参数 $w$(权重向量)、$b$(偏置),目标是最大化几何间隔;
    \item 对偶问题(Dual):优化变量为对偶乘子 $\alpha_i$(对应每个样本),目标是最大化对偶函数;
    \item 等价性:通过KKT条件联结原对偶解,满足 $w = \sum_{i} \alpha_i y_i x_i$、$\sum_{i} \alpha_i y_i = 0$、$0 \leq \alpha_i \leq C$($C$ 为罚系数);
    \item 支持向量特性:仅 $\alpha_i > 0$ 的样本(支持向量)决定 $w$ 与几何间隔 $1/\|w\|$;
    \item 核化能力:对偶问题仅依赖样本内积 $\langle x_i, x_j \rangle$,可直接替换为核函数 $K(x_i, x_j)$,实现高维空间映射。
\end{itemize}

\subsubsection{软间隔SVM的原始问题}
目标:在允许样本"软违约"(用 $\xi_i \geq 0$ 表示违约程度)的前提下,最小化 $\frac{1}{2}\|w\|^2$(等价于最大化几何间隔),同时控制违约惩罚。

\[
\begin{cases}
\min\limits_{w, b, \xi} & \frac{1}{2}\|w\|^2 + C \sum_{i=1}^n \xi_i \\
\text{subject to} & y_i(w^T x_i + b) \geq 1 - \xi_i \quad (i=1, ..., n) \\
& \xi_i \geq 0 \quad (i=1, ..., n)
\end{cases}
\]
其中,$C > 0$ 为罚系数(平衡间隔大小与违约惩罚),$y_i \in \{+1, -1\}$ 为样本标签,$\xi_i$ 为违约变量。

\subsubsection{拉格朗日函数与KKT条件}
引入对偶乘子:
\begin{itemize}
    \item 约束 $y_i(w^T x_i + b) \geq 1 - \xi_i$ 对应 $\alpha_i \geq 0$;
    \item 约束 $\xi_i \geq 0$ 对应 $\beta_i \geq 0$。
\end{itemize}

拉格朗日函数:
\[
\mathcal{L}(w, b, \xi, \alpha, \beta) = \frac{1}{2}\|w\|^2 + C \sum_{i=1}^n \xi_i - \sum_{i=1}^n \alpha_i \left[ y_i(w^T x_i + b) - 1 + \xi_i \right] - \sum_{i=1}^n \beta_i \xi_i
\]

对 $w, b, \xi_i$ 求导并令梯度为零:
\begin{itemize}
    \item 对 $w$:$\nabla_w \mathcal{L} = w - \sum_{i=1}^n \alpha_i y_i x_i = 0 \implies w = \sum_{i=1}^n \alpha_i y_i x_i$;
    \item 对 $b$:$\nabla_b \mathcal{L} = -\sum_{i=1}^n \alpha_i y_i = 0 \implies \sum_{i=1}^n \alpha_i y_i = 0$;
    \item 对 $\xi_i$:$\nabla_{\xi_i} \mathcal{L} = C - \alpha_i - \beta_i = 0 \implies \alpha_i + \beta_i = C$。
\end{itemize}

结合对偶可行性 $\alpha_i \geq 0$、$\beta_i \geq 0$,得 $0 \leq \alpha_i \leq C$。

\subsubsection{软间隔SVM的对偶问题}
消去原始变量 $w, b, \xi_i$,代入拉格朗日函数,最终对偶问题为:
\[
\begin{cases}
\max\limits_{\alpha} & \sum_{i=1}^n \alpha_i - \frac{1}{2} \sum_{i=1}^n \sum_{j=1}^n \alpha_i \alpha_j y_i y_j \langle x_i, x_j \rangle \\
\text{subject to} & 0 \leq \alpha_i \leq C \quad (i=1, ..., n) \\
& \sum_{i=1}^n \alpha_i y_i = 0
\end{cases}
\]

\paragraph{核技巧应用}
将内积 $\langle x_i, x_j \rangle$ 替换为核函数 $K(x_i, x_j)$(如线性核 $K(x_i, x_j) = x_i^T x_j$、RBF核 $K(x_i, x_j) = \exp(-\gamma\|x_i - x_j\|^2)$),即可得到核SVM的对偶形式,解决高维特征空间的优化问题。

\subsubsection{原对偶的意义与决策函数}

\paragraph{原对偶的核心价值}
\begin{itemize}
    \item \textbf{维度优势}:当特征维度 $d$ 极大(如文本分类)、样本数 $n$ 较小时,对偶问题(变量数为 $n$)比原始问题(变量数为 $d+1$)更易求解;
    \item \textbf{可解释性}:$\alpha_i$ 可视为"样本重要性权重"(支持向量的 $\alpha_i > 0$,非支持向量的 $\alpha_i = 0$);
    \item \textbf{最优性判据}:原始问题值 $P(w, b, \xi)$ 与对偶问题值 $D(\alpha)$ 的对偶间隙 $P - D \geq 0$,可作为算法收敛的停机准则。
\end{itemize}

\paragraph{决策函数与偏置计算}
\begin{itemize}
    \item 决策函数:已知对偶最优解 $\alpha^*$,结合核函数 $K$,模型对新样本 $x$ 的预测为:
    \[f(x) = \text{sign}\left( \sum_{i=1}^n \alpha_i^* y_i K(x_i, x) + b^* \right)\]
    \item 偏置 $b^*$:取任意满足 $0 < \alpha_i^* < C$ 的支持向量 $x_i$,代入约束 $y_i(w^* x_i + b^*) = 1$,得:
    \[b^* = y_i - \sum_{j=1}^n \alpha_j^* y_j K(x_j, x_i)\]
\end{itemize}

\chapter{约束优化问题解法}

本章系统介绍常见的带约束优化问题求解思想与典型算法,包括\textbf{外点法(惩罚函数法)}、\textbf{增广拉格朗日法(ALM)}以及在可分结构下广泛应用的\textbf{ADMM(交替方向乘子法)}。

\section{外点法}
外点法是\textbf{惩罚函数法}(Penalty Function Method)的核心分支,属于约束优化的\textbf{间接解法}——通过将约束条件转化为目标函数的惩罚项,将原约束问题转化为一系列无约束优化子问题,迭代求解无约束子问题的最优解,使其逐步逼近原约束问题的最优解。其核心特征是:\textbf{迭代点始终位于可行域外部},通过惩罚项迫使迭代点向可行域边界收敛。

\subsection{核心定义与问题形式}
\begin{definition}[原约束优化问题(标准形式)]
设优化变量 $\boldsymbol{x} \in \mathbb{R}^n$,原问题定义为:
\[
\begin{cases}
\min_{\boldsymbol{x}} & f(\boldsymbol{x}) \quad \text{(目标函数,连续可微)} \\
\text{s.t.} & g_i(\boldsymbol{x}) \leq 0 \quad (i=1,2,\dots,m) \quad \text{(不等式约束)} \\
& h_j(\boldsymbol{x}) = 0 \quad (j=1,2,\dots,p) \quad \text{(等式约束)}
\end{cases}
\]
其中:
\begin{itemize}
    \item 可行域 $\Omega = \{\boldsymbol{x} \in \mathbb{R}^n \mid g_i(\boldsymbol{x}) \leq 0, h_j(\boldsymbol{x}) = 0\}$;
    \item 假设 $\Omega \neq \emptyset$(可行域非空),且原问题存在最优解 $\boldsymbol{x}^*$。
\end{itemize}
\end{definition}

\subsubsection{2. 外点法的核心思想}
\begin{itemize}
    \item 对于\textbf{可行域外部的点}(违反约束的点),通过惩罚项施加“惩罚”,使其目标函数值增大;
    \item 惩罚强度由\textbf{惩罚参数 $\mu > 0$} 控制,且 $\mu$ 随迭代逐步增大($\mu_k \to +\infty$);
    \item 当 $\mu$ 足够大时,无约束子问题的最优解会“被迫”靠近可行域,最终收敛到原问题的最优解 $\boldsymbol{x}^*$。
\end{itemize}

\subsection{数学建模:惩罚函数构造}
外点法的核心是设计\textbf{惩罚函数 $P(\boldsymbol{x}, \mu)$},其通用形式为:
\[
P(\boldsymbol{x}, \mu) = f(\boldsymbol{x}) + \mu \cdot \Phi(\boldsymbol{x})
\]
其中:
\begin{itemize}
    \item $f(\boldsymbol{x})$ 为原目标函数;
    \item $\mu > 0$ 为惩罚参数(迭代中满足 $\mu_{k+1} > \mu_k$,且 $\mu_k \to +\infty$);
    \item $\Phi(\boldsymbol{x})$ 为\textbf{约束违反度量函数}(非负、连续可微),用于量化点 $\boldsymbol{x}$ 对约束的违反程度,满足:
    \[
    \Phi(\boldsymbol{x}) = 0 \iff \boldsymbol{x} \in \Omega \quad \text{(可行点无惩罚)}
    \]
    \[
    \Phi(\boldsymbol{x}) > 0 \iff \boldsymbol{x} \notin \Omega \quad \text{(不可行点有惩罚,违反越严重惩罚越大)}
    \]
\end{itemize}

\subsubsection{3. 约束违反度量函数的具体形式}
根据约束类型(不等式/等式),$\Phi(\boldsymbol{x})$ 通常分解为两部分:
\[
\Phi(\boldsymbol{x}) = \sum_{i=1}^m \phi(g_i(\boldsymbol{x})) + \sum_{j=1}^p \psi(h_j(\boldsymbol{x}))
\]
其中:

\paragraph{(1)不等式约束惩罚项 $\phi(g_i(\boldsymbol{x}))$}
最常用的是\textbf{二次惩罚}(连续可微,便于无约束优化求解):
\[
\phi(t) = \max(0, t)^2 = 
\begin{cases}
t^2 & t > 0 \quad \text{(违反约束,施加惩罚)} \\
0 & t \leq 0 \quad \text{(满足约束,无惩罚)}
\end{cases}
\]
\begin{itemize}
    \item 其他形式:一次惩罚 $\phi(t) = \max(0, t)$(不可微,仅用于简单问题)、指数惩罚 $\phi(t) = e^{\alpha t} - 1$($\alpha > 0$,惩罚增长更快)。
\end{itemize}

\paragraph{(2)等式约束惩罚项 $\psi(h_j(\boldsymbol{x}))$}
等式约束无“满足/违反”的中间状态,直接惩罚偏差:
\[
\psi(t) = t^2 \quad \text{(二次惩罚,最常用)}
\]
\begin{itemize}
    \item 其他形式:$\psi(t) = |t|$(不可微)、$\psi(t) = t^4$(惩罚增长更快)。
\end{itemize}

\subsubsection{4. 完整惩罚函数示例(二次惩罚)}
结合上述形式,二次惩罚的外点法惩罚函数为:
\[
P(\boldsymbol{x}, \mu) = f(\boldsymbol{x}) + \mu \left[ \sum_{i=1}^m \max(0, g_i(\boldsymbol{x}))^2 + \sum_{j=1}^p h_j(\boldsymbol{x})^2 \right]
\]

\subsection{算法流程(严格形式化)}
\begin{algorithm}[外点法算法流程]
\textbf{输入}:
\begin{itemize}
    \item 原问题目标函数 $f(\boldsymbol{x})$、约束 $g_i(\boldsymbol{x})$、$h_j(\boldsymbol{x})$;
    \item 初始参数:初始点 $\boldsymbol{x}_0 \in \mathbb{R}^n$(可在可行域外部)、初始惩罚参数 $\mu_1 > 0$、惩罚参数增长因子 $\beta > 1$(通常取 $10$)、收敛精度 $\epsilon > 0$。
\end{itemize}

\textbf{迭代步骤}:
\begin{enumerate}
    \item \textbf{初始化}:令迭代次数 $k = 1$;
    \item \textbf{构造惩罚函数}:针对当前 $\mu_k$,构造 $P(\boldsymbol{x}, \mu_k)$;
    \item \textbf{求解无约束子问题}:以 $\boldsymbol{x}_{k-1}$ 为初始点,求解 $\min_{\boldsymbol{x}} P(\boldsymbol{x}, \mu_k)$,得到最优解 $\boldsymbol{x}_k$;
    \item \textbf{收敛判断}:若满足以下任一收敛条件,停止迭代,输出 $\boldsymbol{x}_k \approx \boldsymbol{x}^*$:
    \begin{itemize}
        \item 约束违反度量足够小:$\Phi(\boldsymbol{x}_k) < \epsilon$;
        \item 目标函数变化足够小:$\| f(\boldsymbol{x}_k) - f(\boldsymbol{x}_{k-1}) \| < \epsilon$;
        \item 迭代点变化足够小:$\| \boldsymbol{x}_k - \boldsymbol{x}_{k-1} \| < \epsilon$;
    \end{itemize}
    \item \textbf{更新惩罚参数}:令 $\mu_{k+1} = \beta \cdot \mu_k$,$k = k + 1$,返回步骤 2。
\end{enumerate}

\textbf{输出}:
原约束问题的近似最优解 $\boldsymbol{x}_k$。
\end{algorithm}

\begin{remark}[无约束子问题求解]
\begin{center}
\begin{tabular}{|p{0.25\textwidth}|p{0.2\textwidth}|p{0.2\textwidth}|p{0.25\textwidth}|}
\hline
\textbf{场景分类} & \textbf{推荐方法} & \textbf{工具/实现} & \textbf{核心注意事项} \\
\hline
低维 ($n \leq 10$) + 光滑 $P(\boldsymbol{x},\mu)$ & 解析法(求梯度=0解方程组) & 手动推导 & 仅适合简单问题/教学验证 \\
\hline
中高维 ($n \leq 1000$) + 光滑 $P(\boldsymbol{x},\mu)$(工程主流) & 拟牛顿法 (L-BFGS/BFGS), 梯度下降法等等 & Scipy.fmin\_bfgs, Matlab.fminunc & 1. 传前一轮 $\boldsymbol{x}_{k-1}$ 作初始点;2. 配合线搜索;3. 病态时优先 L-BFGS \\
\hline
高维 ($n > 1000$) + 光滑 $P(\boldsymbol{x},\mu)$ & 共轭梯度法 & Scipy.fmin\_cg & 低内存消耗,适合大规模问题 \\
\hline
不可微 $P(\boldsymbol{x},\mu)$(一次惩罚等) & 梯度自由法 (Nelder-Mead) & Scipy.fmin & 收敛慢,仅用于简单非光滑问题 \\
\hline
\end{tabular}
\end{center}
\end{remark}

\begin{remark}[惩罚参数更新方法]
\begin{center}
\begin{tabular}{|p{0.15\textwidth}|p{0.25\textwidth}|p{0.15\textwidth}|p{0.15\textwidth}|p{0.2\textwidth}|}
\hline
\textbf{更新方法} & \textbf{公式/逻辑} & \textbf{优点} & \textbf{缺点} & \textbf{适用场景} \\
\hline
经典倍增法则(基础) & $\mu_{k+1} = \beta\mu_k$ ($\beta=5\sim10$) & 实现最简单,无需额外计算 & 易病态,需调 $\beta$ & 快速验证、简单问题 \\
\hline
自适应更新法(首选) & 阈值触发:$\Phi(\boldsymbol{x}_k)>\eta_k$ 则 $\mu_{k+1}=\beta\mu_k$,否则保持 & 平衡收敛速度与稳定性,鲁棒性强 & 需计算约束违反度量 $\Phi(\boldsymbol{x}_k)$ & 工程实践、中高维问题(优先选) \\
\hline
线性增长法 & $\mu_{k+1} = \mu_k + c$ ($c>0$) & 数值稳定性最好,不易病态 & 收敛慢 & 高维等式约束、易病态问题 \\
\hline
指数增长法 & $\mu_{k+1} = \beta^k\mu_1$ & 收敛极快 & 极易病态 & 约束违反程度下降快的简单问题 \\
\hline
KKT残量法 & 按 KKT 残量 $r_k$ 动态调整 $\mu$ & 精准,收敛速率高 & 复杂,需估计拉格朗日乘子 & 高精度需求、学术研究 \\
\hline
\end{tabular}
\end{center}
\end{remark}

\subsection{收敛性分析(核心结论)}
\begin{theorem}[外点法收敛性]
外点法的收敛性依赖于惩罚参数 $\mu_k \to +\infty$,关键结论如下(严格证明需用到变分不等式或闭映射理论):
\begin{enumerate}
    \item \textbf{序列有界性}:若原问题最优解存在,且惩罚函数 $P(\boldsymbol{x}, \mu)$ 是强制函数(当 $\|\boldsymbol{x}\| \to +\infty$ 时 $P(\boldsymbol{x}, \mu) \to +\infty$),则迭代序列 $\{\boldsymbol{x}_k\}$ 有界;
    \item \textbf{收敛性}:设 $\{\boldsymbol{x}_k\}$ 是迭代序列,其任一聚点 $\boldsymbol{x}^*$ 都是原约束问题的最优解;
    \item \textbf{收敛速率}:二次惩罚外点法的收敛速率为 \textbf{线性收敛}(当 $\mu_k$ 按指数增长时,可达到超线性收敛)。
\end{enumerate}
\end{theorem}

\paragraph{关键直观解释}
当 $\mu_k$ 增大时,惩罚项权重越来越大:
\begin{itemize}
    \item 若 $\boldsymbol{x}_k$ 仍在可行域外部,惩罚项会主导 $P(\boldsymbol{x}, \mu_k)$,迫使 $\boldsymbol{x}_{k+1}$ 向可行域靠近;
    \item 当 $\mu_k \to +\infty$ 时,可行域外部的点会被赋予无穷大惩罚,因此无约束子问题的最优解必须“落在”可行域边界或内部,即收敛到原问题最优解。
\end{itemize}

\subsection{优缺点}
\subsubsection{优点}
\begin{enumerate}
    \item \textbf{初始点灵活}:无需初始点在可行域内(内点法必须初始点可行),尤其适合可行域难以构造初始点的问题;
    \item \textbf{构造简单}:惩罚函数形式直观,无约束子问题可直接用梯度下降、牛顿法等成熟算法求解;
    \item \textbf{兼容性强}:可同时处理不等式约束和等式约束,无需单独设计逻辑。
\end{enumerate}

\subsubsection{缺点}
\begin{enumerate}
    \item \textbf{惩罚病(Penalty Ill-Conditioning)}:当 $\mu_k$ 过大时,惩罚函数 $P(\boldsymbol{x}, \mu_k)$ 的 Hessian 矩阵会呈现“病态”(条件数极大),导致无约束子问题求解困难(梯度下降步长过小、收敛变慢);
    \item \textbf{仅收敛到可行域边界}:对于不等式约束,最优解若在可行域内部(内点),外点法仍会收敛到边界(需结合其他准则修正);
    \item \textbf{线性收敛速率}:相比内点法(超线性收敛),收敛速度较慢,适合中小规模约束优化问题。
\end{enumerate}

\subsubsection{深入分析:$\mu_k$ 过大会让 Hessian 病态?}
\begin{enumerate}
    \item \textbf{Gauss-Newton 近似下的 Hessian 结构}:
    在 Gauss-Newton 近似(忽略约束的二阶项)下,罚函数的 Hessian 可以拆成两部分:
    \[
    H_{\mu_k} := \nabla^2 \Phi_{\mu_k}(x) \approx \underbrace{\nabla^2 f(x)}_{H_f} + \mu_k \underbrace{J_h(x)^T J_h(x)}_{H_h}
    \]
    通俗解释:
    \begin{itemize}
        \item $H_f$ 是\textbf{原目标函数的 Hessian}:反映原目标在当前点的“曲率”;
        \item $H_h$ 是\textbf{等式约束惩罚项的 Hessian}:由约束的雅可比矩阵外积得到,反映约束对惩罚项的“影响强度”;
        \item $\mu_k$ 是惩罚参数:控制约束惩罚项的权重。
    \end{itemize}

    \item \textbf{特征值的变化(矩阵“伸缩能力”的改变)}:
    当 $\mu_k$ 很大时,$H_{\mu_k}$ 的特征值会出现“两极分化”:
    \[
    \lambda_{\text{max}}(H_{\mu_k}) \approx \lambda_{\text{max}}(H_f) + \mu_k \sigma_{\text{max}}^2 \quad (\text{最大特征值被} \mu_k \text{放大})
    \]
    \[
    \lambda_{\text{min}}(H_{\mu_k}) \approx \lambda_{\text{min}}(H_f) \quad (\text{最小特征值基本不变})
    \]

    \item \textbf{条件数暴增 $\to$ Hessian 病态}:
    矩阵的“条件数”是 \textbf{最大特征值 $\div$ 最小特征值},用来衡量矩阵的“病态程度”:
    \[
    \kappa(H_{\mu_k}) = \frac{\lambda_{\text{max}}(H_{\mu_k})}{\lambda_{\text{min}}(H_{\mu_k})} \approx \frac{\lambda_{\text{max}}(H_f) + \mu_k \sigma_{\text{max}}^2}{\lambda_{\text{min}}(H_f)}
    \]
    当 $\mu_k \to +\infty$ 时,条件数 $\kappa(H_{\mu_k}) \to +\infty$ —— 这就是“Hessian 病态”。

    \item \textbf{病态的后果}:
    \begin{itemize}
        \item \textbf{线性系统求解困难}:迭代法收敛慢,直接法数值不稳定。
        \item \textbf{线搜索步长受限}:病态 Hessian 对应的二次模型会变成“很尖的山谷”,导致线搜索只能小步慢挪。
    \end{itemize}

    \item \textbf{结论}:不要一开始把 $\mu_k$ 设太大,应逐步增大。
\end{enumerate}

\begin{remarkinner}
\subsubsection{深入分析:等式约束惩罚项的本质}

等式二次罚函数形式 $ \frac{\mu_k}{2}\|h(x)\|^2 = \frac{\mu_k}{2}\sum_i h_i(x)^2 $,是等式约束下二次罚函数的\textbf{标准形式}(乘以$\frac{1}{2}$是为了推导方便)。

对罚函数 $ \frac{\mu_k}{2}\|h(x)\|^2 $ 求梯度:
根据链式法则,单个 $ h_i(x)^2 $ 的梯度是 $ 2h_i(x)\nabla h_i(x) $,汇总后为 $ \mu_k\sum_i h_i(x)\nabla h_i(x) $。
而雅可比矩阵 $ J_h(x) $ 的定义是“每行对应 $ \nabla h_i(x)^T $”,因此 $ J_h(x)^T $ 的每列对应 $ \nabla h_i(x) $,乘以 $ h(x) $(列向量)恰好得到 $ \sum_i h_i(x)\nabla h_i(x) $。
最终梯度 $ \nabla\left( \frac{\mu_k}{2}\|h\|^2 \right) = \mu_k J_h(x)^T h(x) $。

对梯度 $ \mu_k J_h(x)^T h(x) $ 求 Hessian(二阶导数),需用\textbf{乘积求导法则}:
\[
\nabla^2\left( \mu_k J_h^T h \right) = \mu_k \left( \nabla(J_h^T) \cdot h + J_h^T \cdot \nabla h \right)
\]
\begin{itemize}
    \item 第一项 $ \nabla(J_h^T) \cdot h $ 对应“雅可比矩阵的导数乘以 $ h $”,即 $ \sum_i h_i(x)\nabla^2 h_i(x) $(因为 $ J_h $ 的元素是 $ \partial h_i/\partial x_j $,其导数是 $ \partial^2 h_i/\partial x_j\partial x_k $,即 $ \nabla^2 h_i $ 的元素);
    \item 第二项 $ J_h^T \cdot \nabla h $ 对应“$ J_h^T $ 乘以 $ h $ 的雅可比(即 $ J_h $)”,即 $ J_h^T J_h $。
\end{itemize}

因此 Hessian $ \nabla^2\left( \frac{\mu_k}{2}\|h\|^2 \right) = \mu_k\left( J_h^T J_h + \sum_i h_i \nabla^2 h_i \right) $。

当迭代点处于\textbf{可行邻域}($ h(x) \approx 0 $)时,$ \sum_i h_i \nabla^2 h_i \approx 0 $;
而\textbf{Gauss-Newton 近似}本身就是“忽略残差项的二阶导数(即 $ \sum_i h_i \nabla^2 h_i $)”,因此简化后得到 $ \nabla^2 \Phi_{\mu_k(x) }\approx \nabla^2 f(x) + \mu_k J_h^T J_h $,是合理的近似(工程中广泛使用)。

\begin{itemize}
    \item $ \mu_k J_h^T J_h $ 是\textbf{正半定矩阵}:对任意向量 $ d $,有 $ d^T (\mu_k J_h^T J_h) d = \mu_k \|J_h d\|^2 \geq 0 $,符合“正则项需正半定”的要求;
    \item 特征值与方向的关系:
    \begin{itemize}
        \item 沿\textbf{等式法向}($ J_h $ 的行空间,即不可行方向):$ J_h d \neq 0 $,正则项贡献 $ \mu_k \|J_h d\|^2 $,特征值随 $ \mu_k $ 线性增大,实现“强回拉”;
        \item 沿\textbf{切空间}($ J_h d = 0 $,即可行方向):正则项贡献为 0,由原目标函数的 Hessian $ \nabla^2 f $ 主导,不干扰可行方向的搜索。
    \end{itemize}
\end{itemize}

\subsubsection{深入分析:$(g(x)^+)^2$ 只惩罚越界方向}

\paragraph{一维标量情形}
对于一维变量 $t$,定义惩罚函数:
\[
r(t) = (t_+)^2 = 
\begin{cases} 
t^2, & t > 0 \\
0, & t \leq 0 
\end{cases}
\]
其中 $t_+ = \max(0, t)$,表示只取 $t$ 的非负部分(即“越界”部分)。

\paragraph{导数分析(一阶性质)}
该惩罚函数的导数为:
\[
r'(t) = 
\begin{cases} 
0, & t \leq 0 \\
2t, & t > 0 
\end{cases}
\]
\textbf{关键性质}:
在 $t=0$ 处,左导数($t \to 0^-$)为 0,右导数($t \to 0^+$)也为 0,因此 $r(t)$ 在 $t=0$ 处\textbf{可导}(光滑过渡)。

\textbf{直观理解}:
\begin{itemize}
    \item 当 $t \leq 0$(未越界)时,导数为 0,惩罚函数无变化趋势,即“不惩罚”;
    \item 当 $t > 0$(越界)时,导数为 $2t$(随越界程度增大而增大),惩罚函数随 $t$ 增大而快速增长,即“只惩罚越界方向”。
\end{itemize}

\paragraph{推广到多维不等式约束分量 $g_j(x)$}
对于优化问题中的第 $j$ 个不等式约束 $g_j(x) \leq 0$(可行域要求 $g_j(x) \leq 0$,越界即 $g_j(x) > 0$),其惩罚项 $(g_j(x)^+)^2$ 的梯度为:
\[
\nabla (g_j(x)^+)^2 = 2g_j(x)^+ \nabla g_j(x) = 
\begin{cases} 
0, & g_j(x) \leq 0 \\
2g_j(x) \nabla g_j(x), & g_j(x) > 0 
\end{cases}
\]
\textbf{核心结论}:
\begin{itemize}
    \item 当约束满足时($g_j(x) \leq 0$),梯度为 0,惩罚项对优化方向无影响(不惩罚);
    \item 当约束被违反时($g_j(x) > 0$),梯度非零(与 $\nabla g_j(x)$ 同向),推动迭代点向 $g_j(x)$ 减小的方向移动(即向可行域回归,只惩罚越界方向)。
\end{itemize}
\end{remarkinner}
\paragraph{Hessian 矩阵分析(二阶性质)}
惩罚项 $(g_j(x)^+)^2$ 的二阶导数(Hessian 矩阵)为:
\[
\nabla^2 (g_j(x)^+)^2 = 2\nabla g_j(x) \nabla g_j(x)^T + 2g_j(x) \nabla^2 g_j(x)
\]
\textbf{可行边界附近的近似}:
当迭代点接近可行边界($g_j(x) \approx 0^+$,即刚越界时),$g_j(x) \approx 0$,第二项可忽略,因此 Hessian 近似为:
\[
\nabla^2 (g_j(x)^+)^2 \approx 2\nabla g_j(x) \nabla g_j(x)^T
\]
\textbf{意义}:
近似后的 Hessian 是半正定矩阵(外积形式),其“增强方向”与约束的梯度 $\nabla g_j(x)$ 一致(即越界方向)。这意味着在可行边界附近,惩罚项的二阶特性会“抬升”越界方向的曲率,进一步阻止迭代点向越界方向移动,强化对越界方向的惩罚。

\paragraph{总结}
$(g_j(x)^+)^2$ 作为不等式约束的惩罚项,通过一阶导数(梯度)和二阶导数(Hessian)的设计,实现了“只在约束被违反时生效”的特性:
\begin{itemize}
    \item 可行域内($g_j(x) \leq 0$):惩罚项及其导数均为 0,不干扰优化;
    \item 可行域外($g_j(x) > 0$):惩罚项随越界程度增大而增长,梯度和 Hessian 引导迭代向可行域回归,精准惩罚越界方向。
\end{itemize}
这种特性使其成为不等式约束外点法中最常用的惩罚形式(连续可微且惩罚针对性强)。

\subsection{示例:外点法求解}
我们用\textbf{外点法}求解该约束优化问题,步骤如下:

\subsubsection{一、问题定义}
目标函数:$f(x) = x_1^2 + 2x_2^2 - 2x_1 - 2x_2$
约束:
\begin{itemize}
    \item 等式约束:$h(x) = x_1 + x_2 - 1 = 0$
    \item 不等式约束:$g(x) = x_1 - 0.6 \leq 0$
\end{itemize}

\subsubsection{二、外点法惩罚函数构造}
外点法通过\textbf{惩罚项}将约束转化为无约束问题:
\begin{itemize}
    \item 等式约束用二次惩罚:$h(x)^2$
    \item 不等式约束仅惩罚越界部分:$(\max(0, g(x)))^2$
\end{itemize}
因此,惩罚函数为:
\[
P(x, \mu_k) = f(x) + \mu_k \left[ h(x)^2 + (\max(0, x_1 - 0.6))^2 \right]
\]
其中 $\mu_k > 0$ 是惩罚参数,需逐步增大($\mu_k \to +\infty$)。

\subsubsection{三、求解无约束子问题(对固定 $\mu_k$)}
对 $P(x, \mu_k)$ 求偏导并令其为 0,分\textbf{不等式约束越界($x_1 > 0.6$)}和\textbf{不越界($x_1 \leq 0.6$)}两种情况:

\paragraph{情况 1:$x_1 \leq 0.6$(惩罚项中 $\max(0, x_1-0.6)=0$)}
惩罚函数简化为:
\[
P(x, \mu_k) = x_1^2 + 2x_2^2 - 2x_1 - 2x_2 + \mu_k (x_1 + x_2 - 1)^2
\]
求偏导并令其为 0:
\[
\begin{cases}
\frac{\partial P}{\partial x_1} = 2x_1 - 2 + 2\mu_k(x_1 + x_2 - 1) = 0 \\
\frac{\partial P}{\partial x_2} = 4x_2 - 2 + 2\mu_k(x_1 + x_2 - 1) = 0
\end{cases}
\]
两式相减得 $x_1 = 2x_2$,代入后解得:
\[
x_1 = \frac{2(1+\mu_k)}{2+3\mu_k}, \quad x_2 = \frac{1+\mu_k}{2+3\mu_k}
\]
但当 $\mu_k > 0$ 时,$x_1 = \frac{2(1+\mu_k)}{2+3\mu_k} > 0.6$(验证:$2(1+\mu_k) > 0.6(2+3\mu_k)$ 恒成立),因此\textbf{情况 1 不成立}。

\paragraph{情况 2:$x_1 > 0.6$(惩罚项中 $\max(0, x_1-0.6)=x_1-0.6$)}
惩罚函数为:
\[
P(x, \mu_k) = x_1^2 + 2x_2^2 - 2x_1 - 2x_2 + \mu_k \left[ (x_1 + x_2 - 1)^2 + (x_1 - 0.6)^2 \right]
\]
求偏导并令其为 0:
\[
\begin{cases}
\frac{\partial P}{\partial x_1} = 2x_1 - 2 + 2\mu_k(x_1 + x_2 - 1) + 2\mu_k(x_1 - 0.6) = 0 \\
\frac{\partial P}{\partial x_2} = 4x_2 - 2 + 2\mu_k(x_1 + x_2 - 1) = 0
\end{cases}
\]
两式相减消去 $2\mu_k(x_1 + x_2 - 1)$,得:
\[
x_1(1+\mu_k) - 2x_2 = 0.6\mu_k \implies x_2 = \frac{x_1(1+\mu_k) - 0.6\mu_k}{2}
\]
代入偏导方程,最终解得:
\[
x_1 = \frac{0.6\mu_k^2 + 3.2\mu_k + 2}{\mu_k^2 + 5\mu_k + 2}, \quad x_2 = \frac{0.4\mu_k^2 + 2\mu_k + 1}{\mu_k^2 + 5\mu_k + 2}
\]

\subsubsection{四、令 $\mu_k \to +\infty$,求极限解}
当 $\mu_k$ 足够大时,分子分母的高次项主导,因此:
\[
\lim_{\mu_k \to +\infty} x_1 = \frac{0.6\mu_k^2}{\mu_k^2} = 0.6, \quad \lim_{\mu_k \to +\infty} x_2 = \frac{0.4\mu_k^2}{\mu_k^2} = 0.4
\]

\subsubsection{五、验证最优解}
\begin{itemize}
    \item 约束满足:$x_1 + x_2 = 0.6 + 0.4 = 1$(等式约束),$x_1 - 0.6 = 0$(不等式约束)。
    \item 目标函数值:$f(0.6, 0.4) = 0.6^2 + 2 \times 0.4^2 - 2 \times 0.6 - 2 \times 0.4 = -1.32$。
\end{itemize}
\textbf{最终结果}:原问题的最优解为 $\boldsymbol{x}^* = (0.6, 0.4)$,最优目标函数值为 $f(\boldsymbol{x}^*) = -1.32$。

\subsection{外点法的极限满足 KKT}
\subsubsection{1. 核心定理/引理(基于外点法的推导)}
先将惩罚函数中的 $\frac{\rho}{2}$ 替换为 $\mu_k$,对应惩罚函数为:
\[
\Phi_{\mu_k}(x) = f(x) + \mu_k \|h(x)\|^2 + \mu_k \|g(x)^+\|^2 \quad (g(x)^+ \text{是} g \text{的非负部分})
\]

\begin{lemma}[可行性残差必趋零]
外点法的迭代点 $x^{(k)}$ 满足 $h(x^{(k)}) \to 0$、$g(x^{(k)})^+ \to 0$。
即迭代点会逐渐趋近原问题的可行域(约束违反程度趋近于 0)。
\end{lemma}

\begin{lemma}[构造候选乘子与近似平稳]
定义候选乘子 $v^{(k)} = \mu_k h(x^{(k)})$、$\lambda_j^{(k)} = \mu_k g_j(x^{(k)})^+$($\lambda^{(k)} \geq 0$),则迭代点满足“近似平稳条件”:
\[
\nabla f(x^{(k)}) + J_h(x^{(k)})^T v^{(k)} + \sum_{j=1}^p \lambda_j^{(k)} \nabla g_j(x^{(k)}) = r^{(k)}
\]
其中 $r^{(k)} \to 0$(近似误差趋近于 0)。
\end{lemma}

\begin{lemma}[MFCQ 推出乘子有界]
若 MFCQ(约束规范)成立,则候选乘子 $v^{(k)}$、$\lambda^{(k)}$ 是有界的;且非活跃约束($g_j(x^*) < 0$)对应的 $\lambda_j^{(k)} \to 0$。
\end{lemma}

\begin{theorem}[外点法的极限满足 KKT]
外点法的迭代点极限 $x^*$ 是原问题的 KKT 点。
\end{theorem}

\subsubsection{2. 为什么能证明外点法极限满足 KKT?}
通过三个引理逐步“补全” KKT 条件:
\begin{enumerate}
    \item 引理 1 保证\textbf{极限点满足约束}:$h(x^*) = 0$、$g(x^*) \leq 0$;
    \item 引理 2 构造了候选乘子,得到\textbf{近似平稳条件}(接近 KKT 的梯度条件);
    \item 引理 3 通过 MFCQ 让乘子有界,从而乘子存在收敛子列;
    \item 对近似平稳条件取极限($r^{(k)} \to 0$、乘子收敛),最终得到完整的 KKT 条件。
\end{enumerate}

\subsubsection{3. 外点法极限满足的 KKT 条件描述}
外点法的迭代点极限 $x^*$ 满足以下 KKT 条件:
\begin{enumerate}
    \item \textbf{约束条件}:$h(x^*) = 0$,$g(x^*) \leq 0$(处于可行域内);
    \item \textbf{梯度平稳条件}:存在乘子 $v^*$、$\lambda^* \geq 0$,使得:
    \[
    \nabla f(x^*) + J_h(x^*)^T v^* + J_g(x^*)^T \lambda^* = 0
    \]
    ($J_h, J_g$ 是 $h, g$ 的雅可比矩阵);
    \item \textbf{互补松弛条件}:对每个约束 $j$,$\lambda_j^* g_j(x^*) = 0$(非活跃约束的乘子为 0)。
\end{enumerate}

\section{增广拉格朗日(ALM)}
\subsection{从外点法到增广拉格朗日法(ALM)}
外点法的核心缺陷是 \textbf{惩罚参数 $\mu \to +\infty$ 导致的数值病态}。
增广拉格朗日法(Augmented Lagrangian Method, ALM)的核心改进的是:\textbf{引入拉格朗日乘子近似反映约束的“优先级”,将“单纯增大惩罚”改为“乘子调整+适度惩罚”},使得惩罚参数 $\mu$ 无需趋于无穷大即可实现收敛,从根本上解决了外点法的数值病态问题。

\subsection{拉格朗日函数的基础铺垫}
原约束优化问题的\textbf{拉格朗日函数}定义为:
\[
\mathcal{L}(\boldsymbol{x}, \boldsymbol{\lambda}, \boldsymbol{\nu}) = f(\boldsymbol{x}) + \sum_{i=1}^m \lambda_i g_i(\boldsymbol{x}) + \sum_{j=1}^p \nu_j h_j(\boldsymbol{x})
\]
其中:
\begin{itemize}
    \item $\boldsymbol{\lambda} = (\lambda_1, \dots, \lambda_m)^T \geq \boldsymbol{0}$(不等式约束的拉格朗日乘子);
    \item $\boldsymbol{\nu} = (\nu_1, \dots, \nu_p)^T \in \mathbb{R}^p$(等式约束的拉格朗日乘子)。
\end{itemize}
外点法的本质是“忽略乘子,仅用惩罚强制满足约束”,而 ALM 则是“用乘子近似 KKT 条件中的对偶信息,用惩罚修正约束偏差”。

\subsection{增广拉格朗日函数的构造}
ALM 的核心是\textbf{增广拉格朗日函数},其构造逻辑是:在拉格朗日函数基础上,加入与外点法一致的\textbf{二次惩罚项}。

\subsubsection{1. 标准增广拉格朗日函数}
\[
\mathcal{L}_A(\boldsymbol{x}, \boldsymbol{\lambda}, \boldsymbol{\nu}, \mu) = f(\boldsymbol{x}) + \sum_{i=1}^m \left[ \lambda_i g_i(\boldsymbol{x}) + \frac{\mu}{2} \max\left(0, g_i(\boldsymbol{x}) + \frac{\lambda_i}{\mu}\right)^2 \right] + \sum_{j=1}^p \left[ \nu_j h_j(\boldsymbol{x}) + \frac{\mu}{2} h_j(\boldsymbol{x})^2 \right]
\]

\subsubsection{2. 简化形式与物理意义}
通过代数变形,可将不等式约束的惩罚项简化为更直观的形式(利用 $ \max(0, a)^2 = \max(0, a^2 + 2a \cdot 0)^2 $,但核心是保持“违反约束时惩罚生效”):
$$
\max\left(0, g_i(\boldsymbol{x}) + \frac{\lambda_i}{\mu}\right)^2 = \max\left(0, g_i(\boldsymbol{x})\right)^2 +2 \frac{\lambda_i}{\mu} g_i(\boldsymbol{x}) + \frac{\lambda_i^2}{\mu^2} \quad (\text{仅当} \ g_i(\boldsymbol{x}) > 0 \ 时成立)
$$
代入增广拉格朗日函数后,常数项 $ \sum_{i=1}^m \frac{\lambda_i^2}{2\mu} $ 不影响无约束优化的最优解(对 $ \boldsymbol{x} $ 求导时消失),因此可简化为:
$$
\mathcal{L}_A(\boldsymbol{x}, \boldsymbol{\lambda}, \boldsymbol{\nu}, \mu) = f(\boldsymbol{x}) + \mu \left[ \sum_{i=1}^m \frac{1}{2} \max\left(0, g_i(\boldsymbol{x})\right)^2 + \sum_{j=1}^p \frac{1}{2} h_j(\boldsymbol{x})^2 \right] + \sum_{i=1}^m \lambda_i g_i(\boldsymbol{x}) + \sum_{j=1}^p \nu_j h_j(\boldsymbol{x})
$$
\textbf{核心物理意义}:
\begin{itemize}
    \item 当迭代点 $\boldsymbol{x}$ 违反约束时,惩罚项 $\mu \cdot \Phi(\boldsymbol{x})$ 生效,强制迭代点向可行域靠近;
    \item 拉格朗日乘子 $\boldsymbol{\lambda}, \boldsymbol{\nu}$ 随迭代更新,逐步逼近最优乘子 $\boldsymbol{\lambda}^*, \boldsymbol{\nu}^*$,其作用是“引导”惩罚项的方向。
\end{itemize}

\subsection{乘子更新规则的推导(基于 KKT 条件)}

ALM 的关键是\textbf{乘子迭代更新},其更新规则源于无约束子问题的最优性条件和 KKT 条件的一致性。

\subsubsection{1. 无约束子问题的最优性条件}
对于固定的 $\boldsymbol{\lambda}_k, \boldsymbol{\nu}_k, \mu_k$,求解无约束子问题:
\[
\boldsymbol{x}_{k+1} = \arg\min_{\boldsymbol{x}} \mathcal{L}_A(\boldsymbol{x}, \boldsymbol{\lambda}_k, \boldsymbol{\nu}_k, \mu_k)
\]
其最优性条件为梯度为零:
\[
\nabla_{\boldsymbol{x}} \mathcal{L}_A(\boldsymbol{x}_{k+1}, \boldsymbol{\lambda}_k, \boldsymbol{\nu}_k, \mu_k) = \boldsymbol{0}
\]

\subsubsection{2. 对不等式约束乘子 $\lambda_i$ 的更新}
展开梯度条件中关于 $g_i(\boldsymbol{x})$ 的项:
\[
\nabla f(\boldsymbol{x}_{k+1}) + \sum_{i=1}^m \left[ \lambda_{k,i} + \mu_k \cdot \max\left(0, g_i(\boldsymbol{x}_{k+1})\right) \right] \nabla g_i(\boldsymbol{x}_{k+1}) + \sum_{j=1}^p \left[ \nu_{k,j} + \mu_k h_j(\boldsymbol{x}_{k+1}) \right] \nabla h_j(\boldsymbol{x}_{k+1}) = \boldsymbol{0}
\]

根据 KKT 条件,原问题最优解 $\boldsymbol{x}^*$ 满足:
\[
\nabla f(\boldsymbol{x}^*) + \sum_{i=1}^m \lambda_i^* \nabla g_i(\boldsymbol{x}^*) + \sum_{j=1}^p \nu_j^* \nabla h_j(\boldsymbol{x}^*) = \boldsymbol{0}
\]
且互补松弛条件 $\lambda_i^* g_i(\boldsymbol{x}^*) = 0$(可行点 $g_i(\boldsymbol{x}^*) \leq 0$,故 $\lambda_i^* > 0 \implies g_i(\boldsymbol{x}^*) = 0$)。

对比两式,为使 $\boldsymbol{\lambda}_k$ 逐步逼近 $\lambda_i^*$,自然得到\textbf{乘子更新规则}:
\[
\lambda_{k+1,i} = \max\left( 0, \lambda_{k,i} + \mu_k \cdot g_i(\boldsymbol{x}_{k+1}) \right)
\]
\begin{itemize}
    \item 当 $\boldsymbol{x}_{k+1}$ 满足约束($g_i(\boldsymbol{x}_{k+1}) \leq 0$)且 $\lambda_{k,i} + \mu_k g_i(\boldsymbol{x}_{k+1}) \geq 0$ 时,$\lambda_{k+1,i} = \lambda_{k,i} + \mu_k g_i(\boldsymbol{x}_{k+1})$,逐步逼近最优乘子;
    \item 当 $\lambda_{k,i} + \mu_k g_i(\boldsymbol{x}_{k+1}) < 0$ 时,$\lambda_{k+1,i} = 0$,满足 KKT 条件中 $\lambda_i \geq 0$ 的要求。
\end{itemize}

\subsubsection{3. 对等式约束乘子 $\nu_j$ 的更新}
同理,展开梯度条件中关于 $h_j(\boldsymbol{x})$ 的项,由于等式约束无互补松弛的非负性要求,直接得到:
\[
\nu_{k+1,j} = \nu_{k,j} + \mu_k \cdot h_j(\boldsymbol{x}_{k+1})
\]

\subsection{ALM 算法流程}
\begin{algorithm}[ALM 算法流程]
\textbf{输入}:
\begin{itemize}
    \item 原问题目标函数 $f(\boldsymbol{x})$、约束 $g_i(\boldsymbol{x})$、$h_j(\boldsymbol{x})$;
    \item 初始参数:初始点 $\boldsymbol{x}_0$;初始乘子 $\boldsymbol{\lambda}_0 = \boldsymbol{0}$、$\boldsymbol{\nu}_0 = \boldsymbol{0}$;初始惩罚参数 $\mu_1 > 0$(无需过大,通常取 1 或 10);惩罚参数增长因子 $\beta > 1$;收敛精度 $\epsilon > 0$。
\end{itemize}

\textbf{迭代步骤}:
\begin{enumerate}
    \item \textbf{初始化}:令迭代次数 $k = 1$;
    \item \textbf{构造增广拉格朗日函数}:针对当前 $\boldsymbol{\lambda}_{k-1}, \boldsymbol{\nu}_{k-1}, \mu_k$,构造 $\mathcal{L}_A(\boldsymbol{x}, \boldsymbol{\lambda}_{k-1}, \boldsymbol{\nu}_{k-1}, \mu_k)$;
    \item \textbf{求解无约束子问题}:以 $\boldsymbol{x}_{k-1}$ 为初始点,求解 $\min_{\boldsymbol{x}} \mathcal{L}_A$,得到最优解 $\boldsymbol{x}_k$;
    \item \textbf{更新拉格朗日乘子}:
    \begin{itemize}
        \item 不等式约束乘子:$\lambda_{k,i} = \max\left( 0, \lambda_{k-1,i} + \mu_k \cdot g_i(\boldsymbol{x}_k) \right)$;
        \item 等式约束乘子:$\nu_{k,j} = \nu_{k-1,j} + \mu_k \cdot h_j(\boldsymbol{x}_k)$;
    \end{itemize}
    \item \textbf{收敛判断}:若满足以下所有收敛条件,停止迭代,输出 $\boldsymbol{x}_k \approx \boldsymbol{x}^*$:
    \begin{itemize}
        \item 约束违反度量足够小:$\Phi(\boldsymbol{x}_k) < \epsilon$;
        \item 乘子变化足够小:$\| \boldsymbol{\lambda}_k - \boldsymbol{\lambda}_{k-1} \| < \epsilon$ 且 $\| \boldsymbol{\nu}_k - \boldsymbol{\nu}_{k-1} \| < \epsilon$;
        \item 迭代点变化足够小:$\| \boldsymbol{x}_k - \boldsymbol{x}_{k-1} \| < \epsilon$;
    \end{itemize}
    \item \textbf{更新惩罚参数}:若未收敛,令 $\mu_{k+1} = \beta \cdot \mu_k$(或保持 $\mu_k$ 不变),$k = k + 1$,返回步骤 2。
\end{enumerate}

\textbf{输出}:
原约束问题的近似最优解 $\boldsymbol{x}_k$ 及对应的最优乘子 $\boldsymbol{\lambda}_k, \boldsymbol{\nu}_k$。
\end{algorithm}

\subsection{ALM 与外点法的核心差异(严谨对比)}
\begin{center}
\begin{tabular}{|p{0.15\textwidth}|p{0.35\textwidth}|p{0.35\textwidth}|}
\hline
\textbf{对比维度} & \textbf{外点法 (Penalty Function Method)} & \textbf{增广拉格朗日法 (ALM)} \\
\hline
核心函数 & 惩罚函数 $P(\boldsymbol{x}, \mu) = f(\boldsymbol{x}) + \mu \Phi(\boldsymbol{x})$ & 增广拉格朗日函数 $\mathcal{L}_A(\boldsymbol{x}, \boldsymbol{\lambda}, \boldsymbol{\nu}, \mu)$ \\
\hline
关键参数 & 仅惩罚参数 $\mu$(需 $\mu \to +\infty$) & 惩罚参数 $\mu$(可固定或适度增大)+ 拉格朗日乘子 $\boldsymbol{\lambda}, \boldsymbol{\nu}$ \\
\hline
数值稳定性 & 差($\mu$ 过大导致 Hessian 条件数恶化) & 好(乘子引导惩罚,$\mu$ 无需无穷大) \\
\hline
收敛速度 & 慢(依赖 $\mu$ 逐步增大,迭代后期收敛迟缓) & 快(乘子自适应调整约束权重,迭代前期即可快速靠近最优解) \\
\hline
最优性条件满足 & 仅满足原始可行性($\boldsymbol{x}_k \to \Omega$),对偶信息无保障 & 同时逼近原始最优和对偶最优($\boldsymbol{\lambda}_k \to \boldsymbol{\lambda}^*$,$\boldsymbol{\nu}_k \to \boldsymbol{\nu}^*$) \\
\hline
\end{tabular}
\end{center}

\subsection{收敛性核心结论}
假设原问题满足:$f(\boldsymbol{x})$ 凸、$g_i(\boldsymbol{x})$ 凸、$h_j(\boldsymbol{x})$ 仿射,且 Slater 条件成立(存在严格可行点),则 ALM 的迭代序列 $\{\boldsymbol{x}_k, \boldsymbol{\lambda}_k, \boldsymbol{\nu}_k\}$ 满足:
\begin{enumerate}
    \item $\boldsymbol{x}_k$ 强收敛到原问题的最优解 $\boldsymbol{x}^*$;
    \item $\boldsymbol{\lambda}_k$ 强收敛到最优拉格朗日乘子 $\boldsymbol{\lambda}^*$,$\boldsymbol{\nu}_k$ 强收敛到 $\boldsymbol{\nu}^*$;
    \item 惩罚参数 $\mu$ 可固定为某个常数(无需增大),仍能保证收敛(这是 ALM 相对于外点法的本质优势)。
\end{enumerate}
若原问题非凸,在适当的约束品性下,ALM 仍能保证迭代序列的聚点是原问题的 KKT 点。

\subsection{示例:ALM 应用}
\[
\begin{cases}
\min_{x_1,x_2} & f(x) = x_1^2 + 2x_2^2 - 2x_1 - 2x_2 \\
\text{s.t.} & h(x) = x_1 + x_2 - 1 = 0 \quad (\text{等式约束}) \\
& g(x) = x_1 - 0.6 \leq 0 \quad (\text{不等式约束})
\end{cases}
\]
\begin{itemize}
    \item 初始点:$\boldsymbol{x}_0 = (0, 0)$
    \item 初始乘子:$\lambda_0 = 0$,$\nu_0 = 0$
    \item 初始惩罚参数:$\mu_1 = 1$
    \item 惩罚参数增长因子:$\beta = 2$
\end{itemize}

\subsubsection{第一轮迭代 ($k=1$)}
\paragraph{步骤 1:构造增广拉格朗日函数}
代入 $\lambda_0=0,\ \nu_0=0,\ \mu_1=1$,简化得:
\[
\mathcal{L}_A(\boldsymbol{x}) = x_1^2+2x_2^2-2x_1-2x_2 + \frac{1}{2}\max\left(0, x_1-0.6\right)^2 + \frac{1}{2}(x_1+x_2-1)^2
\]

\paragraph{步骤 2:求解无约束子问题}
对 $x_1, x_2$ 求偏导并令其为 0。若 $x_1 > 0.6$,偏导为:
\[
\begin{cases}
\frac{\partial \mathcal{L}_A}{\partial x_1} = 4x_1 + x_2 - 3.6 = 0 \\
\frac{\partial \mathcal{L}_A}{\partial x_2} = x_1 + 5x_2 - 3 = 0
\end{cases}
\]
解方程组得:
\[
x_1 \approx 0.7895,\ x_2 \approx 0.4421 \quad (\text{满足} \ x_1>0.6)
\]
即第一轮迭代点:$\boldsymbol{x}_1 = (0.7895, 0.4421)$。

\paragraph{步骤 3:更新乘子}
\begin{itemize}
    \item 不等式约束乘子:$\lambda_1 = \max(0, 0 + 1 \cdot (0.7895-0.6)) = 0.1895$
    \item 等式约束乘子:$\nu_1 = 0 + 1 \cdot (0.7895+0.4421-1) = 0.2316$
\end{itemize}

\subsubsection{第二轮迭代 ($k=2$)}
\paragraph{步骤 1:更新惩罚参数 \& 构造增广拉格朗日函数}
$\mu_2 = 2 \times 1 = 2$。代入 $\lambda_1=0.1895,\ \nu_1=0.2316,\ \mu_2=2$,增广拉格朗日函数为:
\[
\mathcal{L}_A(\boldsymbol{x}) = x_1^2+2x_2^2-2x_1-2x_2 + 0.1895(x_1-0.6) + \max\left(0, x_1-0.50525\right)^2 + 0.2316(x_1+x_2-1) + (x_1+x_2-1)^2
\]

\paragraph{步骤 2:求解无约束子问题}
对 $x_1, x_2$ 求偏导并令其为 0,解得:
\[
x_1 \approx 0.6251,\ x_2 \approx 0.4197 \quad (\text{接近} \ x_1=0.6 \text{的约束边界})
\]
即第二轮迭代点:$\boldsymbol{x}_2 = (0.6251, 0.4197)$。

\paragraph{步骤 3:更新乘子}
\begin{itemize}
    \item 不等式约束乘子:$\lambda_2 = \max(0, 0.1895 + 2 \cdot (0.6251-0.6)) = 0.2397$
    \item 等式约束乘子:$\nu_2 = 0.2316 + 2 \cdot (0.6251+0.4197-1) = 0.3212$
\end{itemize}

\subsubsection{迭代结果总结}
\begin{center}
\begin{tabular}{|c|c|c|c|c|}
\hline
迭代轮次 & 迭代点 $\boldsymbol{x}_k$ & 不等式乘子 $\lambda_k$ & 等式乘子 $\nu_k$ & 惩罚参数 $\mu_k$ \\
\hline
初始 & $(0, 0)$ & $0$ & $0$ & $1$ \\
\hline
1 & $(0.7895, 0.4421)$ & $0.1895$ & $0.2316$ & $1$ \\
\hline
2 & $(0.6251, 0.4197)$ & $0.2397$ & $0.3212$ & $2$ \\
\hline
\end{tabular}
\end{center}
可以看到:$\boldsymbol{x}_2$ 已靠近不等式约束边界 $x_1=0.6$,等式约束偏差 $h(\boldsymbol{x}_2) \approx 0.0448$ 也显著减小,说明迭代在向可行域收敛。

\section{约束问题的解法之 ADMM}
\subsection{第一步:先锁定 ALM 的“可分问题特例”(ADMM 的适用场景)}
我们从 ALM 能处理的一般约束问题中,挑一个\textbf{目标可分+线性等式约束}的特例(这正是 ADMM 专门解决的问题):
\[
\begin{cases}
\min_{\boldsymbol{x}, \boldsymbol{z}} & \underbrace{f(\boldsymbol{x})}_{\text{仅依赖}\boldsymbol{x}} + \underbrace{g(\boldsymbol{z})}_{\text{仅依赖}\boldsymbol{z}} \quad \text{(目标可分)} \\
\text{s.t.} & \boldsymbol{A}\boldsymbol{x} + \boldsymbol{B}\boldsymbol{z} = \boldsymbol{c} \quad \text{(线性等式约束)}
\end{cases}
\]

(注:ADMM 只处理这种“目标拆成两个独立部分+线性等式约束”的问题,这是它和 ALM 的第一个区别——ALM 处理更一般的约束)

\subsection{第二步:写出这个特例的 ALM 增广拉格朗日函数}
\[
\mathcal{L}_A^{\text{ALM}}(\boldsymbol{x}, \boldsymbol{z}, \boldsymbol{\nu}, \rho) = f(\boldsymbol{x}) + g(\boldsymbol{z}) + \boldsymbol{\nu}^T(\boldsymbol{A}\boldsymbol{x} + \boldsymbol{B}\boldsymbol{z} - \boldsymbol{c}) + \frac{\rho}{2}\|\boldsymbol{A}\boldsymbol{x} + \boldsymbol{B}\boldsymbol{z} - \boldsymbol{c}\|_2^2
\]
其中:
\begin{itemize}
    \item $\boldsymbol{\nu}$ 是等式约束的拉格朗日乘子;
    \item $\rho > 0$ 是惩罚参数(和 ALM 的 $\mu$ 是同一个东西)。
\end{itemize}

\subsection{第三步:ALM 对这个问题的迭代步骤}
ALM 处理这个问题时,迭代是“\textbf{同时最小化 $\boldsymbol{x},\boldsymbol{z}$,再更新乘子 $\boldsymbol{\nu}$}”:
\begin{enumerate}
    \item 最小化增广拉格朗日:$(\boldsymbol{x}^{k+1}, \boldsymbol{z}^{k+1}) = \arg\min_{\boldsymbol{x},\boldsymbol{z}} \mathcal{L}_A^{\text{ALM}}(\boldsymbol{x}, \boldsymbol{z}, \boldsymbol{\nu}^k, \rho)$
    \item 更新乘子:$\boldsymbol{\nu}^{k+1} = \boldsymbol{\nu}^k + \rho(\boldsymbol{A}\boldsymbol{x}^{k+1} + \boldsymbol{B}\boldsymbol{z}^{k+1} - \boldsymbol{c})$
\end{enumerate}

\subsection{第四步:关键改进——利用“目标可分”拆分最小化步骤}
问题来了:\textbf{同时最小化 $\boldsymbol{x},\boldsymbol{z}$ 可能很难}(比如 $\boldsymbol{x},\boldsymbol{z}$ 维度很大时)。

但我们的目标是“可分的”:$f(\boldsymbol{x})$ 只和 $\boldsymbol{x}$ 有关,$g(\boldsymbol{z})$ 只和 $\boldsymbol{z}$ 有关。所以可以\textbf{交替最小化 $\boldsymbol{x}$ 和 $\boldsymbol{z}$}(先固定 $\boldsymbol{z}$ 求 $\boldsymbol{x}$,再固定 $\boldsymbol{x}$ 求 $\boldsymbol{z}$)——这就是 ADMM 的核心!

\subsubsection{拆分 1:固定 $\boldsymbol{z}^k$,先求 $\boldsymbol{x}^{k+1}$(ADMM 的 $\boldsymbol{x}$-步)}
把 $\boldsymbol{z}=\boldsymbol{z}^k$ 代入增广拉格朗日,此时只有 $\boldsymbol{x}$ 是变量:
\[
\boldsymbol{x}^{k+1} = \arg\min_{\boldsymbol{x}} \left[ f(\boldsymbol{x}) + \boldsymbol{\nu}^{kT}(\boldsymbol{A}\boldsymbol{x} + \boldsymbol{B}\boldsymbol{z}^k - \boldsymbol{c}) + \frac{\rho}{2}\|\boldsymbol{A}\boldsymbol{x} + \boldsymbol{B}\boldsymbol{z}^k - \boldsymbol{c}\|_2^2 \right]
\]

\subsubsection{拆分 2:固定 $\boldsymbol{x}^{k+1}$,再求 $\boldsymbol{z}^{k+1}$(ADMM 的 $\boldsymbol{z}$-步)}
把 $\boldsymbol{x}=\boldsymbol{x}^{k+1}$ 代入增广拉格朗日,此时只有 $\boldsymbol{z}$ 是变量:
\[
\boldsymbol{z}^{k+1} = \arg\min_{\boldsymbol{z}} \left[ g(\boldsymbol{z}) + \boldsymbol{\nu}^{kT}(\boldsymbol{A}\boldsymbol{x}^{k+1} + \boldsymbol{B}\boldsymbol{z} - \boldsymbol{c}) + \frac{\rho}{2}\|\boldsymbol{A}\boldsymbol{x}^{k+1} + \boldsymbol{B}\boldsymbol{z} - \boldsymbol{c}\|_2^2 \right]
\]

\subsection{第五步:简化符号——引入 ADMM 的“对偶残差 \texorpdfstring{$\boldsymbol{u}$}{u}”}
为了让式子更简洁,ADMM 把 ALM 的乘子 $\boldsymbol{\nu}$ 缩放一下:令 $\boldsymbol{u} = \frac{\boldsymbol{\nu}}{\rho}$($\boldsymbol{u}$ 就是 ADMM 里的“对偶变量”)。

把 $\boldsymbol{\nu} = \rho\boldsymbol{u}$ 代入上面的式子,就能消掉 $\boldsymbol{\nu}$ 的线性项,最终得到 ADMM 的标准步骤:

\subsection{最终:从 ALM 拆分得到的 ADMM 迭代(一一对应)}
\begin{center}
\begin{tabular}{|p{0.45\textwidth}|p{0.45\textwidth}|}
\hline
\textbf{ALM 的步骤(可分特例)} & \textbf{对应 ADMM 的步骤(符号简化后)} \\
\hline
1. 固定 $\boldsymbol{z}^k$,求 $\boldsymbol{x}^{k+1}$ & 1. $\boldsymbol{x}$-步:$\boldsymbol{x}^{k+1} = \arg\min_{\boldsymbol{x}} \left[ f(\boldsymbol{x}) + \frac{\rho}{2}\|\boldsymbol{A}\boldsymbol{x} + \boldsymbol{B}\boldsymbol{z}^k - \boldsymbol{c} + \boldsymbol{u}^k\|_2^2 \right]$ \\
\hline
2. 固定 $\boldsymbol{x}^{k+1}$,求 $\boldsymbol{z}^{k+1}$ & 2. $\boldsymbol{z}$-步:$\boldsymbol{z}^{k+1} = \arg\min_{\boldsymbol{z}} \left[ g(\boldsymbol{z}) + \frac{\rho}{2}\|\boldsymbol{A}\boldsymbol{x}^{k+1} + \boldsymbol{B}\boldsymbol{z} - \boldsymbol{c} + \boldsymbol{u}^k\|_2^2 \right]$ \\
\hline
3. 更新乘子 $\boldsymbol{\nu}^{k+1}$ & 3. $\boldsymbol{u}$-步(对应乘子更新):$\boldsymbol{u}^{k+1} = \boldsymbol{u}^k + \boldsymbol{A}\boldsymbol{x}^{k+1} + \boldsymbol{B}\boldsymbol{z}^{k+1} - \boldsymbol{c}$ \\
\hline
\end{tabular}
\end{center}

ADMM 就是\textbf{“可分目标+线性等式约束”问题下的 ALM}——它把 ALM“同时最小化所有变量”的步骤,拆成了“交替最小化可分变量块”的简单步骤,同时缩放了乘子符号让式子更简洁。

\subsection{示例(与前述问题关联的改写)}
要对这个问题用 ADMM 迭代,首先得把它转化为 ADMM 的\textbf{标准形式(可分目标+线性等式约束)}。

\subsubsection{步骤 1:转化为 ADMM 标准形式}
原问题:
\[
\begin{cases}
\min_{x_1,x_2} & f(x_1,x_2) = x_1^2 + 2x_2^2 - 2x_1 - 2x_2 \\
\text{s.t.} & x_1 + x_2 = 1 \quad (\text{等式约束}) \\
& x_1 \leq 0.6 \quad (\text{不等式约束})
\end{cases}
\]
\textbf{拆分变量+转化约束}:
\begin{itemize}
    \item 令\textbf{变量块 1}:$x = x_2$(仅依赖 $x_2$),目标项为 $f(x) = 2x^2 - 2x$;
    \item 令\textbf{变量块 2}:$z = x_1$(仅依赖 $x_1$),目标项为 $g(z) = z^2 - 2z + I(z \leq 0.6)$($I(\cdot)$ 是指示函数);
    \item 约束转化为线性等式:$z + x = 1$(对应 ADMM 的标准约束 $\boldsymbol{A}z + \boldsymbol{B}x = c$,这里 $\boldsymbol{A}=\boldsymbol{B}=1, c=1$)。
\end{itemize}

\subsubsection{步骤 2:ADMM 的增广拉格朗日函数}
\[
\mathcal{L}_A(x, z, u, \rho) = f(x) + g(z) + \frac{\rho}{2}\left(z + x - 1 + u\right)^2 - \frac{\rho}{2}u^2
\]
其中 $u$ 是对偶变量,$\rho=1$。

\subsubsection{步骤 3:ADMM 迭代流程(3 步循环)}
初始化:$x^0 = 0$,$z^0 = 0$,$u^0 = 0$。

\paragraph{第 1 轮迭代 ($k=0 \to k=1$)}
\begin{enumerate}
    \item \textbf{x-步}:固定 $z^0=0, u^0=0$,最小化 $\mathcal{L}_A$ 关于 $x$:
    \[
    \min_x \ 2x^2 - 2x + \frac{1}{2}\left(0 + x - 1 + 0\right)^2
    \]
    求导并令其为 0:$4x - 2 + (x - 1) = 5x - 3 = 0 \implies x^1 = 0.6$。

    \item \textbf{z-步}:固定 $x^1=0.6, u^0=0$,最小化 $\mathcal{L}_A$ 关于 $z$:
    \[
    \min_z \ \underbrace{z^2 - 2z + I(z \leq 0.6)}_{\text{含约束}} + \frac{1}{2}\left(z + 0.6 - 1 + 0\right)^2
    \]
    无约束解:$2z - 2 + (z - 0.4) = 3z - 2.4 = 0 \implies z=0.8$(违反约束);
    约束下最优解:$z^1 = 0.6$(约束边界)。

    \item \textbf{u-步}:更新对偶变量:
    \[
    u^1 = u^0 + z^1 + x^1 - 1 = 0 + 0.6 + 0.6 - 1 = 0.2
    \]
\end{enumerate}

\paragraph{第 2 轮迭代 ($k=1 \to k=2$)}
\begin{enumerate}
    \item \textbf{x-步}:固定 $z^1=0.6, u^1=0.2$:
    \[
    \min_x \ 2x^2 - 2x + \frac{1}{2}\left(0.6 + x - 1 + 0.2\right)^2
    \]
    求导得:$4x - 2 + (x - 0.2) = 5x - 2.2 = 0 \implies x^2 = 0.44$。

    \item \textbf{z-步}:固定 $x^2=0.44, u^1=0.2$:
    \[
    \min_z \ z^2 - 2z + I(z \leq 0.6) + \frac{1}{2}\left(z + 0.44 - 1 + 0.2\right)^2
    \]
    无约束解仍违反 $z \leq 0.6$,故 $z^2 = 0.6$。

    \item \textbf{u-步}:
    \[
    u^2 = 0.2 + 0.6 + 0.44 - 1 = 0.24
    \]
\end{enumerate}

\paragraph{第 3 轮迭代 ($k=2 \to k=3$)}
\begin{enumerate}
    \item \textbf{x-步}:固定 $z^2=0.6, u^2=0.24$:
    \[
    \min_x \ 2x^2 - 2x + \frac{1}{2}\left(0.6 + x - 1 + 0.24\right)^2
    \]
    求导得:$5x - 2.16 = 0 \implies x^3 = 0.432$。

    \item \textbf{z-步}:约束下仍取 $z^3 = 0.6$。

    \item \textbf{u-步}:
    \[
    u^3 = 0.24 + 0.6 + 0.432 - 1 = 0.272
    \]
\end{enumerate}

\subsubsection{迭代趋势}
\begin{center}
\begin{tabular}{|c|c|c|c|c|}
\hline
迭代轮次 & $x_k=x_2$ & $z_k=x_1$ & 对偶变量 $u_k$ & 约束满足度 $z_k+x_k-1$ \\
\hline
0 & 0 & 0 & 0 & -1 \\
\hline
1 & 0.6 & 0.6 & 0.2 & 0.2 \\
\hline
2 & 0.44 & 0.6 & 0.24 & 0.04 \\
\hline
3 & 0.432 & 0.6 & 0.272 & 0.032 \\
\hline
\end{tabular}
\end{center}
可以看到:$z_k=x_1$ 稳定在约束边界 $0.6$,$x_k=x_2$ 逐渐靠近最优值 $0.4$(原问题最优解为 $x_1=0.6, x_2=0.4$)。



\end{document}
