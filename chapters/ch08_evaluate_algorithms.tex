\chapter{优化算法的评价}

\section{预备知识与符号}

\begin{itemize}
    \item \textbf{目标函数}:$f: \mathbb{R}^n \to \mathbb{R}$,可微。
    \item \textbf{L-光滑(Lipschitz梯度)}:存在$L > 0$使
    \[
    \|\nabla f(x) - \nabla f(y)\| \leq L\|x - y\|, \quad \forall x, y.
    \]
    \item \textbf{凸/强凸}:$f$凸;若存在$\mu > 0$使
    \[
    f(y) \geq f(x) + \nabla f(x)^\top (y - x) + \frac{\mu}{2}\|y - x\|^2,
    \]
    则$f$为$\mu$-强凸。
    \item \textbf{PL(Polyak-Łojasiewicz)不等式}:存在$\mu > 0$使
    \[
    \frac{1}{2}\|\nabla f(x)\|^2 \geq \mu\left(f(x) - f^*\right), \quad \forall x.
    \]
    \textit{注}:PL $\nRightarrow$ 凸,但常见于过参数模型/深网络的局部区域。
    \item \textbf{GD更新}:$x_{k+1} = x_k - \eta_k \nabla f(x_k)$,其中步长$\eta_k > 0$。
\end{itemize}

\section{基本工具:下降引理(Descent Lemma)}

\begin{lemma}[下降引理]
若函数$f$为$L$-光滑,则对任意$x, d$与$\eta > 0$,有:
\[
f(x + \eta d) \leq f(x) + \eta \nabla f(x)^\top d + \frac{L}{2}\eta^2 \|d\|^2
\]
\end{lemma}

\begin{proof}
由$L$-光滑性的定义,对任意$y$有:
\[
f(y) \leq f(x) + \nabla f(x)^\top (y - x) + \frac{L}{2}\|y - x\|^2
\]
令$y = x + \eta d$(即沿方向$d$步长$\eta$移动),代入后即得下降引理的不等式。
\end{proof}

\begin{corollary}[沿负梯度下降]
取$d = -\nabla f(x)$(沿\textbf{负梯度方向}更新,梯度下降算法的核心思想),代入下降引理得:
\[
f(x - \eta \nabla f(x)) \leq f(x) - \left( \eta - \frac{L}{2}\eta^2 \right) \|\nabla f(x)\|^2
\]

进一步,当$0 < \eta \leq \frac{1}{L}$时,函数值\textbf{单调下降},且有:
\[
f(x_{k+1}) \leq f(x_k) - \frac{\eta}{2} \|\nabla f(x_k)\|^2
\]

\begin{itemize}
    \item 该推论明确了\textbf{梯度下降的下降性保证}:只要步长$\eta$满足$0 < \eta \leq \frac{1}{L}$,每一步迭代后目标函数值必严格减小,确保算法的"下降"特性。
    \item 为\textbf{步长选择}提供了理论依据(步长不超过$\frac{1}{L}$时算法稳定下降),也为后续收敛速率分析奠定了基础。
\end{itemize}
\end{corollary}

\section{基于Descent Lemma的四类典型收敛结果}

\subsection{非凸 + L-光滑}

由推论2.2,当$0 < \eta \leq \frac{1}{L}$时,梯度下降的单步迭代满足:
\[
f(x_{k+1}) \leq f(x_k) - \frac{\eta}{2} \|\nabla f(x_k)\|^2
\]

将$k = 0, 1, \dots, K-1$的不等式依次展开并累加:
\begin{align*}
f(x_1) &\leq f(x_0) - \frac{\eta}{2} \|\nabla f(x_0)\|^2 \\
f(x_2) &\leq f(x_1) - \frac{\eta}{2} \|\nabla f(x_1)\|^2 \\
&\vdots \\
f(x_K) &\leq f(x_{K-1}) - \frac{\eta}{2} \|\nabla f(x_{K-1})\|^2
\end{align*}
将这些不等式左右两边分别相加,\textbf{中间项$f(x_1), f(x_2), \dots, f(x_{K-1})$会相互抵消},最终得到:
\[
f(x_K) \leq f(x_0) - \frac{\eta}{2} \sum_{k=0}^{K-1} \|\nabla f(x_k)\|^2
\]

已知$f$下有界(即$f_{\text{inf}} = \inf_x f(x) > -\infty$),因此对任意$K$,有$f(x_K) \geq f_{\text{inf}}$。将其代入上式:
\[
f_{\text{inf}} \leq f(x_0) - \frac{\eta}{2} \sum_{k=0}^{K-1} \|\nabla f(x_k)\|^2
\]
整理得梯度范数平方和的上界:
\[
\sum_{k=0}^{K-1} \|\nabla f(x_k)\|^2 \leq \frac{2\left(f(x_0) - f_{\text{inf}}\right)}{\eta}
\]

将上式两边同时除以$K$,得到\textbf{平均梯度范数平方}的上界:
\[
\frac{1}{K} \sum_{k=0}^{K-1} \|\nabla f(x_k)\|^2 \leq \frac{2\left(f(x_0) - f_{\text{inf}}\right)}{\eta K}
\]

进一步分析\textbf{最小梯度范数}的量级:
由于平均值不小于"集合中的最小值"(即$\frac{1}{K} \sum_{k=0}^{K-1} \|\nabla f(x_k)\|^2 \geq \min_{0 \leq k < K} \|\nabla f(x_k)\|^2$),因此:
\[
\min_{0 \leq k < K} \|\nabla f(x_k)\|^2 \leq \frac{2\left(f(x_0) - f_{\text{inf}}\right)}{\eta K} = O\left(\frac{1}{K}\right)
\]

该定理在\textbf{非凸深度学习场景}中具有关键价值:
\begin{enumerate}
    \item 即使损失函数非凸,只要满足L-光滑且下有界,梯度下降的\textbf{平均梯度范数会随迭代次数增加而趋于0},说明算法能逐步逼近"驻点"(最优解的必要条件)。
    \item 收敛速率为$O(1/K)$,明确了"迭代次数越多,平均梯度越小"的量化规律,为训练过程的收敛性分析提供了理论依据。
    \item 解释了"为什么深度网络在梯度下降训练中能逐步收敛"——即使损失函数非凸,L-光滑性和下有界性确保了梯度的整体衰减趋势。
\end{enumerate}

\subsection{凸 + L-光滑}

对\textbf{"凸且L-光滑函数"的梯度下降(GD)算法进行收敛性分析},核心是推导"函数值次优量"与"解的平方距离差分"的关联不等式(即\textbf{望远镜技巧}的核心步骤)

\begin{itemize}
    \item \textbf{函数设定}:$f: \mathbb{R}^n \to \mathbb{R}$是\textbf{凸且L-光滑}的(即满足$\|\nabla f(y) - \nabla f(x)\| \leq L\|y - x\|$),最优解为$x^* \in \arg\min f$。
    \item \textbf{算法设定}:梯度下降取\textbf{固定步长}$\eta = \frac{1}{L}$,迭代规则为$x_{k+1} = x_k - \frac{1}{L}\nabla f(x_k)$。
    \item \textbf{基础工具}:
    \begin{itemize}
        \item 凸函数的一阶性质(3.2.a):$f(y) \geq f(x) + \langle \nabla f(x), y - x \rangle$(凸函数的定义式,用于将"梯度内积"转化为"函数值差距")。
        \item 向量范数展开(3.2.c):$\|a - c\|^2 = \|b - c\|^2 + 2\langle a - b, b - c \rangle + \|a - b\|^2$(用于对"解的距离"$\|x_{k+1} - x^*\|^2$做代数展开)。
    \end{itemize}
\end{itemize}

\subsubsection{步骤1:解的距离展开(3.2.d)}
令$a = x_{k+1}$、$b = x_k$、$c = x^*$,结合梯度下降的更新式$x_{k+1} - x_k = -\frac{1}{L}\nabla f(x_k)$,代入范数展开式(3.2.c)得:
\[
\|x_{k+1} - x^*\|^2 = \|x_k - x^*\|^2 - \frac{2}{L}\langle \nabla f(x_k), x_k - x^* \rangle + \frac{1}{L^2}\|\nabla f(x_k)\|^2
\]

\subsubsection{步骤2:结合凸性,转化梯度内积为函数值差距}
由凸函数的性质(3.2.b):$f(x) - f^* \leq \langle \nabla f(x), x - x^* \rangle$,代入(3.2.d)得:
\[
\|x_{k+1} - x^*\|^2 \leq \|x_k - x^*\|^2 - \frac{2}{L}(f(x_k) - f^*) + \frac{1}{L^2}\|\nabla f(x_k)\|^2
\]

\subsubsection{步骤3:消去梯度范数,关联函数值下降量}
利用L-光滑函数的"下降性质",推导得:
\[
\frac{1}{L^2}\|\nabla f(x_k)\|^2 \leq \frac{2}{L}(f(x_k) - f(x_{k+1}))
\]

\subsubsection{步骤4:关键不等式(望远镜差分,3.2.g)}
将(3.2.f)代回(3.2.e)并整理,最终得到:
\[
\frac{2}{L}(f(x_{k+1}) - f^*) \leq \|x_k - x^*\|^2 - \|x_{k+1} - x^*\|^2
\]
\textbf{意义}:这是"望远镜技巧"的核心——它将"函数值到最优的差距"与"解的距离的差分"直接关联。后续对$k$累加该不等式,可消去中间项,从而推导出\textbf{收敛速率}(如函数值次优量的$O(1/K)$速率)。

\begin{theorem}[$O(1/k)$次优率]
若$f$凸且$L$-光滑,取$\eta = \frac{1}{L}$,则
\[
f(x_k) - f^* \leq \frac{L}{2k}\|x_0 - x^*\|^2, \quad k \geq 1.
\]
\end{theorem}

\begin{proof}[证明(望远镜技巧)]
对$k = 0, \dots, T-1$求和,右侧望远镜展开:
\[
\frac{2}{L}\sum_{k=0}^{T-1} \left(f(x_{k+1}) - f^*\right) \leq \|x_0 - x^*\|^2 - \|x_T - x^*\|^2 \leq \|x_0 - x^*\|^2.
\]

因此
\[
\frac{1}{T}\sum_{k=1}^{T} \left(f(x_k) - f^*\right) \leq \frac{L}{2T}\|x_0 - x^*\|^2. \tag{3.2.h}
\]

又知$\{f(x_k)\}$单调不增,故
\[
f(x_T) - f^* \leq \frac{1}{T}\sum_{k=1}^{T} \left(f(x_k) - f^*\right) \leq \frac{L}{2T}\|x_0 - x^*\|^2.
\]

令$T = k$即得结论。

\textbf{一般步长$\eta \in (0, 1/L]$:}

完全同样的推导(把上文中的$1/L$换为一般$\eta$)给出
\[
f(x_k) - f^* \leq \frac{\|x_0 - x^*\|^2}{2\eta k}, \quad \text{取} \ \eta = \frac{1}{L} \text{恢复定理常数。} \tag{3.2.i}
\]
\end{proof}

\subsection{$\mu$-强凸 + L-光滑:线性(几何)收敛}

强凸与光滑结合下的梯度下降"线性收敛"

\begin{theorem}
若$f$为$\mu$-强凸且$L$-光滑,取$0 < \eta \leq \frac{2}{\mu + L}$,则
\[
\|x_{k+1} - x^*\|^2 \leq \rho^2 \|x_k - x^*\|^2, \quad \rho := \max\{1 - \eta\mu, |1 - \eta L|\} < 1,
\]
进而
\[
f(x_k) - f^* \leq \frac{L}{2} \rho^{2k} \|x_0 - x^*\|^2.
\]
\end{theorem}

\begin{proof}[证明:从"梯度与Hessian"到"几何收缩"]
我们先引入误差项$e_k := x_k - x^*$,梯度下降的更新式为$x_{k+1} = x_k - \eta \nabla f(x_k)$。

\paragraph{第一步:Hessian的"平均积分表示"}
由一维积分形式的平均Hessian,梯度的差可表示为:
\[
\nabla f(x_k) - \nabla f(x^*) = \left( \int_0^1 \nabla^2 f(x^* + t(x_k - x^*)) dt \right) (x_k - x^*) =: H_k e_k,
\]
其中$H_k$是对称矩阵,且满足$\mu I \preceq H_k \preceq L I$(强凸和L-光滑的核心体现:Hessian的特征值被$\mu$和$L$"夹逼")。

\begin{remark}[Hessian积分表示的推导]
令$g(t) = \nabla f\left(x^* + t(x_k - x^*)\right)$,其中$t \in [0,1]$。
\begin{itemize}
    \item 当$t=0$时,$g(0) = \nabla f(x^*)$;
    \item 当$t=1$时,$g(1) = \nabla f(x_k)$。
\end{itemize}

对$g(t)$关于$t$求导(利用\textbf{Hessian的定义}:$\nabla f$的导数是Hessian矩阵$\nabla^2 f$):
\[
g'(t) = \nabla^2 f\left(x^* + t(x_k - x^*)\right) \cdot (x_k - x^*)
\]
根据\textbf{牛顿-莱布尼茨公式}(微积分基本定理),有:
\[
g(1) - g(0) = \int_0^1 g'(t) \, dt
\]
将$g(1) = \nabla f(x_k)$、$g(0) = \nabla f(x^*)$和$g'(t)$的表达式代入,即可得到:
\[
\nabla f(x_k) - \nabla f(x^*) = \left( \int_0^1 \nabla^2 f\left(x^* + t(x_k - x^*)\right) dt \right) (x_k - x^*)
\]
这一步的价值在于\textbf{将"梯度差"转化为"Hessian的积分平均形式"},从而可以利用"$\mu$-强凸(Hessian下界$\mu I$)"和"L-光滑(Hessian上界$L I$)"的条件,分析误差项$x_k - x^*$的收缩性(即后续的几何收敛)。

简单来说,它是"强凸+L-光滑"场景下\textbf{梯度下降线性收敛证明的"第一块基石"}——通过把梯度差和Hessian联系起来,我们才能量化误差的几何收缩速率。
\end{remark}

\paragraph{第二步:误差项的迭代收缩}
将更新式代入误差项$e_{k+1} = x_{k+1} - x^*$,得:
\[
e_{k+1} = e_k - \eta (\nabla f(x_k) - \nabla f(x^*)) = (I - \eta H_k) e_k.
\]

由于$H_k$可正交对角化,矩阵$I - \eta H_k$的\textbf{谱范数}(即最大特征值的绝对值)由$H_k$的特征值范围决定。结合$\mu \leq \lambda(H_k) \leq L$,得:
\[
\|e_{k+1}\| \leq \max_{\lambda \in [\mu, L]} |1 - \eta \lambda| \cdot \|e_k\|.
\]

记$\rho := \max\{1 - \eta\mu, |1 - \eta L|\}$,当$0 < \eta \leq \frac{2}{\mu + L}$时,可验证$1 - \eta\mu < 1$且$|1 - \eta L| < 1$,故$\rho < 1$。因此误差项的平方满足\textbf{几何收缩}:
\[
\|e_{k+1}\|^2 \leq \rho^2 \|e_k\|^2.
\]

\paragraph{第三步:函数值次优量的收敛}
迭代展开误差项得$\|e_k\|^2 \leq \rho^{2k} \|e_0\|^2$。再结合"上二次界"$f(x) - f^* \leq \frac{L}{2} \|x - x^*\|^2$,即可推出函数值次优量的几何收敛:
\[
f(x_k) - f^* \leq \frac{L}{2} \rho^{2k} \|x_0 - x^*\|^2.
\]
\end{proof}

\subsubsection{最优步长的"黄金选择"}
当取$\eta^* = \frac{2}{\mu + L}$时,两端点的绝对值相等:$|1 - \eta^* \mu| = |1 - \eta^* L| = \frac{L - \mu}{L + \mu}$,此时最优收缩因子$\rho^* = \frac{L - \mu}{L + \mu} = \frac{\kappa - 1}{\kappa + 1}$(其中$\kappa = \frac{L}{\mu}$是"条件数",刻画强凸与光滑的平衡)。

之前凸光滑场景是$O(1/k)$的\textbf{次线性收敛},而这里强凸+光滑下是$\rho^{2k}$的\textbf{几何收敛}(也叫线性收敛)——前者是"慢慢悠悠逼近",后者是"指数级收缩",效率天差地别。这解释了为什么"强凸假设"能让优化算法"跑起来"。

强凸($\mu > 0$)让函数"长得更陡",L-光滑($L < \infty$)让函数"长得不突兀"。两者结合时,Hessian的特征值被"夹在$\mu$和$L$之间",这才使得误差项能通过矩阵谱范数实现"几何收缩"。这种"刚柔并济"的结构,是线性收敛的核心密码。

虽然深度学习中损失函数大多非强凸,但这个结论是\textbf{正则化、fine-tuning阶段}的理论参照——当模型接近收敛时,局部可能近似满足强凸性,此时梯度下降的收敛会呈现"加速特性"。同时,"最优步长$\eta^* = \frac{2}{\mu + L}$"也为自适应步长设计提供了灵感。

\subsection{PL条件 + L-光滑:线性收敛(无需凸)}

这部分聚焦\textbf{非凸场景下的线性收敛分析},核心是通过"PL条件(替代凸性)+ L-光滑(保证梯度平滑)"的组合,证明梯度下降在非凸函数上也能实现线性收敛

\begin{theorem}
若$f$满足\textbf{PL条件}且\textbf{L-光滑},取步长$\eta = \frac{1}{L}$,则
\[
f(x_{k+1}) - f^* \leq \left(1 - \frac{\mu}{L}\right)\left(f(x_k) - f^*\right).
\]
\end{theorem}

PL(Polyak-Łojasiewicz)不等式定义:存在$\mu > 0$,使得对所有$x$,
\[
\frac{1}{2}\|\nabla f(x)\|^2 \geq \mu \left(f(x) - f^*\right), \quad \forall x.
\]
\begin{remark}[PL条件的意义]
它把"梯度的大小"和"函数离最优值的差距(次优量$f(x)-f^*$)"直接关联,是"非凸场景下替代凸性"的关键——无需函数全局凸,只要局部满足该不等式,就能量化梯度与优化程度的关系。
\end{remark}

由L-光滑的"沿负梯度下降"推论,当$0 < \eta \leq \frac{1}{L}$时,梯度下降单步迭代满足:
\[
f(x - \eta \nabla f(x)) \leq f(x) - \frac{\eta}{2}\|\nabla f(x)\|^2.
\]
将步长$\eta = \frac{1}{L}$代入不等式,得:
\[
f(x_{k+1}) \leq f(x_k) - \frac{1}{2L}\|\nabla f(x_k)\|^2.
\]

再结合PL不等式$\frac{1}{2}\|\nabla f(x_k)\|^2 \geq \mu \left(f(x_k) - f^*\right)$,两边同乘$\frac{1}{L}$得:
\[
\frac{1}{2L}\|\nabla f(x_k)\|^2 \geq \frac{\mu}{L}\left(f(x_k) - f^*\right).
\]

将其代入函数值下降的不等式,整理后得到:
\[
f(x_{k+1}) - f^* \leq \left(1 - \frac{\mu}{L}\right)\left(f(x_k) - f^*\right).
\]

\begin{itemize}
    \item \textbf{非凸场景的线性收敛突破}:无需凸性假设,仅靠"PL(梯度与次优量挂钩)+ L-光滑(梯度变化平滑)",就能让函数次优量以\textbf{几何级数(线性速率)}收缩——这解释了"为什么非凸的深度模型训练能收敛到较好效果"(过参数化网络局部常满足PL条件)。
    \item \textbf{条件的普适性}:PL条件比凸性弱,更贴合深度学习中损失函数的非凸特性,是分析非凸优化收敛性的核心工具之一。
\end{itemize}
