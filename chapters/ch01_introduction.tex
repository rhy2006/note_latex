\chapter{凸优化简介}

% 将公式按小节编号,格式形如 1.1.1
\numberwithin{equation}{subsection}

\section{数学优化}
数学优化问题可以写为如下形式:
\begin{equation}
\begin{aligned}
\text{minimize} \quad & f_0(x) \\
\text{subject to} \quad & f_i(x) \leq b_i,\quad i=1,2,\dots,m
\end{aligned}
\end{equation}

向量$x = (x_1,x_2,\dots,x_n)$称为问题的\textbf{优化变量},函数$f_0$为目标函数,函数$f_i$为约束函数,常数$b_i$为约束上限或者约束边界。

最优解 $x^*$ 是满足所有约束条件并使目标函数达到最小值的变量取值,即:
\begin{equation}
x^* = \arg\min_{x \in \mathbb{R}^n} f_0(x) \quad \text{满足} \quad f_i(x) \leq b_i,\quad i=1,2,\dots,m
\end{equation}

更具体地说,最优解 $x^*$ 满足以下两个条件:

\begin{enumerate}
  \item \textbf{可行性(Feasibility)}:
  \begin{equation}
  f_i(x^*) \leq b_i,\quad \forall i=1,2,\dots,m
  \end{equation}
  即最优解必须满足所有的约束条件。
  
  \item \textbf{最优性(Optimality)}:
  \begin{equation}
  f_0(x^*) \leq f_0(x),\quad \forall x \in \mathbb{R}^n \text{ 且满足 } f_i(x) \leq b_i,\quad i=1,2,\dots,m
  \end{equation}
  即最优解在所有可行解中使目标函数达到最小值。
\end{enumerate}

如果存在多个满足上述条件的解,则它们都称为最优解,且此时目标函数的最优值是唯一的。

\section{线性规划与非线性规划}
线性规划的核心特征为:\textbf{目标函数和约束函数均为线性函数}。

\begin{definition}[线性函数]
线性函数的定义为:对于任意 $x, y \in \mathbb{R}^n$ 和 $\alpha, \beta \in \mathbb{R}$,均满足:
\begin{equation}
f_i(\alpha x + \beta y) = \alpha f_i(x) + \beta f_i(y)
\end{equation}
\end{definition}


\section{凸优化}
凸优化的核心特征为:\textbf{目标函数和约束函数均为凸函数}。

\begin{definition}[凸函数]
凸函数的定义为:对于任意 $x, y \in \mathbb{R}^n$ 和 $\alpha, \beta \in \mathbb{R}$,且满足 $\alpha + \beta = 1$、$\alpha \geq 0$、$\beta \geq 0$,均有:
\begin{equation}
f_i(\alpha x + \beta y) \leq \alpha f_i(x) + \beta f_i(y)
\end{equation}
\end{definition}

\section{最小二乘和线性规划}
广为人知而且应用广泛的两类凸优化问题:最小二乘和线性规划

\subsection{线性规划(Linear Programming, LP)}
线性规划是一类目标函数与约束函数均为线性的优化问题,是最经典的确定性优化模型之一。

一个函数 $f_i: \mathbb{R}^n \to \mathbb{R}$ 为线性函数,当且仅当对任意 $x, y \in \mathbb{R}^n$ 和任意标量 $\alpha, \beta \in \mathbb{R}$,满足\textbf{线性性}(齐次性+叠加性):
\begin{equation}
f_i(\alpha x + \beta y) = \alpha f_i(x) + \beta f_i(y)
\end{equation}

其具体形式可表示为 $f_i(x) = c_i^T x$,其中 $c_i \in \mathbb{R}^n$ 为常数向量,$x \in \mathbb{R}^n$ 为决策变量。

\begin{definition}[线性规划]
线性规划的目标是在 linear 约束下优化 linear 目标函数,标准形式(以最小化为例)为:
\begin{align}
\min_{x \in \mathbb{R}^n} &\quad f(x) = c^T x \nonumber\\
\text{s.t.} &\quad A x = b \\
&\quad x \geq 0 \nonumber
\end{align}
\end{definition}
其中:
\begin{itemize}
  \item $c \in \mathbb{R}^n$:目标函数系数向量;
  \item $A \in \mathbb{R}^{m \times n}$:约束系数矩阵($m < n$);
  \item $b \in \mathbb{R}^m$:约束右端项向量;
  \item $x \geq 0$:决策变量非负约束。
\end{itemize}


\subsection{最小二乘法(Least Squares Method, LSM)}

对于给定的观测数据 $\{(x_1, y_1), (x_2, y_2), \dots, (x_m, y_m)\}$,假设数据满足某种函数关系 $y = f(x; \theta)$($\theta$ 为待估参数),最小二乘法的目标是寻找参数 $\theta^*$,使得\textbf{误差平方和最小}:
\begin{equation}
\theta^* = \arg\min_{\theta} \sum_{i=1}^m [y_i - f(x_i; \theta)]^2
\end{equation}


当待估参数 $\theta$ 与函数 $f(x; \theta)$ 呈线性关系时,称为线性最小二乘。其典型形式为:
已知线性模型 $y = A \theta + \epsilon$($\epsilon$ 为误差项),目标函数为:
\begin{equation}
\min_{\theta \in \mathbb{R}^n} \quad f(\theta) = \|A \theta - y\|_2^2 = (A \theta - y)^T (A \theta - y)
\end{equation}

其中:
\begin{itemize}
  \item $A \in \mathbb{R}^{m \times n}$:设计矩阵($m > n$,保证超定系统);
  \item $y \in \mathbb{R}^m$:观测值向量;
  \item $\theta \in \mathbb{R}^n$:待估参数向量。
\end{itemize}

该目标函数是关于 $\theta$ 的二次凸函数,无约束条件(或仅含线性约束)。

当然最小二乘还有一些拓展,如加权最小二乘和正则化最小二乘,在此不展开。


\section{仿射集合(Affine Sets)}
仿射集合是线性空间中“平移后的线性子空间”,其核心特征是对\textbf{仿射组合}的封闭性。


\subsection{仿射组合与仿射集合}
\subsubsection{(1)仿射组合(Affine Combination)}
\begin{definition}
设 $x_1, x_2, \dots, x_k \in \mathbb{R}^n$,若标量 $\theta_1, \theta_2, \dots, \theta_k \in \mathbb{R}$ 满足 $\sum_{i=1}^k \theta_i = 1$,则称向量:
\begin{equation}
\theta_1 x_1 + \theta_2 x_2 + \dots + \theta_k x_k
\end{equation}
为 $x_1, x_2, \dots, x_k$ 的\textbf{仿射组合}。
\end{definition}

特别地,当 $k=2$ 时,仿射组合为 $\theta x + (1-\theta) y$($\theta \in \mathbb{R}$),其几何意义是过点 $x$ 和 $y$ 的\textbf{整条直线}(区别于后续凸组合对应的“线段”)。


\subsubsection{(2)仿射集合的定义}
\begin{definition}
一个集合 $A \subseteq \mathbb{R}^n$ 被称为\textbf{仿射集合},当且仅当对任意 $x, y \in A$ 及任意 $\theta \in \mathbb{R}$,$x$ 与 $y$ 的仿射组合仍属于 $A$,即:
\begin{equation}
\theta x + (1-\theta) y \in A
\end{equation}
\end{definition}

推广到 $k$ 个点:若 $A$ 是仿射集合,则对任意 $x_1, \dots, x_k \in A$ 及任意满足 $\sum_{i=1}^k \theta_i = 1$ 的 $\theta_1, \dots, \theta_k \in \mathbb{R}$,有 $\sum_{i=1}^k \theta_i x_i \in A$(可通过数学归纳法证明)。


\subsection{仿射集合与线性子空间的关系}
仿射集合可通过“线性子空间的平移”来等价描述,这是理解仿射集合的关键视角。

\subsubsection{(1)平移与线性子空间}
设 $A \subseteq \mathbb{R}^n$ 是仿射集合,任取 $x_0 \in A$,定义集合:
\begin{equation}
L = A - x_0 = \{ x - x_0 \mid x \in A \}
\end{equation}
则 $L$ 是 $\mathbb{R}^n$ 的\textbf{线性子空间}(满足对线性组合封闭:$\forall u, v \in L, \alpha, \beta \in \mathbb{R}$,$\alpha u + \beta v \in L$)。
证明略


\subsection{仿射包(Affine Hull)}
\begin{definition}
对任意集合 $S \subseteq \mathbb{R}^n$,包含 $S$ 的\textbf{最小仿射集合}称为 $S$ 的仿射包,记为 $\text{aff}(S)$。其数学表达式为:
\begin{equation}
	\text{aff}(S) = \left\{ \sum_{i=1}^k \theta_i x_i \mid x_1, \dots, x_k \in S, \theta_1, \dots, \theta_k \in \mathbb{R}, \sum_{i=1}^k \theta_i = 1 \right\}
\end{equation}
\end{definition}

直观理解:仿射包是“由 $S$ 中所有点的仿射组合构成的集合”,例如 $\mathbb{R}^2$ 中两个点的仿射包是过这两点的直线,三个不共线点的仿射包是整个 $\mathbb{R}^2$。


\subsection{典型实例}

\begin{itemize}
  \item 线性方程组的解空间:设 $A \in \mathbb{R}^{m \times n}$,$b \in \mathbb{R}^m$,则 $A x = b$ 的解集合 $X = \{ x \in \mathbb{R}^n \mid A x = b \}$ 是仿射集合(若 $X$ 非空)。
  
  \begin{proof}
  任取 $x, y \in X$,则 $A(\theta x + (1-\theta) y) = \theta A x + (1-\theta) A y = \theta b + (1-\theta) b = b$,故 $\theta x + (1-\theta) y \in X$。
  \end{proof}
  
  \item 单点集 $\{x_0\}$:是仿射集合(仅含自身,仿射组合仍为自身)。
  
  \item 整个空间 $\mathbb{R}^n$:是仿射集合(线性子空间本身,平移量为0)。
\end{itemize}


\section{凸集(Convex Sets)}
凸集是凸优化的核心结构,其特征是对\textbf{凸组合}的封闭性,这直接保证了“局部最优即全局最优”的关键性质。


\subsection{凸组合与凸集}
\subsubsection{(1)凸组合(Convex Combination)}
\begin{definition}
设 $x_1, x_2, \dots, x_k \in \mathbb{R}^n$,若标量 $\theta_1, \theta_2, \dots, \theta_k \in \mathbb{R}$ 满足 $\sum_{i=1}^k \theta_i = 1$ 且 $\theta_i \geq 0$($i=1, \dots, k$),则称向量:
\begin{equation}
\theta_1 x_1 + \theta_2 x_2 + \dots + \theta_k x_k
\end{equation}
为 $x_1, x_2, \dots, x_k$ 的\textbf{凸组合}。
\end{definition}

特别地,当 $k=2$ 时,凸组合为 $\theta x + (1-\theta) y$($\theta \in [0,1]$),其几何意义是连接 $x$ 和 $y$ 的\textbf{线段}(区别于仿射组合的“直线”)。


\subsubsection{(2)凸集的定义}
\begin{definition}
一个集合 $C \subseteq \mathbb{R}^n$ 被称为\textbf{凸集},当且仅当对任意 $x, y \in C$ 及任意 $\theta \in [0,1]$,$x$ 与 $y$ 的凸组合仍属于 $C$,即:
\begin{equation}
\theta x + (1-\theta) y \in C
\end{equation}
\end{definition}

推广到 $k$ 个点:若 $C$ 是凸集,则对任意 $x_1, \dots, x_k \in C$ 及任意满足 $\sum_{i=1}^k \theta_i = 1$ 且 $\theta_i \geq 0$ 的 $\theta_1, \dots, \theta_k \in \mathbb{R}$,有 $\sum_{i=1}^k \theta_i x_i \in C$(数学归纳法可证)。


\subsection{与仿射集合的关系}
仿射集合是特殊的凸集,但凸集不一定是仿射集合,二者的核心差异在于\textbf{组合系数的约束范围}:

\begin{itemize}
  \item 仿射组合:系数仅要求和为1(可正可负、可大于1);
  \item 凸组合:系数要求和为1且非负(仅在 $[0,1]$ 内取值)。
\end{itemize}

因此,仿射集合对更宽松的组合封闭,自然也对凸组合封闭,即:\textbf{若 $A$ 是仿射集合,则 $A$ 必是凸集}。反之,凸集(如线段、球体)不一定是仿射集合(线段对 $\theta > 1$ 的仿射组合不封闭)。


\subsection{重要概念:凸包(Convex Hull)}
\begin{definition}
对任意集合 $S \subseteq \mathbb{R}^n$,包含 $S$ 的\textbf{最小凸集}称为 $S$ 的凸包,记为 $\text{conv}(S)$。其数学表达式为:
\begin{equation}
	\text{conv}(S) = \left\{ \sum_{i=1}^k \theta_i x_i \mid x_1, \dots, x_k \in S, \theta_i \geq 0, \sum_{i=1}^k \theta_i = 1 \right\}
\end{equation}
\end{definition}

直观理解:凸包是“由 $S$ 中所有点的凸组合构成的集合”,例如 $\mathbb{R}^2$ 中三个不共线点的凸包是三角形,圆上所有点的凸包是闭圆盘。


\subsection{典型例子}

\begin{itemize}
  \item 仿射集合的特例:线性子空间、线性方程组解空间、单点集、$\mathbb{R}^n$ 均为凸集;
  
  \item 标准凸集:
  \begin{itemize}
    \item 闭区间 $[a, b] \subseteq \mathbb{R}$;
    \item 球体 $\{ x \in \mathbb{R}^n \mid \|x - x_0\| \leq r \}$($\|\cdot\|$ 为任意范数);
    \item 正象限 $\mathbb{R}_+^n = \{ x \in \mathbb{R}^n \mid x_i \geq 0, i=1, \dots, n \}$;
    \item 半空间 $\{ x \in \mathbb{R}^n \mid a^T x \leq b \}$($a \neq 0$,本质是线性不等式约束的解空间)。
  \end{itemize}
\end{itemize}


\section{锥(Cones)与凸锥(Convex Cones)}
锥是一类对“正齐次性”封闭的集合,凸锥则进一步结合了凸性,是锥优化(如半定规划、二次锥规划)的核心结构。


\subsection{核心定义:锥与凸锥}
\subsubsection{(1)锥的定义}
\begin{definition}
一个非空集合 $K \subseteq \mathbb{R}^n$ 被称为\textbf{锥},当且仅当对任意 $x \in K$ 及任意 $\alpha \geq 0$(非负标量),有 $\alpha x \in K$,即:
\begin{equation}
x \in K, \alpha \geq 0 \implies \alpha x \in K
\end{equation}
\end{definition}

直观理解:锥是“从原点出发的射线族”,若某条射线包含于 $K$,则射线的所有非负伸缩段也包含于 $K$。


\subsubsection{(2)凸锥的定义}
\begin{definition}
一个非空集合 $K \subseteq \mathbb{R}^n$ 被称为\textbf{凸锥},当且仅当它既是锥,又是凸集。其等价刻画有两种:

$K$ 是凸锥,当且仅当对任意 $x_1, \dots, x_k \in K$ 及任意 $\alpha_1, \dots, \alpha_k \geq 0$,有 $\sum_{i=1}^k \alpha_i x_i \in K$(称为对\textbf{非负线性组合}封闭)。
\end{definition}


\subsection{重要概念:凸锥包(Convex Cone Hull)}
\begin{definition}
对任意集合 $S \subseteq \mathbb{R}^n$,包含 $S$ 的\textbf{最小凸锥}称为 $S$ 的凸锥包,记为 $\text{cone}(S)$。其数学表达式为:
\begin{equation}
	\text{cone}(S) = \left\{ \sum_{i=1}^k \alpha_i x_i \mid x_1, \dots, x_k \in S, \alpha_i \geq 0 \right\}
\end{equation}
\end{definition}

直观理解:凸锥包是“由 $S$ 中所有点的非负线性组合构成的集合”,例如 $\mathbb{R}^2$ 中两个不共线向量的凸锥包是它们张成的“角形区域”。


\subsection{典型实例}

\begin{itemize}
  \item 非凸锥:$\mathbb{R}^2$ 中 $\{ (x_1, x_2) \mid x_1 x_2 \geq 0 \}$(第一、三象限的并集,对凸组合不封闭,如 $(1,0)$ 和 $(0,1)$ 的凸组合 $(1/2, 1/2)$ 不属于该集合);
  
  \item 凸锥:
  \begin{itemize}
    \item 原点 $\{0\}$(平凡凸锥);
    \item 正象限 $\mathbb{R}_+^n = \{ x \mid x_i \geq 0 \}$(非负线性组合仍非负);
    \item 线性子空间 $L \subseteq \mathbb{R}^n$(对任意 $\alpha \in \mathbb{R}$ 封闭,自然对 $\alpha \geq 0$ 封闭,且是凸集);
    \item 二次锥(冰淇淋锥):$Q = \{ (x, t) \in \mathbb{R}^n \times \mathbb{R} \mid \|x\| \leq t \}$(验证:若 $(x_1, t_1), (x_2, t_2) \in Q$,$\alpha_1, \alpha_2 \geq 0$,则 $\|\alpha_1 x_1 + \alpha_2 x_2\| \leq \alpha_1 \|x_1\| + \alpha_2 \|x_2\| \leq \alpha_1 t_1 + \alpha_2 t_2$,故 $\alpha_1(x_1, t_1) + \alpha_2(x_2, t_2) \in Q$)。
  \end{itemize}
\end{itemize}


\section{三类集合的核心对比}
为清晰梳理仿射集合、凸集与凸锥的差异与关联,下表从核心定义、组合类型、关键性质三个维度进行总结:

\begin{table}[h]
    \centering
    % 改为4列,调整列宽适配内容
    \begin{tabular}{p{2.5cm}p{3.5cm}p{3.5cm}p{4cm}}
    \toprule
    维度 & 仿射集合(Affine Set) & 凸集(Convex Set) & 凸锥(Convex Cone) \\
    \midrule
    核心定义 & 对仿射组合封闭 & 对凸组合封闭 & 对非负线性组合封闭 \\
    典型组合形式 & $\theta x + (1-\theta) y$($\theta \in \mathbb{R}$) & $\theta x + (1-\theta) y$($\theta \in [0,1]$) & $\alpha x + \beta y$($\alpha, \beta \geq 0$) \\
    与线性子空间关系 & 平移后的线性子空间 & 包含线性子空间的子集(可非平移) & 线性子空间是特殊凸锥(对任意 $\alpha \in \mathbb{R}$ 封闭) \\
    “最小包含集” & 仿射包(aff(S)) & 凸包(conv(S)) & 凸锥包(cone(S)) \\
    \bottomrule
    \end{tabular}
\end{table}

\section{重要的例子}
\subsection{超平面(Hyperplane)与半空间(Halfspace)}
超平面和半空间是由线性函数定义的基本集合,分别对应“线性等式约束”和“线性不等式约束”的解空间,是构建复杂凸集(如多面体)的基石。

\subsubsection{(1)超平面(Hyperplane)}
\paragraph{定义}
\begin{definition}[超平面]
设 $a \in \mathbb{R}^n$ 且 $a \neq 0$(非零法向量),$b \in \mathbb{R}$(常数),则 $\mathbb{R}^n$ 中的\textbf{超平面}定义为:
\begin{equation}
H = \{ x \in \mathbb{R}^n \mid a^T x = b \}
\end{equation}
\end{definition}

\paragraph{几何意义}
超平面是 $\mathbb{R}^n$ 中“维度为 $n-1$ 的仿射集合”,可理解为“与法向量 $a$ 垂直且到原点的‘距离’为 $|b|/\|a\|$ 的平面”。例如:

\begin{itemize}
  \item 当 $n=2$ 时,$H$ 是直线($a_1 x_1 + a_2 x_2 = b$);
  \item 当 $n=3$ 时,$H$ 是平面($a_1 x_1 + a_2 x_2 + a_3 x_3 = b$)。
\end{itemize}

\paragraph{性质:超平面是仿射集合}


\subsubsection{(2)半空间(Halfspace)}
\paragraph{定义}
\begin{definition}[闭半空间]
设 $a \in \mathbb{R}^n$ 且 $a \neq 0$,$b \in \mathbb{R}$,则 $\mathbb{R}^n$ 中的\textbf{闭半空间}(Closed Halfspace)定义为:
\begin{equation}
H_+ = \{ x \in \mathbb{R}^n \mid a^T x \geq b \}, \quad H_- = \{ x \in \mathbb{R}^n \mid a^T x \leq b \}
\end{equation}
若将不等式改为严格不等号($>$ 或 $<$),则称为\textbf{开半空间}(Open Halfspace)。
\end{definition}

\paragraph{几何意义}
半空间是超平面将 $\mathbb{R}^n$ 分割成的两个“半无限区域”,其中 $H_+$ 是法向量 $a$ 指向的一侧,$H_-$ 是相反侧。

\paragraph{性质:半空间是凸集(非仿射集)}

\subsubsection{(3)超平面与半空间的关系}
超平面是两个闭半空间的交集:$H = H_+ \cap H_-$;反之,每个闭半空间都是超平面的“一侧区域”,二者共同构成线性约束的几何表达。


\subsection{球(Ball)与椭球(Ellipsoid)}
球和椭球是基于“距离”定义的凸集,广泛用于建模“变量取值范围的约束”(如稳健优化中的不确定性集合)。

\subsubsection{(1)球(Euclidean Ball)}
\paragraph{定义}
\begin{definition}[欧氏球]
设 $x_0 \in \mathbb{R}^n$(中心),$r > 0$(半径),基于\textbf{欧氏范数}($\|x\|_2 = \sqrt{x_1^2 + \dots + x_n^2}$)的\textbf{闭球}定义为:
\begin{equation}
B(x_0, r) = \{ x \in \mathbb{R}^n \mid \|x - x_0\|_2 \leq r \}
\end{equation}
若将不等号改为 $<$,则称为\textbf{开球}。
\end{definition}

\paragraph{性质:球是凸集}

\subsubsection{(2)椭球(Ellipsoid)}
椭球是球的“仿射变换”,可描述更一般的“椭圆型区域”,在工程优化中常用于拟合数据分布或刻画变量波动范围。

\paragraph{定义}
\begin{definition}[椭球]
设 $x_0 \in \mathbb{R}^n$(中心),$P \in \mathbb{S}_{++}^n$(正定对称矩阵,控制椭球的形状与方向),$r > 0$(缩放因子),则\textbf{椭球}定义为:
\begin{equation}
\mathcal{E} = \{ x \in \mathbb{R}^n \mid (x - x_0)^T P^{-1} (x - x_0) \leq r^2 \}
\end{equation}

等价表达:通过仿射变换 $x = x_0 + r P^{1/2} z$(其中 $P^{1/2}$ 是 $P$ 的正定平方根,$z \in \mathbb{R}^n$),椭球可表示为球的像:
\begin{equation}
\mathcal{E} = \{ x_0 + r P^{1/2} z \mid \|z\|_2 \leq 1 \}
\end{equation}
当 $P = I$(单位矩阵)时,椭球退化为球 $B(x_0, r)$。
\end{definition}

\paragraph{性质:椭球是凸集}


\subsection{范数球(Norm Ball)与范数锥(Norm Cone)}
范数球和范数锥是基于“一般范数”的扩展集合,将球的“距离约束”与锥的“正齐次性”结合,是范数优化、锥优化的核心结构。

\subsubsection{(1)范数的回顾}
首先明确\textbf{范数}的定义:
\begin{definition}[范数]
函数 $\|\cdot\|: \mathbb{R}^n \to \mathbb{R}$ 称为范数,若对任意 $x, y \in \mathbb{R}^n$ 和 $\alpha \in \mathbb{R}$,满足:

\begin{enumerate}
  \item 非负性:$\|x\| \geq 0$,且 $\|x\| = 0 \iff x = 0$;
  \item 齐次性:$\|\alpha x\| = |\alpha| \|x\|$;
  \item 三角不等式:$\|x + y\| \leq \|x\| + \|y\|$。
\end{enumerate}

\end{definition}
常见范数包括欧氏范数($\|\cdot\|_2$)、1-范数($\|x\|_1 = \sum_{i=1}^n |x_i|$)、无穷范数($\|x\|_\infty = \max_{i=1,\dots,n} |x_i|$)等。


\subsubsection{(2)范数球(Norm Ball)}
\paragraph{定义}
\begin{definition}[范数球]
设 $\|\cdot\|$ 是 $\mathbb{R}^n$ 上的范数,$x_0 \in \mathbb{R}^n$,$r > 0$,则\textbf{范数球}定义为:
\begin{equation}
B_{\|\cdot\|}(x_0, r) = \{ x \in \mathbb{R}^n \mid \|x - x_0\| \leq r \}
\end{equation}

当 $x_0 = 0$ 且 $r = 1$ 时,称为\textbf{单位范数球}(Unit Norm Ball)。
\end{definition}
\paragraph{性质:范数球是凸集}

\paragraph{实例}

\begin{itemize}
  \item 1-范数球($\|x\|_1 \leq 1$):在 $\mathbb{R}^2$ 中是菱形,$\mathbb{R}^3$ 中是正八面体;
  \item 无穷范数球($\|x\|_\infty \leq 1$):在 $\mathbb{R}^2$ 中是正方形,$\mathbb{R}^3$ 中是正方体。
\end{itemize}


\subsubsection{(3)范数锥(Norm Cone)}
\paragraph{定义}
\begin{definition}[范数锥]
设 $\|\cdot\|$ 是 $\mathbb{R}^n$ 上的范数,定义\textbf{范数锥}(又称“冰淇淋锥”的推广)为:
\begin{equation}
K_{\|\cdot\|} = \{ (x, t) \in \mathbb{R}^n \times \mathbb{R} \mid \|x\| \leq t \}
\end{equation}

其中 $(x, t)$ 是 $\mathbb{R}^{n+1}$ 中的向量,$t$ 可理解为“范数的上界”。
\end{definition}
\paragraph{性质:范数锥是凸锥}


\paragraph{实例}

\begin{itemize}
  \item 二次锥(Quadratic Cone):当 $\|\cdot\| = \|\cdot\|_2$ 时,范数锥即为二次锥(又称Lorentz锥):
  \begin{equation}
  Q = \{ (x, t) \in \mathbb{R}^n \times \mathbb{R} \mid \|x\|_2 \leq t \}
  \end{equation}
  是锥优化中最常用的凸锥之一。
  
  \item 1-范数锥:$K_{\|\cdot\|_1} = \{ (x, t) \mid \sum_{i=1}^n |x_i| \leq t \}$,在 $\mathbb{R}^2 \times \mathbb{R}$ 中呈“四棱锥”形状。
\end{itemize}


\subsection{多面体(Polyhedron)与单纯形(Simplex)}
多面体是有限个线性等式与不等式约束的交集,是线性规划可行域的抽象;单纯形则是多面体的特殊情况,是“最低维度”的多面体,在数值优化中常用于构建搜索区域。

\subsubsection{(1)多面体(Polyhedron)}
\paragraph{定义}
\begin{definition}{多面体}
  多面体是有限个线性等式与不等式约束的交集,其数学表达式为: 
\begin{equation}
  \mathcal{P} = \{ x \in \mathbb{R}^n \mid A x \leq b, C x = d \}
\end{equation}
\end{definition}

其中:

\begin{itemize}
  \item $A \in \mathbb{R}^{m \times n},\; b \in \mathbb{R}^m$:对应 $m$ 个半空间约束($A x \leq b$ 即 $a_i^T x \leq b_i,\; i=1,\dots,m$)。
  \item $C \in \mathbb{R}^{p \times n},\; d \in \mathbb{R}^p$:对应 $p$ 个超平面约束($C x = d$ 即 $c_j^T x = d_j,\; j=1,\dots,p$)。
\end{itemize}

\paragraph{性质:多面体是凸集}

\paragraph{实例}

\begin{itemize}
  \item 正象限 $\mathbb{R}_+^n = \{ x \mid x_i \geq 0, i=1,\dots,n \}$:是多面体($A = -I$,$b = 0$);
  \item 线性规划的可行域:$X = \{ x \mid A x = b, x \geq 0 \}$,是多面体。
\end{itemize}


\subsubsection{(2)单纯形(Simplex)}
单纯形是“由 $n+1$ 个仿射无关点生成的凸包”,是维度为 $n$ 的“最简单”多面体(顶点数量最少的多面体)。

\begin{definition}{单纯形}
\paragraph{定义1(基于仿射无关点)}
设 $v_0, v_1, \dots, v_n \in \mathbb{R}^n$ 是\textbf{仿射无关}的点(即向量 $v_1 - v_0, \dots, v_n - v_0$ 线性无关),则由这些点生成的\textbf{单纯形}定义为:
\begin{equation}
\Delta = \text{conv}(v_0, v_1, \dots, v_n) = \left\{ \sum_{i=0}^n \theta_i v_i \mid \theta_i \geq 0, \sum_{i=0}^n \theta_i = 1 \right\}
\end{equation}

\paragraph{定义2(标准单纯形)}
最常用的是\textbf{标准单纯形}(以单位向量为顶点),定义为:
\begin{equation}
\Delta_n = \{ x \in \mathbb{R}^n \mid x_1 + x_2 + \dots + x_n = 1, x_i \geq 0, i=1,\dots,n \}
\end{equation}
\end{definition}
其顶点为 $e_1 = (1,0,\dots,0)^T, e_2 = (0,1,\dots,0)^T, \dots, e_n = (0,\dots,1)^T$($n$ 个标准单位向量),但注意:此处 $n$ 维标准单纯形的顶点数量为 $n$,与定义1中“$n$ 维单纯形需 $n+1$ 个顶点”的差异源于“是否包含原点”——若定义为 $\Delta_n' = \{ (x, 1 - \sum x_i) \mid x \in \Delta_n \}$,则顶点为 $e_1,\dots,e_n, 0$,共 $n+1$ 个,符合定义1。

\paragraph{关键性质}

\begin{enumerate}
  \item \textbf{维度}:由 $n+1$ 个仿射无关点生成的单纯形是 $n$ 维的(与空间维度一致)。
  
  \item \textbf{凸性}:单纯形是有限个点的凸包,而凸包是凸集(由凸包定义:所有凸组合的集合,自然对凸组合封闭),故单纯形是凸集。
  
  \item \textbf{多面体属性}:单纯形可表示为有限个线性等式与不等式的交集(如标准单纯形的约束 $x_1+\dots+x_n=1$ 和 $x_i\geq0$),因此是多面体。
\end{enumerate}

\paragraph{实例}

\begin{itemize}
  \item 1维单纯形:$\Delta_1 = \{ x \in \mathbb{R} \mid x = 1, x \geq 0 \}$ 即点 $\{1\}$;或扩展为 $\Delta_1' = \{ (x, 1-x) \mid x \geq 0 \}$ 即线段 $[0,1]$;
  
  \item 2维单纯形(标准):$\Delta_2 = \{ (x_1, x_2) \mid x_1 + x_2 = 1, x_1, x_2 \geq 0 \}$ 即连接 $(1,0)$ 和 $(0,1)$ 的线段;或扩展为 $\Delta_2' = \{ (x_1, x_2, x_3) \mid x_1+x_2+x_3=1, x_i\geq0 \}$ 即三角形(3个顶点);
  
  \item 3维单纯形:扩展形式为四面体(4个顶点)。
\end{itemize}