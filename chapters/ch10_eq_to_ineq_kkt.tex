\chapter{从等式约束到不等式约束:KKT}

\section{为什么需要KKT}

等式约束优化(拉格朗日乘子法)仅能处理光滑边界、无需区分约束是否"起作用"的场景;而不等式约束的可行集有"内部/边界(活跃约束)",可行方向呈锥形,等式方法无法区分活跃约束、也没法处理约束梯度需非负权重的需求。KKT正是补上这一缺口,将"目标在可行方向不下降"的几何要求,转化为含"活跃约束区分、非负乘子、互补松弛"的代数条件,从而能解不等式约束的优化问题。

\section{不等式约束建模}

\subsection{等式约束优化问题(基础模型)}
设优化目标为最小化\textbf{目标函数}$f(\boldsymbol{x})$,仅受\textbf{等式约束}限制,数学建模如下:
\[
\begin{cases}
\min_{\boldsymbol{x} \in \mathbb{R}^n} \, f(\boldsymbol{x}) \\
\text{subject to} \quad g_i(\boldsymbol{x}) = 0 \quad (i = 1, 2, \dots, m)
\end{cases}
\]
其中:

\begin{itemize}
    \item $\boldsymbol{x} = (x_1, x_2, \dots, x_n)^T$是$n$维决策变量($\mathbb{R}^n$表示$n$维实数空间);
    \item $f: \mathbb{R}^n \to \mathbb{R}$是连续可微的目标函数(映射到实数域);
    \item $g_i: \mathbb{R}^n \to \mathbb{R}$($i=1,\dots,m$)是连续可微的等式约束函数,$m < n$(约束数量少于变量维度,保证可行集非空)。
\end{itemize}

\subsection{含不等式约束的优化问题(扩展模型)}
在等式约束基础上引入\textbf{不等式约束},形成更通用的优化模型:
\[
\begin{cases}
\min_{\boldsymbol{x} \in \mathbb{R}^n} \, f(\boldsymbol{x}) \\
\text{subject to} \quad 
\begin{aligned}
g_i(\boldsymbol{x}) &= 0 \quad (i = 1, 2, \dots, m) \\
h_j(\boldsymbol{x}) &\leq 0 \quad (j = 1, 2, \dots, p)
\end{aligned}
\end{cases}
\]
其中:

\begin{itemize}
    \item 新增$p$个连续可微的\textbf{不等式约束函数}$h_j: \mathbb{R}^n \to \mathbb{R}$($j=1,\dots,p$),约束满足$h_j(\boldsymbol{x}) \leq 0$;
    \item 可行集定义为$\mathcal{X} = \{\boldsymbol{x} \in \mathbb{R}^n \mid g_i(\boldsymbol{x})=0, h_j(\boldsymbol{x}) \leq 0, \forall i,j\}$,最优解需在$\mathcal{X}$内使$f(\boldsymbol{x})$最小;
    \item 需通过\textbf{积极约束集}$\mathcal{A}(\boldsymbol{x}^*) = \{ j \in \{1,\dots,p\} \mid h_j(\boldsymbol{x}^*) = 0 \}$区分"起作用"(边界)与"不起作用"(内部)的不等式约束,为后续KKT条件奠基。
\end{itemize}

\section{KKT条件的推导}

\subsection{一阶必要条件(KKT条件)的严格数学建模}

\subsubsection{前提假设}
设约束优化问题为:
\[
\mathcal{P}: \begin{cases}
\min_{\boldsymbol{x} \in \mathbb{R}^n} \, f(\boldsymbol{x}) \\
\text{s.t.} \quad g_i(\boldsymbol{x}) = 0 \quad (i=1,2,\dots,m) \\
\quad \quad h_j(\boldsymbol{x}) \leq 0 \quad (j=1,2,\dots,p)
\end{cases}
\]
其中:

\begin{itemize}
    \item $f, g_i, h_j \in C^1(\mathbb{R}^n)$(均为一阶连续可微函数);
    \item $\boldsymbol{x}^* \in \mathcal{X}$($\mathcal{X}$为可行集,即$\mathcal{X} = \{\boldsymbol{x} \mid g_i(\boldsymbol{x})=0, h_j(\boldsymbol{x})\leq0\}$),且$\boldsymbol{x}^*$是$\mathcal{P}$的\textbf{局部极小点};
    \item 满足\textbf{约束规格(Constraint Qualification, CQ)}(后续定义)。
\end{itemize}

\subsubsection{数学结论}
存在拉格朗日乘子$\lambda^* = (\lambda_1^*, \lambda_2^*, \dots, \lambda_m^*)^T \in \mathbb{R}^m$和$\mu^* = (\mu_1^*, \mu_2^*, \dots, \mu_p^*)^T \in \mathbb{R}^p$,使得以下4个条件同时成立:

\begin{enumerate}
    \item \textbf{平稳性条件}(目标梯度与约束梯度平衡):
    \[
    \nabla f(\boldsymbol{x}^*) + \sum_{i=1}^m \lambda_i^* \nabla g_i(\boldsymbol{x}^*) + \sum_{j=1}^p \mu_j^* \nabla h_j(\boldsymbol{x}^*) = \boldsymbol{0} \in \mathbb{R}^n
    \]
    
    \item \textbf{原始可行性条件}(解满足所有约束):
    \[
    h_j(\boldsymbol{x}^*) \leq 0 \quad \forall j \in \{1,2,\dots,p\}
    \]
    (注:等式约束$g_i(\boldsymbol{x}^*) = 0$已隐含在$\boldsymbol{x}^* \in \mathcal{X}$中,此处补充不等式约束的显式条件)
    
    \item \textbf{对偶可行性条件}(不等式约束乘子非负):
    \[
    \mu_j^* \geq 0 \quad \forall j \in \{1,2,\dots,p\}
    \]
    
    \item \textbf{互补松弛条件}(非活跃约束乘子为0):
    \[
    \mu_j^* h_j(\boldsymbol{x}^*) = 0 \quad \forall j \in \{1,2,\dots,p\}
    \]
\end{enumerate}

\subsection{约束规格(CQ)的严格数学建模}
约束规格是保证"局部极小点满足KKT条件"的关键假设,以下为两类核心CQ的严格定义:

\subsubsection{线性无关约束规格(Linear Independence Constraint Qualification, LICQ)}
设$\boldsymbol{x}^* \in \mathcal{X}$,记\textbf{积极约束集}$\mathcal{A}(\boldsymbol{x}^*) = \{ j \in \{1,\dots,p\} \mid h_j(\boldsymbol{x}^*) = 0 \}$(即不等式约束中"起作用"的集合)。

若向量集合:
\[
\mathcal{G}(\boldsymbol{x}^*) = \left\{ \nabla g_i(\boldsymbol{x}^*) \mid i=1,\dots,m \right\} \cup \left\{ \nabla h_j(\boldsymbol{x}^*) \mid j \in \mathcal{A}(\boldsymbol{x}^*) \right\}
\]
满足\textbf{线性无关}(即不存在不全为零的常数$\alpha_1,\dots,\alpha_m, \beta_j (j\in\mathcal{A}(\boldsymbol{x}^*))$,使得$\sum_{i=1}^m \alpha_i \nabla g_i(\boldsymbol{x}^*) + \sum_{j\in\mathcal{A}(\boldsymbol{x}^*)} \beta_j \nabla h_j(\boldsymbol{x}^*) = \boldsymbol{0}$),则称在$\boldsymbol{x}^*$处满足LICQ。

\subsubsection{Mangasarian-Fromovitz约束规格(Mangasarian-Fromovitz Constraint Qualification, MFCQ)}
设$\boldsymbol{x}^* \in \mathcal{X}$,若满足以下两个条件:

\begin{enumerate}
    \item 等式约束梯度集$\{ \nabla g_i(\boldsymbol{x}^*) \mid i=1,\dots,m \}$线性无关;
    \item 存在可行方向$\boldsymbol{d} \in \mathbb{R}^n \setminus \{ \boldsymbol{0} \}$(非零方向),使得:
    \[
    \nabla g_i(\boldsymbol{x}^*)^T \boldsymbol{d} = 0 \quad \forall i \in \{1,\dots,m\}
    \]
    \[
    \nabla h_j(\boldsymbol{x}^*)^T \boldsymbol{d} < 0 \quad \forall j \in \mathcal{A}(\boldsymbol{x}^*)
    \]
\end{enumerate}

则称在$\boldsymbol{x}^*$处满足MFCQ。

\begin{remark}
MFCQ是弱于LICQ的约束规格,即LICQ成立可推出MFCQ成立,但反之不成立。
\end{remark}

\subsection{LICQ下KKT条件的严格证明建模}

\subsubsection{第一步:定义线性化可行方向锥}
设$\boldsymbol{x}^*$是$\mathcal{P}$的局部极小点,且在$\boldsymbol{x}^*$处满足LICQ。定义\textbf{线性化可行方向锥}:
\[
\mathcal{F}(\boldsymbol{x}^*) = \left\{ \boldsymbol{d} \in \mathbb{R}^n \mid \nabla g_i(\boldsymbol{x}^*)^T \boldsymbol{d} = 0 \ (\forall i=1,\dots,m), \ \nabla h_j(\boldsymbol{x}^*)^T \boldsymbol{d} \leq 0 \ (\forall j \in \mathcal{A}(\boldsymbol{x}^*)) \right\}
\]
几何意义:$\mathcal{F}(\boldsymbol{x}^*)$是在$\boldsymbol{x}^*$处"沿该方向移动,线性近似下仍可行"的所有方向集合。

\subsubsection{第二步:证明核心引理}

\begin{lemma}
对所有$\boldsymbol{d} \in \mathcal{F}(\boldsymbol{x}^*)$,有$\nabla f(\boldsymbol{x}^*)^T \boldsymbol{d} \geq 0$。
\end{lemma}

\begin{proof}[证明(反证法)]
假设存在$\boldsymbol{d}_0 \in \mathcal{F}(\boldsymbol{x}^*)$使得$\nabla f(\boldsymbol{x}^*)^T \boldsymbol{d}_0 < 0$。

由LICQ成立,根据隐函数定理,可构造\textbf{可行曲线}:
\[
\boldsymbol{x}(t) = \boldsymbol{x}^* + t \boldsymbol{d}_0 + o(t) \quad (t \geq 0)
\]
其中$o(t)$是高阶无穷小(满足$\lim_{t \to 0^+} \frac{\|o(t)\|}{t} = 0$),且曲线满足:

\begin{enumerate}
    \item 对等式约束:$g_i(\boldsymbol{x}(t)) = 0 + t \nabla g_i(\boldsymbol{x}^*)^T \boldsymbol{d}_0 + o(t) = o(t)$(因$\boldsymbol{d}_0 \in \mathcal{F}(\boldsymbol{x}^*)$,故$\nabla g_i(\boldsymbol{x}^*)^T \boldsymbol{d}_0 = 0$);
    \item 对积极不等式约束($j \in \mathcal{A}(\boldsymbol{x}^*)$):$h_j(\boldsymbol{x}(t)) = 0 + t \nabla h_j(\boldsymbol{x}^*)^T \boldsymbol{d}_0 + o(t) \leq 0 + o(t)$(因$\boldsymbol{d}_0 \in \mathcal{F}(\boldsymbol{x}^*)$,故$\nabla h_j(\boldsymbol{x}^*)^T \boldsymbol{d}_0 \leq 0$)。
\end{enumerate}

取充分小的$t_0 > 0$,当$t \in (0, t_0)$时:
\begin{itemize}
    \item $|o(t)| < t \cdot \min\left\{ 1, \left| \nabla h_j(\boldsymbol{x}^*)^T \boldsymbol{d}_0 \right| \ (\forall j \in \mathcal{A}(\boldsymbol{x}^*)) \right\}$,故$g_i(\boldsymbol{x}(t)) \approx 0$、$h_j(\boldsymbol{x}(t)) \leq 0$,即$\boldsymbol{x}(t) \in \mathcal{X}$(可行);
    \item 目标函数泰勒展开:$f(\boldsymbol{x}(t)) = f(\boldsymbol{x}^*) + t \nabla f(\boldsymbol{x}^*)^T \boldsymbol{d}_0 + o(t) < f(\boldsymbol{x}^*)$(因$\nabla f(\boldsymbol{x}^*)^T \boldsymbol{d}_0 < 0$,且$t$充分小)。
\end{itemize}

这与$\boldsymbol{x}^*$是局部极小点矛盾,故假设不成立,引理得证。
\end{proof}

\subsubsection{第三步:应用Farkas引理推导乘子存在性}

\begin{lemma}[Farkas引理]
设$\boldsymbol{A} \in \mathbb{R}^{k \times n}$,$\boldsymbol{b} \in \mathbb{R}^n$,则以下两个系统\textbf{有且仅有一个}有解:
\begin{itemize}
    \item 系统1:$\boldsymbol{A} \boldsymbol{d} \leq \boldsymbol{0}$,$\boldsymbol{b}^T \boldsymbol{d} > 0$($\boldsymbol{d} \in \mathbb{R}^n$);
    \item 系统2:$\boldsymbol{A}^T \boldsymbol{y} = \boldsymbol{b}$,$\boldsymbol{y} \geq \boldsymbol{0}$($\boldsymbol{y} \in \mathbb{R}^k$)。
\end{itemize}
\end{lemma}

将第二步引理转化为Farkas引理的"系统1无解"场景:

构造矩阵$\boldsymbol{A}$和向量$\boldsymbol{b}$如下:
\begin{itemize}
    \item $\boldsymbol{A}$的行由$\nabla g_i(\boldsymbol{x}^*)^T$、$-\nabla g_i(\boldsymbol{x}^*)^T$(对应$\nabla g_i(\boldsymbol{x}^*)^T \boldsymbol{d} = 0$拆分为$\leq 0$和$\geq 0$)、$\nabla h_j(\boldsymbol{x}^*)^T$($j \in \mathcal{A}(\boldsymbol{x}^*)$)组成;
    \item $\boldsymbol{b} = -\nabla f(\boldsymbol{x}^*)$。
\end{itemize}

由第二步引理,"系统1:$\boldsymbol{A} \boldsymbol{d} \leq \boldsymbol{0}$,$\boldsymbol{b}^T \boldsymbol{d} > 0$"无解,故由Farkas引理,"系统2:$\boldsymbol{A}^T \boldsymbol{y} = \boldsymbol{b}$,$\boldsymbol{y} \geq \boldsymbol{0}$"有解。

整理系统2的解,可得:
\begin{itemize}
    \item 存在$\lambda_i^* \in \mathbb{R}$(对应等式约束的乘子,由$\nabla g_i(\boldsymbol{x}^*)^T$和$-\nabla g_i(\boldsymbol{x}^*)^T$的系数合成);
    \item 存在$\mu_j^* \geq 0$(对应积极不等式约束的乘子,由$\nabla h_j(\boldsymbol{x}^*)^T$的系数给出);
    \item 对非积极约束($j \notin \mathcal{A}(\boldsymbol{x}^*)$),令$\mu_j^* = 0$,则平稳性条件成立。
\end{itemize}

同时,互补松弛条件$\mu_j^* h_j(\boldsymbol{x}^*) = 0$自然满足(非积极约束$\mu_j^* = 0$,积极约束$h_j(\boldsymbol{x}^*) = 0$),对偶可行性条件$\mu_j^* \geq 0$由Farkas引理的$\boldsymbol{y} \geq \boldsymbol{0}$保证。

综上,KKT条件在LICQ下得证。

\subsection{Farkas引理证明}

\begin{theorem}[Farkas引理]
设$A \in \mathbb{R}^{m \times n}$,$b \in \mathbb{R}^{n}$,则以下两个系统\textbf{有且仅有一个}有解:
\begin{itemize}
    \item 系统1:$A x \leq 0$,$b^T x > 0$($x \in \mathbb{R}^{n}$);
    \item 系统2:$A^T y = b$,$y \geq 0$($y \in \mathbb{R}^{m}$)。
\end{itemize}
\end{theorem}

\begin{proof}[证明(分两步核心逻辑)]

\paragraph{第一步:证明"两系统不能同时有解"(矛盾法)}

假设系统1、系统2同时存在解,即存在$x \in \mathbb{R}^n$满足$A x \leq 0$且$b^T x > 0$,同时存在$y \in \mathbb{R}^m$满足$A^T y = b$且$y \geq 0$。

对$b^T x$做代数变形:

由$A^T y = b$,两边转置得$b^T = y^T A$,因此$b^T x = y^T (A x)$。

结合已知条件分析:
\begin{itemize}
    \item 因$A x \leq 0$(系统1)且$y \geq 0$(系统2),向量内积$y^T (A x) \leq 0$,即$b^T x \leq 0$;
    \item 但系统1要求$b^T x > 0$,二者矛盾。故两系统不能同时有解。
\end{itemize}

\paragraph{第二步:证明"若系统1无解,则系统2必有解"(凸集分离定理+矛盾法)}

\begin{enumerate}
    \item \textbf{定义闭凸锥}:设集合$C = \{ A^T y \mid y \geq 0 \}$,易证$C$是$\mathbb{R}^n$中的\textbf{闭凸锥}(闭性由线性映射连续性+非负锥闭性保证,凸性由线性映射凸性+非负锥凸性保证)。
    
    \item \textbf{反证假设与凸集分离}:
    
    假设系统1无解,且系统2也无解(即$b \notin C$,若$b \in C$则存在$y \geq 0$使$A^T y = b$,系统2有解)。
    
    因$C$是闭凸集且$b \notin C$,由\textbf{凸集分离定理},存在非零向量$x \in \mathbb{R}^n$和实数$\alpha \in \mathbb{R}$,使得:
    \[
    b^T x > \alpha \quad \text{且} \quad (A^T y)^T x \leq \alpha \quad \forall y \geq 0
    \]
    
    \item \textbf{推导系统1有解(矛盾)}:
    \begin{itemize}
        \item 令$y = 0$(因$0 \in \{ y \mid y \geq 0 \}$,故$A^T 0 = 0 \in C$),代入右边不等式得$0^T x \leq \alpha$,即$\alpha \geq 0$,因此$b^T x > \alpha \geq 0$,即$b^T x > 0$。
        \item 若存在某个分量$(A x)_k > 0$,取$y = t e_k$($e_k$为第$k$个单位向量,$t > 0$),则$(A^T y)^T x = t (A x)_k$。当$t \to +\infty$时,$t (A x)_k \to +\infty$,与$(A^T y)^T x \leq \alpha$($\alpha$是固定实数)矛盾。故必须$A x \leq 0$。
    \end{itemize}
    
    此时$x$满足$A x \leq 0$且$b^T x > 0$,即系统1有解——与"系统1无解"的初始假设矛盾。故系统2必有解。
\end{enumerate}
\end{proof}

\section{二阶最优性条件}

在学习优化问题时,\textbf{一阶KKT条件}告诉我们"最优解处的梯度要平衡约束"(相当于"坡度为零"),但这只够判断"可能是极值点"——就像走到平地上,分不清是山顶、山谷还是半山腰的平台。而\textbf{二阶最优性条件}是在一阶条件基础上,通过判断"曲率"(相当于地面的弯曲方向),明确这个"平地"到底是不是真正的极小点。

要判断"曲率",首先得明确:在KKT点$x^*$处,还有可能让目标函数下降的方向有哪些?这个方向集合就是"临界锥"。

\subsection{临界锥}

\begin{definition}[临界锥]
在KKT点$x^{*}$处,临界锥定义为:
\[
\mathcal{C}\left(x^{*}, \mu^{*}\right)=\left\{d \in \mathbb{R}^{n}: \begin{aligned}
& \nabla g_{i} \left(x^{*}\right)^{T} d=0, \quad i=1, \dots, m \\
& \nabla h_{j}\left(x^{*}\right)^{T} d=0, \quad j \in \mathcal{A}\left(x^{*}\right) \text{ 且 } \mu_{j}^{*}>0 \\
& \nabla h_{j}\left(x^{*}\right)^{T} d \leq 0, \quad j \in \mathcal{A}\left(x^{*}\right) \text{ 且 } \mu_{j}^{*}=0
\end{aligned}\right\}
\]
\end{definition}

临界锥包含了在$x^{*}$处所有可能的"候选下降方向",即"在不违反任何约束的前提下,可能让目标函数下降的所有方向"。

\subsection{二阶必要条件}

\begin{theorem}[二阶必要条件]
设$x^*$是优化问题的\textbf{局部极小点},且满足线性无关约束规格(LICQ),$(\lambda^*,\mu^*)$是对应的KKT乘子,则对所有方向$d \in \mathcal{C}(x^*,\mu^*)$,有:
\[
d^T \nabla_{xx}^2 \mathcal{L}(x^*,\lambda^*,\mu^*) d \geq 0
\]
\end{theorem}

其中:
\begin{itemize}
    \item \textbf{拉格朗日函数}:$\mathcal{L}(x,\lambda,\mu) = f(x) + \sum_{i=1}^m \lambda_i g_i(x) + \sum_{j=1}^p \mu_j h_j(x)$,其中$\lambda \in \mathbb{R}^m$、$\mu \in \mathbb{R}^p$为拉格朗日乘子;
    \item \textbf{拉格朗日Hessian矩阵}:$\nabla_{xx}^2 \mathcal{L}(x^*,\lambda^*,\mu^*)$是拉格朗日函数在$(x^*,\lambda^*,\mu^*)$处关于$x$的二阶偏导数矩阵(Hessian矩阵),严格定义为:
    \[
    \nabla_{xx}^2 \mathcal{L}(x^*,\lambda^*,\mu^*) = \nabla^2 f(x^*) + \sum_{i=1}^m \lambda_i^* \nabla^2 g_i(x^*) + \sum_{j=1}^p \mu_j^* \nabla^2 h_j(x^*)
    \]
    其中$\nabla^2 f(x^*)$、$\nabla^2 g_i(x^*)$、$\nabla^2 h_j(x^*)$分别表示$f$、$g_i$、$h_j$在$x^*$处的Hessian矩阵($n \times n$对称矩阵)。
\end{itemize}

\begin{remark}
"必要条件"的意思是:\textbf{如果$x^*$是局部极小点,那么在所有"可能下降的方向"(临界锥内),综合曲率必须≥0}。
\end{remark}

\subsection{二阶充分条件}

\begin{theorem}[二阶充分条件]
设$x^*$是优化问题的\textbf{可行点}(满足$g_i(x^*)=0$、$h_j(x^*)\leq0$),存在乘子$(\lambda^*,\mu^*)$使得$(x^*,\lambda^*,\mu^*)$满足KKT条件,且对所有\textbf{非零方向}$d \in \mathcal{C}(x^*,\mu^*)$,有:
\[
d^T \nabla_{xx}^2 \mathcal{L}(x^*,\lambda^*,\mu^*) d > 0
\]
则$x^*$是优化问题的\textbf{严格局部极小点}。
\end{theorem}

\begin{remark}
"充分条件"的意思是:\textbf{只要在所有"可能下降的方向"(临界锥内的非零方向),综合曲率都>0,那么$x^*$一定是局部极小点}。
\end{remark}
