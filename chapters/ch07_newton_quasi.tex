\chapter{牛顿法和拟牛顿法}

\section{符号说明}

\begin{table}[H]
\centering
\begin{tabular}{|l|l|p{8cm}|}
\hline
\textbf{符号} & \textbf{类型} & \textbf{定义与用途} \\ \hline
$f(\mathbf{x})$ & 标量函数 & 无约束优化问题的目标函数,输入为$n$维优化变量$\mathbf{x}$,输出为实数值。 \\ \hline
$\nabla f(\mathbf{x})$ & 向量函数 & 目标函数$f(\mathbf{x})$的梯度,输入为$\mathbf{x}$,输出为$n$维梯度向量。 \\ \hline
$\mathbf{x}_k$ & 向量 & 第$k$步迭代点,$\mathbf{x}_k \in \mathbb{R}^n$;$\mathbf{x}_0$为初始迭代点。 \\ \hline
$\mathbf{x}^*$ & 向量 & 目标函数$f(\mathbf{x})$的最优解(近似)。 \\ \hline
$\mathbf{g}_k$ & 向量 & 第$k$步迭代点的梯度,即$\mathbf{g}_k = \nabla f(\mathbf{x}_k)$,$\mathbf{g}_k \in \mathbb{R}^n$。 \\ \hline
$\mathbf{H}$ & 矩阵 & 目标函数$f(\mathbf{x})$的Hessian矩阵(二阶导数矩阵),$\mathbf{H} \in \mathbb{R}^{n \times n}$;$\mathbf{H}_k = \nabla^2 f(\mathbf{x}_k)$为第$k$步迭代点的Hessian矩阵。 \\ \hline
$\mathbf{B}_k$ & 矩阵 & BFGS方法中第$k$步的Hessian矩阵近似,$\mathbf{B}_k \in \mathbb{R}^{n \times n}$,满足拟牛顿条件$\mathbf{B}_{k+1}\mathbf{s}_k = \mathbf{y}_k$。 \\ \hline
$\mathbf{G}_k$ & 矩阵 & $\mathbf{B}_k$的逆矩阵近似(即$\mathbf{G}_k \approx \mathbf{B}_k^{-1}$),$\mathbf{G}_k \in \mathbb{R}^{n \times n}$,用于直接计算搜索方向,避免矩阵求逆。 \\ \hline
$\mathbf{p}_k$ & 向量 & 第$k$步的搜索方向(下降方向),牛顿法中$\mathbf{p}_k = -\mathbf{H}_k^{-1}\mathbf{g}_k$,BFGS中$\mathbf{p}_k = -\mathbf{G}_k \mathbf{g}_k$。 \\ \hline
$\alpha_k$ & 标量 & 第$k$步的迭代步长,通过Wolfe条件(Armijo条件+曲率条件)确定,$\alpha_k > 0$。 \\ \hline
$\mathbf{s}_k$ & 向量 & 第$k$步的变量增量,即$\mathbf{s}_k = \mathbf{x}_{k+1} - \mathbf{x}_k$,反映迭代点的变化量。 \\ \hline
$\mathbf{y}_k$ & 向量 & 第$k$步的梯度增量,即$\mathbf{y}_k = \mathbf{g}_{k+1} - \mathbf{g}_k$,反映梯度的变化量。 \\ \hline
\end{tabular}
\caption{符号说明}
\end{table}

\section{牛顿法的严格数学建模(复习)}

Newton法是比梯度下降更高效的优化方法,核心是通过\textbf{函数的局部二次近似}确定搜索方向,兼具“局部快速收敛”与“全局有效下降”的特性,其核心逻辑围绕“二阶展开$\to$牛顿步$\to$收敛性”展开:

\subsection{核心思路:函数的局部二次近似}

梯度下降仅用“一阶信息(梯度)”将函数局部近似为线性函数,而Newton法引入“二阶信息(Hessian矩阵)”,将函数局部近似为\textbf{二次函数}(更贴合非凸函数的局部曲率),具体如下:

对迭代点$x_k$,将目标函数$f(x_k+p)$在$x_k$处做二阶泰勒展开($p$为搜索方向向量):
\[
f(x_k+p) \approx f(x_k) + g_k^\top p + \frac{1}{2}p^\top H_k p
\]
其中:
\begin{itemize}
    \item $g_k = \nabla f(x_k)$是$f(x)$在$x_k$处的梯度(一阶导数);
    \item $H_k = \nabla^2 f(x_k)$是$f(x)$在$x_k$处的Hessian矩阵(二阶导数矩阵),反映函数在$x_k$处的局部曲率。
\end{itemize}

Newton法的核心是:\textbf{最小化上述二次近似函数},直接求解使近似函数最小的搜索方向$p$。

\subsection{牛顿步(Newton Step)的推导}

对二阶近似函数关于$p$求导,并令导数为0(二次函数的极值点条件):
\[
\frac{\partial}{\partial p}\left[ f(x_k) + g_k^\top p + \frac{1}{2}p^\top H_k p \right] = g_k + H_k p = 0
\]
若Hessian矩阵\textbf{正定}($H_k \succ 0$,保证二次近似函数是凸函数,极值点为最小值点),则可解出唯一的搜索方向——\textbf{牛顿步}:
\[
p_k = -H_k^{-1} g_k
\]

\subsection{牛顿步的下降性}

牛顿步能保证是“下降方向”的前提是$H_k \succ 0$,证明如下:
计算梯度与牛顿步的内积(判断方向是否下降的核心指标,内积$<0$则为下降方向):
\[
g_k^\top p_k = g_k^\top \left( -H_k^{-1} g_k \right)
\]
因$H_k \succ 0$,其逆矩阵$H_k^{-1}$也正定,故对任意非零向量$g_k$,有$g_k^\top H_k^{-1} g_k > 0$,因此:
\[
g_k^\top p_k < 0
\]
即牛顿步满足“下降方向”的核心条件。

\subsection{局部二次收敛:牛顿法的核心优势}

当迭代点足够靠近最优解$x^*$时,Newton法会呈现\textbf{二次收敛}(收敛速度远快于梯度下降的线性收敛),严格定义如下:

\subsubsection{收敛条件}
若满足以下两个前提:
\begin{enumerate}
    \item Hessian矩阵$H(x)$在$x^*$的邻域内\textbf{Lipschitz连续}(曲率变化平缓);
    \item 初始迭代点$x_0$足够靠近$x^*$(进入“局部收敛域”)。
\end{enumerate}

\subsubsection{二次收敛公式}
此时存在常数$C > 0$,使得迭代误差满足:
\[
\| x_{k+1} - x^* \| \leq C \cdot \| x_k - x^* \|^2
\]

\subsubsection{直观意义}
二次收敛意味着“每次迭代后,误差的有效位数会翻倍”——例如:若第$k$步误差为$10^{-2}$,第$k+1$步误差可降至$10^{-4}$,第$k+2$步可降至$10^{-8}$,接近最优解时收敛极快。

牛顿法也有缺陷,数学层面的问题包括:
\begin{enumerate}
    \item \textbf{Hessian构造与求解成本高}:
    \begin{itemize}
        \item 构造Hessian矩阵$H_k$需计算$n(n+1)/2$个二阶偏导数(复杂度$O(n^2)$);
        \item 求解线性方程组$H_k p_k = -g_k$需$O(n^3)$复杂度(如LU分解),当$n$较大(如大规模优化)时计算成本不可承受。
    \end{itemize}
    \item \textbf{Hessian不定导致方向非下降}:
    若$H_k$不定(非正定时),牛顿方向$p_k = -H_k^{-1} g_k$可能不满足“下降方向”条件(即$g_k^\top p_k \geq 0$),导致迭代点$x_{k+1}$处$f(x_{k+1}) > f(x_k)$,算法不稳定。
\end{enumerate}

\section{拟牛顿法}

BFGS(Broyden-Fletcher-Goldfarb-Shanno)是最常用的拟牛顿方法之一,主要用于求解无约束优化问题。拟牛顿法的目标是通过逐步逼近目标函数的Hessian矩阵来优化目标函数,而无需显式计算二阶导数。其核心思想是通过逐步更新一个近似Hessian矩阵,使得每一步的更新尽可能接近真实的Hessian矩阵。

\subsection{拟牛顿法的基本框架}

考虑一个无约束优化问题:
\[
\min_{x} f(x),
\]
其中$f(x)$是一个可微的目标函数,$x \in \mathbb{R}^n$是优化变量。拟牛顿法的基本思路是通过计算一系列梯度来逼近目标函数的Hessian矩阵,而不是直接计算Hessian矩阵。

牛顿法的更新公式为:
\[
x_{k+1} = x_k - H_k^{-1} g_k,
\]
其中$H_k$是$f(x)$在$x_k$处的Hessian矩阵。由于计算Hessian矩阵代价较高,拟牛顿法通过构造一个近似矩阵$B_k$来代替真实的$H_k$,从而避免显式计算它。

\subsection{BFGS方法}

BFGS是一种常用的拟牛顿方法,给出了如何通过一系列梯度信息更新一个Hessian近似矩阵$B_k$的规则。其核心思想是通过引入变量$s_k = x_{k+1} - x_k$和$y_k = g_{k+1} - g_k$来更新近似Hessian矩阵。

\subsubsection{BFGS 更新公式}

BFGS方法通过以下更新公式计算$B_{k+1}$:
\[
B_{k+1} = B_k + \frac{y_k y_k^\top}{y_k^\top s_k} - \frac{B_k s_k s_k^\top B_k}{s_k^\top B_k s_k},
\]
其中:
\begin{itemize}
    \item $B_k$是第$k$步的Hessian近似。
    \item $s_k = x_{k+1} - x_k$ 是变量的变化。
    \item $y_k = g_{k+1} - g_k$ 是梯度的变化。
\end{itemize}

该公式满足\textbf{割线方程}$B_{k+1}s_k = y_k$,这是对Hessian矩阵$H$性质$H s \approx y$的近似。

公式中两个修正项的直观解释:
\begin{itemize}
    \item 第一项$\frac{y_k y_k^\top}{y_k^\top s_k}$:引入了最新的梯度变化信息$y_k$和步长信息$s_k$,是对当前Hessian近似的主要校正。
    \item 第二项$-\frac{B_k s_k s_k^\top B_k}{s_k^\top B_k s_k}$:移除了旧的、与新步长$s_k$方向相关的信息,为新的信息腾出空间。
\end{itemize}

\begin{remark}[从矩阵秩的角度理解这个式子]
\begin{enumerate}
    \item \textbf{秩1矩阵的基本性质}:若 $ a $ 是一个非零向量,则矩阵 $ a a^\top $ 是\textbf{秩1矩阵}(因为其列向量都可由 $ a $ 线性表示,秩最多为1)。除以一个非零标量(内积 $ a^\top b $ 是标量)不会改变其秩,因此形如 $ \frac{a a^\top}{a^\top b} $ 的矩阵仍为\textbf{秩1矩阵}。
    \item \textbf{分解式子中的秩1项}:
    对于更新式 $ B_{k+1} = B_k + \frac{y_k y_k^\top}{y_k^\top s_k} - \frac{B_k s_k s_k^\top B_k}{s_k^\top B_k s_k} $,我们分别分析两个修正项:
    \begin{itemize}
        \item 项1:$ \frac{y_k y_k^\top}{y_k^\top s_k} $ 是\textbf{秩1矩阵}(因 $ y_k $ 是非零向量,且 $ y_k^\top s_k \neq 0 $)。
        \item 项2:令 $ v_k = B_k s_k $($ v_k $ 是向量),则项2可表示为 $ \frac{v_k v_k^\top}{v_k^\top s_k} $,也是\textbf{秩1矩阵}(因 $ v_k $ 非零且 $ v_k^\top s_k \neq 0 $)。
    \end{itemize}
    \item \textbf{秩的变化:“秩2修正”}:整个更新式可看作对 $ B_k $ 进行\textbf{两个秩1矩阵的加减操作},即:
    \[ B_{k+1} = B_k + (\text{秩1矩阵}) - (\text{秩1矩阵}) \]
    根据矩阵秩的不等式:若 $ A, C $ 是秩分别为 $ r_A, r_C $ 的矩阵,则 $ \text{rank}(A + C) \leq r_A + r_C $,且 $ \text{rank}(A - C) \geq |r_A - r_C| $。
    这里两个修正项都是秩1矩阵,且在拟牛顿法的背景下(满足拟牛顿条件 $ B_{k+1} s_k = y_k $),这两个秩1矩阵\textbf{线性无关},因此对 $ B_k $ 的修正属于\textbf{秩2修正}(即 $ B_{k+1} $ 与 $ B_k $ 的秩差最多为2)。
    \item \textbf{拟牛顿法中的秩意义}:在拟牛顿法中,$ B_k $ 是Hessian矩阵的近似。通过这种\textbf{秩2更新},可以在保持矩阵对称性(或正定性,在一定条件下)的同时,逐步调整 $ B_k $ 的秩结构,使其逼近真实的Hessian矩阵(在算法收敛时)。
    初始时 $ B_0 $ 通常取满秩矩阵(如单位矩阵),经过每次秩2修正后,$ B_k $ 的秩会逐渐适应Hessian的秩特性,从而实现高效的梯度近似迭代。
\end{enumerate}
综上,从秩的角度看,该式子是对 $ B_k $ 进行\textbf{秩2修正}(由两个秩1矩阵的加减构成),通过这种修正来近似Hessian矩阵,同时满足拟牛顿条件,保证迭代的有效性。
\end{remark}

\subsubsection{推导过程}

\begin{itemize}
    \item \textbf{目标}:构造一个新的$B_{k+1}$来近似Hessian矩阵,使得在每一步迭代中,拟牛顿方法的更新步长和真实牛顿法的更新步长尽量相似。
    \item \textbf{确保更新的对称性和正定性}:
    \begin{itemize}
        \item 对称性:若$B_k$对称,则$B_{k+1}$也是对称的,因为$y_k y_k^\top$和$B_k s_k s_k^\top B_k$都是对称矩阵。
        \item 正定性:可以证明,如果初始矩阵$B_0$是正定的,并且线搜索满足Wolfe条件,那么后续所有的$B_k$都会保持正定。
    \end{itemize}
\end{itemize}

\begin{remark}[\texorpdfstring{$\mathbf{B}_k \to \mathbf{B}_{k+1}$}{Bk -> Bk+1}构造推导(BFGS框架)]
\textbf{一、前提与核心定义(统一符号)}
首先明确推导所需的基础记号、约束条件与优化目标:
\begin{enumerate}
    \item \textbf{迭代核心变量}:
    \begin{itemize}
        \item 位移向量:$\mathbf{s} = \mathbf{x}_{k+1} - \mathbf{x}_k$(第$k$到$k+1$步的迭代位移);
        \item 梯度差分:$\mathbf{y} = \nabla f(\mathbf{x}_{k+1}) - \nabla f(\mathbf{x}_k) = \mathbf{g}_{k+1} - \mathbf{g}_k$(对应Hessian作用于位移的近似);
    \end{itemize}
    \item \textbf{初始近似矩阵}:
    \begin{itemize}
        \item 已知$\mathbf{B}_k \succ 0$(对称正定的Hessian近似矩阵),其逆矩阵为$\mathbf{G}_k = \mathbf{B}_k^{-1}$(对称正定的逆Hessian近似);
    \end{itemize}
    \item \textbf{加权范数(度量“矩阵接近度”)}:
    为量化$\mathbf{B}_{k+1}$与$\mathbf{B}_k$的“改变量”,定义基于$\mathbf{G}_k$的加权内积与范数(保证更新对\textbf{线性变换(单位缩放、坐标变化)不变},优于仅对正交变换不变的Frobenius范数$\|\cdot\|_F$):
    \[
    \langle \mathbf{X}, \mathbf{Z} \rangle_{\mathbf{G}_k} = \text{tr}(\mathbf{G}_k \mathbf{X} \mathbf{G}_k \mathbf{Z}), \quad \|\mathbf{X}\|_{\mathbf{G}_k}^2 = \langle \mathbf{X}, \mathbf{X} \rangle_{\mathbf{G}_k}
    \]
    其中$\text{tr}(\cdot)$为矩阵迹算子;
    \item \textbf{优化目标}:
    寻找对称矩阵$\mathbf{B}_{k+1}$,满足两大核心约束:
    \begin{itemize}
        \item 割线约束:$\mathbf{B}_{k+1} \mathbf{s} = \mathbf{y}$(拟合“平均曲率”,即拟牛顿条件);
        \item 最小改变量:在满足割线约束的所有对称矩阵中,$\mathbf{B}_{k+1}$与$\mathbf{B}_k$在$\|\cdot\|_{\mathbf{G}_k}$范数下最接近。
    \end{itemize}
\end{enumerate}

\textbf{二、第一步:“先忘”——构造子空间最优矩阵$\mathbf{B}'$}
首先在\textbf{子空间$\mathcal{S}_0 = \{\mathbf{B} \mid \mathbf{B} = \mathbf{B}^\top, \mathbf{B} \mathbf{s} = 0\}$} 中,找到与$\mathbf{B}_k$最接近的矩阵$\mathbf{B}'$(即“擦除”$\mathbf{B}_k$中沿$\mathbf{s}$方向的曲率信息,使其满足$\mathbf{B}' \mathbf{s} = 0$)。

\textbf{2.1 优化问题转化(白化变换)}
直接求解$\mathbf{B}$的优化问题较复杂,通过“白化变换”将其转化为Frobenius范数下的简化问题(利用$\mathbf{G}_k = \mathbf{B}_k^{-1}$的正定性):
\begin{enumerate}
    \item \textbf{白化矩阵定义}:
    令$\mathbf{C} = \mathbf{G}_k^{1/2} \mathbf{B} \mathbf{G}_k^{1/2}$,其中$\mathbf{G}_k^{1/2}$是$\mathbf{G}_k$的对称平方根(因$\mathbf{G}_k \succ 0$,平方根存在且对称正定);
    \item \textbf{初始白化矩阵}:
    代入$\mathbf{G}_k = \mathbf{B}_k^{-1}$,得初始白化矩阵:
    \[
    \mathbf{C}_k = \mathbf{G}_k^{1/2} \mathbf{B}_k \mathbf{G}_k^{1/2} = \mathbf{G}_k^{1/2} \mathbf{G}_k^{-1} \mathbf{G}_k^{1/2} = \mathbf{I}
    \]
    其中$\mathbf{I}$为单位矩阵;
    \item \textbf{位移白化}:
    令$\tilde{\mathbf{s}} = \mathbf{G}_k^{-1/2} \mathbf{s}$(将位移向量转化到白化空间)。
\end{enumerate}

\textbf{(1)范数等价转化}
利用\textbf{迹的循环不变性}($\text{tr}(\mathbf{AB}) = \text{tr}(\mathbf{BA})$),可证明$\mathbf{B}$与$\mathbf{B}_k$的加权距离等价于白化矩阵的Frobenius距离:
\[
\|\mathbf{B} - \mathbf{B}_k\|_{\mathbf{G}_k}^2 = \text{tr}(\mathbf{G}_k (\mathbf{B}-\mathbf{B}_k) \mathbf{G}_k (\mathbf{B}-\mathbf{B}_k))
\]
将$\mathbf{B} - \mathbf{B}_k = \mathbf{G}_k^{-1/2} (\mathbf{C} - \mathbf{C}_k) \mathbf{G}_k^{-1/2}$(由$\mathbf{C} = \mathbf{G}_k^{1/2} \mathbf{B} \mathbf{G}_k^{1/2}$变形得)代入,展开后利用迹的循环不变性抵消$\mathbf{G}_k^{1/2}$与$\mathbf{G}_k^{-1/2}$,最终得:
\[
\|\mathbf{B} - \mathbf{B}_k\|_{\mathbf{G}_k}^2 = \|\mathbf{C} - \mathbf{I}\|_F^2
\]

\textbf{(2)约束等价转化}
割线约束$\mathbf{B} \mathbf{s} = 0$可转化为白化空间的约束:
\[
\mathbf{B} \mathbf{s} = 0 \implies \mathbf{G}_k^{1/2} \mathbf{B} \mathbf{G}_k^{1/2} \cdot \mathbf{G}_k^{-1/2} \mathbf{s} = \mathbf{G}_k^{1/2} \mathbf{B} \mathbf{s} = 0 \implies \mathbf{C} \tilde{\mathbf{s}} = 0
\]

\textbf{(3)转化后的优化问题}
原问题(加权范数下的约束优化)等价为Frobenius范数下的简化问题:
\[
\min_{\mathbf{C} = \mathbf{C}^\top} \frac{1}{2} \|\mathbf{C} - \mathbf{I}\|_F^2 \quad \text{s.t.} \quad \mathbf{C} \tilde{\mathbf{s}} = 0
\]

\textbf{2.2 求解白化空间优化问题(正交相似变换)}
通过构造正交基对齐约束,简化$\mathbf{C}$的结构并求解最小值:
\begin{enumerate}
    \item \textbf{构造正交矩阵$\mathbf{Q}$}:
    取正交矩阵$\mathbf{Q} = [\mathbf{q}_1, \mathbf{q}_2, \dots, \mathbf{q}_n]$(满足$\mathbf{Q}^\top \mathbf{Q} = \mathbf{Q} \mathbf{Q}^\top = \mathbf{I}$),其中:
    \begin{itemize}
        \item 第一列$\mathbf{q}_1 = \frac{\tilde{\mathbf{s}}}{\|\tilde{\mathbf{s}}\|}$($\tilde{\mathbf{s}}$的单位向量,因$\mathbf{s} \neq 0$且$\mathbf{G}_k^{-1/2}$可逆,故$\|\tilde{\mathbf{s}}\| \neq 0$);
        \item 其余列$\mathbf{q}_2, \dots, \mathbf{q}_n$为$\mathbf{q}_1$正交补空间的标准正交基。
    \end{itemize}
    此时$\mathbf{Q}^\top \tilde{\mathbf{s}} = \|\tilde{\mathbf{s}}\| \mathbf{e}_1$($\mathbf{e}_1 = (1,0,\dots,0)^\top$为标准基向量),因$\mathbf{q}_1^\top \tilde{\mathbf{s}} = \|\tilde{\mathbf{s}}\|$,$\mathbf{q}_j^\top \tilde{\mathbf{s}} = 0$($j \geq 2$)。

    \item \textbf{分块对角化$\mathbf{C}$}:
    令$\mathbf{D} = \mathbf{Q}^\top \mathbf{C} \mathbf{Q}$(正交相似变换),则:
    \begin{itemize}
        \item 对称性:因$\mathbf{C} = \mathbf{C}^\top$且$\mathbf{Q}$正交,故$\mathbf{D} = \mathbf{D}^\top$;
        \item 范数不变性:Frobenius范数在正交变换下不变,即$\|\mathbf{C} - \mathbf{I}\|_F^2 = \|\mathbf{D} - \mathbf{I}\|_F^2$;
        \item 约束转化:$\mathbf{C} \tilde{\mathbf{s}} = 0 \implies \mathbf{D} (\mathbf{Q}^\top \tilde{\mathbf{s}}) = 0 \implies \mathbf{D} (\|\tilde{\mathbf{s}}\| \mathbf{e}_1) = 0 \implies \mathbf{D} \mathbf{e}_1 = 0$(因$\|\tilde{\mathbf{s}}\| \neq 0$)。
    \end{itemize}

    \item \textbf{最小化目标函数}:
    由$\mathbf{D} = \mathbf{D}^\top$且$\mathbf{D} \mathbf{e}_1 = 0$,可知$\mathbf{D}$的第一列全为0,对称性导致第一行也全为0,故$\mathbf{D}$可分块为:
    \[
    \mathbf{D} = \begin{bmatrix} 0 & \mathbf{0}^\top \\ \mathbf{0} & \mathbf{M} \end{bmatrix}, \quad \mathbf{M} = \mathbf{M}^\top \in \mathbb{R}^{(n-1) \times (n-1)}
    \]
    代入目标函数:
    \[
    \|\mathbf{D} - \mathbf{I}\|_F^2 = \left\| \begin{bmatrix} -1 & \mathbf{0}^\top \\ \mathbf{0} & \mathbf{M} - \mathbf{I}_{n-1} \end{bmatrix} \right\|_F^2 = 1 + \|\mathbf{M} - \mathbf{I}_{n-1}\|_F^2
    \]
    要最小化该式,需$\|\mathbf{M} - \mathbf{I}_{n-1}\|_F^2 = 0$,即$\mathbf{M} = \mathbf{I}_{n-1}$。因此,白化空间的最优解为:
    \[
    \mathbf{D}^* = \begin{bmatrix} 0 & \mathbf{0}^\top \\ \mathbf{0} & \mathbf{I}_{n-1} \end{bmatrix} = \mathbf{I} - \mathbf{e}_1 \mathbf{e}_1^\top
    \]
\end{enumerate}

\textbf{2.3 反变换回$\mathbf{B}'$(从白化空间到原空间)}
\begin{enumerate}
    \item \textbf{从$\mathbf{D}^*$到$\mathbf{C}^*$}:
    由$\mathbf{D}^* = \mathbf{Q}^\top \mathbf{C}^* \mathbf{Q}$,得$\mathbf{C}^* = \mathbf{Q} \mathbf{D}^* \mathbf{Q}^\top$。代入$\mathbf{D}^* = \mathbf{I} - \mathbf{e}_1 \mathbf{e}_1^\top$,并利用$\mathbf{Q} \mathbf{e}_1 = \mathbf{q}_1 = \frac{\tilde{\mathbf{s}}}{\|\tilde{\mathbf{s}}\|}$,得:
    \[
    \mathbf{C}^* = \mathbf{I} - \mathbf{Q} \mathbf{e}_1 \mathbf{e}_1^\top \mathbf{Q}^\top = \mathbf{I} - \frac{\tilde{\mathbf{s}} \tilde{\mathbf{s}}^\top}{\|\tilde{\mathbf{s}}\|^2}
    \]

    \item \textbf{从$\mathbf{C}^*$到$\mathbf{B}'$}:
    由$\mathbf{C}^* = \mathbf{G}_k^{1/2} \mathbf{B}' \mathbf{G}_k^{1/2}$,得$\mathbf{B}' = \mathbf{G}_k^{-1/2} \mathbf{C}^* \mathbf{G}_k^{-1/2}$。代入$\mathbf{C}^*$并分拆两项展开:
    \begin{itemize}
        \item 第一项:$\mathbf{G}_k^{-1/2} \mathbf{I} \mathbf{G}_k^{-1/2} = \mathbf{G}_k^{-1} = \mathbf{B}_k$(因$\mathbf{G}_k = \mathbf{B}_k^{-1}$);
        \item 第二项:$\mathbf{G}_k^{-1/2} \cdot \frac{\tilde{\mathbf{s}} \tilde{\mathbf{s}}^\top}{\|\tilde{\mathbf{s}}\|^2} \cdot \mathbf{G}_k^{-1/2}$。
    \end{itemize}
    进一步计算第二项的分子与分母:
    \begin{itemize}
        \item 分子:$\mathbf{G}_k^{-1/2} \tilde{\mathbf{s}} = \mathbf{G}_k^{-1/2} \cdot \mathbf{G}_k^{-1/2} \mathbf{s} = \mathbf{G}_k^{-1} \mathbf{s} = \mathbf{B}_k \mathbf{s}$,故分子为$(\mathbf{B}_k \mathbf{s})(\mathbf{B}_k \mathbf{s})^\top = \mathbf{B}_k \mathbf{s} \mathbf{s}^\top \mathbf{B}_k$;
        \item 分母:$\|\tilde{\mathbf{s}}\|^2 = \tilde{\mathbf{s}}^\top \tilde{\mathbf{s}} = (\mathbf{G}_k^{-1/2} \mathbf{s})^\top (\mathbf{G}_k^{-1/2} \mathbf{s}) = \mathbf{s}^\top \mathbf{G}_k^{-1} \mathbf{s} = \mathbf{s}^\top \mathbf{B}_k \mathbf{s}$。
    \end{itemize}
    因此,“先忘”步骤的结果为:
    \[
    \mathbf{B}' = \mathbf{B}_k - \frac{\mathbf{B}_k \mathbf{s} \mathbf{s}^\top \mathbf{B}_k}{\mathbf{s}^\top \mathbf{B}_k \mathbf{s}}
    \]
    验证约束:$\mathbf{B}' \mathbf{s} = \mathbf{B}_k \mathbf{s} - \frac{\mathbf{B}_k \mathbf{s} \mathbf{s}^\top \mathbf{B}_k \mathbf{s}}{\mathbf{s}^\top \mathbf{B}_k \mathbf{s}} = \mathbf{B}_k \mathbf{s} - \mathbf{B}_k \mathbf{s} = 0$,满足子空间约束。
\end{enumerate}

\textbf{三、第二步:“再写”——构造修正项$\Delta^+$}
在$\mathbf{B}'$的基础上,添加最小改动的对称矩阵$\Delta^+$,使$\mathbf{B}_{k+1} = \mathbf{B}' + \Delta^+$满足割线约束$\mathbf{B}_{k+1} \mathbf{s} = \mathbf{y}$。

\textbf{3.1 白化空间的修正项$\Delta_{\mathbf{C}}^+$}
因$\mathbf{B}' \mathbf{s} = 0$,故割线约束等价于$\Delta^+ \mathbf{s} = \mathbf{y}$。转化到白化空间:
\begin{enumerate}
    \item \textbf{梯度差分白化}:令$\tilde{\mathbf{y}} = \mathbf{G}_k^{1/2} \mathbf{y}$(与位移白化对应);
    \item \textbf{约束转化}:$\Delta_{\mathbf{C}}^+ \tilde{\mathbf{s}} = \tilde{\mathbf{y}}$(推导同2.1.2,利用$\Delta^+ = \mathbf{G}_k^{-1/2} \Delta_{\mathbf{C}}^+ \mathbf{G}_k^{-1/2}$)。
\end{enumerate}
为满足约束且最小化改动,选择\textbf{对称秩-1矩阵}作为$\Delta_{\mathbf{C}}^+$:
\[
\Delta_{\mathbf{C}}^+ = \frac{\tilde{\mathbf{y}} \tilde{\mathbf{y}}^\top}{\tilde{\mathbf{y}}^\top \tilde{\mathbf{s}}}
\]
验证约束:$\Delta_{\mathbf{C}}^+ \tilde{\mathbf{s}} = \frac{\tilde{\mathbf{y}} (\tilde{\mathbf{y}}^\top \tilde{\mathbf{s}})}{\tilde{\mathbf{y}}^\top \tilde{\mathbf{s}}} = \tilde{\mathbf{y}}$,完全满足。

\textbf{3.2 反变换回$\Delta^+$(原空间修正项)}
由$\Delta^+ = \mathbf{G}_k^{-1/2} \Delta_{\mathbf{C}}^+ \mathbf{G}_k^{-1/2}$,代入$\Delta_{\mathbf{C}}^+$展开:
\begin{itemize}
    \item 分子:$\mathbf{G}_k^{-1/2} \tilde{\mathbf{y}} = \mathbf{G}_k^{-1/2} \cdot \mathbf{G}_k^{1/2} \mathbf{y} = \mathbf{y}$,故分子为$\mathbf{y} \mathbf{y}^\top$;
    \item 分母:$\tilde{\mathbf{y}}^\top \tilde{\mathbf{s}} = (\mathbf{G}_k^{1/2} \mathbf{y})^\top (\mathbf{G}_k^{-1/2} \mathbf{s}) = \mathbf{y}^\top \mathbf{G}_k^{1/2} \mathbf{G}_k^{-1/2} \mathbf{s} = \mathbf{y}^\top \mathbf{s}$。
\end{itemize}
因此,原空间的修正项为:
\[
\Delta^+ = \frac{\mathbf{y} \mathbf{y}^\top}{\mathbf{y}^\top \mathbf{s}}
\]
验证约束:$\Delta^+ \mathbf{s} = \frac{\mathbf{y} (\mathbf{y}^\top \mathbf{s})}{\mathbf{y}^\top \mathbf{s}} = \mathbf{y}$,满足割线约束要求。

\textbf{四、最终构造:$\mathbf{B}_{k+1}$的表达式与正定性验证}
\textbf{4.1 $\mathbf{B}_{k+1}$的最终公式}
合并“先忘”步骤的$\mathbf{B}'$与“再写”步骤的$\Delta^+$,得到BFGS方法中Hessian近似矩阵的更新公式(B-form):
\[
\boxed{\mathbf{B}_{k+1} = \mathbf{B}_k - \frac{\mathbf{B}_k \mathbf{s} \mathbf{s}^\top \mathbf{B}_k}{\mathbf{s}^\top \mathbf{B}_k \mathbf{s}} + \frac{\mathbf{y} \mathbf{y}^\top}{\mathbf{y}^\top \mathbf{s}}}
\]

\textbf{4.2 正定性验证(保证下降方向)}
若满足\textbf{曲率条件$\mathbf{s}^\top \mathbf{y} > 0$}(强Wolfe线搜索可保证),则$\mathbf{B}_{k+1} \succ 0$(对称正定),证明如下:
对任意非零向量$\mathbf{z}$,代入$\mathbf{B}_{k+1}$的表达式得:
\[
\mathbf{z}^\top \mathbf{B}_{k+1} \mathbf{z} = \mathbf{z}^\top \mathbf{B}_k \mathbf{z} - \frac{(\mathbf{z}^\top \mathbf{B}_k \mathbf{s})^2}{\mathbf{s}^\top \mathbf{B}_k \mathbf{s}} + \frac{(\mathbf{z}^\top \mathbf{y})^2}{\mathbf{y}^\top \mathbf{s}}
\]
由\textbf{Cauchy-Schwarz不等式},$\frac{(\mathbf{z}^\top \mathbf{B}_k \mathbf{s})^2}{\mathbf{s}^\top \mathbf{B}_k \mathbf{s}} \leq \mathbf{z}^\top \mathbf{B}_k \mathbf{z}$(等号仅当$\mathbf{z}$与$\mathbf{s}$线性相关时成立)。结合$\mathbf{s}^\top \mathbf{y} > 0$,第二项$\frac{(\mathbf{z}^\top \mathbf{y})^2}{\mathbf{y}^\top \mathbf{s}} \geq 0$,且仅当$\mathbf{z}^\top \mathbf{y} = 0$时为0。
\begin{itemize}
    \item 若$\mathbf{z}^\top \mathbf{y} \neq 0$:$\mathbf{z}^\top \mathbf{B}_{k+1} \mathbf{z} > 0$;
    \item 若$\mathbf{z}^\top \mathbf{y} = 0$:$\mathbf{z}^\top \mathbf{B}_{k+1} \mathbf{z} = \mathbf{z}^\top \mathbf{B}_k \mathbf{z} - \frac{(\mathbf{z}^\top \mathbf{B}_k \mathbf{s})^2}{\mathbf{s}^\top \mathbf{B}_k \mathbf{s}} \geq 0$,且仅当$\mathbf{z} = 0$时取等号(因$\mathbf{B}_k \succ 0$)。
\end{itemize}
综上,$\mathbf{B}_{k+1} \succ 0$,保证后续搜索方向$\mathbf{p}_k = -\mathbf{G}_{k+1} \mathbf{g}_k$($\mathbf{G}_{k+1} = \mathbf{B}_{k+1}^{-1}$)为下降方向。
\end{remark}

\subsection{准牛顿法的其他变种}

除了BFGS,拟牛顿法还有其他变种,例如DFP(Davidon-Fletcher-Powell)方法。DFP的更新规则与BFGS相似,但它直接更新Hessian的逆矩阵近似$G_k$。DFP方法的更新公式为:
\[
G_{k+1} = G_k + \frac{s_k s_k^\top}{s_k^\top y_k} - \frac{G_k y_k y_k^\top G_k}{y_k^\top G_k y_k}
\]

DFP与BFGS实际上是\textbf{对偶关系}。BFGS更新Hessian的近似$B_k$,而DFP更新其逆的近似$G_k$。

\section{收敛性}

\subsection{BFGS的收敛性}
BFGS的收敛性需分\textbf{目标函数类型}(二次/非二次)和\textbf{线搜索策略}(精确/不精确)讨论,核心依赖"拟牛顿条件"和"近似Hessian矩阵的正定性"。

\subsubsection{1. 二次函数下的收敛性}
若目标函数为\textbf{严格凸二次函数}(即Hessian矩阵$H$正定且恒定),且采用\textbf{精确线搜索}(即每次步长选择使目标函数沿搜索方向最小化),BFGS具有以下收敛性质:
\begin{itemize}
    \item \textbf{有限终止性}:对于$n$维二次函数,BFGS最多迭代$n$步即可收敛到全局最优解。\\
    原因:二次函数的Hessian恒定,BFGS通过迭代更新的$B_k$会逐步逼近真实$H$,当$B_k = H$时,一步即可达到最优,而理论上最多$n$步可完成逼近。
    \item \textbf{正定性保持}:若初始近似矩阵$B_0$正定,则所有迭代过程中的$B_k$均保持正定,确保搜索方向始终为"下降方向"(即$-B_k^{-1} g_k$与梯度反向),避免迭代发散。
\end{itemize}

\subsubsection{2. 非二次函数下的收敛性}
实际优化问题多为非二次函数(如机器学习中的损失函数),此时需假设目标函数满足\textbf{光滑性和凸性条件}(如二阶导数Lipschitz连续、Hessian在最优解附近正定),BFGS的收敛性质为:
\begin{itemize}
    \item \textbf{局部收敛性}:若初始点$x_0$足够接近全局最优解$x^*$,且采用精确/不精确线搜索(如Wolfe条件),BFGS会收敛到$x^*$。\\
    关键前提:最优解$x^*$处的Hessian$H^* = \nabla^2 f(x^*)$正定,确保迭代过程中梯度方向的"有效性"。
    \item \textbf{全局收敛性(凸函数下)}:若目标函数为\textbf{严格凸函数},且采用精确线搜索或满足Wolfe条件的不精确线搜索,BFGS可实现\textbf{全局收敛}(即无论初始点$x_0$如何,最终均收敛到全局最优解)。\\
    非凸函数下:仅能保证\textbf{局部收敛}(可能收敛到局部最优解),这是多数无约束优化方法的共性(除非结合全局优化策略,如随机初始化)。
\end{itemize}

\subsection{BFGS的收敛速度:介于梯度下降与牛顿法之间,逼近牛顿法}
收敛速度衡量迭代序列$\{x_k\}$趋近最优解$x^*$的快慢,常用"收敛阶"定义(如线性收敛、超线性收敛、二次收敛)。BFGS的收敛速度需结合函数类型分析:

\subsubsection{1. 收敛阶的定义(参考基准)}
\begin{itemize}
    \item \textbf{线性收敛}:存在常数$0 < c < 1$,使$\lim_{k \to \infty} \frac{\|x_{k+1} - x^*\|}{\|x_k - x^*\|} = c$(如梯度下降法)。
    \item \textbf{超线性收敛}:$\lim_{k \to \infty} \frac{\|x_{k+1} - x^*\|}{\|x_k - x^*\|} = 0$(比线性快)。
    \item \textbf{二次收敛}:存在常数$c > 0$,使$\lim_{k \to \infty} \frac{\|x_{k+1} - x^*\|}{\|x_k - x^*\|^2} = c$(如牛顿法,最快)。
\end{itemize}

\subsubsection{2. BFGS的收敛速度}
\begin{itemize}
    \item \textbf{二次函数下}:若采用精确线搜索,BFGS对$n$维二次函数是\textbf{有限收敛}(最多$n$步),本质上比二次收敛更快(无需无限迭代)。
    \item \textbf{非二次函数下}:在最优解$x^*$附近(满足Hessian正定且Lipschitz连续),BFGS是\textbf{超线性收敛}。\\
    BFGS的近似Hessian$B_k$满足$\lim_{k \to \infty} \frac{\|(B_k - H^*)p_k\|}{\|p_k\|} = 0$(Dennis-Moré条件),使得迭代步长逐渐接近牛顿法的最优步长,因此收敛速度接近牛顿法,但无需显式计算Hessian。
\end{itemize}

\subsection{BFGS的最优性:理论性质与实际性能的"最优平衡"}
BFGS的"最优性"并非指它在所有场景下都是"最好"的优化方法,而是指它在\textbf{计算成本、数值稳定性、收敛性能}三者间达到了工程应用中的"最优权衡",具体体现在以下方面:

\subsubsection{1. 理论最优性:满足拟牛顿法的核心目标}
拟牛顿法的核心目标是"用梯度信息逼近Hessian,以降低牛顿法的计算成本",BFGS在这一目标下满足:
\begin{itemize}
    \item \textbf{拟牛顿条件的严格满足}:每次更新的$B_{k+1}$均严格满足$B_{k+1} s_k = y_k$(这是逼近Hessian的核心条件),确保$B_k$对真实Hessian的逼近是"有效"的。
    \item \textbf{正定性的严格保持}:若$B_0$正定且线搜索满足"充分下降条件"(如Wolfe条件),则所有$B_k$均正定,避免搜索方向变为"上升方向"(这是DFP等方法有时难以保证的,BFGS的数值稳定性更优)。
\end{itemize}

\subsubsection{2. 实际应用中的最优性:兼顾效率与稳定性}
\begin{itemize}
    \item \textbf{计算成本最优}:
    \begin{itemize}
        \item 每次迭代仅需计算1次梯度(成本$O(n)$),更新$B_k$的成本为$O(n^2)$(无需计算Hessian的$O(n^3)$成本)。
        \item 存储成本为$O(n^2)$(仅需存储$B_k$或其逆矩阵$G_k$),适用于中大规模问题($n$从几百到几万)。
    \end{itemize}
    \item \textbf{数值稳定性最优}:\\
    相比DFP、SR1等其他拟牛顿法,BFGS对"线搜索误差"和"梯度噪声"的容忍度更高,即使采用不精确线搜索(实际应用中常用),也不易出现$B_k$奇异或迭代发散的情况。
    \item \textbf{收敛性能最优}:\\
    在相同计算成本下,BFGS的收敛速度远快于梯度下降(线性收敛),且接近牛顿法(二次收敛),同时避免了牛顿法计算Hessian和求解线性方程组的高昂成本,因此在机器学习、工程优化等领域成为"首选方法"之一。
\end{itemize}

\begin{definition}[Wolfe条件]
Wolfe条件是\textbf{不精确线搜索}的核心准则。设目标函数为$f(x)$,梯度为$g(x)$,搜索方向为$p_k$,则步长$\alpha_k$需满足:
\begin{enumerate}
    \item \textbf{Armijo条件(充分下降条件)}:
    \[ f(x_k + \alpha_k p_k) \leq f(x_k) + c_1 \alpha_k g_k^\top p_k \]
    \item \textbf{曲率条件(步长不太小条件)}:
    \[ g(x_k + \alpha_k p_k)^\top p_k \geq c_2 g_k^\top p_k \]
\end{enumerate}
其中 $0 < c_1 < c_2 < 1$(典型值 $c_1 = 10^{-4}, c_2 = 0.9$)。
\end{definition}

\section{伪代码}

\begin{algorithm}[BFGS拟牛顿优化算法]
\textbf{输入}:目标函数$f(x)$,梯度函数$\nabla f(x)$,初始点$x_0 \in \mathbb{R}^n$,收敛阈值$\epsilon > 0$,最大迭代次数$T$ \\
\textbf{输出}:最优解近似$x^*$

\begin{enumerate}
    \item \textbf{初始化}:
    \begin{itemize}
        \item 令$k = 0$,$x_k = x_0$
        \item 计算梯度$g_k = \nabla f(x_k)$
        \item 初始化Hessian逆矩阵近似$G_k = I_n$($n \times n$单位矩阵)
    \end{itemize}

    \item \textbf{收敛判断}:
    若$\|g_k\| < \epsilon$,则输出$x^* = x_k$并终止

    \item \textbf{迭代主循环}(当$k < T$时):
    \begin{enumerate}
        \item \textbf{计算搜索方向}:$p_k = -G_k g_k$(下降方向)
        \item \textbf{线搜索}:寻找步长$\alpha_k > 0$,使其满足Wolfe条件:
        \[
        \begin{cases}
        f(x_k + \alpha_k p_k) \leq f(x_k) + c_1 \alpha_k g_k^\top p_k \\
        \nabla f(x_k + \alpha_k p_k)^\top p_k \geq c_2 g_k^\top p_k
        \end{cases}
        \]
        \item \textbf{更新迭代点}:$x_{k+1} = x_k + \alpha_k p_k$
        \item \textbf{更新梯度}:$g_{k+1} = \nabla f(x_{k+1})$
        \item \textbf{计算增量}:
        \begin{itemize}
            \item 变量增量:$s_k = x_{k+1} - x_k$
            \item 梯度增量:$y_k = g_{k+1} - g_k$
        \end{itemize}
        \item \textbf{检查正定性条件}:
        若$y_k^\top s_k \leq 0$(不满足曲率条件),则令$G_{k+1} = G_k$(跳过更新);
        否则,按BFGS公式更新Hessian逆近似:
        \[
        G_{k+1} = \left(I - \frac{s_k y_k^\top}{y_k^\top s_k}\right) G_k \left(I - \frac{y_k s_k^\top}{y_k^\top s_k}\right) + \frac{s_k s_k^\top}{y_k^\top s_k}
        \]
        \item \textbf{迭代更新}:$k = k + 1$,返回步骤2
    \end{enumerate}

    \item \textbf{终止}:若达到最大迭代次数,输出$x^* = x_k$
\end{enumerate}
\end{algorithm}

\begin{remark}[逆Hessian近似形式(G-form)的推导与使用原因]
在BFGS拟牛顿法中,\textbf{逆Hessian近似形式(G-form)} 指直接对逆Hessian近似矩阵$\mathbf{G}_k = \mathbf{B}_k^{-1}$($\mathbf{B}_k$为Hessian近似矩阵)进行更新,而非对$\mathbf{B}_k$(B-form)更新。以下先推导G-form的数学表达式,再详细说明为何优先使用逆BFGS形式。

\textbf{一、G-form(逆Hessian近似)的数学推导}
G-form的更新公式需从B-form($\mathbf{B}_{k+1}$的更新)出发,利用\textbf{Woodbury矩阵求逆公式}(适用于秩修正矩阵的逆计算)推导$\mathbf{G}_{k+1} = \mathbf{B}_{k+1}^{-1}$的表达式,核心步骤如下:

\textbf{1. 已知前提与工具}
\begin{itemize}
    \item \textbf{B-form更新公式}(已推导的Hessian近似更新):
    \[
    \mathbf{B}_{k+1} = \mathbf{B}_k - \frac{\mathbf{B}_k \mathbf{s}_k \mathbf{s}_k^\top \mathbf{B}_k}{\mathbf{s}_k^\top \mathbf{B}_k \mathbf{s}_k} + \frac{\mathbf{y}_k \mathbf{y}_k^\top}{\mathbf{y}_k^\top \mathbf{s}_k}
    \]
    其中$\mathbf{s}_k = \mathbf{x}_{k+1} - \mathbf{x}_k$(位移),$\mathbf{y}_k = \mathbf{g}_{k+1} - \mathbf{g}_k$(梯度差分),且$\mathbf{B}_k \succ 0$(对称正定,故$\mathbf{G}_k = \mathbf{B}_k^{-1}$存在)。

    \item \textbf{Woodbury公式}(秩-$r$修正矩阵的逆):
    若矩阵$\mathbf{A}$可逆,$\mathbf{U},\mathbf{V}$为$n \times r$矩阵,$\mathbf{C}$为$r \times r$可逆矩阵,则:
    \[
    (\mathbf{A} + \mathbf{U} \mathbf{C} \mathbf{V}^\top)^{-1} = \mathbf{A}^{-1} - \mathbf{A}^{-1} \mathbf{U} (\mathbf{C}^{-1} + \mathbf{V}^\top \mathbf{A}^{-1} \mathbf{U})^{-1} \mathbf{V}^\top \mathbf{A}^{-1}
    \]
    本推导中$\mathbf{B}_{k+1} = \mathbf{B}_k + \mathbf{U} \mathbf{M} \mathbf{U}^\top$(秩-2修正,$\mathbf{U}$为$n \times 2$矩阵,$\mathbf{M}$为$2 \times 2$对角矩阵),符合Woodbury公式适用场景。
\end{itemize}

\textbf{2. 步骤1:将B-form改写为“原矩阵+秩修正”形式}
令:
\begin{itemize}
    \item 秩修正矩阵的列向量:$\mathbf{U} = [\mathbf{y}_k, \mathbf{B}_k \mathbf{s}_k]$($n \times 2$矩阵);
    \item 对角系数矩阵:$\mathbf{M} = \text{diag}\left( \frac{1}{\mathbf{y}_k^\top \mathbf{s}_k}, -\frac{1}{\mathbf{s}_k^\top \mathbf{B}_k \mathbf{s}_k} \right)$($2 \times 2$矩阵)。
\end{itemize}
则B-form可改写为:
\[
\mathbf{B}_{k+1} = \mathbf{B}_k + \mathbf{U} \mathbf{M} \mathbf{U}^\top
\]
(验证:$\mathbf{U} \mathbf{M} \mathbf{U}^\top = \frac{\mathbf{y}_k \mathbf{y}_k^\top}{\mathbf{y}_k^\top \mathbf{s}_k} - \frac{\mathbf{B}_k \mathbf{s}_k \mathbf{s}_k^\top \mathbf{B}_k}{\mathbf{s}_k^\top \mathbf{B}_k \mathbf{s}_k}$,与B-form一致)。

\textbf{3. 步骤2:应用Woodbury公式求$\mathbf{G}_{k+1} = \mathbf{B}_{k+1}^{-1}$}
将$\mathbf{A} = \mathbf{B}_k$、$\mathbf{V} = \mathbf{U}$、$\mathbf{C} = \mathbf{M}$代入Woodbury公式,且$\mathbf{G}_k = \mathbf{B}_k^{-1}$,展开计算:
\begin{enumerate}
    \item 计算$\mathbf{C}^{-1} = \mathbf{M}^{-1}$(对角矩阵逆为对角元素倒数):
    \[
    \mathbf{M}^{-1} = \text{diag}\left( \mathbf{y}_k^\top \mathbf{s}_k, -\mathbf{s}_k^\top \mathbf{B}_k \mathbf{s}_k \right)
    \]
    \item 计算$\mathbf{V}^\top \mathbf{A}^{-1} \mathbf{U} = \mathbf{U}^\top \mathbf{G}_k \mathbf{U}$($2 \times 2$矩阵):
    \[
    \mathbf{U}^\top \mathbf{G}_k \mathbf{U} = \begin{bmatrix} \mathbf{y}_k^\top \mathbf{G}_k \mathbf{y}_k & \mathbf{y}_k^\top \mathbf{s}_k \\ \mathbf{s}_k^\top \mathbf{y}_k & \mathbf{s}_k^\top \mathbf{B}_k \mathbf{s}_k \end{bmatrix}
    \]
    (因$\mathbf{G}_k \mathbf{B}_k = \mathbf{I}$,故$\mathbf{G}_k \mathbf{B}_k \mathbf{s}_k = \mathbf{s}_k$)。
    \item 计算$\mathbf{C}^{-1} + \mathbf{U}^\top \mathbf{G}_k \mathbf{U}$(记为$\mathbf{S}$,$2 \times 2$矩阵):
    令$\rho_k = \frac{1}{\mathbf{y}_k^\top \mathbf{s}_k}$(简化符号),则$\mathbf{y}_k^\top \mathbf{s}_k = \frac{1}{\rho_k}$,代入得:
    \[
    \mathbf{S} = \begin{bmatrix} \mathbf{y}_k^\top \mathbf{G}_k \mathbf{y}_k + \frac{1}{\rho_k} & \frac{1}{\rho_k} \\ \frac{1}{\rho_k} & 0 \end{bmatrix}
    \]
    求$\mathbf{S}^{-1}$(2阶矩阵逆公式),并代入Woodbury公式展开、化简后,最终得到\textbf{G-form的紧凑更新公式}:
    \[
    \boxed{\mathbf{G}_{k+1} = \left( \mathbf{I} - \rho_k \mathbf{s}_k \mathbf{y}_k^\top \right) \mathbf{G}_k \left( \mathbf{I} - \rho_k \mathbf{y}_k \mathbf{s}_k^\top \right) + \rho_k \mathbf{s}_k \mathbf{s}_k^\top}
    \]
\end{enumerate}

\textbf{4. G-form的关键性质验证}
\begin{itemize}
    \item \textbf{割线条件}:$\mathbf{G}_{k+1} \mathbf{y}_k = \mathbf{s}_k$(与B-form的$\mathbf{B}_{k+1} \mathbf{s}_k = \mathbf{y}_k$等价,保证曲率拟合);
    \item \textbf{对称正定}:若$\mathbf{G}_k \succ 0$且$\mathbf{s}_k^\top \mathbf{y}_k > 0$(曲率条件),则$\mathbf{G}_{k+1} \succ 0$(保证搜索方向为下降方向)。
\end{itemize}

\textbf{二、总结}
逆BFGS形式(G-form)的核心价值在于\textbf{“以更低的计算/存储成本,保留牛顿方向的高质量性”}:
\begin{enumerate}
    \item 推导上,通过Woodbury公式从B-form转化而来,严格满足拟牛顿的割线条件与正定性质;
    \item 实践上,其搜索方向计算仅需矩阵-向量乘法,且L-BFGS基于G-form实现了“有限记忆+低复杂度”,成为大规模无约束优化(如深度学习、科学计算)的默认基线方法。
\end{enumerate}
相比之下,B-form因需解线性方程组、存储成本高,仅在小规模问题($n \ll 10^3$)中偶尔使用,工程价值远低于G-form。
\end{remark}

\section{L-BFGS(有限记忆二阶近似)}

L-BFGS(Limited-Memory BFGS)的核心是\textbf{放弃显式存储$n \times n$的逆Hessian近似$\mathbf{G}_k$},仅保留最近$m$对曲率信息$(\mathbf{s}_i, \mathbf{y}_i)$($m \in [5,20]$,远小于变量维度$n$),通过“两环递推”在线构造$\mathbf{G}_k \mathbf{v}$($\mathbf{v}$为任意向量,通常取梯度$\mathbf{g}_k$)的结果,实现“低内存+低复杂度”的二阶优化。

\subsection{推导起点:BFGS的G-form递推关系}
L-BFGS源于BFGS的逆Hessian更新公式(G-form)。回顾已推导的G-form:
\[
\mathbf{G}_{k+1} = \underbrace{(\mathbf{I} - \rho_k \mathbf{s}_k \mathbf{y}_k^\top)}_{V_k} \mathbf{G}_k \underbrace{(\mathbf{I} - \rho_k \mathbf{y}_k \mathbf{s}_k^\top)}_{W_k} + \underbrace{\rho_k \mathbf{s}_k \mathbf{s}_k^\top}_{R_k} \tag{1}
\]
其中:
\begin{itemize}
    \item $V_k, W_k$为“投影矩阵”(秩-$n-1$,用于消除旧曲率中与$\mathbf{s}_k$相关的冗余信息);
    \item $R_k$为“秩-1修正项”(用于注入新曲率$(\mathbf{s}_k, \mathbf{y}_k)$)。
\end{itemize}

\textbf{关键观察:$\mathbf{G}_k$的“乘向量算子”属性}
L-BFGS不直接存储$\mathbf{G}_k$,而是关注$\mathbf{G}_k$对任意向量$\mathbf{v}$的作用(记为$\mathbf{G}_k \mathbf{v}$)。对式(1)两边同时右乘$\mathbf{v}$,展开得:
\[
\mathbf{G}_{k+1} \mathbf{v} = V_k \cdot \left( \mathbf{G}_k \cdot (W_k^\top \mathbf{v}) \right) + R_k \mathbf{v} \tag{2}
\]
上式揭示:$\mathbf{G}_{k+1} \mathbf{v}$可由$\mathbf{G}_k$对“预处理后的$\mathbf{v}$”(即$W_k^\top \mathbf{v}$)的作用,再经$V_k$投影和$R_k$修正得到。\textbf{递推展开该式},即可用历史$\{(\mathbf{s}_i, \mathbf{y}_i)\}$和初始$\mathbf{G}_0$表示$\mathbf{G}_k \mathbf{v}$,无需显式$\mathbf{G}_k$。

\subsection{Step 1:递推展开\texorpdfstring{$\mathbf{G}_k \mathbf{v}$}{G_k v}}
假设保留最近$m$对曲率信息$(\mathbf{s}_{k-m}, \mathbf{y}_{k-m}), \dots, (\mathbf{s}_{k-1}, \mathbf{y}_{k-1})$,对式(2)从$\mathbf{G}_k$反向递推至$\mathbf{G}_{k-m}$(初始逆Hessian近似,通常取缩放单位矩阵$\mathbf{G}_0 = \gamma_k \mathbf{I}$):
\begin{enumerate}
    \item 对$\mathbf{G}_k$:$\mathbf{G}_k \mathbf{v} = V_{k-1} \cdot \left( \mathbf{G}_{k-1} \cdot (W_{k-1}^\top \mathbf{v}) \right) + R_{k-1} \mathbf{v}$
    \item 对$\mathbf{G}_{k-1}$:$\mathbf{G}_{k-1} \cdot (W_{k-1}^\top \mathbf{v}) = V_{k-2} \cdot \left( \mathbf{G}_{k-2} \cdot (W_{k-2}^\top \cdot W_{k-1}^\top \mathbf{v}) \right) + R_{k-2} \cdot (W_{k-1}^\top \mathbf{v})$
    \item 以此类推,直到$\mathbf{G}_{k-m}$:
    \[
    \mathbf{G}_k \mathbf{v} = V_{k-1} V_{k-2} \dots V_{k-m} \cdot \left( \mathbf{G}_{k-m} \cdot (W_{k-m}^\top \dots W_{k-2}^\top W_{k-1}^\top \mathbf{v}) \right) + \text{秩-1修正项总和} \tag{3}
    \]
\end{enumerate}
式(3)可拆分为两部分:
\begin{itemize}
    \item \textbf{右乘链}:$W_{k-m}^\top \dots W_{k-1}^\top \mathbf{v}$(对应“反向环”,处理所有$W_i^\top$对$\mathbf{v}$的预处理);
    \item \textbf{左乘链+修正项}:$V_{k-m} \dots V_{k-1} \cdot (\mathbf{G}_0 \cdot \text{右乘结果}) + \text{修正项}$(对应“正向环”,处理$V_i$投影和$R_i$修正)。
\end{itemize}

\subsection{Step 2:反向环(Right Loop)——处理右乘链}
反向环的目标是计算“预处理后的向量”$\mathbf{q}$,即式(3)中$\mathbf{G}_0$的输入:$\mathbf{q} = W_{k-m}^\top \dots W_{k-1}^\top \mathbf{v}$。

\textbf{3.1 单个$W_i^\top$的作用}
由$W_i = \mathbf{I} - \rho_i \mathbf{y}_i \mathbf{s}_i^\top$,其转置为$W_i^\top = \mathbf{I} - \rho_i \mathbf{s}_i \mathbf{y}_i^\top$(因$\mathbf{s}_i \mathbf{y}_i^\top$的转置为$\mathbf{y}_i \mathbf{s}_i^\top$)。对任意向量$\mathbf{z}$,$W_i^\top \mathbf{z}$的计算为:
\[
W_i^\top \mathbf{z} = \mathbf{z} - \rho_i \mathbf{s}_i (\mathbf{y}_i^\top \mathbf{z}) \tag{4}
\]
但结合递推顺序(从$i=k-1$到$i=k-m$),需定义\textbf{中间系数$\alpha_i$} 简化计算:令$\alpha_i = \rho_i (\mathbf{s}_i^\top \mathbf{z})$,则式(4)可改写为:
\[
\mathbf{z} \leftarrow \mathbf{z} - \alpha_i \mathbf{y}_i \tag{5}
\]
(注:$\alpha_i$记录了$\mathbf{s}_i$与当前$\mathbf{z}$的内积信息,后续正向环需复用该系数,避免重复计算。)

\textbf{3.2 反向环完整流程}
初始化$\mathbf{q} = \mathbf{v}$(初始向量,若计算搜索方向则$\mathbf{v} = \mathbf{g}_k$),从最近的曲率对开始,自后向前迭代($i = k-1, k-2, \dots, k-m$):
\[
\begin{cases}
\alpha_i = \rho_i \cdot \mathbf{s}_i^\top \mathbf{q} \\
\mathbf{q} = \mathbf{q} - \alpha_i \cdot \mathbf{y}_i
\end{cases} \tag{6}
\]
\textbf{作用}:通过$m$次迭代,将所有$W_i^\top$的作用“吸收”到$\mathbf{q}$中,得到$\mathbf{q} = W_{k-m}^\top \dots W_{k-1}^\top \mathbf{v}$,为后续$\mathbf{G}_0$作用做准备。

\subsection{Step 3:初始缩放(Initial Scaling)——\texorpdfstring{$\mathbf{G}_0$}{G_0}的作用}
L-BFGS的初始逆Hessian近似$\mathbf{G}_0$不直接取$\mathbf{I}$(单位矩阵),而是取\textbf{缩放单位矩阵},目的是让$\mathbf{G}_0$的尺度接近真实逆Hessian$\nabla^2 f(\mathbf{x}_k)^{-1}$的尺度,提升方向质量。

\textbf{4.1 缩放系数$\gamma_k$的选择}
缩放系数$\gamma_k$由最近一次的曲率对$(\mathbf{s}_{k-1}, \mathbf{y}_{k-1})$计算,满足“模拟Hessian的对角尺度”:
\[
\gamma_k = \frac{\mathbf{s}_{k-1}^\top \mathbf{y}_{k-1}}{\mathbf{y}_{k-1}^\top \mathbf{y}_{k-1}} \tag{7}
\]
\textbf{物理意义}:$\mathbf{s}_{k-1}^\top \mathbf{y}_{k-1}$是“平均曲率”的近似($\mathbf{y}_{k-1} \approx \nabla^2 f(\bar{\mathbf{x}}) \mathbf{s}_{k-1}$,故$\mathbf{s}_{k-1}^\top \mathbf{y}_{k-1} \approx \mathbf{s}_{k-1}^\top \nabla^2 f(\bar{\mathbf{x}}) \mathbf{s}_{k-1}$),$\gamma_k$相当于$\nabla^2 f(\bar{\mathbf{x}})^{-1}$的“对角平均”,确保$\mathbf{G}_0 = \gamma_k \mathbf{I}$的尺度合理。

\textbf{4.2 初始缩放计算}
对反向环得到的$\mathbf{q}$,施加$\mathbf{G}_0$的作用:
\[
\mathbf{r} = \gamma_k \cdot \mathbf{q} \tag{8}
\]
此时$\mathbf{r} = \mathbf{G}_0 \cdot \mathbf{q} = \gamma_k W_{k-m}^\top \dots W_{k-1}^\top \mathbf{v}$,对应式(3)中$\mathbf{G}_{k-m} \cdot (\text{右乘结果})$。

\subsection{Step 4:正向环(Left Loop)——处理左乘链与修正项}
正向环的目标是将式(3)中的“左乘链$V_{k-m} \dots V_{k-1}$”和“秩-1修正项总和”融入$\mathbf{r}$,最终得到$\mathbf{G}_k \mathbf{v}$。

\textbf{5.1 单个$V_i$与$R_i$的作用}
由$V_i = \mathbf{I} - \rho_i \mathbf{s}_i \mathbf{y}_i^\top$和$R_i = \rho_i \mathbf{s}_i \mathbf{s}_i^\top$,结合式(2)的递推逻辑,对任意向量$\mathbf{z}$,$V_i \mathbf{z} + R_i \mathbf{v}$的计算为:
\[
V_i \mathbf{z} + R_i \mathbf{v} = \mathbf{z} - \rho_i \mathbf{s}_i (\mathbf{y}_i^\top \mathbf{z}) + \rho_i \mathbf{s}_i (\mathbf{s}_i^\top \mathbf{v}) \tag{9}
\]
利用反向环中已存储的$\alpha_i = \rho_i (\mathbf{s}_i^\top \mathbf{v})$(式(6)),定义\textbf{中间系数$\beta_i = \rho_i (\mathbf{y}_i^\top \mathbf{z})$},则式(9)可简化为:
\[
\mathbf{z} \leftarrow \mathbf{z} + \mathbf{s}_i (\alpha_i - \beta_i) \tag{10}
\]
\textbf{推导验证}:
\[
\mathbf{z} - \beta_i \mathbf{s}_i + \alpha_i \mathbf{s}_i = \mathbf{z} + \mathbf{s}_i (\alpha_i - \beta_i)
\]
完全匹配式(9),且复用了反向环的$\alpha_i$,避免重复计算$\mathbf{s}_i^\top \mathbf{v}$。

\textbf{5.2 正向环完整流程}
从最早保留的曲率对开始,自前向后迭代($i = k-m, k-m+1, \dots, k-1$):
\[
\begin{cases}
\beta_i = \rho_i \cdot \mathbf{y}_i^\top \mathbf{r} \\
\mathbf{r} = \mathbf{r} + \mathbf{s}_i \cdot (\alpha_i - \beta_i)
\end{cases} \tag{11}
\]
\textbf{作用}:通过$m$次迭代,将所有$V_i$的左乘作用和$R_i$的秩-1修正融入$\mathbf{r}$,最终$\mathbf{r}$即为$\mathbf{G}_k \mathbf{v}$的结果:
\[
\mathbf{r} = \mathbf{G}_k \mathbf{v} \tag{12}
\]

\subsection{Step 5:L-BFGS搜索方向与迭代流程}
当$\mathbf{v} = \mathbf{g}_k$(当前梯度)时,由式(12)得$\mathbf{r} = \mathbf{G}_k \mathbf{g}_k$,因此\textbf{L-BFGS的搜索方向}为:
\[
\mathbf{p}_k = -\mathbf{r} = -\mathbf{G}_k \mathbf{g}_k \tag{13}
\]
(与BFGS方向一致,保证是下降方向,因$\mathbf{G}_k \succ 0$且$\mathbf{g}_k^\top \mathbf{p}_k = -\mathbf{g}_k^\top \mathbf{G}_k \mathbf{g}_k < 0$)。

\begin{algorithm}[L-BFGS完整迭代流程]
\begin{enumerate}
    \item \textbf{初始化}:
    \begin{itemize}
        \item 初始点$\mathbf{x}_0$,最大记忆数$m$,线搜索参数(强Wolfe),容忍度$\text{tol}$;
        \item 清空存储队列$\mathcal{S} = []$(存$\mathbf{s}_i$)、$\mathcal{Y} = []$(存$\mathbf{y}_i$),$k=0$;
        \item 计算$\mathbf{g}_0 = \nabla f(\mathbf{x}_0)$,若$\|\mathbf{g}_0\| \leq \text{tol}$,停止。
    \end{itemize}

    \item \textbf{方向计算}:
    \begin{itemize}
        \item 若$k=0$(无历史曲率):取$\mathbf{p}_0 = -\gamma_0 \mathbf{g}_0$($\gamma_0$取1或经验值);
        \item 若$k \geq 1$:执行“反向环(式6)$\to$ 初始缩放(式8)$\to$ 正向环(式11)”,得$\mathbf{p}_k = -\mathbf{r}$。
    \end{itemize}

    \item \textbf{线搜索}:
    用强Wolfe条件求步长$\alpha_k > 0$,满足:
    \[
    f(\mathbf{x}_k + \alpha_k \mathbf{p}_k) \leq f(\mathbf{x}_k) + c_1 \alpha_k \mathbf{g}_k^\top \mathbf{p}_k, \quad |\mathbf{g}(\mathbf{x}_k + \alpha_k \mathbf{p}_k)^\top \mathbf{p}_k| \leq c_2 |\mathbf{g}_k^\top \mathbf{p}_k|
    \]
    ($c_1 \in (0,1), c_2 \in (c_1,1)$,强Wolfe保证$\mathbf{s}_k^\top \mathbf{y}_k > 0$,即曲率条件成立)。

    \item \textbf{更新与存储管理}:
    \begin{itemize}
        \item 计算$\mathbf{x}_{k+1} = \mathbf{x}_k + \alpha_k \mathbf{p}_k$,$\mathbf{s}_k = \alpha_k \mathbf{p}_k$,$\mathbf{g}_{k+1} = \nabla f(\mathbf{x}_{k+1})$,$\mathbf{y}_k = \mathbf{g}_{k+1} - \mathbf{g}_k$;
        \item 若$\mathbf{s}_k^\top \mathbf{y}_k > 0$(曲率条件):将$\mathbf{s}_k$加入$\mathcal{S}$,$\mathbf{y}_k$加入$\mathcal{Y}$;若队列长度$>m$,删除最早的$\mathbf{s}_{k-m}$和$\mathbf{y}_{k-m}$;
        \item 计算$\gamma_{k+1} = \mathbf{s}_k^\top \mathbf{y}_k / (\mathbf{y}_k^\top \mathbf{y}_k)$(供下次初始缩放)。
    \end{itemize}

    \item \textbf{终止判断}:
    若$\|\mathbf{g}_{k+1}\| \leq \text{tol}$,停止;否则$k=k+1$,返回步骤2。
\end{enumerate}
\end{algorithm}

\subsection{关键性质验证}
\begin{enumerate}
    \item \textbf{割线条件保持}:L-BFGS的两环递推严格继承BFGS的割线条件$\mathbf{G}_k \mathbf{y}_i = \mathbf{s}_i$($i = k-m, \dots, k-1$),确保曲率拟合的准确性;
    \item \textbf{正定性保证}:若所有保留的$\mathbf{s}_i^\top \mathbf{y}_i > 0$,则$\mathbf{G}_k \succ 0$,搜索方向$\mathbf{p}_k$必为下降方向;
    \item \textbf{复杂度优势}:
    \begin{itemize}
        \item 内存复杂度:仅存储$2m$个$n$维向量($\mathcal{S}$和$\mathcal{Y}$),为$O(nm)$(标准BFGS为$O(n^2)$);
        \item 时间复杂度:每次方向计算需$2m$次向量内积和$2m$次向量加法,为$O(nm)$(标准BFGS为$O(n^2)$),适配大规模问题($n \gtrsim 10^5$)。
    \end{itemize}
\end{enumerate}

L-BFGS的推导核心是\textbf{“将G-form的矩阵递推转化为向量操作”}:通过反向环吸收右乘投影、正向环融合左乘投影与秩-1修正,仅用$m$对历史曲率对在线模拟$\mathbf{G}_k \mathbf{v}$的计算,既保留了BFGS的二阶收敛性,又解决了标准BFGS的内存瓶颈。其本质是“用少量历史信息近似逆Hessian”,是大规模无约束优化(如深度学习、科学计算)的默认基线方法。
