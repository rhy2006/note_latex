\chapter{无约束优化问题}

\numberwithin{equation}{subsection}

\section{无约束优化问题}

\subsection{问题基本形式}
\begin{definition}[无约束优化问题]
针对\textbf{凸且二阶可微}的目标函数,无约束优化问题的数学形式定义为:
\begin{equation}
\min f(x)
\end{equation}
其中,$f(x)$满足“凸性”与“二阶可微性”这两个核心前提假设。
\end{definition}

\subsection{最优性条件}
\begin{proposition}[最优性条件]
无约束优化问题达到最优解$x^*$的核心必要条件为:
\begin{equation}
\nabla f(x^*) = 0
\end{equation}
即最优解处函数的梯度(一阶导数)等于零向量;若该等式无法直接求解,则需通过迭代方法逼近最优解。
\end{proposition}

\subsection{核心求解框架(下山法迭代格式)}
下山法通过迭代构造严格递减的函数值序列以逼近最优解,其数学化迭代流程如下:
\begin{enumerate}
    \item \textbf{初始设定}:给定初始迭代点$x^{(0)}$,初始化迭代次数$k=0$;
    \item \textbf{方向选取}:确定第$k$次迭代的搜索方向$\Delta x^{(k)}$(后续需进一步优化方向选取规则);
    \item \textbf{步长选取}:确定第$k$次迭代的步长$\alpha > 0$(后续需进一步优化步长选取规则);
    \item \textbf{迭代更新}:按以下公式更新迭代点:
    \begin{equation}
    x^{(k+1)} = x^{(k)} + \alpha \Delta x^{(k)}
    \end{equation}
    同时更新迭代次数$k \leftarrow k+1$,且需满足函数值严格递减条件:
    \begin{equation}
    f(x^{(0)}) > f(x^{(1)}) > f(x^{(2)}) > \dots
    \end{equation}
\end{enumerate}

综上,无约束优化问题的核心数学转化目标为:通过数学规则确定“搜索方向$\Delta x^{(k)}$”与“步长$\alpha$”,使上述迭代流程满足严格递减性并收敛至最优解。

\section{搜索方向的确定}

\subsection{视角1:线性化与下降条件}

\begin{definition}[梯度L-Lipschitz条件]
若函数$f$的梯度满足L-Lipschitz连续性,等价于:
\begin{equation}
\| \nabla f(x)-\nabla f(y)\| _{2} \leq L\| x-y\| _{2}
\end{equation}
其中$L$为Lipschitz常数,$\|\cdot\|_2$表示欧氏范数。
\end{definition}

\begin{proposition}[函数线性化不等式]
利用基本定理与Cauchy–Schwarz公式,可推导出函数在点$x$处的线性化上界:
\begin{equation}
f(x+\Delta) \leq f(x)+\nabla f(x)^{\top} \Delta+\frac{L}{2}\| \Delta\| _{2}^{2}
\end{equation}
\end{proposition}

\begin{proof}[推导:函数线性化不等式]
\textbf{梯度L-Lipschitz连续性}(推导的基础,保证梯度变化有界):
\begin{equation}
\|\nabla f(x) - \nabla f(x + t\Delta)\|_2 \leq L \cdot \|t\Delta\|_2 = L t \|\Delta\|_2 \quad (t \in [0,1])
\end{equation}

\textbf{多元微积分基本定理}:
\begin{equation}
f(x+\Delta) - f(x) = \int_{0}^{1} \nabla f\left(x + t\Delta\right)^{\top} \Delta \, dt
\end{equation}

\textbf{积分拆分与线性项提取}:
\begin{equation}
\int_{0}^{1} \nabla f(x + t\Delta)^{\top}\Delta dt = \nabla f(x)^{\top}\Delta \cdot \int_{0}^{1} dt + \int_{0}^{1} \left[\nabla f(x + t\Delta) - \nabla f(x)\right]^{\top}\Delta dt
\end{equation}
其中第一部分积分结果直接为线性项:
\begin{equation}
\nabla f(x)^{\top}\Delta \cdot \int_{0}^{1} dt = \nabla f(x)^{\top}\Delta
\end{equation}

\textbf{Cauchy-Schwarz不等式+L-Lipschitz条件}(控制第二部分积分,得到“二次项$\frac{L}{2}\|\Delta\|_2^2$”):
由Cauchy-Schwarz不等式:
\begin{equation}
\left[\nabla f(x + t\Delta) - \nabla f(x)\right]^{\top}\Delta \leq \|\nabla f(x + t\Delta) - \nabla f(x)\|_2 \cdot \|\Delta\|_2
\end{equation}
代入L-Lipschitz条件并积分:
\begin{equation}
\int_{0}^{1} \|\nabla f(x + t\Delta) - \nabla f(x)\|_2 \cdot \|\Delta\|_2 dt \leq \int_{0}^{1} L t \|\Delta\|_2^2 dt = \frac{L}{2}\|\Delta\|_2^2
\end{equation}

\textbf{合并得到最终不等式}:
\begin{equation}
f(x+\Delta) - f(x) \leq \nabla f(x)^{\top}\Delta + \frac{L}{2}\|\Delta\|_2^2 \implies f(x+\Delta) \leq f(x) + \nabla f(x)^{\top}\Delta + \frac{L}{2}\|\Delta\|_2^2
\end{equation}
\end{proof}

\begin{definition}[下降方向]
令$\Delta=\alpha p$($\alpha>0$为步长,$p$为搜索方向),只要满足特定条件,就存在足够小的$\alpha>0$使函数值降低——这是下降方向的充要条件。其中,最自然的搜索方向选择为\textbf{负梯度方向}:
\begin{equation}
p_{gd}=-\nabla f(x)
\end{equation}
\end{definition}

\subsection{视角2:一般范数下的“最速下降”}

\begin{definition}[最速下降方向]
对任意范数$\|\cdot\|$及其对应的对偶范数$\|\cdot\|_*$,“最速下降方向”需通过求解以下优化问题得到:
\begin{equation}
p^{*}=\arg \min _{\| d\| _{*} \leq 1} \nabla f(x)^{\top} d
\end{equation}
上述优化问题的解为:
\begin{equation}
p^{*}=-\frac{\nabla f(x)}{\| \nabla f(x)\| }
\end{equation}
\end{definition}

\begin{example}[特殊范数案例]
\begin{itemize}
    \item 当使用\textbf{欧氏范数}时,对偶范数与原范数一致,此时$p^*$即为负梯度方向(与前文中结论一致);
    \item 若使用\textbf{Hessian度量}(定义为$\|d\|_{H(x)}=\sqrt{d^{\top} H(x) d}$,$H(x)$为函数$f$在$x$处的Hessian矩阵),则“最速”方向等价于Newton步(详见后续Newton法相关内容)。
\end{itemize}
\end{example}

\begin{quote}
\textit{解释:}
函数$f$在点$x$沿方向$d$的\textbf{局部下降速度},由梯度与方向的内积$\nabla f(x)^\top d$决定:
\begin{itemize}
    \item 内积越小(越负),函数沿$d$下降越快;
    \item 因此“找最速下降方向”,等价于在约束$\|d\|_* \leq 1$(对偶范数单位球)下,求解:
    \begin{equation}
    p^* = \arg\min_{\|d\|_* \leq 1} \nabla f(x)^\top d
    \end{equation}
\end{itemize}
设$a = \nabla f(x)$,需先确定$a^\top d$(即$\nabla f(x)^\top d$)的最小可能值。
根据\textbf{对偶范数的核心性质}:对任意满足$\|d\|_* \leq 1$的$d$,有:
\begin{equation}
a^\top d \geq -\|a\|
\end{equation}
方向$p^* = -\frac{\nabla f(x)}{\|\nabla f(x)\|}$,既满足对偶范数单位球约束,又能让内积达到最小(下降最快),因此它就是单位球上的最速下降方向。
\end{quote}

\subsection{预条件化的作用}

\textbf{最速下降方向是 “当前范数下,使内积$\nabla f(x)^\top d$最小(下降最快)的方向”}。预条件化的本质是\textbf{改变 “范数度量标准”}:不再用欧氏范数,而是用 “预条件范数”。

\begin{definition}[预条件化]
考虑二次目标函数 $f(x)=\frac{1}{2} x^{\top} A x-b^{\top} x \quad (A \succ 0)$,其等高线为椭圆。
通过坐标变换$y=A^{1/2} x$,可将原椭圆等高线“拉成”圆形(即消除椭圆的“细长”特性),这一过程称为\textbf{预条件化}。
\end{definition}

\textbf{直观意义}:预条件化相当于将优化问题中的“细长谷”地形转化为“圆形洼地”,从而缓解负梯度下降时容易出现的“楼梯形”迂回路径,提升迭代效率。

\section{如何确定步长}

设线搜索的核心目标函数为$\phi(\alpha) = f(x + \alpha p)$,其中$x$为当前迭代点,$p$为已确定的搜索方向,$\alpha \geq 0$为待求解的步长(需满足函数值下降条件$f(x + \alpha p) < f(x)$)。以下分别介绍三种主流线搜索方法:

\subsection{精确线搜索(Exact Line Search)}

\begin{definition}[精确线搜索]
精确线搜索直接求解“使$\phi(\alpha)$最小化”的步长$\alpha^*$,数学表达为:
\begin{equation}
\alpha^{*} = \arg \min _{\alpha \geq 0} \phi(\alpha)
\end{equation}
其目标是找到“当前方向下最优的步长”,理论上能让单次迭代的函数值下降幅度最大。
\end{definition}

\begin{example}[二次函数的封闭解]
若目标函数为二次函数$f(x) = \frac{1}{2} x^\top A x - b^\top x$(其中$A \succ 0$,即$A$为正定矩阵),且搜索方向$p = -\nabla f(x) = -g$($g = \nabla f(x)$为当前梯度),则精确线搜索的步长有\textbf{封闭解}:
\begin{equation}
\alpha^{*} = \frac{g^\top g}{g^\top A g}
\end{equation}
该解的本质是:在二次函数的近似下,使“更新后的梯度$\nabla f(x + \alpha p)$与搜索方向$p$垂直”(即一次更新在二次近似意义上最有效),等价于求解$\arg \min _{\alpha} \|g - \alpha A g\|_2^2$。
\end{example}

\textbf{与固定步长的比较}:在函数满足 \textbf{L-光滑性}(梯度L-Lipschitz连续)的前提下,精确线搜索得到的$\alpha^*$至少不劣于固定步长$\alpha = 1/L$(固定步长仅能保证“函数值下降”,但无法保证下降幅度最优)。

\subsection{黄金分割(Golden Section Search)}

\textbf{适用场景}:当目标函数$\phi(\alpha)$是\textbf{单峰函数}(即区间内仅有一个最小值点),且计算梯度(或导数$\phi'(\alpha)$)代价高、难度大时,采用黄金分割法通过“区间收缩”逼近最优步长$\alpha^*$。

\textbf{核心流程}:
\begin{enumerate}
    \item \textbf{初始区间设定}:确定初始搜索区间$[a, b]$,满足$\phi(0) < \phi(b)$(保证最小值点在区间内,因$\alpha=0$对应当前点,函数值最大),初始令$a=0$;
    \item \textbf{区间收缩规则}:
    \begin{itemize}
        \item 在区间$[a, b]$内选取两个对称点$t_1$和$t_2$($a < t_1 < t_2 < b$),两点间距与区间总长的比例为“黄金分割系数”$c = \frac{1}{2}(\sqrt{5} - 1) \approx 0.618$,即:
        \begin{equation}
        t_1 = a + (1 - c)(b - a), \quad t_2 = a + c(b - a)
        \end{equation}
        \item 比较函数值:
        \begin{itemize}
            \item 若$\phi(t_2) > \phi(t_1)$:说明最小值点$\alpha^* \in [a, t_2]$,令新区间为$[a, t_2]$;
            \item 若$\phi(t_1) > \phi(t_2)$:说明最小值点$\alpha^* \in [t_1, b]$,令新区间为$[t_1, b]$;
        \end{itemize}
    \end{itemize}
    \item \textbf{迭代收敛}:重复步骤2,不断收缩区间,直到区间长度小于预设精度。最终区间内的任意点均可作为$\alpha^*$的近似值,收敛误差上界与$0.618^k$成正比($k$为迭代次数)。
\end{enumerate}

\textbf{特点}:优点是无需计算梯度/导数,仅通过函数值比较即可迭代;缺点是收敛速度较慢(线性收敛),仅适用于单峰函数。

\subsection{回溯搜索(Backtracking Line Search)与 Armijo–Wolfe 准则}

回溯搜索通过“先试后调”的方式确定步长,需满足两个核心条件(保证步长既“足够大”以加速收敛,又“足够小”以保证函数值下降):

\begin{definition}[Armijo 条件(充分下降条件)]
确保步长能使函数值显著下降,数学表达为:
\begin{equation}
f(x + \alpha p) \leq f(x) + c_1 \alpha \nabla f(x)^\top p
\end{equation}
其中$0 < c_1 < 1$(通常取$c_1 = 10^{-4}$),$\nabla f(x)^\top p < 0$(因$p$为下降方向),右边项为函数值的“预期下降下限”。
\end{definition}

\begin{definition}[Wolfe 条件(曲率条件)]
确保步长不会过小(避免收敛过慢),数学表达为:
\begin{equation}
\nabla f(x + \alpha p)^\top p \geq c_2 \nabla f(x)^\top p
\end{equation}
其中$c_1 < c_2 < 1$(通常取$c_2 = 0.9$),该条件要求“更新后的梯度与搜索方向的内积”不小于“初始梯度与搜索方向内积”的$c_2$倍,避免步长停留在“函数值下降缓慢的区域”。
\end{definition}

\textbf{回溯搜索算法流程}:
给定后退因子$\beta \in (0, 1)$(通常取$\beta = 0.5$或$0.8$),步骤如下:
\begin{enumerate}
    \item \textbf{初始步长尝试}:令初始步长$\alpha \leftarrow 1$(默认从“单位步长”开始,适配 Newton 法等需要大步长的场景);
    \item \textbf{条件判断与步长调整}:若当前$\alpha$不满足 Armijo 条件( Armijo–Wolfe 联合条件),则按比例缩小步长:$\alpha \leftarrow \beta \alpha$;
    \item \textbf{终止}:重复步骤2,直到$\alpha$满足预设条件,输出最终步长$\alpha$。
\end{enumerate}

\textbf{关键性质与应用}:
\begin{itemize}
    \item \textbf{终止性}:由 Descent Lemma 可证明:当$\alpha$足够小时,Armijo 条件必成立,因此回溯搜索一定能终止;
    \item \textbf{与 Newton 法的结合(两阶段收敛)}:
    \begin{itemize}
        \item 阶段 I(远离最优解时):$\alpha < 1$,通过回溯调整步长进入“可接受域”(满足 Armijo–Wolfe 条件);
        \item 阶段 II(靠近最优解时):步长会触发$\alpha = 1$(单位步长),此时 Newton 法可实现二次收敛(收敛速度远快于梯度下降)。
    \end{itemize}
\end{itemize}

\section{收敛率:强凸 / PL 条件与“楼梯现象”}

\subsection{强凸 + L-光滑:线性收敛}

\begin{definition}[强凸与L-光滑]
目标函数$f(x)$需同时满足两大性质:
\begin{enumerate}
    \item \textbf{强凸性}:存在常数$\mu > 0$,对任意迭代点$x$,其Hessian矩阵(二阶导数矩阵)满足下界约束:
    \begin{equation}
    \mu I \preceq \nabla^2 f(x)
    \end{equation}
    (“强凸”保证函数有唯一最小值点,且函数形态“下凸程度”可控);
    \item \textbf{L-光滑性}:存在常数$L > 0$,对任意迭代点$x$,其Hessian矩阵满足上界约束:
    \begin{equation}
    \nabla^2 f(x) \preceq L I
    \end{equation}
    (“L-光滑”保证函数曲率不超过阈值,梯度变化平缓,避免局部剧烈波动)。
\end{enumerate}
\end{definition}

\begin{theorem}[线性收敛性]
当步长取$\alpha = 1/L$时,梯度下降迭代满足严格的线性收敛性质:
\begin{enumerate}
    \item 函数值下降界(每次迭代函数值必递减且幅度可控):
    \begin{equation}
    f(x_{k+1}) \leq f(x_k) - \frac{1}{2L} \|\nabla f(x_k)\|_2^2
    \end{equation}
    \item 迭代点误差界($x^*$为函数最优解,与最优解的距离按固定比例缩小):
    \begin{equation}
    \|x_{k+1} - x^*\|_2^2 \leq \left(1 - \frac{\mu}{L}\right) \|x_k - x^*\|_2^2
    \end{equation}
\end{enumerate}
\end{theorem}

\begin{proof}[证明:线性收敛性]
\textbf{一、明确核心前提与已有结论}

在证明前,需明确2个关键性质(强凸+L-光滑)的推论,及1个已证结论:

1. \textbf{强凸性的核心推论}:
若函数$f$强凸(存在$\mu>0$,使$\mu I \preceq \nabla^2 f(x)$),则对任意迭代点$x$与最优解$x^*$(满足$\nabla f(x^*)=0$),有:
\begin{equation}
f(x) - f(x^*) \geq \frac{\mu}{2}\|x - x^*\|_2^2 \tag{1}
\end{equation}
(强凸性保证“函数值与最优值的差距”不小于“迭代点与最优解距离平方”的固定倍数,建立误差与函数值差的关联)

同时,强凸性还可推出“梯度范数与函数值差的关系”:
因为$\nabla f(x^*)=0$,
\begin{equation}
f(x^*)\ge f(x)+\nabla f(x)^{\top}(x^*-x)+\frac{\mu}{2}|x^*-x|^2.
\end{equation}
移项:
\begin{equation}
f(x)-f(x^*)\le\nabla f(x)^{\top}(x-x^*)-\frac{\mu}{2}|x-x^*|^2. \tag{B}
\end{equation}
由 Cauchy–Schwarz:
\begin{equation}
\nabla f(x)^{\top}(x-x^*) \le |\nabla f(x)|\cdot|x-x^*|.
\end{equation}
令右侧关于$|x-x^*|$的表达最小化,可视为二次函数
\begin{equation}
|\nabla f(x)|\cdot|x-x^*|-\frac{\mu}{2}|x-x^*|^2.
\end{equation}
其最大值出现在$|x-x^*|=\frac{|\nabla f(x)|}{\mu}$,代入得
\begin{equation}
f(x)-f(x^*)\le \frac{1}{2\mu}|\nabla f(x)|^2.
\end{equation}
即:
\begin{equation}
\|\nabla f(x)\|_2^2 \geq 2\mu(f(x) - f(x^*)) \tag{2}
\end{equation}
(梯度大小能反映函数值与最优值的差距,为后续替换梯度项做准备)

2. \textbf{已证的函数值下降界}:
由L-光滑性(梯度L-Lipschitz连续)及步长$\alpha=1/L$,已证明函数值满足:
\begin{equation}
f(x_{k+1}) \leq f(x_k) - \frac{1}{2L}\|\nabla f(x_k)\|_2^2 \tag{3}
\end{equation}

\textbf{二、步骤1:推导函数值差的线性衰减关系}

将式(3)变形为“相邻迭代的函数值差下界”:
\begin{equation}
f(x_k) - f(x_{k+1}) \geq \frac{1}{2L}\|\nabla f(x_k)\|_2^2
\end{equation}
将强凸性推论式(2)($\|\nabla f(x_k)\|_2^2 \geq 2\mu(f(x_k) - f(x^*))$)代入上式,替换梯度范数项:
\begin{equation}
f(x_k) - f(x_{k+1}) \geq \frac{1}{2L} \cdot 2\mu(f(x_k) - f(x^*))
\end{equation}
化简后得到“函数值差的衰减关系”:
\begin{equation}
f(x_k) - f(x_{k+1}) \geq \frac{\mu}{L}(f(x_k) - f(x^*))
\end{equation}
进一步整理,将函数值差聚焦到“与最优值的差距”:
\begin{equation}
f(x_{k+1}) - f(x^*) \leq f(x_k) - f(x^*) - \frac{\mu}{L}(f(x_k) - f(x^*))
\end{equation}
\begin{equation}
f(x_{k+1}) - f(x^*) \leq \left(1 - \frac{\mu}{L}\right)\left(f(x_k) - f(x^*)\right) \tag{4}
\end{equation}
式(4)表明:迭代中“函数值与最优值的差距”按$(1 - \mu/L)$的比例线性衰减。

\textbf{三、步骤2:将函数值差衰减转化为迭代误差衰减}

利用强凸性推论式(1),分别对$x_{k+1}$和$x_k$建立“迭代误差与函数值差的关联”:

- 对$x_{k+1}$:
\begin{equation}
\|x_{k+1} - x^*\|_2^2 \leq \frac{2}{\mu}\left(f(x_{k+1}) - f(x^*)\right) \tag{5}
\end{equation}
(由式(1)变形:两边同乘$2/\mu$,不等号方向不变)

- 对$x_k$:
\begin{equation}
f(x_k) - f(x^*) \geq \frac{\mu}{2}\|x_k - x^*\|_2^2 \implies \frac{2}{\mu}\left(f(x_k) - f(x^*)\right) \geq \|x_k - x^*\|_2^2 \tag{6}
\end{equation}

\textbf{四、步骤3:合并推导得到迭代误差界}

将式(4)(函数值差衰减)代入式(5),再结合式(6)(函数值差与$x_k$误差的关联):
\begin{equation}
\|x_{k+1} - x^*\|_2^2 \leq \frac{2}{\mu} \cdot \left(1 - \frac{\mu}{L}\right)\left(f(x_k) - f(x^*)\right)
\end{equation}
由式(6)可知$\frac{2}{\mu}\left(f(x_k) - f(x^*)\right) \geq \|x_k - x^*\|_2^2$,因此:
\begin{equation}
\|x_{k+1} - x^*\|_2^2 \leq \left(1 - \frac{\mu}{L}\right) \cdot \frac{2}{\mu}\left(f(x_k) - f(x^*)\right) \leq \left(1 - \frac{\mu}{L}\right)\|x_k - x^*\|_2^2
\end{equation}

\textbf{五、结论}

最终证得迭代误差界公式:
\begin{equation}
\| x_{k+1}-x^{*}\| _{2}^{2} \leq\left(1-\frac{\mu}{L}\right)\left\| x_{k}-x^{*}\right\| _{2}^{2}
\end{equation}
该公式表明:强凸+L-光滑条件下,梯度下降的迭代误差按$(1 - \mu/L)$的线性因子衰减,收敛速度由条件数$\kappa=L/\mu$决定——$\kappa$越大,$(1 - 1/\kappa)$越接近1,误差衰减越慢,且易因等高线“细长”出现“楼梯形”迂回路径。
\end{proof}

\begin{definition}[条件数]
定义条件数$\kappa = L/\mu$(L与μ的比值),线性收敛的“衰减速度”由$\kappa$决定:
\begin{itemize}
    \item $\kappa$越小(L与μ接近):衰减因子$(1 - 1/\kappa)$越接近0,收敛越快;
    \item $\kappa$越大(L远大于μ):衰减因子越接近1,收敛越慢,且极易出现“楼梯现象”。
\end{itemize}
\end{definition}

\subsection{PL(Polyak–Lojasiewicz)不等式}

\begin{definition}[PL条件]
若存在常数$\mu > 0$,对任意迭代点$x$,函数值差与梯度范数满足以下关系:
\begin{equation}
f(x) - f(x^*) \leq \frac{1}{2\mu} \|\nabla f(x)\|_2^2
\end{equation}
($f(x^*)$为函数最小值,PL条件是强凸性的“弱化版本”——无需函数严格强凸,仅通过“函数值差距”与“梯度大小”的关联约束函数形态)。
\end{definition}

\begin{theorem}[PL条件下的收敛性]
即使$f(x)$非强凸,只要满足PL条件,梯度下降仍能实现\textbf{线性型收敛},函数值差的衰减公式为:
\begin{equation}
f(x_k) - f(x^*) \leq \left(1 - \frac{\mu}{L}\right)^k \left[f(x_0) - f(x^*)\right]
\end{equation}
($x_0$为初始迭代点,$k$为迭代次数)。
\end{theorem}

\textbf{实际意义}:解释了深度学习训练中的常见现象——深度网络的损失函数通常非强凸,但可能满足PL条件,因此训练时损失曲线会呈现“近似线性下降”的稳定趋势。

\subsection{“楼梯现象”:成因与缓解}
\begin{itemize}
    \item \textbf{现象描述}:当函数条件数$\kappa = L/\mu$过大时,负梯度下降的迭代路径会呈现“楼梯形”:沿细长的等高线(如二次函数的椭圆等高线)迂回前进,每次仅沿等高线短轴方向小幅下降,无法直接逼近最小值点,迭代效率极低。
    \item \textbf{核心成因}:条件数$\kappa$过大导致函数等高线“细长扁平”,负梯度方向(沿等高线法向)与“最优下降方向”(沿等高线长轴方向)偏差极大,梯度下降陷入“来回震荡、缓慢逼近”的困境。
    \item \textbf{缓解方法}:\textbf{预条件化}(或输入/参数归一化)——通过线性变换(如文档中二次函数的坐标变换$y = A^{1/2}x$)将“细长谷”地形转化为“圆形洼地”,本质是减小条件数$\kappa$,使负梯度方向更接近最优下降方向,从而消除楼梯现象。
\end{itemize}

\subsection{实践提示}
\begin{enumerate}
    \item \textbf{预条件化的核心价值}:通过调整优化空间的度量规则(如引入预条件矩阵),显著降低条件数$\kappa$,从根本上改善收敛速度;
    \item \textbf{迭代停止准则}:无需迭代至完全收敛,满足以下任一条件即可终止:
    \begin{itemize}
        \item 梯度范数足够小(函数接近平稳):$\|\nabla f(x_k)\|_2 \leq \varepsilon$($\varepsilon$为预设精度,如$10^{-6}$);
        \item 函数值相对下降不足(继续迭代收益极低):$\frac{f(x_{k-1}) - f(x_k)}{\max(1, f(x_{k-1}))} \leq \varepsilon$。
    \end{itemize}
\end{enumerate}

\section{Newton法:局部二次近似与两阶段收敛}
Newton法是比梯度下降更高效的优化方法,核心是通过\textbf{函数的局部二次近似}确定搜索方向,兼具“局部快速收敛”与“全局有效下降”的特性,其核心逻辑围绕“二阶展开→牛顿步→收敛性”展开。

\subsection{核心思路:函数的局部二次近似}
梯度下降仅用“一阶信息(梯度)”将函数局部近似为线性函数,而Newton法引入“二阶信息(Hessian矩阵)”,将函数局部近似为\textbf{二次函数}(更贴合非凸函数的局部曲率)。

\begin{definition}[局部二次近似]
对迭代点$x$,将目标函数$f(x+\Delta)$在$x$处做二阶泰勒展开($\Delta$为搜索方向向量):
\begin{equation}
f(x+\Delta) \approx f(x) + \nabla f(x)^\top \Delta + \frac{1}{2}\Delta^\top H(x) \Delta
\end{equation}
其中:
\begin{itemize}
    \item $\nabla f(x)$是$f(x)$的梯度(一阶导数);
    \item $H(x) = \nabla^2 f(x)$是$f(x)$的Hessian矩阵(二阶导数矩阵),反映函数在$x$处的局部曲率。
\end{itemize}
Newton法的核心是:\textbf{最小化上述二次近似函数},直接求解使近似函数最小的搜索方向$\Delta$。
\end{definition}

\subsection{牛顿步(Newton Step)的推导}
对二阶近似函数关于$\Delta$求导,并令导数为0(二次函数的极值点条件):
\begin{equation}
\frac{\partial}{\partial \Delta}\left[ f(x) + \nabla f(x)^\top \Delta + \frac{1}{2}\Delta^\top H(x) \Delta \right] = \nabla f(x) + H(x) \Delta = 0
\end{equation}

\begin{definition}[牛顿步]
若Hessian矩阵\textbf{正定}($H(x) \succ 0$,保证二次近似函数是凸函数,极值点为最小值点),则可解出唯一的搜索方向——\textbf{牛顿步}:
\begin{equation}
\Delta_{nt} = -H(x)^{-1} \nabla f(x)
\end{equation}
\end{definition}

\subsection{牛顿步的下降性}
牛顿步能保证是“下降方向”的前提是$H(x) \succ 0$,证明如下:
计算梯度与牛顿步的内积(判断方向是否下降的核心指标,内积<0则为下降方向):
\begin{equation}
\nabla f(x)^\top \Delta_{nt} = \nabla f(x)^\top \left( -H(x)^{-1} \nabla f(x) \right)
\end{equation}
因$H(x) \succ 0$,其逆矩阵$H(x)^{-1}$也正定,故对任意非零向量$\nabla f(x)$,有$\nabla f(x)^\top H(x)^{-1} \nabla f(x) > 0$,因此:
\begin{equation}
\nabla f(x)^\top \Delta_{nt} < 0
\end{equation}
即牛顿步满足“下降方向”的核心条件。

\subsection{局部二次收敛:牛顿法的核心优势}
当迭代点足够靠近最优解$x^*$时,Newton法会呈现\textbf{二次收敛}(收敛速度远快于梯度下降的线性收敛)。

\begin{theorem}[牛顿法局部二次收敛]
若满足以下两个前提:
\begin{enumerate}
    \item Hessian矩阵$H(x)$在$x^*$的邻域内\textbf{Lipschitz连续}(曲率变化平缓);
    \item 初始迭代点$x^{(0)}$足够靠近$x^*$(进入“局部收敛域”)。
\end{enumerate}
此时存在常数$C > 0$,使得迭代误差满足:
\begin{equation}
\| x_{k+1} - x^* \| \leq C \cdot \| x_k - x^* \|^2
\end{equation}
\end{theorem}

\begin{proof}[证明:局部二次收敛]
\textbf{结论:}
在$H$在某邻域内满足 Lipschitz(存在常数$M$使得$|H(x)-H(y)| \le M|x-y|$)且$H(x^*)$非奇异的情形,从足够近的初值出发,牛顿迭代局部二次收敛,即存在常数$C>0$和半径$r>0$,当$|x_k-x^*| \le r$时
\begin{equation}
|x_{k+1}-x^*| \le C|x_k-x^*|^2.
\end{equation}

\textbf{证明:}
设误差$e_k := x_k - x^*$。由$\nabla f(x^*) = 0$和一维积分形式的泰勒公式:
\begin{equation}
\nabla f(x_k) = \int_0^1 H(x^* + t e_k) e_k \, dt.
\end{equation}
牛顿更新写作:
\begin{equation}
e_{k+1} = x_k - x^* - H(x_k)^{-1} \nabla f(x_k) = H(x_k)^{-1} \left( H(x_k) - \int_0^1 H(x^* + t e_k) dt \right) e_k.
\end{equation}
取范数并用三角不等式得:
\begin{equation}
|e_{k+1}| \le |H(x_k)^{-1}| \int_0^1 |H(x_k) - H(x^* + t e_k)| dt \cdot |e_k|.
\end{equation}
利用 Hessian 的 Lipschitz 性质(常数记为$M$):
\begin{equation}
|H(x_k) - H(x^* + t e_k)| \le M |x_k - (x^* + t e_k)| = M(1 - t)|e_k|.
\end{equation}
代入并对$t$积分:
\begin{equation}
|e_{k+1}| \le |H(x_k)^{-1}| \cdot M \left( \int_0^1 (1 - t) dt \right) |e_k|^2 = \frac{M}{2} |H(x_k)^{-1}| |e_k|^2.
\end{equation}
由于$H$连续且$H(x^*)$非奇异,存在半径$r>0$,使得对所有$|x - x^*| \le r$,$H(x)$可逆,且$|H(x)^{-1}| \le B$($B$为该闭球上逆的上界)。因此当$|e_k| \le r$时:
\begin{equation}
|e_{k+1}| \le \frac{M B}{2} |e_k|^2.
\end{equation}
令常数$C := \frac{M B}{2}$,即得局部二次收敛估计。
\end{proof}

\textbf{直观意义}:二次收敛意味着“每次迭代后,误差的有效位数会翻倍”——例如:若第$k$步误差为$10^{-2}$,第$k+1$步误差可降至$10^{-4}$,第$k+2$步可降至$10^{-8}$,接近最优解时收敛极快。
