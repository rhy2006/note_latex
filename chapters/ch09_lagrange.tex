\chapter{从无约束到等式约束:拉格朗日乘子}

\section{拉格朗日乘子数学建模}

\subsection{1. 无约束优化的数学模型}
无约束优化的基本模型为:
\[
\min_{x \in \mathbb{R}^n} f(x)
\]
其中$f: \mathbb{R}^n \to \mathbb{R}$是可微函数。其\textbf{一阶最优性条件}(极值点的必要条件)为:
\[
\nabla f(x^*) = 0
\]
即目标函数在最优解$x^*$处的梯度为零(可微函数的无约束极值点必是梯度为零的点)。

\subsection{2. 等式约束优化的数学模型}
当引入\textbf{等式约束}后,模型变为:
\[
\begin{cases} \min_{x \in \mathbb{R}^n} & f(x) \\ \text{s.t.} & h_i(x) = 0,\ i=1,\dots,m \end{cases}
\]
其中$h_i: \mathbb{R}^n \to \mathbb{R}$是可微的约束函数,$m$是约束个数(通常$m < n$)。

此时,变量$x$被限制在由$m$个等式约束定义的\textbf{可行域}上,无法直接应用无约束的"梯度为零"条件——因为可行域是约束的"限制空间",需考虑约束对优化的"限制作用"。

\subsection{3. 拉格朗日函数的建模思想}
为了将\textbf{等式约束"融入"目标函数},我们引入\textbf{拉格朗日乘子}$\lambda = (\lambda_1, \dots, \lambda_m)^\top \in \mathbb{R}^m$,构造\textbf{拉格朗日函数}:
\[
\mathcal{L}(x, \lambda) = f(x) + \lambda^\top h(x)
\]
其中$h(x) = (h_1(x), \dots, h_m(x))^\top$是约束函数的向量形式,$\lambda^\top h(x) = \sum_{i=1}^m \lambda_i h_i(x)$是"约束项与乘子的耦合项"。

\subsection{4. 拉格朗日函数的最优性条件(一阶KKT条件)}
对拉格朗日函数$\mathcal{L}(x, \lambda)$分别关于$x$和$\lambda$求偏导,并令其为零,得到\textbf{一阶最优性条件}(等式约束下的KKT条件核心):

\begin{itemize}
    \item \textbf{对$x$求梯度并令其为零}:
    \[
    \nabla_x \mathcal{L}(x^*, \lambda^*) = \nabla f(x^*) + \sum_{i=1}^p \lambda_i^* \nabla h_i(x^*) = 0
    \]
    即目标函数的梯度可表示为约束函数梯度的\textbf{线性组合},组合系数就是拉格朗日乘子$\lambda_i^*$。这一条件体现了"目标函数与约束的梯度平衡"——在最优解处,目标函数的梯度被约束的梯度"抵消",使得在可行域的切空间内无下降方向。

    \item \textbf{对$\lambda$求梯度并令其为零}:
    \[
    \nabla_\lambda \mathcal{L}(x^*, \lambda^*) = h(x^*) = 0
    \]
    这就是原问题的\textbf{等式约束条件},保证最优解满足约束。
\end{itemize}

\subsection{5. 几何意义(直观理解)}
等式约束$h_i(x)=0$定义了一个\textbf{可行域(流形)},其在最优解$x^*$处的\textbf{切空间}由"与所有$\nabla h_i(x^*)$正交的方向"构成。

目标函数的梯度$\nabla f(x^*)$若要使$x^*$是极值点,必须"无法在切空间内找到下降方向",即$\nabla f(x^*)$必须位于可行域的\textbf{法空间}中(法空间由$\nabla h_1(x^*), \dots, \nabla h_m(x^*)$张成)。因此,$\nabla f(x^*)$可表示为这些法向量的线性组合,这正是$\nabla_x \mathcal{L} = 0$所表达的"梯度线性组合"关系。

综上,拉格朗日函数的建模是通过\textbf{引入乘子变量$\lambda$},将\textbf{等式约束转化为无约束优化的梯度条件},从而把"带约束的优化"转化为"对$(x, \lambda)$的无约束优化(在一阶条件意义下)",实现了从无约束到等式约束优化的数学衔接。

\section{拉格朗日乘子法的必要条件}

\subsection{一阶必要条件及其证明}

\begin{theorem}[一阶必要条件]
设$x^*$是等式约束优化问题的局部极小点,且约束梯度$\nabla g_1(x^*), \dots, \nabla g_m(x^*)$线性无关(即\textbf{约束规格满足})。则存在唯一的拉格朗日乘子$\lambda^* = (\lambda_1^*, \dots, \lambda_m^*)^T$使得:
\[
\begin{cases}
\nabla_x \mathcal{L}(x^*, \lambda^*) = 0 \\
\nabla_\lambda \mathcal{L}(x^*, \lambda^*) = 0
\end{cases}
\]
\end{theorem}

\subsubsection{证明}

\begin{proof}
考虑等式约束优化问题:
\[
\begin{cases} \min_{x \in \mathbb{R}^n} & f(x) \\ \text{s.t.} & h_i(x) = 0,\ i=1,\dots,m \end{cases}
\]
其中$f, h_i$连续可微,$x^*$是局部极小点,且\textbf{约束梯度线性无关}(即约束规格满足):$\nabla h_1(x^*), \dots, \nabla h_m(x^*)$线性无关。

记约束函数的向量形式为$h(x) = (h_1(x), \dots, h_m(x))^\top$,其雅可比矩阵为$J_h(x) \in \mathbb{R}^{m \times n}$,第$i$行为$\nabla h_i(x)^\top$。由约束梯度线性无关,$J_h(x^*)$的\textbf{秩为$m$}(列满秩)。

将变量$x$分块为$x = (y, z)$,其中$y \in \mathbb{R}^m$,$z \in \mathbb{R}^{n-m}$(通过变量重排,可假设$J_h(x^*)$的前$m$列构成的子矩阵$\nabla_y h(x^*)$是\textbf{可逆的$m \times m$矩阵})。

根据\textbf{隐函数定理},存在$x^*$的邻域和可微函数$g: \mathbb{R}^{n-m} \to \mathbb{R}^m$,使得在该邻域内,约束$h(y, z) = 0$可唯一表示为$y = g(z)$,且$g(z^*) = y^*$(即$x^* = (y^*, z^*)$)。

原约束问题可转化为关于$z$的\textbf{无约束优化问题}:
\[
\min_{z \in \mathbb{R}^{n-m}} \ f(g(z), z)
\]

由于$x^*$是原问题的局部极小点,$z^*$是上述无约束问题的局部极小点。根据\textbf{无约束优化的一阶必要条件},对$z$的梯度为0:
\[
\nabla_z f(x^*) + \nabla_y f(x^*) \cdot \nabla_z g(z^*) = 0 \tag{1}
\]

对$h(g(z), z) = 0$关于$z$求导,由链式法则得:
\[
\nabla_y h(x^*) \cdot \nabla_z g(z^*) + \nabla_z h(x^*) = 0
\]

由于$\nabla_y h(x^*)$可逆,解得:
\[
\nabla_z g(z^*) = - \left( \nabla_y h(x^*) \right)^{-1} \nabla_z h(x^*) \tag{2}
\]

将式(2)代入式(1):
\[
\nabla_z f(x^*) - \nabla_y f(x^*) \left( \nabla_y h(x^*) \right)^{-1} \nabla_z h(x^*) = 0
\]

定义\textbf{拉格朗日乘子}$\lambda^* = \left( \nabla_y h(x^*) \right)^{-T} \nabla_y f(x^*)^\top$(转置是因为矩阵逆的转置等于转置的逆)。

拉格朗日函数为$\mathcal{L}(x, \lambda) = f(x) + \lambda^\top h(x)$,其对$x$的梯度为:
\[
\nabla_x \mathcal{L}(x, \lambda) = \nabla f(x) + J_h(x)^\top \lambda
\]

将$\lambda^*$代入,分块验证:
\begin{itemize}
    \item \textbf{$y$分量}:$\nabla_y f(x^*) + \nabla_y h(x^*)^\top \lambda^*$。代入$\lambda^* = \left( \nabla_y h(x^*) \right)^{-T} \nabla_y f(x^*)^\top$,得$\nabla_y f(x^*) + \nabla_y f(x^*) = 0$(转置后等式成立)。
    \item \textbf{$z$分量}:结合式(1)(2)的推导,可验证$\nabla_z f(x^*) + \nabla_z h(x^*)^\top \lambda^* = 0$。
\end{itemize}

因此,$\nabla_x \mathcal{L}(x^*, \lambda^*) = 0$;同时,$\nabla_\lambda \mathcal{L}(x^*, \lambda^*) = h(x^*) = 0$(因$x^*$是可行点)。

\textbf{唯一性:}

假设存在两个乘子$\lambda^*$和$\lambda'^*$,满足:
\[
\nabla f(x^*) + J_h(x^*)^\top \lambda^* = 0, \quad \nabla f(x^*) + J_h(x^*)^\top \lambda'^* = 0
\]

两式相减得$J_h(x^*)^\top (\lambda^* - \lambda'^*) = 0$。由于$J_h(x^*)$列满秩,$J_h(x^*)^\top$的零空间仅含零向量,故$\lambda^* = \lambda'^*$,唯一性得证。
\end{proof}

\subsection{二阶最优性条件及其证明}

\begin{theorem}[二阶必要条件]
设$x^*$是等式约束优化问题的局部极小点,$f$和$g_i$在$x^*$处二阶连续可微,且约束梯度$\nabla g_1(x^*), \dots, \nabla g_m(x^*)$线性无关。则存在$\lambda^*$使得一阶条件成立,且对任意满足$J_g(x^*)d = 0$的$d \in \mathbb{R}^n$,有:
\[
d^T \nabla^2_{xx} \mathcal{L}(x^*, \lambda^*) d \geq 0
\]
其中$\nabla^2_{xx} \mathcal{L}(x^*, \lambda^*)$是拉格朗日函数关于$x$的Hessian矩阵。
\end{theorem}

\begin{theorem}[二阶充分条件]
设$f$和$g_i$在$x^*$处二阶连续可微,存在$\lambda^*$满足一阶条件,且对任意非零向量$d$满足$J_g(x^*)d = 0$,有:
\[
d^T \nabla^2_{xx} \mathcal{L}(x^*, \lambda^*) d > 0
\]
则$x^*$是严格局部极小点。
\end{theorem}

\begin{proof}[二阶必要条件证明]
要证明\textbf{等式约束优化的二阶必要条件},我们通过\textbf{泰勒展开}结合\textbf{一阶必要条件}推导,步骤如下:

考虑等式约束优化问题:
\[
\begin{cases} \min_{x \in \mathbb{R}^n} & f(x) \\ \text{s.t.} & h_i(x) = 0,\ i=1,\dots,m \end{cases}
\]
其中$x^*$是局部极小点,$f, h_i$二阶连续可微,且约束梯度$\nabla h_1(x^*), \dots, \nabla h_m(x^*)$线性无关(约束规格满足)。

根据\textbf{一阶必要条件(拉格朗日条件)},存在拉格朗日乘子$\lambda^* = (\lambda_1^*, \dots, \lambda_m^*)^\top$,使得:
\[
\begin{cases} \nabla_x \mathcal{L}(x^*, \lambda^*) = \nabla f(x^*) + \sum_{i=1}^m \lambda_i^* \nabla h_i(x^*) = 0 \\ h(x^*) = 0 \end{cases}
\]
其中拉格朗日函数$\mathcal{L}(x, \lambda) = f(x) + \lambda^\top h(x)$。

考虑\textbf{切空间中的方向}$d \in \mathbb{R}^n$,即满足约束雅可比矩阵$J_h(x^*)$零空间的方向:
\[
J_h(x^*) d = 0 \implies \nabla h_i(x^*)^\top d = 0,\ \forall i=1,\dots,m
\]
(该方向$d$是"不违反约束的微小移动方向",即当$t$充分小时,$x^* + t d$是可行点)。

对$f(x^* + t d)$做\textbf{二阶泰勒展开}:
\[
f(x^* + t d) = f(x^*) + t \nabla f(x^*)^\top d + \frac{t^2}{2} d^\top \nabla^2 f(x^*) d + o(t^2)
\]

对每个约束$h_i(x^* + t d)$做\textbf{一阶泰勒展开}(因$h_i(x^*) = 0$且$\nabla h_i(x^*)^\top d = 0$,一阶项为0):
\[
h_i(x^* + t d) = 0 + t \nabla h_i(x^*)^\top d + \frac{t^2}{2} d^\top \nabla^2 h_i(x^*) d + o(t^2) = 0 \quad (\text{可行点})
\]

由一阶条件$\nabla f(x^*) = - \sum_{i=1}^m \lambda_i^* \nabla h_i(x^*)$,代入$\nabla f(x^*)^\top d$得:
\[
\nabla f(x^*)^\top d = - \sum_{i=1}^m \lambda_i^* \nabla h_i(x^*)^\top d = 0 \quad (\text{因 } \nabla h_i(x^*)^\top d = 0)
\]

因此,$f(x^* + t d) - f(x^*)$的展开式可简化为:
\[
f(x^* + t d) - f(x^*) = \frac{t^2}{2} d^\top \left( \nabla^2 f(x^*) + \sum_{i=1}^m \lambda_i^* \nabla^2 h_i(x^*) \right) d + o(t^2)
\]

注意到拉格朗日函数关于$x$的\textbf{二阶Hessian矩阵}为:
\[
\nabla^2_{xx} \mathcal{L}(x^*, \lambda^*) = \nabla^2 f(x^*) + \sum_{i=1}^m \lambda_i^* \nabla^2 h_i(x^*)
\]

因此,上式可写为:
\[
f(x^* + t d) - f(x^*) = \frac{t^2}{2} d^\top \nabla^2_{xx} \mathcal{L}(x^*, \lambda^*) d + o(t^2)
\]

由于$x^*$是\textbf{局部极小点},存在$\delta > 0$,使得对所有$t \in (0, \delta)$,若$x^* + t d$可行,则$f(x^* + t d) \geq f(x^*)$。

因此,对充分小的$t > 0$,有:
\[
\frac{t^2}{2} d^\top \nabla^2_{xx} \mathcal{L}(x^*, \lambda^*) d + o(t^2) \geq 0
\]

两边除以$\frac{t^2}{2}$($t > 0$,故$\frac{t^2}{2} > 0$),并令$t \to 0$,得:
\[
d^\top \nabla^2_{xx} \mathcal{L}(x^*, \lambda^*) d \geq 0
\]

对任意满足$J_h(x^*) d = 0$的$d \in \mathbb{R}^n$,有$d^\top \nabla^2_{xx} \mathcal{L}(x^*, \lambda^*) d \geq 0$,即二阶必要条件成立。
\end{proof}

\begin{proof}[二阶充分条件证明]
证明二阶充分条件,通过泰勒展开和严格局部极小的定义推导,步骤如下:

考虑等式约束优化问题:
\[
\begin{cases} \min_{x \in \mathbb{R}^n} & f(x) \\ \text{s.t.} & h_i(x) = 0,\ i=1,\dots,m \end{cases}
\]

已知:
\begin{enumerate}
    \item $f, h_i$在$x^*$处\textbf{二阶连续可微};
    \item 存在拉格朗日乘子$\lambda^*$,满足\textbf{一阶拉格朗日条件}:$\nabla_x \mathcal{L}(x^*, \lambda^*) = 0$且$h(x^*) = 0$(其中$\mathcal{L}(x, \lambda) = f(x) + \lambda^\top h(x)$是拉格朗日函数);
    \item 对任意\textbf{非零}向量$d \in \mathbb{R}^n$满足$J_h(x^*) d = 0$(即$\nabla h_i(x^*)^\top d = 0,\ \forall i=1,\dots,m$),有$d^\top \nabla^2_{xx} \mathcal{L}(x^*, \lambda^*) d > 0$(拉格朗日函数关于$x$的Hessian矩阵在切空间方向上正定)。
\end{enumerate}

\paragraph{证明目标}
需证明$x^*$是\textbf{严格局部极小点},即存在邻域$\mathcal{N}(x^*)$,使得对所有可行点$x \in \mathcal{N}(x^*)$且$x \neq x^*$,有$f(x) > f(x^*)$。

\paragraph{证明步骤}

\textbf{步骤1:可行点的局部参数化(切空间方向)}

设$d \in \mathbb{R}^n$满足$J_h(x^*) d = 0$(称此类$d$为\textbf{切空间方向},沿该方向移动不违反约束)。对充分小的$t$,定义$x(t) = x^* + t d$。

对约束$h_i(x(t))$做\textbf{二阶泰勒展开}:
\[
h_i(x(t)) = h_i(x^*) + t \nabla h_i(x^*)^\top d + \frac{t^2}{2} d^\top \nabla^2 h_i(x^*) d + o(t^2)
\]

由$h_i(x^*) = 0$且$\nabla h_i(x^*)^\top d = 0$,得:
\[
h_i(x(t)) = \frac{t^2}{2} d^\top \nabla^2 h_i(x^*) d + o(t^2)
\]

当$t$充分小时,$h_i(x(t)) = 0$(高阶小量可忽略),故$x(t)$是\textbf{可行点}。

\textbf{步骤2:拉格朗日函数的二阶泰勒展开}

拉格朗日函数$\mathcal{L}(x, \lambda^*) = f(x) + (\lambda^*)^\top h(x)$,对$x(t)$做\textbf{二阶泰勒展开}:
\[
\mathcal{L}(x(t), \lambda^*) = \mathcal{L}(x^*, \lambda^*) + t \nabla_x \mathcal{L}(x^*, \lambda^*)^\top d + \frac{t^2}{2} d^\top \nabla^2_{xx} \mathcal{L}(x^*, \lambda^*) d + o(t^2)
\]

由\textbf{一阶拉格朗日条件}$\nabla_x \mathcal{L}(x^*, \lambda^*) = 0$,上式简化为:
\[
\mathcal{L}(x(t), \lambda^*) = \mathcal{L}(x^*, \lambda^*) + \frac{t^2}{2} d^\top \nabla^2_{xx} \mathcal{L}(x^*, \lambda^*) d + o(t^2) \tag{1}
\]

\textbf{步骤3:结合可行点的目标函数}

因$x(t)$是可行点($h(x(t)) = 0$),故$\mathcal{L}(x(t), \lambda^*) = f(x(t))$。同时,$\mathcal{L}(x^*, \lambda^*) = f(x^*)$(因$h(x^*) = 0$)。

将式(1)改写为:
\[
f(x(t)) = f(x^*) + \frac{t^2}{2} d^\top \nabla^2_{xx} \mathcal{L}(x^*, \lambda^*) d + o(t^2)
\]

\textbf{步骤4:利用二阶正定性推导严格不等式}

由已知条件,对\textbf{非零}$d$,有$d^\top \nabla^2_{xx} \mathcal{L}(x^*, \lambda^*) d > 0$。因此,存在$t_0 > 0$,当$0 < |t| < t_0$时:
\[
\frac{t^2}{2} d^\top \nabla^2_{xx} \mathcal{L}(x^*, \lambda^*) d + o(t^2) > 0
\]
即$f(x(t)) > f(x^*)$。

\textbf{步骤5:验证严格局部极小}

由隐函数定理,可行域在$x^*$附近的局部结构可由所有切空间方向$d$生成的$x(t)$覆盖。因此,存在$x^*$的邻域$\mathcal{N}(x^*)$,使得对所有可行点$x \in \mathcal{N}(x^*)$且$x \neq x^*$,必有$f(x) > f(x^*)$,即$x^*$是\textbf{严格局部极小点}。
\end{proof}
