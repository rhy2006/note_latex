\chapter{从等式约束到不等式约束:对偶}

\section{从KKT条件到对偶问题的严格数学建模}

\subsection{原始问题(Primal Problem)建模}

\subsubsection{定义(原始约束优化问题)}
给定变量空间 $\mathbb{R}^n$,目标函数与约束函数满足:
\begin{itemize}
    \item 目标函数:$f: \mathbb{R}^n \to \mathbb{R}$(从n维欧氏空间到实数域的映射)
    \item 等式约束函数:$g_i: \mathbb{R}^n \to \mathbb{R}$($i=1,2,\dots,m$),共$m$个等式约束
    \item 不等式约束函数:$h_j: \mathbb{R}^n \to \mathbb{R}$($j=1,2,\dots,p$),共$p$个不等式约束
\end{itemize}

原始问题的数学表达式为:
\[
\begin{cases}
\min\limits_{x \in \mathbb{R}^n} & f(x) \\
\text{subject to} & g_i(x) = 0, \quad i=1,2,\dots,m \\
& h_j(x) \leq 0, \quad j=1,2,\dots,p
\end{cases}
\]
其中,$x = (x_1,x_2,\dots,x_n)^T$ 为原始问题的优化变量,原始问题的最优值记为 $p^* = \inf\{f(x) \mid x \text{ 满足所有约束}\}$。


\subsection{拉格朗日函数建模}

\subsubsection{定义(拉格朗日函数)}
引入拉格朗日乘子:
\begin{itemize}
    \item 等式约束乘子:$\lambda = (\lambda_1,\lambda_2,\dots,\lambda_m)^T \in \mathbb{R}^m$(无符号限制)
    \item 不等式约束乘子:$\mu = (\mu_1,\mu_2,\dots,\mu_p)^T \in \mathbb{R}^p$(后续将限定非负)
\end{itemize}

拉格朗日函数定义为:
\[
\mathcal{L}(x, \lambda, \mu) = f(x) + \sum_{i=1}^m \lambda_i g_i(x) + \sum_{j=1}^p \mu_j h_j(x)
\]
其中,$\mathcal{L}: \mathbb{R}^n \times \mathbb{R}^m \times \mathbb{R}^p \to \mathbb{R}$(从$n+m+p$维乘积空间到实数域的映射)。


\subsection{拉格朗日对偶函数建模}

\begin{definition}[拉格朗日对偶函数]
拉格朗日对偶函数是拉格朗日函数关于原始变量$x$的\textbf{逐点下确界},定义为:
\[
d(\lambda, \mu) = \inf_{x \in \mathbb{R}^n} \mathcal{L}(x, \lambda, \mu)
\]
其中,$d: \mathbb{R}^m \times \mathbb{R}^p \to \mathbb{R} \cup \{-\infty\}$(对偶函数的值域可包含负无穷,当下确界不存在时取$-\infty$)。
\end{definition}


\subsection{对偶函数的核心性质建模}

\begin{theorem}[对偶函数的凹性]
对偶函数 $d(\lambda, \mu)$ 是关于 $(\lambda, \mu)$ 的凹函数。
\end{theorem}

\begin{proof}
\begin{enumerate}
    \item 对任意固定的 $x \in \mathbb{R}^n$,构造函数 $\phi_{x}(\lambda, \mu) = \mathcal{L}(x, \lambda, \mu)$。由拉格朗日函数的定义可知:
    \[
    \phi_{x}(\lambda, \mu) = f(x) + \sum_{i=1}^m \lambda_i g_i(x) + \sum_{j=1}^p \mu_j h_j(x)
    \]
    其中 $f(x), g_i(x), h_j(x)$ 均与 $(\lambda, \mu)$ 无关,因此 $\phi_{x}(\lambda, \mu)$ 是关于 $(\lambda, \mu)$ 的\textbf{仿射函数}(线性函数加常数项)。
    
    \item 对偶函数 $d(\lambda, \mu) = \inf_{x \in \mathbb{R}^n} \phi_{x}(\lambda, \mu)$,即对偶函数是一族仿射函数 $\{\phi_{x}(\lambda, \mu) \mid x \in \mathbb{R}^n\}$ 的逐点下确界。
    
    \item 由凸分析基本性质:\textbf{一族仿射函数的逐点下确界是凹函数},因此 $d(\lambda, \mu)$ 是凹函数。
\end{enumerate}
\end{proof}

\begin{theorem}[对偶函数的下界性质]
对任意 $\lambda \in \mathbb{R}^m$ 和 $\mu \geq 0$(即 $\mu_j \geq 0$ 对所有 $j=1,2,\dots,p$ 成立),有:
\[
d(\lambda, \mu) \leq p^*
\]
其中 $p^*$ 是原始问题的最优值。
\end{theorem}

\begin{proof}
\begin{enumerate}
    \item 设 $x$ 是原始问题的任意可行点,即满足:
    \[
    g_i(x) = 0 \quad (i=1,\dots,m), \quad h_j(x) \leq 0 \quad (j=1,\dots,p)
    \]
    
    \item 由于 $\mu \geq 0$ 且 $h_j(x) \leq 0$,可得:
    \[
    \sum_{j=1}^p \mu_j h_j(x) \leq 0
    \]
    又因为 $g_i(x) = 0$,故:
    \[
    \sum_{i=1}^m \lambda_i g_i(x) + \sum_{j=1}^p \mu_j h_j(x) = 0 + \sum_{j=1}^p \mu_j h_j(x) \leq 0
    \]
    
    \item 代入拉格朗日函数得:
    \[
    \mathcal{L}(x, \lambda, \mu) = f(x) + \sum_{i=1}^m \lambda_i g_i(x) + \sum_{j=1}^p \mu_j h_j(x) \leq f(x)
    \]
    
    \item 由对偶函数的定义(下确界性质),对任意 $x \in \mathbb{R}^n$ 有:
    \[
    d(\lambda, \mu) = \inf_{y \in \mathbb{R}^n} \mathcal{L}(y, \lambda, \mu) \leq \mathcal{L}(x, \lambda, \mu)
    \]
    
    \item 结合步骤3和步骤4,得:
    \[
    d(\lambda, \mu) \leq \mathcal{L}(x, \lambda, \mu) \leq f(x)
    \]
    
    \item 由于该不等式对所有原始可行点 $x$ 成立,而 \textbf{$p^*$ 是所有可行点对应的 $f(x)$ 的下确界},因此:
    \[
    d(\lambda, \mu) \leq p^*
    \]
\end{enumerate}
\end{proof}


\subsection{拉格朗日对偶问题建模}

\subsubsection{定义(对偶问题)}
基于对偶函数的下界性质,对偶问题的目标是\textbf{最大化对偶函数的下界},同时满足乘子约束 $\mu \geq 0$。其数学表达式为:
\[
\begin{cases}
\max\limits_{\lambda \in \mathbb{R}^m, \mu \in \mathbb{R}^p} & d(\lambda, \mu) \\
\text{subject to} & \mu_j \geq 0, \quad j=1,2,\dots,p
\end{cases}
\]

\subsubsection{关键定义补充}
\begin{enumerate}
    \item 对偶问题的最优值:$d^* = \sup\{d(\lambda, \mu) \mid \mu \geq 0\}$(上确界,因对偶函数是凹函数,最大值若存在则上确界等于最大值)。
    \item 对偶间隙:原始问题最优值与对偶问题最优值的差值,即 $\Delta = p^* - d^*$。
    \item 弱对偶性:由定理4.2直接可得 $d^* \leq p^*$,即对偶间隙非负($\Delta \geq 0$),该性质对所有原始-对偶问题对恒成立。
\end{enumerate}


\section{强对偶}

当对偶间隙消失(即原始问题与对偶问题的最优值相等)时,称\textbf{强对偶成立},这是对偶理论中"原始-对偶等价"的关键条件。

\begin{definition}[强对偶]
若对偶间隙$\Delta = 0$,即:
\[
p^* = d^*
\]
则称原始问题与对偶问题满足强对偶性。
\end{definition}

需特别说明:强对偶并非对所有约束优化问题恒成立,仅在满足特定条件(如Slater条件)时可被保证。


\subsection{Slater条件}

\begin{theorem}[Slater条件]
若原始问题是凸问题,且存在\textbf{严格可行点}$x_{\text{strict}} \in \mathbb{R}^n$,满足:
\[
\begin{cases}
g_i(x_{\text{strict}}) = 0, & i=1,\dots,m \quad (\text{等式约束仍严格满足}) \\
h_j(x_{\text{strict}}) < 0, & j=1,\dots,p \quad (\text{不等式约束严格满足,无"紧约束"})
\end{cases}
\]
则原始问题与对偶问题满足强对偶性($p^* = d^*$)。
\end{theorem}

\begin{remark}
Slater条件是强对偶成立的\textbf{充分条件,而非必要条件}——即满足Slater条件一定有强对偶,但强对偶成立时未必满足Slater条件。
\end{remark}


\subsection{几何解释}

构造三维乘积空间$\mathbb{R}^m \times \mathbb{R}^p \times \mathbb{R}$中的集合$\mathcal{G}$,其元素为满足"约束与目标函数不等式"的三元组$(u, v, t)$,数学定义为:
\[
\mathcal{G} = \left\{ (u, v, t) \in \mathbb{R}^m \times \mathbb{R}^p \times \mathbb{R} \mid \exists x \in \mathbb{R}^n, \ g_i(x)=u_i\ (i=1,\dots,m),\ h_j(x)\leq v_j\ (j=1,\dots,p),\ f(x)\leq t \right\}
\]
其中:
\begin{itemize}
    \item $u = (u_1,\dots,u_m)^T$(等式约束$g_i(x)$的取值);
    \item $v = (v_1,\dots,v_p)^T$(不等式约束$h_j(x)$的上界);
    \item $t$(目标函数$f(x)$的上界)。
\end{itemize}

原始问题的最优值$p^*$是"使$(0,0,t) \in \mathcal{G}$的最小$t$"——即当等式约束取$u=0$、不等式约束取$v=0$(满足原始约束)时,目标函数上界$t$的下确界,数学表达式为:
\[
p^* = \inf \left\{ t \mid (0, 0, t) \in \mathcal{G} \right\}
\]
几何意义:$\mathcal{G}$中所有"第一分量为0、第二分量为0"的点,其第三分量的最小值即为$p^*$。

对偶问题的最优值$d^*$是"对$\mu \geq 0$,在$\mathcal{G}$上最小化$t+\lambda^T u + \mu^T v$的最大值",数学表达式为:
\[
d^* = \sup_{\mu \geq 0} \inf_{(u, v, t) \in \mathcal{G}} \left\{ t + \lambda^T u + \mu^T v \right\}
\]
其中,内层下确界对应对偶函数$d(\lambda,\mu) = \inf_{(u,v,t)\in\mathcal{G}} \{t+\lambda^T u + \mu^T v\}$,外层上界对应对偶问题的最大化目标。

强对偶成立($p^* = d^*$)的几何意义是:原始最优值的"下确界"与对偶最优值的"上确界-下确界"相等,即$\mathcal{G}$的极值特性满足"对偶无间隙"。


\subsection{强对偶定理证明}

\begin{theorem}[强对偶定理]
若原始问题是凸问题($f$、$h_j$凸,$g_i$仿射),且满足Slater条件(存在严格可行点$x_{\text{strict}}$),则强对偶成立,即$p^* = d^*$。
\end{theorem}

\begin{proof}
定义集合$A = \mathcal{G}$(即前述几何建模中的凸集,因原始问题是凸问题,$A$是凸集);定义集合$B = \left\{ (0, 0, s) \in \mathbb{R}^m \times \mathbb{R}^p \times \mathbb{R} \mid s < p^* \right\}$(所有"前两分量为0、第三分量小于$p^*$"的点,$B$是凸集)。

\paragraph{关键引理:}$A$与$B$不相交。

若存在$(0,0,s) \in A \cap B$,则$s < p^*$且$(0,0,s) \in \mathcal{G}$——由$\mathcal{G}$的定义,存在$x$满足$g_i(x)=0$、$h_j(x)\leq0$、$f(x)\leq s < p^*$,这与$p^*$是原始问题最优值(最小$f(x)$)矛盾,故$A \cap B = \emptyset$。

由于$A$、$B$是$\mathbb{R}^{m+p+1}$中的不相交凸集,根据\textbf{凸集分离定理},存在非零向量$(\tilde{\lambda}, \tilde{\mu}, \nu) \in \mathbb{R}^m \times \mathbb{R}^p \times \mathbb{R}$和实数$\alpha \in \mathbb{R}$,使得对所有$(u, v, t) \in A$、$(0,0,s) \in B$,有:
\[
\begin{cases}
\tilde{\lambda}^T u + \tilde{\mu}^T v + \nu t \geq \alpha \quad (\text{超平面上方包含} \ A) \\
\tilde{\lambda}^T \cdot 0 + \tilde{\mu}^T \cdot 0 + \nu s \leq \alpha \quad (\text{超平面下方包含} \ B)
\end{cases}
\]

\paragraph{第一步:证明$\nu \geq 0$}

假设$\nu < 0$,对任意$(u, v, t) \in A$,当$t \to +\infty$时,$\tilde{\lambda}^T u + \tilde{\mu}^T v + \nu t \to -\infty$,与"$\geq \alpha$"矛盾,故$\nu \geq 0$。

\paragraph{第二步:证明$\nu \neq 0$}

假设$\nu = 0$,则分离不等式变为$\tilde{\lambda}^T u + \tilde{\mu}^T v \geq \alpha$(对所有$(u, v, t) \in A$)。由Slater条件,存在严格可行点$x_{\text{strict}}$,使得$g_i(x_{\text{strict}})=0$、$h_j(x_{\text{strict}})=v_j < 0$,即$(0, v, t) \in A$。代入得$\tilde{\mu}^T v \geq \alpha$。若令$v_j \to -\infty$(通过调整$x$),则$\tilde{\mu}^T v \to -\infty$,与"$\geq \alpha$"矛盾,故$\nu \neq 0$。

由于$\nu > 0$,可对分离向量$(\tilde{\lambda}, \tilde{\mu}, \nu)$进行缩放(不影响分离性质),令$\nu = 1$。此时:

\begin{enumerate}
    \item 分离不等式变为$\tilde{\lambda}^T u + \tilde{\mu}^T v + t \geq \alpha$(对所有$(u, v, t) \in A$)。对原始问题的任意可行点$x$,取$u_i = g_i(x)=0$、$v_j = h_j(x)\leq0$、$t = f(x)$,代入得:
    \[\tilde{\lambda}^T \cdot 0 + \tilde{\mu}^T h(x) + f(x) \geq \alpha\]
    
    \item 对$B$的不等式($\nu s \leq \alpha$,$s < p^*$),令$s \to p^*$,得$p^* \leq \alpha$(因$\nu=1$)。
    
    \item 结合1和2,得$f(x) + \tilde{\mu}^T h(x) \geq \alpha \geq p^*$。对所有可行$x$取下确界(即对偶函数定义):
    \[
    d(\tilde{\lambda}, \tilde{\mu}) = \inf_x \left\{ f(x) + \tilde{\lambda}^T g(x) + \tilde{\mu}^T h(x) \right\} \geq p^*
    \]
    
    \item 由弱对偶性($d(\lambda,\mu) \leq p^*$),得$d(\tilde{\lambda}, \tilde{\mu}) = p^*$。
\end{enumerate}

假设存在$j$使得$\tilde{\mu}_j < 0$,则选择$x$使$h_j(x)$足够大(正值),可令$\tilde{\lambda}^T g(x) + \tilde{\mu}^T h(x) + f(x) \to -\infty$,与"$\geq p^*$"矛盾,故$\tilde{\mu} \geq 0$。

$(\tilde{\lambda}, \tilde{\mu})$是对偶问题的可行解($\tilde{\mu} \geq 0$),且$d(\tilde{\lambda}, \tilde{\mu}) = p^*$。由对偶最优值$d^* = \sup\{d(\lambda,\mu) \mid \mu \geq 0\}$,得$d^* \geq p^*$;结合弱对偶性$d^* \leq p^*$,故$d^* = p^*$,强对偶成立。
\end{proof}


\section{互补松弛条件}

设原始问题与对偶问题满足:
\begin{enumerate}
    \item 原始问题最优解$x^*$存在,对偶问题最优解$(\lambda^*, \mu^*)$存在;
    \item 强对偶成立($p^* = d^*$)。
\end{enumerate}

则必有以下两个等价结论:
\begin{enumerate}
    \item \textbf{拉格朗日函数极值等式}:$\mathcal{L}(x^*, \lambda^*, \mu^*) = f(x^*)$;
    \item \textbf{乘子-约束乘积为零}:对所有不等式约束的下标$j=1,2,\dots,p$,有
    \[\mu_j^* \cdot h_j(x^*) = 0\]
\end{enumerate}

(注:结论2是互补松弛条件的"核心量化形式",也是实际应用中最常用的表述。)

\begin{proof}
由强对偶成立($p^* = d^*$),结合原始最优值与对偶函数的定义,可得:
\[
f(x^*) = p^* = d^* = d(\lambda^*, \mu^*) \tag{1}
\]

根据对偶函数的定义($d(\lambda,\mu) = \inf_{x \in \mathbb{R}^n} \mathcal{L}(x,\lambda,\mu)$),对任意$x \in \mathbb{R}^n$,对偶函数是拉格朗日函数的下确界,因此:
\[
d(\lambda^*, \mu^*) \leq \mathcal{L}(x^*, \lambda^*, \mu^*) \tag{2}
\]

结合式(1)与式(2),得:
\[
f(x^*) \leq \mathcal{L}(x^*, \lambda^*, \mu^*) \tag{3}
\]

将拉格朗日函数在$(x^*, \lambda^*, \mu^*)$处展开:
\[
\mathcal{L}(x^*, \lambda^*, \mu^*) = f(x^*) + \sum_{i=1}^m \lambda_i^* g_i(x^*) + \sum_{j=1}^p \mu_j^* h_j(x^*) \tag{4}
\]

由\textbf{原始可行性}($x^*$满足等式约束),对所有$i=1,\dots,m$,有$g_i(x^*) = 0$,因此式(4)中的等式约束乘子项消失:
\[
\mathcal{L}(x^*, \lambda^*, \mu^*) = f(x^*) + \sum_{j=1}^p \mu_j^* h_j(x^*) \tag{5}
\]

由\textbf{对偶可行性}($\mu^* \geq 0$),对所有$j=1,\dots,p$,有$\mu_j^* \geq 0$;由\textbf{原始可行性}($x^*$满足不等式约束),对所有$j=1,\dots,p$,有$h_j(x^*) \leq 0$。

因此,对每个$j$,$\mu_j^* \cdot h_j(x^*) \leq 0$(非正数),求和后仍为非正数:
\[
\sum_{j=1}^p \mu_j^* h_j(x^*) \leq 0 \tag{6}
\]

将式(6)代入式(5),得:
\[
\mathcal{L}(x^*, \lambda^*, \mu^*) \leq f(x^*) \tag{7}
\]

结合式(3)($f(x^*) \leq \mathcal{L}(x^*, \lambda^*, \mu^*)$)与式(7)($\mathcal{L}(x^*, \lambda^*, \mu^*) \leq f(x^*)$),所有不等式变为等式:
\[
\mathcal{L}(x^*, \lambda^*, \mu^*) = f(x^*) \tag{8}
\]

将式(8)代入式(5),得:
\[
f(x^*) = f(x^*) + \sum_{j=1}^p \mu_j^* h_j(x^*) \implies \sum_{j=1}^p \mu_j^* h_j(x^*) = 0 \tag{9}
\]

由步骤3可知,每个$\mu_j^* h_j(x^*) \leq 0$(非正数),而\textbf{非正数的和为零,当且仅当每个非正数均为零}。因此:
\[
\mu_j^* h_j(x^*) = 0 \quad \forall j=1,2,\dots,p \tag{10}
\]

式(8)与式(10)共同构成互补松弛条件。
\end{proof}

\begin{remark}
互补松弛条件$\mu_j^* h_j(x^*) = 0$的本质是\textbf{判断不等式约束对最优解的"活性"},可分为两种互斥情况,直观反映约束是否影响最优解:

\begin{itemize}
    \item \textbf{情况1}:$\mu_j^* > 0$(对偶乘子为正)。对应约束$h_j(x) \leq 0$是\textbf{紧约束}(起作用的约束)。最优解$x^*$恰好落在约束边界上($h_j(x^*) = 0$),该约束限制了目标函数的进一步优化。
    \item \textbf{情况2}:$\mu_j^* = 0$(对偶乘子为零)。对应约束$h_j(x) \leq 0$是\textbf{非紧约束}(不起作用的约束)。最优解$x^*$落在约束内部($h_j(x^*) < 0$),即使移除该约束,最优解也不会改变。
\end{itemize}

(注:等式约束$g_i(x) = 0$始终为"紧约束",无互补松弛判断,因其对偶乘子$\lambda_i^*$无符号限制,无需通过乘积为零判断活性。)
\end{remark}

互补松弛条件是KKT(Karush-Kuhn-Tucker)最优性条件的重要组成部分。在\textbf{凸问题+强对偶成立+函数可微}的前提下,KKT条件是原始-对偶最优解的充要条件,其结构如下:

\subsection{KKT条件的完整构成(含互补松弛)}
设原始问题为凸问题($f$、$h_j$凸,$g_i$仿射),$f$、$g_i$、$h_j$可微,$x^*$为原始最优解,$(\lambda^*, \mu^*)$为对偶最优解,则:
\begin{enumerate}
    \item \textbf{平稳性}:$\nabla_x \mathcal{L}(x^*, \lambda^*, \mu^*) = 0$(拉格朗日函数在$x^*$处梯度为零,即无改进方向);
    \item \textbf{原始可行性}:$g_i(x^*) = 0\ (i=1,\dots,m)$,$h_j(x^*) \leq 0\ (j=1,\dots,p)$;
    \item \textbf{对偶可行性}:$\mu_j^* \geq 0\ (j=1,\dots,p)$;
    \item \textbf{互补松弛}:$\mu_j^* h_j(x^*) = 0\ (j=1,\dots,p)$。
\end{enumerate}

可见,互补松弛条件是KKT条件的"闭环环节"——它连接了原始约束的可行性($h_j(x^*) \leq 0$)与对偶乘子的可行性($\mu_j^* \geq 0$),确保原始-对偶最优解的一致性。


\section{对偶理论的应用实例}

\subsection{线性规划的对偶}

考虑线性规划问题(原始问题):
\[
\begin{cases}
\min\limits_{x} & c^{T} x \\
\text{subject to} & A x = b \\
& x \geq 0
\end{cases}
\]
其中,$x \in \mathbb{R}^n$ 为原始优化变量,$c \in \mathbb{R}^n$、$A \in \mathbb{R}^{m \times n}$、$b \in \mathbb{R}^m$ 为已知参数。

引入拉格朗日乘子:
\begin{itemize}
    \item 等式约束 $A x = b$ 对应的乘子:$\lambda \in \mathbb{R}^m$(无符号限制);
    \item 不等式约束 $x \geq 0$ 对应的乘子:$\mu \in \mathbb{R}^n$(满足 $\mu \geq 0$)。
\end{itemize}

拉格朗日函数定义为:
\[
\mathcal{L}(x, \lambda, \mu) = c^{T} x + \lambda^{T}(b - A x) - \mu^{T} x
\]

对偶函数 $d(\lambda, \mu)$ 是拉格朗日函数关于 $x$ 的下确界,即:
\[
d(\lambda, \mu) = \inf_{x} \left[ c^{T} x + \lambda^{T}(b - A x) - \mu^{T} x \right]
\]
将函数整理为关于 $x$ 的线性形式:
\[
d(\lambda, \mu) = \inf_{x} \left[ (c - A^{T} \lambda - \mu)^{T} x + \lambda^{T} b \right]
\]

为使下确界有限(避免取值为 $-\infty$),需满足线性项系数为零:
\[
c - A^{T} \lambda - \mu = 0 \implies \mu = c - A^{T} \lambda
\]
结合 $\mu \geq 0$ 的约束,可得:
\[
A^{T} \lambda \leq c
\]

将 $\mu = c - A^{T} \lambda$ 代入拉格朗日函数,对偶函数简化为:
\[
d(\lambda) = \lambda^{T} b
\]

对偶问题的目标是最大化对偶函数 $d(\lambda)$,同时满足对偶可行性约束,即:
\[
\begin{cases}
\max\limits_{\lambda} & b^{T} \lambda \\
\text{subject to} & A^{T} \lambda \leq c
\end{cases}
\]
此为线性规划的标准对偶形式。


\subsection{二次规划的对偶}

考虑二次规划问题(原始问题):
\[
\begin{cases}
\min\limits_{x} & \frac{1}{2} x^{T} Q x + c^{T} x \\
\text{subject to} & A x = b \\
& x \geq 0
\end{cases}
\]
其中,$x \in \mathbb{R}^n$ 为原始优化变量,$Q \in \mathbb{R}^{n \times n}$ 为\textbf{半正定矩阵}($Q \succeq 0$,保证目标函数凸),$c \in \mathbb{R}^n$、$A \in \mathbb{R}^{m \times n}$、$b \in \mathbb{R}^m$ 为已知参数。

引入拉格朗日乘子:
\begin{itemize}
    \item 等式约束 $A x = b$ 对应的乘子:$\lambda \in \mathbb{R}^m$(无符号限制);
    \item 不等式约束 $x \geq 0$ 对应的乘子:$\mu \in \mathbb{R}^n$(满足 $\mu \geq 0$)。
\end{itemize}

拉格朗日函数定义为:
\[
\mathcal{L}(x, \lambda, \mu) = \frac{1}{2} x^{T} Q x + c^{T} x + \lambda^{T}(b - A x) - \mu^{T} x
\]

对偶函数 $d(\lambda, \mu)$ 是拉格朗日函数关于 $x$ 的下确界,即:
\[
d(\lambda, \mu) = \inf_{x} \left[ \frac{1}{2} x^{T} Q x + \left( c - A^{T} \lambda - \mu \right)^{T} x + \lambda^{T} b \right]
\]

由于目标函数是关于 $x$ 的二次函数,且 $Q \succeq 0$(凸函数),当 $Q$ \textbf{正定}($Q \succ 0$)时,函数存在唯一极小值。对 $x$ 求导并令梯度为零,得最优 $x$:
\[
\nabla_x \mathcal{L} = Q x + (c - A^{T} \lambda - \mu) = 0 \implies x = -Q^{-1}(c - A^{T} \lambda - \mu)
\]

将最优 $x$ 代入拉格朗日函数,化简得对偶函数:
\[
d(\lambda, \mu) = -\frac{1}{2} \left( c - A^{T} \lambda - \mu \right)^{T} Q^{-1} \left( c - A^{T} \lambda - \mu \right) + \lambda^{T} b
\]

对偶问题的目标是最大化对偶函数 $d(\lambda, \mu)$,同时满足对偶可行性约束,即:
\[
\begin{cases}
\max\limits_{\lambda, \mu} & -\frac{1}{2} \left( c - A^{T} \lambda - \mu \right)^{T} Q^{-1} \left( c - A^{T} \lambda - \mu \right) + \lambda^{T} b \\
\text{subject to} & \mu \geq 0
\end{cases}
\]


\subsection{数值例题(二次规划对偶求解)}

以具体二次规划问题为例,验证对偶理论的应用及强对偶性。

\[
\begin{cases}
\min\limits_{x_1, x_2} & \frac{1}{2}(x_1^2 + x_2^2) \\
\text{subject to} & x_1 + x_2 = 1 \quad (\text{等式约束}) \\
& x_1 \geq 0, \ x_2 \geq 0 \quad (\text{不等式约束})
\end{cases}
\]

\subsubsection{步骤1:构造拉格朗日函数}
引入乘子:
\begin{itemize}
    \item 等式约束 $x_1 + x_2 = 1$ 对应 $\lambda \in \mathbb{R}$;
    \item 不等式约束 $x_1 \geq 0$、$x_2 \geq 0$ 对应 $\mu_1 \geq 0$、$\mu_2 \geq 0$。
\end{itemize}

拉格朗日函数:
\[
\mathcal{L}(x, \lambda, \mu) = \frac{1}{2}(x_1^2 + x_2^2) + \lambda(1 - x_1 - x_2) - \mu_1 x_1 - \mu_2 x_2
\]

\subsubsection{步骤2:列写KKT条件}
\begin{enumerate}
    \item \textbf{平稳性}:$\nabla_x \mathcal{L} = 0$
    \[\frac{\partial \mathcal{L}}{\partial x_1} = x_1 - \lambda - \mu_1 = 0\]
    \[\frac{\partial \mathcal{L}}{\partial x_2} = x_2 - \lambda - \mu_2 = 0\]
    \item \textbf{原始可行性}:$x_1 + x_2 = 1$,$x_1 \geq 0$,$x_2 \geq 0$;
    \item \textbf{对偶可行性}:$\mu_1 \geq 0$,$\mu_2 \geq 0$;
    \item \textbf{互补松弛}:$\mu_1 x_1 = 0$,$\mu_2 x_2 = 0$。
\end{enumerate}

\subsubsection{步骤3:分析最优解}
假设 $x_1 > 0$ 且 $x_2 > 0$,由互补松弛条件得 $\mu_1 = \mu_2 = 0$。代入平稳性条件:
\[x_1 = \lambda, \quad x_2 = \lambda\]
结合等式约束 $x_1 + x_2 = 1$,得 $\lambda = 0.5$,$x_1 = x_2 = 0.5$。

验证所有约束均满足,因此原始最优解为 $x^* = (0.5, 0.5)$,原始最优值 $p^* = \frac{1}{2}(0.5^2 + 0.5^2) = 0.25$。

\subsubsection{步骤4:对偶问题求解}

对偶函数是拉格朗日函数关于 $x_1, x_2$ 的下确界:
\[
d(\lambda, \mu) = \inf_{x_1, x_2} \left[ \frac{1}{2}(x_1^2 + x_2^2) + \lambda(1 - x_1 - x_2) - \mu_1 x_1 - \mu_2 x_2 \right]
\]

对 $x_1, x_2$ 求导并令梯度为零,得:
\[x_1 = \lambda + \mu_1, \quad x_2 = \lambda + \mu_2\]

将 $x_1, x_2$ 代入拉格朗日函数,展开化简:
\begin{align*}
d(\lambda, \mu) &= \frac{1}{2}\left[ (\lambda + \mu_1)^2 + (\lambda + \mu_2)^2 \right] + \lambda\left(1 - 2\lambda - \mu_1 - \mu_2\right) \\
&\quad - \mu_1(\lambda + \mu_1) - \mu_2(\lambda + \mu_2) \\
&= -\lambda^2 - \lambda \mu_1 - \lambda \mu_2 - \frac{1}{2}\mu_1^2 - \frac{1}{2}\mu_2^2 + \lambda
\end{align*}

对偶问题为:
\[
\begin{cases}
\max\limits_{\lambda, \mu_1, \mu_2} & -\lambda^2 - \lambda \mu_1 - \lambda \mu_2 - \frac{1}{2}\mu_1^2 - \frac{1}{2}\mu_2^2 + \lambda \\
\text{subject to} & \mu_1 \geq 0, \ \mu_2 \geq 0
\end{cases}
\]

由原始最优解的互补松弛条件($\mu_1 = \mu_2 = 0$),代入对偶函数:
\[
d(\lambda, 0, 0) = -\lambda^2 + \lambda
\]

对 $\lambda$ 求导并令导数为零,得 $\lambda = 0.5$,此时对偶最优值 $d^* = -(0.5)^2 + 0.5 = 0.25$。

原始最优值 $p^* = 0.25$,对偶最优值 $d^* = 0.25$,满足 $p^* = d^*$,强对偶成立。


\subsection{SVM中的原问题与对偶问题}

支持向量机(SVM)是对偶理论在机器学习中的典型应用,核心是通过对偶问题简化高维特征空间中的优化求解。

\subsubsection{关键结论(原对偶关系)}
\begin{itemize}
    \item 原始问题(Primal):优化变量为模型参数 $w$(权重向量)、$b$(偏置),目标是最大化几何间隔;
    \item 对偶问题(Dual):优化变量为对偶乘子 $\alpha_i$(对应每个样本),目标是最大化对偶函数;
    \item 等价性:通过KKT条件联结原对偶解,满足 $w = \sum_{i} \alpha_i y_i x_i$、$\sum_{i} \alpha_i y_i = 0$、$0 \leq \alpha_i \leq C$($C$ 为罚系数);
    \item 支持向量特性:仅 $\alpha_i > 0$ 的样本(支持向量)决定 $w$ 与几何间隔 $1/\|w\|$;
    \item 核化能力:对偶问题仅依赖样本内积 $\langle x_i, x_j \rangle$,可直接替换为核函数 $K(x_i, x_j)$,实现高维空间映射。
\end{itemize}

\subsubsection{软间隔SVM的原始问题}
目标:在允许样本"软违约"(用 $\xi_i \geq 0$ 表示违约程度)的前提下,最小化 $\frac{1}{2}\|w\|^2$(等价于最大化几何间隔),同时控制违约惩罚。

\[
\begin{cases}
\min\limits_{w, b, \xi} & \frac{1}{2}\|w\|^2 + C \sum_{i=1}^n \xi_i \\
\text{subject to} & y_i(w^T x_i + b) \geq 1 - \xi_i \quad (i=1, ..., n) \\
& \xi_i \geq 0 \quad (i=1, ..., n)
\end{cases}
\]
其中,$C > 0$ 为罚系数(平衡间隔大小与违约惩罚),$y_i \in \{+1, -1\}$ 为样本标签,$\xi_i$ 为违约变量。

\subsubsection{拉格朗日函数与KKT条件}
引入对偶乘子:
\begin{itemize}
    \item 约束 $y_i(w^T x_i + b) \geq 1 - \xi_i$ 对应 $\alpha_i \geq 0$;
    \item 约束 $\xi_i \geq 0$ 对应 $\beta_i \geq 0$。
\end{itemize}

拉格朗日函数:
\[
\mathcal{L}(w, b, \xi, \alpha, \beta) = \frac{1}{2}\|w\|^2 + C \sum_{i=1}^n \xi_i - \sum_{i=1}^n \alpha_i \left[ y_i(w^T x_i + b) - 1 + \xi_i \right] - \sum_{i=1}^n \beta_i \xi_i
\]

对 $w, b, \xi_i$ 求导并令梯度为零:
\begin{itemize}
    \item 对 $w$:$\nabla_w \mathcal{L} = w - \sum_{i=1}^n \alpha_i y_i x_i = 0 \implies w = \sum_{i=1}^n \alpha_i y_i x_i$;
    \item 对 $b$:$\nabla_b \mathcal{L} = -\sum_{i=1}^n \alpha_i y_i = 0 \implies \sum_{i=1}^n \alpha_i y_i = 0$;
    \item 对 $\xi_i$:$\nabla_{\xi_i} \mathcal{L} = C - \alpha_i - \beta_i = 0 \implies \alpha_i + \beta_i = C$。
\end{itemize}

结合对偶可行性 $\alpha_i \geq 0$、$\beta_i \geq 0$,得 $0 \leq \alpha_i \leq C$。

\subsubsection{软间隔SVM的对偶问题}
消去原始变量 $w, b, \xi_i$,代入拉格朗日函数,最终对偶问题为:
\[
\begin{cases}
\max\limits_{\alpha} & \sum_{i=1}^n \alpha_i - \frac{1}{2} \sum_{i=1}^n \sum_{j=1}^n \alpha_i \alpha_j y_i y_j \langle x_i, x_j \rangle \\
\text{subject to} & 0 \leq \alpha_i \leq C \quad (i=1, ..., n) \\
& \sum_{i=1}^n \alpha_i y_i = 0
\end{cases}
\]

\paragraph{核技巧应用}
将内积 $\langle x_i, x_j \rangle$ 替换为核函数 $K(x_i, x_j)$(如线性核 $K(x_i, x_j) = x_i^T x_j$、RBF核 $K(x_i, x_j) = \exp(-\gamma\|x_i - x_j\|^2)$),即可得到核SVM的对偶形式,解决高维特征空间的优化问题。

\subsubsection{原对偶的意义与决策函数}

\paragraph{原对偶的核心价值}
\begin{itemize}
    \item \textbf{维度优势}:当特征维度 $d$ 极大(如文本分类)、样本数 $n$ 较小时,对偶问题(变量数为 $n$)比原始问题(变量数为 $d+1$)更易求解;
    \item \textbf{可解释性}:$\alpha_i$ 可视为"样本重要性权重"(支持向量的 $\alpha_i > 0$,非支持向量的 $\alpha_i = 0$);
    \item \textbf{最优性判据}:原始问题值 $P(w, b, \xi)$ 与对偶问题值 $D(\alpha)$ 的对偶间隙 $P - D \geq 0$,可作为算法收敛的停机准则。
\end{itemize}

\paragraph{决策函数与偏置计算}
\begin{itemize}
    \item 决策函数:已知对偶最优解 $\alpha^*$,结合核函数 $K$,模型对新样本 $x$ 的预测为:
    \[f(x) = \text{sign}\left( \sum_{i=1}^n \alpha_i^* y_i K(x_i, x) + b^* \right)\]
    \item 偏置 $b^*$:取任意满足 $0 < \alpha_i^* < C$ 的支持向量 $x_i$,代入约束 $y_i(w^* x_i + b^*) = 1$,得:
    \[b^* = y_i - \sum_{j=1}^n \alpha_j^* y_j K(x_j, x_i)\]
\end{itemize}
