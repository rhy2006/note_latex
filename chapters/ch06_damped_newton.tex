\chapter{阻尼牛顿法}

\section{牛顿法复习}

这是牛顿法区别于梯度下降的核心,用二次函数拟合目标函数局部形态:
\[
f(x+\Delta) \approx f(x) + \nabla f(x)^\top \Delta + \frac{1}{2}\Delta^\top H(x) \Delta
\]

对二次近似函数求极值(导数为0),解出的搜索方向即“牛顿步”:
\[
\Delta_{nt} = -H(x)^{-1} \nabla f(x)
\]

\begin{itemize}
    \item \textbf{作用}:直接给出使局部二次函数最小的方向,无需像梯度下降那样手动调整步长(理论上步长为1时局部最优)。
    \item \textbf{前提}:$H(x) \succ 0$(Hessian正定),保证解唯一且为最小值点。
\end{itemize}

通过梯度与牛顿步的内积,证明牛顿步是“有效下降方向”:
\[
\nabla f(x)^\top \Delta_{nt} = -\nabla f(x)^\top H(x)^{-1} \nabla f(x) < 0
\]

\begin{itemize}
    \item \textbf{作用}:确保沿牛顿步迭代时,目标函数值会减小(内积$<0$,方向与梯度相反且符合曲率)。
    \item \textbf{关键}:因 $H(x) \succ 0$,其逆矩阵也正定,故 $\nabla f(x)^\top H(x)^{-1} \nabla f(x) > 0$,最终内积为负。
\end{itemize}

靠近最优解 $x^*$ 时,迭代误差呈“平方级”减小,收敛速度远快于梯度下降:
\[
\| x_{k+1} - x^* \| \leq C \cdot \| x_k - x^* \|^2 \quad (C>0)
\]

\begin{itemize}
    \item \textbf{作用}:体现牛顿法的核心优势——一旦进入“局部收敛域”,迭代会快速收敛到最优解。
    \item \textbf{前提}:$H(x)$ 在 $x^*$ 邻域 Lipschitz 连续,且初始点 $x^{(0)}$ 足够靠近 $x^*$。
\end{itemize}

\section{阻尼牛顿法}

阻尼牛顿法(Damped Newton Method)是为解决\textbf{纯牛顿法在远离最优解时可能不下降、发散}的问题而提出的改进方法,核心思路是:\textbf{保留牛顿方向的优势,通过引入线搜索(Line Search)确定合适的步长},而非纯牛顿法中默认的步长1,从而保证每次迭代都能使目标函数值下降,兼顾“局部快速收敛”与“全局有效下降”。

\subsection{纯牛顿法的局限性(阻尼牛顿法的必要性)}
纯牛顿法的迭代公式为 $x_{k+1} = x_k + \Delta_{nt}$(步长固定为1),但存在两个关键问题:
\begin{enumerate}
    \item \textbf{Hessian矩阵非正定}:当 $H(x_k)$ 不正定时,牛顿步 $\Delta_{nt} = -H(x_k)^{-1}\nabla f(x_k)$ 可能不是下降方向(甚至是上升方向),导致 $f(x_{k+1}) > f(x_k)$。
    \item \textbf{步长1过大}:即使 $H(x_k)$ 正定(牛顿步是下降方向),但远离最优解时,二次近似的误差较大,步长1可能导致迭代点“迈过”最优解,反而使函数值上升。
\end{enumerate}

\subsection{阻尼牛顿法的核心改进:牛顿方向 + 线搜索}
阻尼牛顿法保留“牛顿步”作为搜索方向(利用二阶信息的高效性),但通过\textbf{线搜索}动态调整步长 $\alpha_k$($\alpha_k > 0$),确保每次迭代满足 $f(x_{k+1}) < f(x_k)$。

\textbf{迭代公式:}
\[
x_{k+1} = x_k + \alpha_k \cdot \Delta_{nt}
\]
其中:
\begin{itemize}
    \item $\Delta_{nt} = -H(x_k)^{-1}\nabla f(x_k)$ 是牛顿方向(与纯牛顿法一致);
    \item $\alpha_k$ 是通过线搜索确定的步长(核心改进点,替代固定步长1)。
\end{itemize}

\subsection{线搜索:如何确定步长 $\alpha_k$?}
线搜索的目标是找到最小的 $\alpha_k$(通常从1开始尝试),使得目标函数值“充分下降”。常用准则为\textbf{Armijo准则}(保证下降性的同时避免步长过小):

\begin{definition}[Armijo准则]
Armijo准则(Armijo Criterion)是\textbf{线搜索(Line Search)中用于确定合适步长}的经典准则。

设当前迭代点为 $x_k$,搜索方向为 $d_k$(需满足“下降方向”,即 $\nabla f(x_k)^\top d_k < 0$),步长为 $\alpha > 0$。Armijo准则要求步长 $\alpha$ 满足:
\[
f(x_k + \alpha \cdot d_k) \leq f(x_k) + c \cdot \alpha \cdot \nabla f(x_k)^\top d_k
\]
其中:
\begin{itemize}
    \item $f(\cdot)$ 是目标函数;
    \item $\nabla f(x_k)$ 是 $f$ 在 $x_k$ 处的梯度;
    \item $c$ 是预设常数,满足 $0 < c < 1$(通常取 $c=10^{-4}$,控制“最小可接受的下降量”)。
\end{itemize}
\end{definition}

\textbf{实际计算步骤(回溯线搜索):}
\begin{enumerate}
    \item 初始尝试步长 $\alpha = 1$(纯牛顿法的默认步长,靠近最优解时通常有效);
    \item 检查是否满足 Armijo 准则:若 $f(x_k + \alpha d_k) \leq f(x_k) + c \cdot \alpha \cdot \nabla f(x_k)^\top d_k$,则接受该 $\alpha$;
    \item 若不满足,缩小步长(如乘以收缩因子 $\beta$,$0 < \beta < 1$,常用 $\beta=0.5$),即 $\alpha = \beta \cdot \alpha$,重复步骤2;
    \item 直到找到满足准则的 $\alpha$(理论上,因 $d_k$ 是下降方向,当 $\alpha \to 0$ 时准则必然满足,故一定能找到)。
\end{enumerate}

对于阻尼牛顿法,步长 $\alpha_k$ 需满足:
\[
f(x_k + \alpha_k \Delta_{nt}) \leq f(x_k) + c \cdot \alpha_k \cdot \nabla f(x_k)^\top \Delta_{nt}
\]

\textbf{说明:}
\begin{itemize}
    \item 不等式右边是函数值的“预期下降量”(基于一阶近似),左边是实际下降量。
    \item 因牛顿步是下降方向(若 $H$ 正定,则 $\nabla f(x_k)^\top \Delta_{nt} < 0$),$c \cdot \alpha_k \cdot (\text{负数})$ 会使右边小于 $f(x_k)$,从而强制左边(实际函数值)必须下降。
    \item 若 $\alpha=1$ 满足 Armijo 准则,则直接使用(接近最优解时通常成立,保持二次收敛);否则按比例缩小 $\alpha$(如乘以0.5),直到满足条件。
\end{itemize}

\subsection{阻尼牛顿法的优势}
\begin{enumerate}
    \item \textbf{全局下降保证}:通过线搜索,无论 $H(x_k)$ 是否正定(即使牛顿步不是下降方向,线搜索会筛选出使函数下降的步长),都能确保 $f(x_{k+1}) < f(x_k)$,避免发散。
    \item \textbf{保留局部快速收敛}:当迭代点靠近最优解 $x^*$ 时,二次近似误差很小,$\alpha_k$ 会趋近于1,此时阻尼牛顿法退化为纯牛顿法,仍保持二次收敛速度。
    \item \textbf{适用性更广}:相比纯牛顿法,对初始点的要求更低(无需“足够靠近最优解”),在非凸问题或远离最优解的场景中更稳定。
\end{enumerate}

\subsection{阻尼牛顿法的步骤总结}
\begin{algorithm}[阻尼牛顿法]
\begin{enumerate}
    \item 初始化迭代点 $x_0$,设置精度阈值 $\epsilon > 0$;
    \item 计算梯度 $\nabla f(x_k)$,若 $\|\nabla f(x_k)\| < \epsilon$,停止迭代(已收敛);
    \item 计算 Hessian 矩阵 $H(x_k)$,求解牛顿步 $\Delta_{nt} = -H(x_k)^{-1}\nabla f(x_k)$;
    \item 通过 Armijo 准则线搜索确定步长 $\alpha_k$;
    \item 更新迭代点:$x_{k+1} = x_k + \alpha_k \Delta_{nt}$,返回步骤2。
\end{enumerate}
\end{algorithm}

简言之,阻尼牛顿法通过“方向用牛顿(高效),步长靠搜索(稳定)”的策略,完美弥补了纯牛顿法的缺陷,是实际中更常用的牛顿类优化方法。

\section{阻尼牛顿法性质}

\subsection{一些假设}

\begin{itemize}
    \item (A1) $ f \in C^2(\mathbb{R}^n) $,且梯度 \textbf{Lipschitz连续}:存在常数 $ L > 0 $ 使
    \[ \|\nabla f(x) - \nabla f(y)\| \leq L\|x - y\|, \quad \forall x, y. \]
    等价于 $ f $ 为 $ L $-光滑($ L $-smooth)。
    \item (A2) $ f $ \textbf{下有界}:$ \inf_x f(x) > -\infty $。
    \item (A3) 方向矩阵 \textbf{一致有界且正定}:存在常数 $ 0 < m \leq M < \infty $,对所有 $ k $ 有
    \[ mI \preceq H_k \preceq MI. \]
    (若直接用 Hessian,则这相当于全局强凸与曲率上界;在一般非凸情形,可通过\textbf{正定化}保证。)
    \item (A4) 步长 $ \alpha_k $ 由\textbf{回溯}满足 Armijo 条件(或更强的 Wolfe 条件)。
\end{itemize}

\subsection{全局收敛到临界点}

要证明阻尼牛顿法的\textbf{全局收敛性}($\|\nabla f(x_k)\| \to 0$),我们基于上述两个引理和假设(A1)-(A4),分步骤推导:

\begin{lemma}[牛顿方向的下降性与有界性]
设 $g_k = \nabla f(x_k)$,牛顿方向 $p_k = -H_k^{-1} g_k$(由 $H_k p_k = -g_k$ 定义)。

\begin{enumerate}
    \item \textbf{下降方向证明}:
    因 $H_k \succ 0$,其逆矩阵 $H_k^{-1} \succ 0$,故
    \[
    -g_k^\top p_k = g_k^\top H_k^{-1} g_k \geq \frac{1}{\|H_k\|} \|g_k\|^2 \geq \frac{1}{M} \|g_k\|^2 > 0
    \]
    这说明 $p_k$ 是\textbf{下降方向}(梯度与方向的内积为正,即方向与负梯度同向)。

    \item \textbf{方向有界性与夹角估计}:
    由 $H_k \succeq mI$,得 $\|H_k^{-1}\| \leq \frac{1}{m}$,因此
    \[
    \|p_k\| = \|H_k^{-1} g_k\| \leq \|H_k^{-1}\| \|g_k\| \leq \frac{1}{m} \|g_k\|
    \]
    方向与负梯度的夹角余弦为:
    \[
    \cos\theta_k = \frac{-g_k^\top p_k}{\|g_k\| \|p_k\|} \geq \frac{1/M}{1/m} = \frac{m}{M} > 0
    \]
    即方向与负梯度夹角远离 $90^\circ$,是“有效下降方向”。
\end{enumerate}
\end{lemma}

\begin{lemma}[回溯线搜索的步长下界]
设(A1)-(A3)成立,定义
\[
\alpha_0 := \frac{2(1 - c_1) m^2}{L M}
\]
则回溯法接受的步长 $\alpha_k$ 满足
\[
\alpha_k \geq \underline{\alpha} := \min\{1, \beta \alpha_0\} > 0
\]
\end{lemma}

\begin{proof}
由 $f$ 是 $L$-光滑(A1),其泰勒展开满足上界:
\[
f(x_k + \alpha p_k) \leq f(x_k) + \alpha g_k^\top p_k + \frac{L}{2} \alpha^2 \|p_k\|^2
\]
Armijo 条件要求:
\[
f(x_k + \alpha p_k) \leq f(x_k) + c_1 \alpha g_k^\top p_k
\]
结合上式,只需:
\[
(1 - c_1) \alpha g_k^\top p_k + \frac{L}{2} \alpha^2 \|p_k\|^2 \leq 0
\]
令 $\delta_k = -g_k^\top p_k > 0$(因 $p_k$ 是下降方向),上式等价于:
\[
-(1 - c_1) \alpha \delta_k + \frac{L}{2} \alpha^2 \|p_k\|^2 \leq 0
\]
当 $\alpha \leq \frac{2(1 - c_1) \delta_k}{L \|p_k\|^2}$ 时成立。

由引理1,$\delta_k \geq \frac{1}{M} \|g_k\|^2$ 且 $\|p_k\| \leq \frac{1}{m} \|g_k\|$,代入得:
\[
\frac{2(1 - c_1) \delta_k}{L \|p_k\|^2} \geq \frac{2(1 - c_1)(1/M)\|g_k\|^2}{L(1/m^2)\|g_k\|^2} = \frac{2(1 - c_1) m^2}{L M} = \alpha_0
\]
因此,当 $\alpha \leq \alpha_0$ 时 Armijo 条件必成立。

回溯法从 $\alpha=1$ 开始按因子 $\beta$($0 < \beta < 1$)递减,首次满足条件的步长不少于 $\beta \alpha_0$,故 $\alpha_k \geq \min\{1, \beta \alpha_0\} = \underline{\alpha} > 0$。
\end{proof}

\begin{theorem}[全局收敛性]
在假设(A1)-(A4)下,阻尼牛顿法生成的序列满足 $\|\nabla f(x_k)\| \to 0$,即全局收敛。
\end{theorem}

\begin{proof}
\begin{enumerate}
    \item \textbf{函数值序列的单调性与有界性}:
    由 Armijo 条件(A4),$f(x_{k+1}) < f(x_k)$,即 $\{f(x_k)\}$ 单调递减。又 $f$ 下有界(A2),故 $\{f(x_k)\}$ 收敛,即
    \[
    \lim_{k\to\infty} \left( f(x_k) - f(x_{k+1}) \right) = 0
    \]

    \item \textbf{结合步长下界与下降量的关系}:
    由引理1,$-g_k^\top p_k \geq \frac{1}{M} \|g_k\|^2$;由引理2,$\alpha_k \geq \underline{\alpha} > 0$。结合 Armijo 条件的下降量:
    \[
    f(x_k) - f(x_{k+1}) \geq c_1 \alpha_k (-g_k^\top p_k) \geq c_1 \underline{\alpha} \cdot \frac{1}{M} \|g_k\|^2
    \]

    \item \textbf{级数收敛推导出梯度范数收敛}:
    由于 $\sum_{k=0}^\infty \left( f(x_k) - f(x_{k+1}) \right)$ 收敛,其通项必须趋于0,即
    \[
    \lim_{k\to\infty} \|g_k\|^2 = \lim_{k\to\infty} \|\nabla f(x_k)\|^2 = 0
    \]
    因此
    \[
    \lim_{k\to\infty} \|\nabla f(x_k)\| = 0
    \]
\end{enumerate}
\end{proof}

\subsection{局部收敛阶段的退化}

这部分内容是在\textbf{分析阻尼牛顿法在“局部收敛阶段”的行为},核心是证明:当迭代点足够靠近最优解时,阻尼牛顿法的步长会恢复为 $\alpha_k = 1$(即“单位步”,退化为纯牛顿法),从而\textbf{恢复局部二次收敛的快速速率}。

\begin{lemma}
目的是给出一个\textbf{充分条件},使得步长 $\alpha_k = 1$ 满足 Armijo 条件(即无需缩小步长,直接用纯牛顿步迭代)。

\begin{itemize}
    \item \textbf{条件}:若牛顿步 $p_k = -H_k^{-1}g_k$($g_k = \nabla f(x_k)$)满足
    \[ \|p_k\| \leq \frac{3(1 - 2c_1)}{L_H} \lambda_{\min}(H_k), \quad c_1 \in (0, \frac{1}{2}) \]
    则 $\alpha_k = 1$ 满足 Armijo 条件。
    \item \textbf{意义}:当该条件成立时,阻尼牛顿法的步长不再需要“回溯缩小”,直接用纯牛顿步迭代,从而继承纯牛顿法的\textbf{局部二次收敛速率}。
\end{itemize}
\end{lemma}

通过\textbf{带三阶余项的泰勒展开}分析函数值变化,验证 $\alpha=1$ 时 Armijo 条件成立:
\begin{itemize}
    \item 对 $f(x_k + p_k)$ 做三阶泰勒展开(余项由 Hessian 的 Lipschitz 常数 $L_H$ 控制):
    \[ f(x_k + p_k) \leq f(x_k) + g_k^\top p_k + \frac{1}{2} p_k^\top H_k p_k + \frac{L_H}{6} \|p_k\|^3 \]
    \item 代入牛顿步的定义 $g_k = -H_k p_k$,化简得:
    \[ f(x_k + p_k) \leq f(x_k) - \frac{1}{2} p_k^\top H_k p_k + \frac{L_H}{6} \|p_k\|^3 \]
    \item 对比 Armijo 条件的要求($f(x_k + p_k) \leq f(x_k) - c_1 p_k^\top H_k p_k$),推导得:当
    \[ \left( \frac{1}{2} - c_1 \right) \lambda_{\min}(H_k) \|p_k\|^2 \geq \frac{L_H}{6} \|p_k\|^3 \]
    时,Armijo 条件成立。整理后即得到引理中的条件 $\|p_k\| \leq \frac{3(1 - 2c_1)}{L_H} \lambda_{\min}(H_k)$。
\end{itemize}

当迭代点足够靠近最优解 $x^*$ 时:
\begin{itemize}
    \item 牛顿步的长度 $\|p_k\| = O(\|e_k\|)$($e_k = x_k - x^*$ 是迭代误差),且 Hessian 的最小特征值 $\lambda_{\min}(H_k) \to \lambda_{\min}(H^*)$(因 Hessian 在邻域 Lipschitz 连续)。
    \item 此时引理的条件会被满足,回溯线搜索将以 $\alpha_k = 1$ 终止,迭代退化为\textbf{纯牛顿步}:$x_{k+1} = x_k + p_k$。
    \item 进而恢复纯牛顿法的\textbf{局部二次收敛速率},误差满足:
    \[ \|e_{k+1}\| \leq C \|e_k\|^2, \quad C = \frac{L_H}{2m} \]
\end{itemize}

这部分内容的核心是\textbf{桥接阻尼牛顿法的“全局收敛稳定性”与纯牛顿法的“局部二次收敛快速性”}:通过证明“局部邻域内步长恢复为1”,说明阻尼牛顿法在全局收敛后,会快速进入二次收敛阶段,从而兼具“全局稳定下降”和“局部快速收敛”的优势。

\section{非显式求逆方法}

\subsection{Cholesky 分解(Cholesky Decomposition)}

\textbf{定义}:
若矩阵 $ A \in \mathbb{R}^{n \times n} $ 是\textbf{对称正定矩阵(symmetric positive definite)},则存在唯一的下三角矩阵 $ L $ 满足:
\[
A = L L^{\mathrm{T}}
\]
其中 $ L $ 的对角元素全为正。

\textbf{含义}:
\begin{itemize}
    \item 把正定矩阵看作“平方”出来的结果。
    \item 等价于高斯消元的稳定版本。
    \item 数值优化中,常用于求解 $ A\mathbf{x} = \mathbf{b} $ 时避免直接求逆:
    \[
    L L^{\mathrm{T}} \mathbf{x} = \mathbf{b} \Rightarrow L \mathbf{y} = \mathbf{b},\ L^{\mathrm{T}} \mathbf{x} = \mathbf{y}
    \]
\end{itemize}

\textbf{计算方式}:
对每个元素递推:
\[
L_{ii} = \sqrt{A_{ii} - \sum_{k=1}^{i-1} L_{ik}^2}, \quad
L_{ij} = \frac{1}{L_{jj}}\Big(A_{ij} - \sum_{k=1}^{j-1} L_{ik}L_{jk}\Big)\ (i>j)
\]

\textbf{条件}:
必须是正定矩阵,否则根号项出现负数,分解失败。

\subsection{LDLᵀ 分解(LDL Decomposition)}

\textbf{定义}:
若矩阵 $ A \in \mathbb{R}^{n \times n} $ 是\textbf{对称矩阵}(不要求正定),则可以分解为:
\[
A = L D L^{\mathrm{T}}
\]
其中:
\begin{itemize}
    \item $ L $ 是单位下三角矩阵(对角元为 1),
    \item $ D $ 是对角矩阵(可以含正或负元素)。
\end{itemize}

\textbf{含义}:
\begin{itemize}
    \item 是对称矩阵的广义 Cholesky 分解。
    \item 适用于\textbf{半正定或不定矩阵},不会出现在 Cholesky 分解中那样的平方根问题。
    \item 若 $ A $ 正定,则 $ D $ 全为正,退化为标准 Cholesky:
    \[
    A = (L \sqrt{D})(L \sqrt{D})^{\mathrm{T}}
    \]
\end{itemize}

\textbf{算法形式}:
递推式:
\[
D_{jj} = A_{jj} - \sum_{k=1}^{j-1} L_{jk}^2 D_{kk}, \quad
L_{ij} = \frac{1}{D_{jj}}\Big(A_{ij} - \sum_{k=1}^{j-1} L_{ik} D_{kk} L_{jk}\Big)
\]

\textbf{优点}:
\begin{itemize}
    \item 不需要取平方根,数值更稳定。
    \item 可处理不定矩阵。
\end{itemize}
