\chapter{随机梯度下降(Stochastic Gradient Descent, SGD)}

\section{随机梯度下降基础}

当面对大规模数据集(数据量记为$N$,单个数据为$x_{i}$,$i=1, \dots, N$),需要优化目标函数 $\min _{x} \sum_{i=1}^{N} f_{i}(x)$ 时,若无法一次性获取所有数据 $x_{i}$ 或对应函数 $f_{i}$,则可通过随机梯度下降(SGD)实现优化。

\subsection{核心思路:用“部分数据”估算梯度}
由于无法计算全部数据的完整梯度 $\nabla f$,SGD通过\textbf{随机选取部分数据(称为“小批量”,记为 $\mathcal{B}^{(k)}$,其数据量记为 $|\mathcal{B}^{(k)}|$)},用这部分数据的梯度平均值近似整体梯度,即:
\begin{equation}
\nabla f \approx \frac{1}{|\mathcal{B}^{(k)}|} \sum_{i \in \mathcal{B}^{(k)}} \nabla \ell_{i}\left(x^{(k)}\right)
\end{equation}
其中 $\nabla \ell_{i}\left(x^{(k)}\right)$ 是单个数据 $i$ 在当前参数 $x^{(k)}$ 下的梯度,近似得到的整体梯度记为 $g^{(k)}$,即 $g^{(k)}=\frac{1}{|\mathcal{B}^{(k)}|} \sum_{i \in \mathcal{B}^{(k)}} \nabla \ell_{i}\left(x^{(k)}\right)$。

\subsection{完整更新流程}
SGD的优化流程是在传统梯度下降(GD)基础上,修改“梯度计算方式”,具体步骤如下:
\begin{enumerate}
    \item \textbf{初始值设定}:从初始参数 $x^{(0)}$ 开始,迭代次数 $k=0$;
    \item \textbf{确定下降方向}:基于随机选取的小批量数据 $\mathcal{B}^{(k)}$,计算近似梯度 $g^{(k)}$,下降方向为 $\Delta x^{(k)}=-g^{(k)}$(负梯度方向,保证函数值下降);
    \item \textbf{选择步长(学习率)}:步长 $\alpha^{(k)}$ 可设为常数,也可随迭代次数动态调整(如后期逐步减小,避免参数震荡);
    \item \textbf{参数更新}:按以下公式更新参数,使新参数对应的函数值更小(即 $f(x^{(k+1)}) < f(x^{(k)})$),之后迭代次数 $k=k+1$,重复步骤2-4:
    \begin{equation}
    x^{(k+1)}=x^{(k)}-\alpha^{(k)} g^{(k)}
    \end{equation}
\end{enumerate}

\subsection{关键超参数}
SGD的效果依赖两个核心超参数的设置,需根据数据和任务调整:
\begin{itemize}
    \item \textbf{批量大小(Batch Size)}:即小批量数据 $\mathcal{B}^{(k)}$ 包含的数据量 $|\mathcal{B}^{(k)}|$。批量越大,梯度估算越精准(噪声越小),但计算速度越慢;批量越小,计算越快,但梯度噪声越大,参数易震荡。
    \item \textbf{学习率(Learning Rate)}:即步长 $\alpha^{(k)}$。学习率过大可能导致参数“越过”最优解,函数值不下降反而上升;学习率过小则参数更新缓慢,需更多迭代次数才能收敛。
\end{itemize}

\section{一个随机估计问题}

先来看一个随机估计问题。

\subsection{有限样本下的均值计算}
\begin{itemize}
    \item 当我们采样得到 $n$ 个样本时,均值可表示为:
    \begin{equation}
    f_{n}(x)=\frac{1}{n} \sum_{i \in[1, n]} f\left(x_{i}\right)
    \end{equation}
    \item 当继续采样到第 $n+1$ 个样本时,新的均值为:
    \begin{equation}
    f_{n+1}(x)=\frac{1}{n+1} \sum_{i \in[1, n+1]} f\left(x_{i}\right)
    \end{equation}
\end{itemize}

\subsection{前后均值的递推关系}
通过数学变形,可建立 $f_{n}(x)$ 与 $f_{n+1}(x)$ 的关联,避免重复计算所有样本:
\begin{equation}
\begin{aligned}
f_{n+1}(x) & =\frac{1}{n+1}\left(f\left(x_{n+1}\right)+\sum_{i \in[1, n]} f\left(x_{i}\right)\right) \\
& =\frac{1}{n+1}\left(f\left(x_{n+1}\right)+n f_{n}(x)\right) \\
& =\left(1-\frac{1}{n+1}\right) f_{n}(x)+\frac{1}{n+1} f\left(x_{n+1}\right)
\end{aligned}
\end{equation}
若令步长 $\alpha=\frac{1}{n+1}$,则递推式可简化为更通用的形式:
\begin{equation}
f_{n+1}(x)=f_{n}(x)+\alpha\left(f\left(x_{n+1}\right)-f_{n}(x)\right)
\end{equation}
这意味着新均值=旧均值+步长$\times$(新样本值-旧均值),无需存储所有历史样本,仅需保留旧均值即可更新。

\subsection{均值收敛的条件}
要保证当样本数量 $n \to \infty$ 时,均值 $f_{n}(x)$ 能稳定收敛到真实期望,需满足 Robbins-Monro(1951)提出的步长条件:
\begin{equation}
\sum_{n=1}^{\infty} \alpha_{n}=\infty, \quad \sum_{n=1}^{\infty} \alpha_{n}^{2}<\infty
\end{equation}
\begin{itemize}
    \item 第一个条件 $\sum_{n=1}^{\infty} \alpha_{n}=\infty$:保证步长累积足够大,均值能持续向真实期望靠近,避免“半途停滞”;
    \item 第二个条件 $\sum_{n=1}^{\infty} \alpha_{n}^{2}<\infty$:保证步长衰减足够快,避免后期新样本对均值的干扰过大,导致结果震荡。
\end{itemize}

\section{Robbins-Monro(RM)算法}

首先回顾 Robbins-Monro(RM)算法的基础,它是推导 SGD 的起点。

\subsection{RM算法的目标}
RM算法用于求解\textbf{黑箱函数的根},即找到 $w^*$ 满足:
\begin{equation}
g(w^*) = 0
\end{equation}
其中 $g: \mathbb{R}^d \to \mathbb{R}^d$ 是未知函数(黑箱),仅能通过带噪声的观测获取信息:
\begin{equation}
\tilde{g}(w, \eta) = g(w) + \eta
\end{equation}
$\eta$ 是观测噪声,满足 $\mathbb{E}[\eta \mid H_k] = 0$($H_k = \{w_k, w_{k-1}, \dots\}$ 为历史信息),且方差有界 $\mathbb{E}[\eta^2 \mid H_k] < \infty$。

\subsection{RM算法的迭代公式}
为求解 $g(w) = 0$,RM算法的迭代更新规则为:
\begin{equation}
w_{k+1} = w_k - a_k \cdot \tilde{g}(w_k, \eta_k)
\end{equation}
其中 $a_k > 0$ 是步长序列,$w_k$ 是第 $k$ 次迭代的估计值。

\subsection{RM算法的收敛条件}
要保证 $w_k \to w^*$(几乎必然收敛),需满足3个核心条件:
\begin{enumerate}
    \item \textbf{函数单调性}:$g(w)$ 单调递增,且梯度有界 $0 < c_1 \leq \nabla_w g(w) \leq c_2$(确保根唯一);
    \item \textbf{步长条件}:$\sum_{k=1}^\infty a_k = \infty$(步长不收敛太快,保证能逼近根)且 $\sum_{k=1}^\infty a_k^2 < \infty$(步长趋于0,避免震荡);
    \item \textbf{噪声条件}:$\mathbb{E}[\eta_k \mid H_k] = 0$ 且 $\mathbb{E}[\eta_k^2 \mid H_k] < \infty$(噪声无偏且方差有界)。
\end{enumerate}

\section{SGD之问:为何能够收敛?}

在此之前,我们先来看一个引理。

\begin{lemma}[Robbins–Siegmund 超鞅收敛引理]
给定非负可测序列 $(X_k)$,满足条件:
\begin{equation}
\mathbb{E}[X_{k+1}\mid \mathcal F_k] \le (1 - a_k) X_k + b_k,
\end{equation}
其中:
\begin{itemize}
    \item $0 \le a_k \le 1$,控制“衰减比例”;
    \item $b_k \ge 0$,表示小的扰动或噪声;
    \item $\sum a_k = \infty$,保证长期衰减足够;
    \item $\sum b_k < \infty$,保证扰动总量有限。
\end{itemize}

\textbf{结论:}
\begin{enumerate}
    \item $(X_k)$ 几乎处处收敛;
    \item $\sum a_k X_k < \infty$ 几乎处处成立。
\end{enumerate}

\textbf{直观理解:}
\begin{itemize}
    \item $(X_k)$ 类似“衰减量 + 小扰动”的随机过程;
    \item $(a_k)$ 保证每步都有“收敛拉力”,而 $(b_k)$ 干扰有限;
    \item 因此 $(X_k)$ 不会发散,最终收敛,并且累计衰减量 $(\sum a_k X_k)$ 有限。
\end{itemize}
\end{lemma}

Robbins–Siegmund 引理提供了\textbf{在随机衰减 + 有限扰动下的序列收敛保证},是随机优化与在线算法理论分析的核心工具。

\subsection{设定与记号}
\begin{itemize}
    \item \textbf{数据与参数}:数据(或小批量数据)为 $x^{(k)}$,模型参数为 $\theta \in \mathbb{R}^{d}$($d$ 为参数维度)。
    \item \textbf{目标函数}:目标函数定义为期望损失,即 $f(\theta) \triangleq \mathbb{E}_{x}[L(x, \theta)]$,其中 $L(x, \theta)$ 是单个数据(或小批量数据)的损失函数。
    \item \textbf{SGD更新公式}:参数更新遵循 $\pmb{\theta}^{(k+1)}=\pmb{\theta}^{(k)}-\eta_{k} g^{(k)}$,其中 $\eta_k$ 是第 $k$ 步的学习率,$g^{(k)} \equiv \nabla_{\theta} L\left(x^{(k)}, \theta^{(k)}\right)$ 是第 $k$ 步的随机梯度(基于小批量数据计算)。
    \item \textbf{噪声分解}:将随机梯度拆分为“真实梯度”与“噪声”两部分,即 $g^{(k)}=\nabla f(\theta^{(k)})+\xi^{(k)}$。其中 $\nabla f(\theta^{(k)})$ 是目标函数在 $\theta^{(k)}$ 处的真实梯度,$\xi^{(k)}$ 是随机噪声,且满足条件 $\mathbb{E}[\xi^{(k)} | \mathcal{F}_{k}]=0$($\mathcal{F}_{k}$ 表示到第 $k$ 步的所有观测信息集合,即“自然滤子”)。
\end{itemize}
这一分解恰好契合 \textbf{Robbins–Monro框架}:该框架旨在寻找方程 $h(\theta)=0$ 的根(即目标函数极小值点,此时 $\nabla f(\theta^*)=0$),但仅能获得带噪声的观测 $H(\theta, x)$(对应此处的随机梯度 $g^{(k)}$),且观测的期望等于真实函数(即 $\mathbb{E}[g^{(k)} | \mathcal{F}_k] = \nabla f(\theta^{(k)})$)。令 $h(\theta)=\nabla f(\theta)$,即可将 SGD 纳入该框架分析收敛性。

\subsection{收敛性证明的核心假设}
要证明 SGD 收敛,需满足以下5个关键假设(记 $\theta^*$ 为目标函数极小值点,即 $\nabla f(\theta^*)=0$):
\begin{itemize}
    \item \textbf{(A1) 无偏噪声}:随机梯度的条件期望等于真实梯度,即 $\mathbb{E}[g^{(k)} | \mathcal{F}_{k}]=\nabla f(\theta^{(k)})$。
    \item \textbf{(A2) 有界二阶矩}:噪声的条件二阶矩有上限,即 $\mathbb{E}[\left\|\xi^{(k)}\right\|^{2} | \mathcal{F}_{k}] \leq\sigma^{2}+c\left\|\nabla f(\theta^{(k)})\right\|^{2}$。其中 $\sigma^2$ 是常数,$c$ 是系数,该假设限制了噪声的“强度”,避免噪声过大导致参数震荡不收敛。常用特例为“常数方差”,即 $\mathbb{E}[\left\|\xi^{(k)}\right\|^{2} | \mathcal{F}_{k}] \leq\sigma^{2}$。
    \item \textbf{(A3) L平滑}:目标函数的梯度满足 Lipschitz 连续条件,即 $\|\nabla f(\theta)-\nabla f(\phi)\| \leq L\|\theta-\phi\|$($L$ 为 Lipschitz 常数)。
    \item \textbf{(A4) $\mu$-强凸}:目标函数是 $\mu$-强凸的,即 $(\nabla f(\theta)-\nabla f(\phi))^{\top}(\theta-\phi) \geq \mu\|\theta-\phi\|^{2}$($\mu>0$ 为强凸系数)。强凸性保证目标函数有唯一极小值点 $\theta^*$,且参数会“持续向极小值点靠近”,不会在多个局部极小值间徘徊。
    \item \textbf{(A5) Robbins–Monro步长条件}:学习率序列 $\{\eta_k\}$ 需满足两个条件:
    \begin{enumerate}
        \item $\sum_{k=1}^{\infty} \eta_{k}=\infty$(学习率累积和为无穷大):保证参数有足够的“推进力”,能持续向极小值点靠近,避免因步长过小而“半途停滞”;
        \item $\sum_{k=1}^{\infty} \eta_{k}^{2}<\infty$(学习率平方的累积和有限):保证后期步长足够小,避免参数在极小值点附近“来回震荡”。
    \end{enumerate}
    典型的满足该条件的学习率形式为 $\eta_{k}=\frac{\alpha}{k+\beta}$($\alpha>0$,$\beta \geq 0$)。
\end{itemize}

\subsection{收敛性结论}

\subsubsection{结论一:几乎处处收敛(基于Robbins–Siegmund引理)}
\begin{theorem}[a.s. 收敛]
在假设(A1)–(A5)成立的前提下,SGD生成的参数序列满足:
\begin{equation}
\theta^{(k)} \underset{k \to \infty}{\stackrel{a.s.}{\to}} \theta^{*}, \quad \sum_{k=1}^{\infty} \eta_{k}\left\|\nabla f\left(\theta^{(k)}\right)\right\|^{2}<\infty \quad a.s.
\end{equation}
其中“a.s.”表示“几乎必然”(即除了概率为0的特殊情况外,参数序列一定收敛到 $\theta^*$)。
\end{theorem}

\textbf{证明核心思路(关键不等式与引理应用):}
\begin{enumerate}
    \item \textbf{定义距离变量}:令 $\Delta^{(k)} \triangleq \theta^{(k)}-\theta^*$(即当前参数与极小值点的距离向量),需证明 $\|\Delta^{(k)}\| \to 0$(距离趋近于0)。
    \item \textbf{展开距离平方的递推关系}:根据 SGD 更新公式,展开 $\|\Delta^{(k+1)}\|^2$(第 $k+1$ 步的距离平方):
    \begin{equation}
    \begin{aligned}
    \left\|\Delta^{(k+1)}\right\|^{2} &= \left\|\theta^{(k+1)}-\theta^*\right\|^{2} \\
    &= \left\|\theta^{(k)} - \eta_k g^{(k)} - \theta^*\right\|^{2} \\
    &= \left\|\Delta^{(k)} - \eta_k g^{(k)}\right\|^{2} \\
    &= \left\|\Delta^{(k)}\right\|^{2} - 2\eta_k \Delta^{(k)\top} g^{(k)} + \eta_k^2 \left\|g^{(k)}\right\|^{2}
    \end{aligned}
    \end{equation}
    \item \textbf{取条件期望并代入假设}:对等式两侧关于 $\mathcal{F}_k$ 取条件期望,结合(A1)(无偏噪声)和(A2)(有界二阶矩),将 $g^{(k)}=\nabla f(\theta^{(k)})+\xi^{(k)}$ 代入,可化简得到:
    \begin{equation}
    \begin{aligned}
    \mathbb{E}\left[\left\|\Delta^{(k+1)}\right\|^{2} | \mathcal{F}_k\right] \leq \mathbb{E}\left[\left\|\Delta^{(k)}\right\|^{2} | \mathcal{F}_k\right] - 2\eta_k \Delta^{(k)\top} \nabla f(\theta^{(k)}) \\
    + \eta_k^2 \left( \left\|\nabla f(\theta^{(k)})\right\|^2 + \sigma^2 + c\left\|\nabla f(\theta^{(k)})\right\|^2 \right)
    \end{aligned}
    \end{equation}
    \item \textbf{利用强凸与L平滑简化}:由(A4)($\mu$-强凸)可得 $\Delta^{(k)\top} \nabla f(\theta^{(k)}) \geq \mu \left\|\Delta^{(k)}\right\|^2$;由(A3)(L平滑)可得 $\left\|\nabla f(\theta^{(k)})\right\| \leq L \left\|\Delta^{(k)}\right\|$(因 $\nabla f(\theta^*)=0$)。代入上式后,可整理得到:
    \begin{equation}
    \mathbb{E}\left[\left\|\Delta^{(k+1)}\right\|^{2} | \mathcal{F}_k\right] \leq \left(1 - 2\mu \eta_k + C \eta_k^2\right) \left\|\Delta^{(k)}\right\|^2 + \sigma^2 \eta_k^2
    \end{equation}
    其中 $C \triangleq (1+c)L^2$(常数)。当 $k$ 足够大时,$\eta_k$ 足够小,可满足 $1 - 2\mu \eta_k + C \eta_k^2 \leq 1 - \mu \eta_k$。
    \item \textbf{应用Robbins–Siegmund引理}:令 $X_k = \left\|\Delta^{(k)}\right\|^2$(待分析的非负序列),$a_k = \mu \eta_k$,$b_k = \sigma^2 \eta_k^2$,则上述不等式可化为引理要求的形式:
    \begin{equation}
    \mathbb{E}\left[X_{k+1} | \mathcal{F}_k\right] \leq (1 - a_k)X_k + b_k
    \end{equation}
    结合(A5),$\sum a_k = \mu \sum \eta_k = \infty$,$\sum b_k = \sigma^2 \sum \eta_k^2 < \infty$,满足引理条件。根据引理可得出:$X_k$ 几乎必然收敛,且 $\sum a_k X_k < \infty$。再结合 $\sum a_k = \infty$,可推出 $\liminf_{k \to \infty} X_k = 0$;又因目标函数强凸((A4)),最终可得 $X_k \to 0$(即 $\theta^{(k)} \to \theta^*$)几乎必然成立。
\end{enumerate}

\subsubsection{结论二:强凸下的收敛速率与Polyak–Ruppert平均}
\begin{theorem}[期望二次误差 $O(1/k)$]
假设(A1)–(A4)成立,若取学习率 $\eta_k = \frac{\alpha}{k+\beta}$(其中 $\alpha > \frac{1}{\mu}$,$\beta \geq 1$),则存在常数 $K$,使得:
\begin{equation}
\mathbb{E}\left[\left\|\theta^{(k)} - \theta^*\right\|^2\right] \leq \frac{K}{k+\beta}
\end{equation}
即参数与极小值点的“期望平方距离”随迭代次数 $k$ 增长,以 $O(1/k)$ 的速率衰减。
\end{theorem}

\begin{proof}
将结论一证明中的“距离平方的条件期望递推式”取全期望,令 $u_k = \mathbb{E}\left[\left\|\Delta^{(k)}\right\|^2\right]$(期望平方距离),代入 $\eta_k = \frac{\alpha}{k+\beta}$ 后可得到:
\begin{equation}
u_{k+1} \leq \left(1 - \frac{2\mu \alpha}{k+\beta} + \frac{C \alpha^2}{(k+\beta)^2}\right) u_k + \frac{\sigma^2 \alpha^2}{(k+\beta)^2}
\end{equation}
通过定义辅助变量 $v_k = (k+\beta) u_k$,利用“差分比较法”可证明 $u_k = O(1/k)$,进而得到上述期望误差界。
\end{proof}

\textbf{随着迭代次数 $k$ 增加,参数与最优解的 “平均距离平方” 会以 $1/k$ 的速度变小}。

\begin{theorem}[Polyak–Ruppert迭代平均的最优渐近方差]
定义参数的迭代平均为:
\begin{equation}
\overline{\theta}^{(T)} \triangleq \frac{1}{T} \sum_{k=1}^{T} \theta^{(k)}
\end{equation}
在假设(A1)–(A4)与(A5)(步长 $\eta_k = \frac{\alpha}{k+\beta}$)成立的前提下,有:
\begin{equation}
\sqrt{T}\left(\overline{\theta}^{(T)} - \theta^*\right) \Rightarrow \mathcal{N}\left(0, A^{-1} S A^{-\top}\right)
\end{equation}
其中 $A \triangleq \nabla^2 f(\theta^*)$(目标函数在极小值点处的 Hessian 矩阵),$S$ 是噪声协方差的极限值。
\end{theorem}

该结论表明:通过对参数序列做“迭代平均”,可使 SGD 达到随机逼近(SA)框架下的“最优渐近效率”——即平均后的参数估计量,其渐近方差是最小的,在实践中能显著减小噪声导致的参数波动,提升收敛稳定性。

普通 SGD 是“每步更新一个参数,最后用最后一步的参数”;而 Polyak–Ruppert 方法是“先迭代 $T$ 步,得到 $T$ 个参数 $\theta^{(1)},\theta^{(2)},\dots,\theta^{(T)}$,再求它们的平均值 $\overline{\theta}^{(T)} = \frac{1}{T}\sum_{k=1}^T \theta^{(k)}$”。

用“迭代平均”后的参数 $\overline{\theta}^{(T)}$,\textbf{随着迭代次数 $T$ 增加,它与最优解的差距会服从“正态分布”,且这个差距的“波动范围(方差)是最小的”}(即“最优渐近方差”)。

SGD 的收敛性本质上源于 \textbf{Robbins–Monro 随机逼近框架}与 \textbf{Robbins–Siegmund 超鞅引理}的支撑:
\begin{enumerate}
    \item 强凸目标函数+满足 Robbins–Monro 条件的学习率(如 $\eta_k \propto 1/k$),可保证参数“几乎处处收敛”到极小值点,且期望二次误差以 $O(1/k)$ 速率衰减;
    \item 对参数做 Polyak–Ruppert 迭代平均,能进一步优化渐近方差,提升收敛精度与稳定性。
\end{enumerate}

\subsection{非强凸场景下的SGD收敛性(仅凸/一般非凸)}

在之前的分析中,我们默认目标函数满足“$\mu$-强凸”条件(假设(A4)),但实际场景中很多目标函数不具备强凸性(如仅凸函数、非凸函数),因此需要单独分析这类场景下 SGD 的收敛表现。

\subsubsection{1. 仅凸场景(无强凸性,仅满足凸性)}
\textbf{核心设定}:此时目标函数可能存在“平坦区域”或“多个最优解(构成凸集)”,无法保证参数收敛到唯一极小值点,但可保证“函数值收敛到最优值”。

\textbf{收敛性结论}:若调整学习率为 $\eta_k = \frac{1}{k^\alpha}$(其中 $\alpha \in (1/2, 1]$,满足 Robbins–Monro 条件 $\sum \eta_k = \infty$ 且 $\sum \eta_k^2 < \infty$),则对参数的迭代平均值 $\overline{\theta}^{(T)} = \frac{1}{T}\sum_{k=1}^T \theta^{(k)}$,有:
\begin{equation}
f(\overline{\theta}^{(T)}) - f(\theta^*) = o(1)
\end{equation}
即随着迭代次数 $T$ 增大,平均参数对应的函数值会“逐步逼近最优函数值 $f(\theta^*)$”,最终趋近于 0。

若进一步量化收敛速率,通常为 $O\left(\frac{1}{T^{1-\alpha}}\right)$ 量级(如 $\alpha=0.8$ 时,速率为 $O\left(\frac{1}{T^{0.2}}\right)$)——相比强凸场景下的 $O\left(\frac{1}{T}\right)$,仅凸场景的收敛更慢,这是因为缺少强凸性带来的“强制向最优解靠近”的约束。

\subsubsection{2. 一般非凸场景(无凸性,仅满足L-平滑)}
\textbf{核心设定}:“一般非凸”指目标函数既不满足强凸性,也不满足凸性,仅满足 L-平滑条件(假设(A3):梯度变化平缓,$\|\nabla f(\theta)-\nabla f(\phi)\| \leq L\|\theta-\phi\|$)。
这类场景在深度学习中最常见(如神经网络的损失函数),目标函数可能存在大量局部极小值、鞍点,无法保证参数收敛到全局最优解,只能退而求其次——保证参数收敛到“一阶驻点”(即梯度趋近于 0 的点,$\nabla f(\theta) \approx 0$,此时参数再更新也难以显著降低函数值)。

\textbf{收敛性结论}:在无偏噪声(假设(A1))和噪声方差有界(假设(A2))的前提下,若采用“分段常数步长”或“$\eta_k = \frac{1}{\sqrt{k}}$ 步长”(注:$\frac{1}{\sqrt{k}}$ 不满足 Robbins–Monro 的 $\sum \eta_k^2 < \infty$,因此不具备“几乎处处收敛”性质,仅能保证“梯度的期望有界”),则有:
\begin{equation}
\min_{1 \leq k \leq T} \mathbb{E}\left[\left\|\nabla f(\theta^{(k)})\right\|^2\right] = \mathcal{O}\left(\frac{1}{\sqrt{T}}\right)
\end{equation}
该结论的含义是:在 $T$ 次迭代中,\textbf{至少存在某一步的参数 $\theta^{(k)}$,其梯度的期望平方值不超过 $\frac{C}{\sqrt{T}}$($C$ 为常数)},且随着 $T$ 增大,这个“最小梯度期望”会以 $\frac{1}{\sqrt{T}}$ 的速率减小,逐步趋近于 0。

需要特别注意:
\begin{enumerate}
    \item 该速率是“到一阶驻点”的速率,而非“到全局最优解”的速率——最终参数可能停在局部极小值或鞍点,但这些点的梯度已足够小,函数值难以继续下降;
    \item 与强凸/仅凸场景不同,非凸场景的收敛结论不涉及“参数是否收敛”或“函数值是否收敛到最优”,仅保证“梯度足够小”,这是因为非凸函数的全局最优解难以通过 SGD 的随机搜索触及,“找到驻点”已是实际能达到的目标。
\end{enumerate}

\subsubsection{3. 非强凸场景与强凸场景的核心差异}
为了更清晰理解不同场景的收敛特性,可通过下表对比:

\begin{table}[htbp]
\centering
\small
\begin{tabular}{@{}llllll@{}}
\toprule
\textbf{场景} & \textbf{目标函数性质} & \textbf{收敛目标} & \textbf{学习率要求} & \textbf{收敛速率(期望)} & \textbf{关键限制} \\ \midrule
强凸 & 强凸+L-平滑 & 全局最优解 $\theta^*$ & $\eta_k \propto \frac{1}{k}$ (满足RM) & $O(1/T)$ & 需强凸性,适用场景有限 \\
仅凸 & 凸+L-平滑 & 最优函数值 $f(\theta^*)$ & $\eta_k = \frac{1}{k^\alpha}$ ($\alpha \in (0.5,1]$) & $O(1/T^{1-\alpha})$ & 收敛慢,无唯一最优参数 \\
一般非凸 & L-平滑(无凸性) & 一阶驻点 ($\nabla f \approx 0$) & 分段常数/$\eta_k \propto \frac{1}{\sqrt{k}}$ & $O(1/\sqrt{T})$ (梯度期望) & 仅能到驻点,可能是局部最优 \\ \bottomrule
\end{tabular}
\caption{不同场景下的收敛特性对比}
\end{table}

\subsection{两种不同目标下的步长设计及收敛策略差异}

\subsubsection{1. OGD/Regret(在线学习/长期平均性能)}
\begin{itemize}
    \item \textbf{目标}:保证长期平均损失接近最优,即 \textbf{后悔(regret)界} 小。
    \item \textbf{对应公式常见形式}:
    \begin{equation}
    \text{Regret}(T) = \sum_{t=1}^T f_t(x_t) - \min_x \sum_{t=1}^T f_t(x) \le O(\sqrt{T}) \text{ 或 } O(\log T)
    \end{equation}
    \item \textbf{步长选择}:通常用 \textbf{非递减或者 $1/\sqrt{t}$ 形式},保证平均损失下降快。
\end{itemize}

\subsubsection{2. RM/SA(Robbins–Monro / Stochastic Approximation)}
\begin{itemize}
    \item \textbf{目标}:保证\textbf{参数序列几乎处处收敛到最优点}(a.s. convergence),属于点估计/统计意义。
    \item \textbf{收敛条件}:
    \begin{equation}
    \sum_{k=1}^{\infty} a_k = \infty,\quad \sum_{k=1}^{\infty} a_k^2 < \infty
    \end{equation}
    \item \textbf{常用步长}:$a_k = 1/k$ 或 $1/k^\gamma$ ($0.5<\gamma\le1$)。
    \item \textbf{意义}:每步衰减足够慢以保证探索,但衰减快以抑制噪声,满足 Robbins–Siegmund 引理条件。
\end{itemize}

\begin{itemize}
    \item 如果使用 \textbf{OGD/Regret 的步长策略} 来保证 \textbf{几乎处处收敛},可能违反 RM/SA 的平方可积条件($\sum < \infty$),因此不能保证 a.s. 收敛。
    \item 反之,如果严格使用 RM/SA 的条件($\sum < \infty$)来优化在线 regret,可能收敛太慢,导致平均损失下降慢。
\end{itemize}
\textbf{关键点}:两种方法的目标不同,不能直接互换步长策略。

\subsubsection{3. 折中 / 统一策略}
\textbf{选用 $a_k = 1/k^\gamma$ ($0.5< \gamma < 1$)}
\begin{itemize}
    \item 满足 RM/SA 条件:$\sum a_k = \infty$ 且 $\sum a_k^2 < \infty$,保证 a.s. 收敛。
    \item 同时保持较慢衰减,平均性能也不错(在线学习效果可接受)。
\end{itemize}

\textbf{阶段常数步长 + Polyak–Ruppert 平均 + Doubling Trick}
\begin{itemize}
    \item \textbf{阶段常数步长}:将迭代分阶段,每阶段使用\textbf{近似最优常数步长},提升该阶段的平均损失性能(降低 regret)。
    \item \textbf{Doubling Trick}:阶段长度每次加倍(doubling trick),保证整体步长衰减满足 RM/SA 条件,确保 a.s. 收敛。
    \item \textbf{Polyak–Ruppert 平均}:阶段内取参数平均,进一步稳定收敛。
\end{itemize}

\begin{remark}
\textbf{OGD/Regret 关注长期平均损失;RM/SA 关注参数几乎处处收敛。两者步长策略冲突,但可以通过衰减指数、阶段常数步长和 Polyak–Ruppert 平均实现折中,兼顾在线性能和几乎处处收敛。}
\end{remark}

\section{从随机估计到动力学}

从动力学视角分析 SGD,核心是将“离散的参数更新过程”与“连续的物理运动方程”建立关联——通过极限近似,把梯度下降(GD)对应到确定性的常微分方程(ODE),把随机梯度下降(SGD)对应到含噪声的随机微分方程(SDE),从而用物理运动规律解释 SGD 的收敛行为、噪声影响及参数调整逻辑。

\subsection{从GD到ODE:离散更新是梯度流的“显式欧拉积分”}
梯度下降(GD)的参数更新是离散步骤,而通过“时间标度转换”和“连续极限”,可将其转化为描述“确定性下降运动”的常微分方程(ODE),即“梯度流”。

\subsubsection{1.1 离散更新与时间标度定义}
GD 的离散更新公式为:
\begin{equation}
\theta^{(k+1)} = \theta^{(k)} - \eta \nabla L(\theta^{(k)})
\end{equation}
其中:
\begin{itemize}
    \item $\theta^{(k)}$:第 $k$ 步的参数;
    \item $\eta$:步长(学习率);
    \item $\nabla L(\theta^{(k)})$:目标函数 $L$ 在 $\theta^{(k)}$ 处的梯度(确定性,无噪声)。
\end{itemize}
为建立连续关联,定义“连续时间” $t_k = k \cdot \eta$——即把每一步更新的“步长 $\eta$”视为“时间增量”,迭代次数 $k$ 越多,对应的连续时间 $t_k$ 越大。

\subsubsection{1.2 连续极限:从离散更新到梯度流ODE}
当步长 $\eta \to 0$(时间增量无限小)、且 $t_k \to t$(连续时间趋近于某个值)时,对 GD 的离散更新公式做“差分近似”:
左边参数增量除以时间增量,近似为连续时间下的参数变化率(导数):
\begin{equation}
\frac{\theta^{(k+1)} - \theta^{(k)}}{\eta} \Rightarrow \dot{\theta}(t)
\end{equation}
右边代入 GD 的更新规则,可得连续时间下的“梯度流方程”(ODE):
\begin{equation}
\dot{\theta}(t) = -\nabla L(\theta(t))
\end{equation}

\textbf{物理意义}:GD 的离散更新,本质是对“梯度流 ODE”的“显式欧拉数值积分”——每一步按当前梯度方向“迈一小步”,步长越小,离散的参数轨迹越贴近 ODE 描述的“连续下降路径”(类似下山时“小步慢走”更贴近顺滑的山坡轨迹)。

\subsubsection{1.3 数值稳定性与曲率的关系}
GD 的收敛稳定性(是否会“震荡不收敛”),与目标函数的“曲率”直接相关,可通过二次函数案例直观理解:
\begin{itemize}
    \item 若目标函数为二次形式 $L(\theta) = \frac{1}{2}\theta^\top H \theta$($H \succeq 0$ 为 Hessian 矩阵,代表函数曲率),则 GD 的更新公式可改写为:
    \begin{equation}
    \theta^{(k+1)} = (I - \eta H) \theta^{(k)}
    \end{equation}
    其中 $I$ 为单位矩阵。
    \item 收敛条件:该线性迭代收敛的充要条件是“矩阵 $I - \eta H$ 的谱半径 $\rho(I - \eta H) < 1$”,等价于步长需满足:
    \begin{equation}
    0 < \eta < \frac{2}{\lambda_{\text{max}}(H)}
    \end{equation}
    ($\lambda_{\text{max}}(H)$ 是 Hessian 矩阵的最大特征值,代表函数的“最大曲率”)。
    \item 一般 L-平滑场景:若目标函数的梯度满足 L-Lipschitz 连续(L 为平滑常数,可理解为“梯度变化的最大速率”),则取 $0 < \eta < \frac{2}{L}$ 可保证每步更新后函数值下降;若同时满足强凸性,取 $0 < \eta \leq \frac{1}{L}$ 还能获得“线性收敛速率”(参数快速靠近最优解)。
\end{itemize}

\textbf{核心启发(A)}:可将学习率 $\eta$ 视为“时间步长”——函数曲率越大($\lambda_{\text{max}}(H)$ 或 $L$ 越大),“显式欧拉积分”的稳定范围越窄,GD 需要更小的学习率才能避免震荡;实际中“分段调整学习率”“周期衰减学习率”,本质是通过“细化时间网格”提升数值稳定性,让参数更新更贴合梯度流的顺滑路径。

\subsection{从SGD到SDE:扩散极限与朗之万动力学}
SGD 的核心是“用随机小批量梯度近似真实梯度”,存在噪声干扰。通过类似的连续极限,可将其转化为含噪声的随机微分方程(SDE),即“朗之万动力学”,从而用“扩散运动”解释 SGD 的噪声探索与收敛平衡。

\subsubsection{2.1 噪声分解:随机梯度的构成}
SGD 的小批量梯度包含“真实梯度”和“噪声”两部分,分解公式为:
\begin{equation}
\nabla L_{\mathcal{B}}(\theta^{(k)}) = \nabla L(\theta^{(k)}) + \xi^{(k)}
\end{equation}
其中:
\begin{itemize}
    \item $\nabla L_{\mathcal{B}}(\theta^{(k)})$:基于小批量 $\mathcal{B}$ 计算的随机梯度;
    \item $\nabla L(\theta^{(k)})$:目标函数的真实梯度(确定性部分);
    \item $\xi^{(k)}$:小批量采样引入的噪声,满足 $\mathbb{E}[\xi^{(k)} | \theta^{(k)}] = 0$(无偏噪声),其协方差 $\text{Cov}[\xi^{(k)}] \approx \Sigma(\theta^{(k)})$(随参数变化的噪声强度)。
\end{itemize}
基于此,SGD 的离散更新公式可改写为:
\begin{equation}
\theta^{(k+1)} = \theta^{(k)} - \eta \left( \nabla L(\theta^{(k)}) + \xi^{(k)} \right)
\end{equation}

\subsubsection{2.2 扩散极限:从离散SGD到SDE(欧拉–丸山连续化)}
同样定义连续时间 $t_k = k \cdot \eta$,当步长 $\eta \to 0$(时间增量无限小)、且小批量噪声近似高斯分布时,可将 SGD 的离散更新转化为“随机微分方程(SDE)”:
\begin{equation}
d\theta_t = -\nabla L(\theta_t) dt + G(\theta_t) dW_t
\end{equation}
其中:
\begin{itemize}
    \item $\theta_t$:连续时间 $t$ 下的参数;
    \item $dt$:连续时间增量;
    \item $dW_t$:多维布朗运动(Wiener 过程),代表连续时间下的随机噪声(均值为 0,方差为 $dt$);
    \item $G(\theta_t)$:噪声强度矩阵,满足 $G(\theta_t) G(\theta_t)^\top \approx \eta \cdot \Sigma(\theta_t)$(将离散噪声的协方差与连续时间的噪声强度关联)。
\end{itemize}

\textbf{规范朗之万形式}:
若令噪声强度为“各向同性常数”(即不同参数方向的噪声强度相同),设 $G = \sqrt{2T}$($T$ 为“温度”参数,控制噪声整体强度),则 SDE 可简化为标准的“朗之万动力学方程”:
\begin{equation}
d\theta_t = -\nabla L(\theta_t) dt + \sqrt{2T} dW_t
\end{equation}
其核心性质是:若存在平稳分布(参数长期运动的稳定概率分布),则该分布与目标函数 $L$ 的 Gibbs 权重成正比,即 $\propto \exp\left(-\frac{L}{T}\right)$——“温度” $T$ 越高,噪声越强,参数探索范围越广(更易跳出局部极小值);$T$ 越低,噪声越弱,参数越容易收敛到目标函数的低价值区域(极小值附近)。

\subsubsection{2.3 两类极限:消噪极限与扩散极限}
SGD 的连续极限存在两种典型场景,对应不同的训练阶段目标:
\begin{itemize}
    \item \textbf{消噪极限}:若步长 $\eta \to 0$,同时小批量大小 $B$ 增大(使噪声协方差 $\Sigma \to 0$),则 SDE 中的扩散项(噪声部分)$G(\theta_t) dW_t$ 会逐渐消失,SDE 退化为 GD 对应的“梯度流 ODE”——这对应训练后期“增大批量、减小学习率”的策略,目的是“消除噪声,精准收敛到最优解”。
    \item \textbf{扩散极限}:若按比例调整步长 $\eta$ 和批量大小 $B$(如保持 $\frac{\eta}{B}$ 为常数),使“有效噪声强度”($\eta \cdot \Sigma$)保持不变,则 SDE 的扩散项非平凡(噪声持续存在)——这对应训练前期“小批量、稍大学习率”的策略,目的是“保留噪声,通过随机探索找到更优的参数区域”。
\end{itemize}

\subsubsection{2.4 Fokker–Planck视角:参数分布的演化}
SGD 的参数在连续时间下的概率密度 $p_t(\theta)$(即参数在时刻 $t$ 处于某个值的概率),满足“Fokker–Planck 方程”:
\begin{equation}
\partial_t p_t = \nabla \cdot \left( p_t \nabla L \right) + \frac{1}{2} \sum_{i,j} \partial_i \partial_j \left( [D(\theta)]_{ij} p_t \right)
\end{equation}
其中 $D(\theta) = G(\theta) G(\theta)^\top$ 是扩散系数矩阵(代表噪声在不同参数方向的强度)。

该方程的意义是:参数密度的变化由两部分驱动——
\begin{enumerate}
    \item 确定性漂移项($\nabla \cdot (p_t \nabla L)$):由目标函数梯度主导,使参数密度向 $L$ 的低价值区域聚集(类似水流向低处);
    \item 随机扩散项(二阶导数项):由噪声主导,使参数密度向周围扩散(类似墨水在水中扩散)。
\end{enumerate}
若 $D(\theta)$ 为常数且各向同性(噪声在所有参数方向强度相同),则平稳密度为 Gibbs 分布;若 $D(\theta)$ 随参数变化或各向异性(不同方向噪声强度不同),则平稳密度会偏离简单的 Gibbs 分布——这解释了实际 SGD 中“噪声具有方向性”的现象:某些参数方向的噪声更强,参数在这些方向的探索更活跃,最终收敛位置也会偏向噪声影响更小的“平坦区域”(与“平坦极小值泛化更好”的经验观察一致)。

\subsection{局部二次近似与OU过程:常步长下的方差-曲率权衡}
在目标函数的极小值点 $\theta^*$ 附近,可将函数近似为二次形式(局部二次近似),此时 SGD 的连续极限(SDE)可简化为“Ornstein–Uhlenbeck(OU)过程”——通过分析 OU 过程的平稳分布,能清晰理解“参数曲率”与“噪声方差”的平衡关系。

\subsubsection{3.1 局部二次近似}
在 $\theta^*$ 附近,对目标函数 $L(\theta)$ 做泰勒展开并忽略高阶项,得到二次近似:
\begin{equation}
L(\theta) \approx L(\theta^*) + \frac{1}{2} (\theta - \theta^*)^\top H (\theta - \theta^*)
\end{equation}
其中 $H = \nabla^2 L(\theta^*)$ 是目标函数在 $\theta^*$ 处的 Hessian 矩阵($H \succ 0$,因 $\theta^*$ 是极小值点),代表函数在极小值附近的“局部曲率”——$H$ 的特征值越大,对应参数方向的曲率越大(函数在该方向越“陡峭”)。

\subsubsection{3.2 OU过程与平稳协方差}
令 $\vartheta_t = \theta_t - \theta^*$(参数与极小值点的偏差),代入朗之万 SDE,结合局部二次近似($\nabla L(\theta_t) \approx H \vartheta_t$),可得偏差 $\vartheta_t$ 满足的 OU 过程:
\begin{equation}
d\vartheta_t = -H \vartheta_t dt + \sqrt{2T} dW_t
\end{equation}
OU 过程是“带阻尼的线性随机过程”,其核心性质是存在\textbf{平稳分布}(当时间 $t \to \infty$ 时,$\vartheta_t$ 的分布不再变化):
\begin{itemize}
    \item 平稳分布为高斯分布 $\mathcal{N}(0, P)$,其中 $P$ 是协方差矩阵,满足“Lyapunov 方程”:
    \begin{equation}
    H P + P H = 2T I
    \end{equation}
    ($I$ 为单位矩阵,$T$ 为温度参数)。
    \item 若噪声为各向异性常数扩散($D = G G^\top$,非单位矩阵),则 Lyapunov 方程推广为:
    \begin{equation}
    H P + P H = D
    \end{equation}
\end{itemize}

\subsubsection{3.3 核心启发(B):“宽谷偏好”的物理解释}
从 Lyapunov 方程可直接推导“曲率”与“平稳方差”的关系:对 Hessian 矩阵 $H$ 的某个特征值 $\lambda_i$(对应第 $i$ 个参数方向的曲率),其对应的平稳方差 $P_{ii}$(参数在该方向的波动范围)满足:
\begin{equation}
P_{ii} = \frac{T}{\lambda_i}
\end{equation}
这意味着:在相同噪声强度(温度 $T$)下,\textbf{曲率越小($\lambda_i$ 越小)的参数方向,平稳方差越大}——即目标函数的“宽谷区域”(曲率小)对参数的“吸引概率”更高,参数更易在宽谷中稳定下来。

这一结论完美解释了深度学习中的经验观察:“更平坦的极小值泛化性能更好”——因为 SGD 的噪声会使参数自然偏向宽谷区域,而宽谷区域的参数对数据扰动更不敏感,泛化能力更强。

\subsection{学习率、批量与“温度”的定量关系}
通过动力学分析,可建立 SGD 中“学习率($\eta$)”“批量大小($B$)”与“温度($T$,噪声强度)”的明确关联,为超参数调整提供理论依据。

\subsubsection{4.1 小批量梯度的方差尺度}
设数据集总大小为 $N$,小批量大小为 $B$,在“样本独立同分布(IID)”的近似下,小批量梯度的协方差满足:
\begin{equation}
\text{Cov}\left[ \nabla L_{\mathcal{B}}(\theta) \right] \approx \left( \frac{1}{B} - \frac{1}{N} \right) C(\theta)
\end{equation}
其中 $C(\theta)$ 是单个样本梯度的协方差(与参数 $\theta$ 相关,代表数据本身的梯度波动)。
当 $B \ll N$(小批量远小于总数据量)时,$\frac{1}{N}$ 可忽略,协方差近似为 $\frac{1}{B} C(\theta)$——即批量越大,随机梯度的噪声越小(方差与批量大小成反比)。

\subsubsection{4.2 有效温度与“噪声刻度”}
结合 SDE 的扩散系数定义($D \approx \eta \cdot \text{Cov}[\nabla L_{\mathcal{B}}(\theta)]$)和 OU 过程的平稳协方差($P \propto \frac{T}{\lambda}$),可推导出“有效温度” $T$ 与学习率 $\eta$、批量大小 $B$ 的关系:
\begin{equation}
T \propto \eta \cdot \left( \frac{1}{B} - \frac{1}{N} \right)
\end{equation}
(比例系数由问题本身的尺度决定,如 $C(\theta)$ 的大小)。

这一关系揭示了“保持温度不变(噪声强度不变)”的两种等效策略:
\begin{enumerate}
    \item 减小学习率 $\eta$(降温):若批量 $B$ 不变,减小 $\eta$ 会降低有效温度,使参数更稳定收敛;
    \item 增大批量 $B$(降温):若学习率 $\eta$ 不变,增大 $B$ 会减小 $\frac{1}{B}$,同样降低有效温度,且增大批量更利于并行计算(比减小学习率更高效)。
\end{enumerate}

\textbf{核心启发(C):等效退火策略}
训练后期需要“降低噪声,精准收敛”,可采用“逐步增大批量”的“等效退火”策略——相比传统的“学习率衰减”,增大批量能在不降低更新速度的前提下减小噪声,同时利用并行计算提升训练效率,是更优的超参数调整方案。

\subsection{训练策略:将动力学结论落地到实践}
基于上述动力学分析,可总结出5条切实可行的 SGD 训练策略,直接指导实际调参与优化:

\begin{enumerate}
    \item \textbf{两阶段训练日程}:
    \begin{itemize}
        \item \textbf{探索阶段(前期)}:采用“小批量+稍大学习率”——对应扩散极限,保持较高的有效温度(噪声强度),让参数通过随机探索跳出局部极小值,找到更优的参数区域;
        \item \textbf{精调阶段(后期)}:采用“逐步增大批量或衰减学习率”——对应消噪极限,降低有效温度,使参数在优质区域内精准收敛到极小值点。
    \end{itemize}
    \item \textbf{学习率的经验上界}:若目标函数满足 L-平滑条件,优先保证学习率 $\eta < \frac{2}{L}$(避免 GD 的显式欧拉积分不稳定);若在极小值附近(局部二次区域),可通过 Hessian 矩阵的最大特征值 $\lambda_{\text{max}}$ 估算学习率上界($\eta < \frac{2}{\lambda_{\text{max}}}$),进一步提升稳定性。
    \item \textbf{常步长+迭代平均}:常步长 SGD 在局部二次近似下易形成稳定的平稳分布(OU 过程的平稳态),但参数会因噪声存在波动;对参数序列做“迭代平均”(如 Polyak–Ruppert 平均),可显著降低噪声导致的抖动,同时保留平稳分布的“宽谷偏好”,提升收敛精度与泛化能力。
    \item \textbf{预条件缓解各向异性}:目标函数的各向异性(不同参数方向曲率差异大)会导致 SGD 在大曲率方向震荡、小曲率方向推进缓慢;通过“预条件”(如对不同参数方向设置不同的有效步长,或使用 Adam 等自适应优化器),可平衡不同方向的曲率与噪声强度,缓解“快慢维”问题,加快整体收敛速度。
    \item \textbf{早停并非悖论}:虽然朗之万动力学的平稳分布需要“长时间混合”(参数充分探索),但扩散近似表明:SGD 的“先探索后收束”是宏观规律——训练前期参数快速向优质区域移动,后期若继续训练,参数可能因噪声在平稳区域内波动,反而导致泛化性能下降。因此,结合验证集监控的“早停”策略,本质是在“探索充分”与“收敛稳定”之间找最优平衡点,并非与动力学规律矛盾。
\end{enumerate}

\subsection{动力学近似的失效场景}
需注意,上述 ODE/SDE 近似并非万能,在以下4种场景中会失效,需结合实际情况调整策略:
\begin{enumerate}
    \item \textbf{大步长或强非线性区域}:步长过大时,显式欧拉积分的误差增大,ODE/SDE 的连续近似失真;目标函数强非线性(如激活函数导致的非光滑区域)会破坏局部二次近似,OU 过程的假设不成立;
    \item \textbf{重尾或异方差噪声}:若小批量梯度的噪声不满足高斯分布(重尾分布),或噪声方差随参数剧烈变化(异方差),扩散近似的偏差会显著增大;
    \item \textbf{非IID/强自相关采样}:若小批量采样非独立同分布(如时序数据的连续采样),会破坏噪声的无偏性与方差稳定性,噪声项存在“记忆效应”,SDE 的布朗运动假设(无记忆性)失效;
    \item \textbf{强各向异性}:若目标函数的曲率与噪声强度在不同方向差异极大(强各向异性),单一温度参数无法刻画噪声的方向性,需更精细的“随机平均场(SME)”或“Fokker–Planck 方程”分析,或通过预条件技术针对性优化。
\end{enumerate}

\subsection{核心关系总结}
\begin{table}[htbp]
\centering
\small
\begin{tabular}{@{}llll@{}}
\toprule
\textbf{离散算法} & \textbf{连续动力学模型} & \textbf{核心方程} & \textbf{关键参数/概念} \\ \midrule
梯度下降(GD) & 梯度流(ODE) & $\dot{\theta}(t) = -\nabla L(\theta(t))$ & 学习率 $\eta$(时间步长), L-平滑常数 $L$ \\
随机梯度下降(SGD) & 朗之万动力学(SDE) & $d\theta_t = -\nabla L dt + \sqrt{2T}dW_t$ & 有效温度 $T$(噪声强度), 批量 $B$, 扩散系数 $D$ \\
极小值附近SGD & OU过程 & $d\vartheta_t = -H\vartheta_t dt + \sqrt{2T}dW_t$ & Hessian $H$(曲率), 平稳协方差 $P$, Lyapunov方程 \\ \bottomrule
\end{tabular}
\caption{离散算法与连续动力学模型的对应关系}
\end{table}

\begin{remark}[一些简单的总结]
\textbf{1. GD $\to$ ODE}
离散更新:$\theta_{k+1} = \theta_k - \eta \nabla L(\theta_k)$ 是连续方程 $\dot{\theta}(t) = -\nabla L(\theta(t))$ 的显式欧拉近似。学习率 $\eta$ 对应时间步长。$\eta$ 过大会导致数值不稳定。稳定区间与 Hessian 最大特征值 $\lambda_{\max}$ 相关:$0<\eta<2/\lambda_{\max}(H)$。

\textbf{2. SGD $\to$ SDE}
随机梯度 $\nabla L_B = \nabla L + \xi$ 在 $\eta \to 0$ 时可连续化为 $d\theta_t = -\nabla L(\theta_t) dt + G(\theta_t) dW_t$,即朗之万动力学。噪声强度矩阵 $G G^\top \approx \eta \Sigma$。

\textbf{3. 两类极限}
\begin{itemize}
    \item \textbf{消噪极限($\eta \to 0$, $B$ 大)}:SDE $\to$ ODE,收敛。
    \item \textbf{扩散极限($\eta, B$ 按比例缩放)}:保持噪声强度,探索。
\end{itemize}

\textbf{4. Fokker–Planck 方程}
描述参数分布演化,平衡确定性漂移与随机扩散。噪声方向性解释了“平坦极小值泛化更好”。

\textbf{5. 局部二次近似 $\to$ OU 过程}
在极小值附近:$d\vartheta_t = -H\vartheta_t dt + \sqrt{2T} dW_t$。平稳协方差 $P$ 满足 $HP + PH = 2T I$。各方向方差 $P_{ii} = T/\lambda_i$,曲率越小波动越大 $\to$ SGD 自然偏向宽谷。

\textbf{6. 温度刻度}
小批量方差 $\propto 1/B$。有效温度 $T \propto \eta \left( \frac{1}{B} - \frac{1}{N} \right)$。增大 $B$ 或减小 $\eta$ 均可“降温”,即退火。

\textbf{7. 实践策略}
\begin{itemize}
    \item \textbf{前期}:小批量 + 大学习率(高温扩散探索)。
    \item \textbf{后期}:增大批量或衰减学习率(降温收敛)。
    \item \textbf{常步长 + 迭代平均} 降低噪声。
    \item \textbf{预条件处理} 各向异性。
    \item \textbf{早停} 平衡探索与收敛。
\end{itemize}

\textbf{8. 失效条件}
步长过大、非高斯噪声、非 IID 采样、强各向异性。此时需改用随机平均场或 Fokker–Planck 分析。

\textbf{核心洞见}:\textbf{SGD 是朗之万扩散在能量地形 $L(\theta)$ 上的近似积分}。学习率控制时间步,批量控制温度。广义目标是利用噪声探索宽谷、再逐步降温精调。
\end{remark}

\section{SGD之问:为什么需要动量?}

\subsection{从SGD出发:我们到底缺什么?}
SGD 的核心更新公式为:
\begin{equation}
\theta^{(k+1)}=\theta^{(k)}-\eta g^{(k)}, \quad g^{(k)} \equiv \nabla_{\theta} L\left(x^{(k)}, \theta^{(k)}\right)
\end{equation}
其中 $\theta^{(k)}$ 是第 $k$ 步参数,$\eta$ 是学习率,$g^{(k)}$ 是基于小批量数据计算的随机梯度。

在实际训练中,纯 SGD 会暴露三个典型“痛点”,这正是动量(如 Heavy-Ball、NAG)要解决的核心问题:
\begin{enumerate}
    \item \textbf{痛点1:收敛慢}
    在 L-平滑凸目标函数上,纯 SGD 即使用最优常数步长,函数值收敛速率也只能达到 $O(1/k)$;若目标函数是强凸的,收敛速率还会受“条件数 $\kappa=L/\mu$”($L$ 为平滑常数,$\mu$ 为强凸系数)控制——条件数越大(如高维模型),收敛越慢,甚至出现“硬问题”(迭代数千步仍无明显下降)。
    \item \textbf{痛点2:“峡谷之字形”震荡}
    当目标函数存在“各向异性曲率”(Hessian 矩阵特征值跨度大,即“峡谷地形”)时,纯 SGD 会在峡谷两侧来回震荡:大曲率方向(峡谷壁)的梯度大,迫使步长被“钳制”得很小;而小曲率方向(峡谷底)的梯度小,小步长导致推进缓慢,整体呈现“之字形”路径,严重浪费迭代次数。
    \item \textbf{痛点3:噪声底难以突破}
    小批量数据带来的梯度噪声,会使纯 SGD 在训练后期陷入“噪声主导”状态——参数围绕极小值点反复波动,无法继续降低函数值,形成“噪声底”,难以收敛到更优解。
\end{enumerate}

\subsection{Heavy-Ball(HB):用“惯性”优化SGD的核心痛点}
Heavy-Ball 是最经典的动量方法,核心是给 SGD 加入“惯性记忆”,通过累积历史更新方向,实现“抑制震荡、加快收敛、抵抗噪声”的效果。

\subsubsection{两种等价实现形式}
\begin{itemize}
    \item \textbf{位移形式(经典 Polyak 公式)}:直接通过历史参数差引入惯性
    \begin{equation}
    \theta^{(k+1)}=\theta^{(k)}-\eta g^{(k)}+\beta\left(\theta^{(k)}-\theta^{(k-1)}\right)
    \end{equation}
    其中 $\beta \in [0,1)$ 是动量系数,$\theta^{(k)}-\theta^{(k-1)}$ 是上一步的参数更新量(历史方向),$\beta$ 越大,惯性越强。
    \item \textbf{速度-EMA形式(深度学习常用)}:通过“指数移动平均(EMA)”维护一个“速度”变量,间接引入惯性
    \begin{equation}
    v^{(k+1)}=\beta v^{(k)}+(1-\beta) g^{(k)}, \quad \theta^{(k+1)}=\theta^{(k)}-\eta v^{(k+1)}
    \end{equation}
    其中 $v^{(k)}$ 是速度变量(可理解为“加权平均后的梯度”),$(1-\beta)$ 是当前梯度的权重,$\beta$ 是历史速度的权重——本质是对梯度做平滑,降低噪声影响。
\end{itemize}

\subsubsection{Heavy-Ball为什么有效?(基于一维/特征方向的直觉)}
以“二次目标函数”(最易理解的凸函数)为例,设 $L(\theta)=\frac{1}{2}\lambda(\theta-\theta^*)^2$($\theta^*$ 是最优解,$\lambda$ 是曲率),此时 HB 的误差($e^{(k)}=\theta^{(k)}-\theta^*$)满足二阶递推关系:
\begin{equation}
e^{(k+1)}=(1-\eta \lambda+\beta) e^{(k)}-\beta e^{(k-1)}
\end{equation}
通过选择合适的 $(\eta, \beta)$(如 $\beta=\frac{\sqrt{\kappa}-1}{\sqrt{\kappa}+1}$,$\kappa=L/\mu$ 为条件数),可实现三大优化:
\begin{itemize}
    \item \textbf{加速收敛}:将强凸二次函数上的收敛复杂度从纯 SGD 的 $O(\kappa \log \frac{1}{\varepsilon})$($\varepsilon$ 为精度要求)降低到 $O(\sqrt{\kappa} \log \frac{1}{\varepsilon})$——条件数越大,加速效果越明显;
    \item \textbf{抑制“之字形”震荡}:动量对“高频反向梯度”(如峡谷壁的来回震荡方向)提供阻尼(历史方向与当前方向相反时,惯性会抵消部分更新),对“低频一致梯度”(如峡谷底的前进方向)做累积放大,使参数沿谷底顺滑推进;
    \item \textbf{抵抗噪声}:EMA 形式的速度变量会将梯度噪声的方差按系数 $\frac{1-\beta}{1+\beta}$ 压低——例如 $\beta=0.9$ 时,噪声方差仅为纯 SGD 的约 5\%,轨迹更平滑,后期更易突破噪声底。
\end{itemize}

\subsubsection{局限性}
HB 在“二次函数”或“局部强凸目标”上效果显著,但对“一般凸目标”(无强凸性)的全局加速速率,缺乏像 NAG 那样的普适理论保证——在非强凸场景下,HB 的加速效果可能不稳定。

\subsection{Nesterov(NAG):“前瞻-校正”实现更稳健的加速}
Nesterov 动量(简称 NAG)是对 HB 的改进,核心是加入“前瞻步骤”:先根据历史惯性“预判”下一步的参数位置,再用该位置的梯度做校正,避免 HB 可能出现的“过冲”问题,实现更稳健的全局加速。

\subsubsection{3.1 两种常见实现形式}
\begin{itemize}
    \item \textbf{原始 Nesterov 加速梯度(FISTA/FGM 形态)}:先计算前瞻点,再用前瞻点的梯度更新
    \begin{equation}
    y^{(k)}=\theta^{(k)}+\beta\left(\theta^{(k)}-\theta^{(k-1)}\right), \quad \theta^{(k+1)}=y^{(k)}-\eta \nabla L\left(y^{(k)}\right)
    \end{equation}
    其中 $y^{(k)}$ 是“前瞻点”(基于历史惯性预判的下一步参数),$\nabla L(y^{(k)})$ 是前瞻点的梯度——相比 HB 直接用当前点梯度,NAG 用前瞻点梯度能更精准地捕捉“下一步的真实坡度”。
    \item \textbf{深度学习 NAG(look-ahead 梯度形态)}:结合速度变量的前瞻更新
    \begin{equation}
    v^{(k+1)}=\beta v^{(k)}+g\left(\theta^{(k)}-\eta \beta v^{(k)}\right), \quad \theta^{(k+1)}=\theta^{(k)}-\eta v^{(k+1)}
    \end{equation}
    其中 $\theta^{(k)}-\eta \beta v^{(k)}$ 是前瞻点(用历史速度预判的位置),$g(\cdot)$ 是前瞻点的梯度——本质与原始形态一致,只是用速度变量简化了计算。
\end{itemize}

\subsubsection{Nesterov为什么更优?(核心是“前瞻-校正”)}
\begin{itemize}
    \item \textbf{凸问题的全局加速保证}:在 L-平滑凸目标函数上,NAG 能通过“前瞻-校正”实现 $O(1/k^2)$ 的函数值收敛速率(纯 SGD 是 $O(1/k)$);在强凸目标上,能达到与 HB 相同的线性收敛率(最坏收敛因子 $\propto 1-\frac{1}{\sqrt{\kappa}}$),但全局理论保证更普适。
    \item \textbf{减少“过冲”与误判}:在强各向异性的“峡谷地形”中,HB 的惯性可能导致参数“冲过”谷底(因用当前点梯度判断方向,未考虑下一步的坡度变化);而 NAG 的前瞻点梯度更贴近“下一步真正会到的位置”的曲率,能提前校正惯性方向,减少反复横跳,稳定性更强。
\end{itemize}

\subsection{HB与NAG的噪声鲁棒性对比}
在相同步幅(有效更新量)下,HB 和 NAG 都能抵抗梯度噪声,但机制与适用场景略有差异:
\begin{itemize}
    \item \textbf{HB 的优势}:EMA 形式的速度平滑对“低噪声、强曲率”场景更友好——例如小批量较大(噪声小)的强凸任务,HB 的惯性能快速累积前进方向,且实现简单、调参成本低;
    \item \textbf{NAG 的优势}:前瞻-校正对“高噪声、强非线性”场景更稳健——例如小批量较小(噪声大)的深度学习任务,NAG 能减少“陈旧梯度”(当前点梯度与下一步实际梯度的偏差)导致的误判,在弯曲剧烈的噪声场中轨迹更稳定。
\end{itemize}
两者的共同局限是:若训练后期不“降温”(减学习率或增批量),都会因噪声存在“噪声底”——因此需配合“迭代平均”或“等效退火”策略,进一步提升收敛精度。

\subsection{实践选择:何时用HB,何时用NAG?}
基于目标函数特性与工程需求,可按以下原则选择:

\begin{table}[htbp]
\centering
\small
\begin{tabular}{@{}lll@{}}
\toprule
\textbf{场景特征} & \textbf{优先选择} & \textbf{理由} \\ \midrule
有理论保证需求(凸/强凸任务) & NAG & 能提供 $O(1/k^2)$ 或线性收敛的全局理论保证,结果更可控 \\
明显“峡谷地形”、过冲严重 & NAG & 前瞻-校正能减少惯性过冲,抑制横跳 \\
调参简洁、兼容现有SGD流程 & HB & EMA形式易集成到现有代码,仅需新增一个动量系数 $\beta$,调参成本低 \\
大批量、低噪声、强凸任务 & HB & 噪声小,无需复杂的前瞻校正,HB的惯性加速更直接 \\
极大条件数、低噪声(确定性任务) & 两者均可 & 都能实现 $\sqrt{\kappa}$ 级加速,NAG理论保证更优,HB实现更简单 \\ \bottomrule
\end{tabular}
\caption{HB与NAG的选择指南}
\end{table}

\subsection{核心总结}
动量的本质是给 SGD 加入“历史方向记忆”,解决纯 SGD“慢、晃、抖”的痛点:
\begin{itemize}
    \item HB 通过“惯性累积”实现加速与震荡抑制,适合简单场景与低噪声任务;
    \item NAG 通过“前瞻-校正”实现更稳健的全局加速,适合复杂场景与高噪声任务;
    \item 无论选择哪种动量,训练后期都需配合“降温”(减学习率/增批量)或“迭代平均”,才能突破噪声底,实现精准收敛。
\end{itemize}
