% 凸优化3(章节已存在,无需重复)

\chapter{凸优化的基本性质}
\section{凸优化问题的形式}

\subsection{一般优化模型}

一个一般优化问题的数学表达式为:
\begin{equation}
\min _{x \in \mathbb{R}^{n}} f_{0}(x) \quad \text{s.t.} \quad f_{i}(x) \leq 0, \, i=1, \ldots, m ; \quad h_{j}(x)=0, \, j=1, \ldots, p
\end{equation}
其中各部分含义如下:

\begin{itemize}
\item \textbf{目标函数}:$f_{0}(x)$,即需要最小化的函数;
\item \textbf{约束函数}:
  \begin{itemize}
  \item 不等式约束:$f_{i}(x) \leq 0$(共$m$个,$i$取值为1到$m$);
  \item 等式约束:$h_{j}(x) = 0$(共$p$个,$j$取值为1到$p$);
  \end{itemize}
\item \textbf{可行域}:满足所有约束条件的变量集合,定义为 $X = \{x \mid f_{i}(x) \leq 0, \, h_{j}(x) = 0\}$。
\end{itemize}

\subsection{凸优化的定义}

\begin{definition}[凸优化问题]
当且仅当满足以下三个条件时,一般优化问题是凸优化问题:

\begin{itemize}
\item 目标函数 $f_{0}(x)$ 为\textbf{凸函数};
\item 不等式约束函数 $f_{i}(x)$($i=1,\ldots,m$)均为\textbf{凸函数};
\item 等式约束函数 $h_{j}(x)$($j=1,\ldots,p$)均为\textbf{仿射函数},即满足形式 $h_{j}(x) = a_{j}^{T} x + b_{j}$(其中$a_{j}$为向量,$b_{j}$为常数)。
\end{itemize}
\end{definition}

此时,凸优化问题可简化描述为:\textbf{minimize a convex function over a convex set}(在凸集上最小化凸函数)。


\section{凸函数的定义以及性质}

\subsection{定义与几何意义}

\subsubsection{数学定义}
\begin{definition}
函数 $f: \mathbb{R}^n \to \mathbb{R}$ 是\textbf{凸函数},当且仅当对任意 $x_1, x_2 \in \text{dom}(f)$($\text{dom}(f)$表示函数$f$的定义域)和任意 $\theta \in [0,1]$,满足以下不等式:
\begin{equation}
f\left(\theta x_1 + (1-\theta) x_2\right) \leq \theta f\left(x_1\right) + (1-\theta) f\left(x_2\right)
\end{equation}
\end{definition}

\subsubsection{几何意义}

凸函数的曲线始终位于连接两点 $(x_1, f(x_1))$和 $(x_2, f(x_2))$的割线之下。

\begin{quote}
\textit{直观理解:在函数定义域内任取两点,两点间的割线不会低于函数曲线本身。}
\end{quote}

\subsection{Jensen 不等式}
Jensen 不等式是凸函数最重要的性质之一,具体表述如下:
\begin{theorem}[Jensen 不等式]
若 $f$为凸函数,$\{x_i\}$为 $\text{dom}(f)$内的任意点集,$\{\theta_i\}$为非负权重且满足 $\sum_{i} \theta_i = 1$,则:
\begin{equation}
f\left(\sum_{i} \theta_i x_i\right) \leq \sum_{i} \theta_i f\left(x_i\right)
\end{equation}
\end{theorem}

\subsubsection{应用场景}

Jensen 不等式不仅是判定函数凸性的重要依据,还广泛应用于:

\begin{itemize}
\item 概率论(如期望相关不等式推导);
\item 信息论(如 KL 散度、熵函数的性质分析);
\item 期望下界的证明。
\end{itemize}

\subsubsection{典型示例}

若 $f(x) = x^2$(已知为凸函数),对随机变量 $X$应用 Jensen 不等式,可得:
\[
(\mathbb{E}[X])^2 \leq \mathbb{E}[X^2]
\]
(其中 $\mathbb{E}[X]$表示 $X$的期望,$\mathbb{E}[X^2]$表示 $X^2$的期望)。

\subsection{一阶条件(First-order condition)}

\paragraph{适用前提}

函数 $f$可微(即梯度 $\nabla f(x)$在定义域内存在)。

\paragraph{判定准则}
\begin{proposition}
函数 $f$是凸函数,当且仅当对任意 $x, y \in \text{dom}(f)$,满足:
\begin{equation}
f(y) \geq f(x) + \nabla f(x)^T (y - x)
\end{equation}
\end{proposition}

\paragraph{几何与直观意义}

\begin{itemize}
\item 数学层面:函数在任意点 $x$处的切平面(或切线,当 $n=1$时)是函数的\textbf{全局下界};
\item 梯度意义:梯度 $\nabla f(x)$始终指向函数的上升方向。
\end{itemize}

\subsection{二阶条件(Hessian 判定)}

\paragraph{适用前提}

函数 $f$二阶可微(即 Hessian 矩阵 $\nabla^2 f(x)$在定义域内存在)。

\paragraph{判定准则}
\begin{proposition}
函数 $f$是凸函数,当且仅当对任意 $x \in \text{dom}(f)$,其 Hessian 矩阵满足\textbf{半正定}:
\begin{equation}
f \text{ 是凸的} \quad \Leftrightarrow \quad \nabla^2 f(x) \succeq 0, \, \forall x
\end{equation}
\end{proposition}

\paragraph{强凸函数的延伸判定}
\begin{definition}
当且仅当对任意 $x \in \text{dom}(f)$,Hessian 矩阵满足\textbf{正定}($\nabla^2 f(x) \succ 0$)时,函数 $f$为\textbf{强凸函数}(强凸是凸函数的更强形式)。
\end{definition}

\paragraph{强凸性定义}

\begin{definition}[强凸函数(等价定义)]
函数 $f$称为强凸函数,若存在常数 $\mu>0$,对所有 $x, y$:
\begin{equation}
f(y) \geq f(x)+\nabla f(x)^{T}(y-x)+\frac{\mu}{2}\| y-x\| ^{2}
\end{equation}

其中 $\mu$称为强凸系数。
\end{definition}

这意味着:
函数不仅“向上弯”,而且弯曲程度有下界;
强凸函数具有唯一最优解。

\begin{itemize}
\item 性质1:强凸函数有且仅有一个极小点;
\item 性质2:梯度法在强凸问题上以线性速率收敛
\end{itemize}

\paragraph{证明}

\begin{proof}[证明:强凸函数梯度法线性收敛]

\subparagraph{证明前提}

\begin{definition}[Lipschitz连续]
梯度Lipschitz连续(工程中常用假设):存在$L>0$,对所有$x,y$有$\|\nabla f(x) - \nabla f(y)\| \leq L\|x - y\|$。
\end{definition}

\begin{itemize}
\item 强凸函数$f$满足\textbf{梯度Lipschitz连续};
\item 梯度法迭代公式:$x_{k+1} = x_k - \alpha \nabla f(x_k)$,其中步长$\alpha$取$\alpha = \frac{1}{L}$(最优步长选择);
\item $x^*$为强凸函数的唯一极小点,故$\nabla f(x^*) = 0$。
\end{itemize}

\subparagraph{证明过程(核心推导)}

1. \textbf{展开迭代误差范数}:  
对$x_{k+1} = x_k - \alpha \nabla f(x_k)$,两边减去$x^*$得:  
\[x_{k+1} - x^* = (x_k - x^*) - \alpha \nabla f(x_k)\]  
取平方范数(利用$\|a - b\|^2 = \|a\|^2 - 2a^T b + \|b\|^2$):  
\begin{equation}
\|x_{k+1} - x^*\|^2 = \|x_k - x^*\|^2 - 2\alpha \nabla f(x_k)^T(x_k - x^*) + \alpha^2\|\nabla f(x_k)\|^2 \tag{1}
\end{equation}

2. \textbf{利用强凸性放缩梯度项}:  
对$x=x_k, y=x^*$应用强凸定义,代入$\nabla f(x^*) = 0$:  
\[f(x^*) \geq f(x_k) + \nabla f(x_k)^T(x^* - x_k) + \frac{\mu}{2}\|x^* - x_k\|^2\]  
由于$f(x^*) \leq f(x_k)$($x^*$是最优解),整理得:  
\begin{equation}
\nabla f(x_k)^T(x_k - x^*) \geq \frac{\mu}{2}\|x_k - x^*\|^2 \tag{2}
\end{equation}

3. \textbf{利用Lipschitz连续放缩梯度范数}:  
对$x=x_k, y=x^*$应用梯度Lipschitz连续,代入$\nabla f(x^*) = 0$:  
\[
\|\nabla f(x_k)\| \leq L\|x_k - x^*\| \implies \|\nabla f(x_k)\|^2 \leq L^2\|x_k - x^*\|^2 \tag{3}
\]

4. \textbf{代入迭代公式证线性收敛}:  
将(2)(3)代入(1),并取$\alpha = \frac{1}{L}$:  
\begin{align}
\|x_{k+1} - x^*\|^2 &\leq \|x_k - x^*\|^2 - 2\alpha \cdot \frac{\mu}{2}\|x_k - x^*\|^2 + \alpha^2 L^2\|x_k - x^*\|^2 \nonumber \\
&= \|x_k - x^*\|^2 \left(1 - \alpha \mu + \alpha^2 L^2\right) \nonumber \\
&= \|x_k - x^*\|^2 \left(1 - \frac{\mu}{L} + \frac{L^2}{L^2}\right) \nonumber \\
&= \|x_k - x^*\|^2 \left(1 - \frac{\mu}{L}\right)
\end{align}  
其中$0 < 1 - \frac{\mu}{L} < 1$(因$\mu < L$,强凸系数小于Lipschitz常数)。

\subparagraph{结论}

梯度法迭代误差满足$\|x_{k+1} - x^*\|^2 \leq \gamma \|x_k - x^*\|^2$($\gamma = 1 - \frac{\mu}{L}$为收敛因子),即\textbf{以线性速率收敛}。
\end{proof}


\section{全局与局部最优}

\subsection{定义回顾}

对于一般优化问题,局部最优解与全局最优解的定义如下:

\begin{definition}[局部与全局最优解]
\begin{itemize}
\item \textbf{局部最优解}:存在$\epsilon>0$,使得对所有满足$\|x - x^{*}\| < \epsilon$的可行点$x$,均有$f(x) \geq f(x^{*})$(即$x^{*}$在自身邻域内是最优的)。
\item \textbf{全局最优解}:对所有可行域内的点$x$,均有$f(x) \geq f(x^{*})$(即$x^{*}$在整个可行域内是最优的)。
\end{itemize}
\end{definition}

\subsection{凸优化的核心定理}

\begin{theorem}[凸优化局部最优即全局最优]
若目标函数$f$是凸函数,且优化问题的可行域$\mathcal{X}$是凸集,则\textbf{任何局部最优解都是全局最优解}。
\end{theorem}

\begin{proof}[证明(反证法)]
\begin{enumerate}
\item \textbf{假设前提}:假设$x^{*}$是局部最优解,但非全局最优解。根据全局最优解的定义,此时存在可行点$x' \in \mathcal{X}$,使得$f(x') < f(x^{*})$。

\item \textbf{构造凸组合}:定义凸组合$x_{\theta} = (1-\theta)x^{*} + \theta x'$,其中$\theta \in (0,1)$(即$x_{\theta}$是$x^{*}$与$x'$连线上的点)。

\item  \textbf{利用凸集性质}:由于可行域$\mathcal{X}$是凸集,根据凸集的定义,$x_{\theta} \in \mathcal{X}$(即$x_{\theta}$是可行点)。

\item \textbf{利用凸函数性质}:由于$f$是凸函数,根据凸函数的定义:  
\[
f(x_{\theta}) = f\left((1-\theta)x^{*} + \theta x'\right) \leq (1-\theta)f(x^{*}) + \theta f(x')
\]

\item  \textbf{推出矛盾}:结合假设$f(x') < f(x^{*})$,代入上式得:  
\[
f(x_{\theta}) \leq (1-\theta)f(x^{*}) + \theta f(x') < (1-\theta)f(x^{*}) + \theta f(x^{*}) = f(x^{*})
\]  
即$f(x_{\theta}) < f(x^{*})$。又因为当$\theta$足够小时,$x_{\theta}$满足$\|x_{\theta} - x^{*}\| = \theta\|x' - x^{*}\| < \epsilon$(符合局部最优解的邻域条件),这与$x^{*}$是局部最优解的定义矛盾。
\end{enumerate}
\end{proof}

\subsection{强凸函数的唯一最优性}

若$f$为$\mu$-强凸函数($\mu>0$为强凸系数),则:

\begin{itemize}
\item \textbf{最优解唯一性}:强凸函数的最优解有且仅有一个(不存在多个最优解);
\item \textbf{函数值差的二次下界}:距离最优点的函数值差满足二次下界关系(具体表现为$f(x) - f(x^{*}) \geq \frac{\mu}{2}\|x - x^{*}\|^2$)。
\end{itemize}

该性质为算法收敛性分析(如梯度下降法、牛顿法等)提供了重要的理论基础。


\section{凸优化的几何意义}

\subsection{可行域与等高线}

\begin{definition}[等高线与凸优化几何特征]
凸优化的几何特征具有明确的直观性:\textbf{目标函数的等高线(level set)与凸形可行域(convex feasible region)相切的点,即为全局最优点}。
\end{definition}

几何补充:等高线是目标函数值等于某一常数的点的集合(如二次函数的椭圆等高线);由于可行域是凸集,其边界呈“凸向外侧”的形态,二者相切时仅存在唯一接触点,该点即为整个可行域内使目标函数最小的点。

\subsection{法向条件(支撑超平面)}

\begin{definition}[支撑超平面]
从几何角度分析,凸优化问题的最优点 $x^*$处必然存在一个\textbf{支撑超平面(supporting hyperplane)},其数学表达式为:
\begin{equation}
\mathbf{a}^T \left( x - x^* \right) \geq 0, \quad \forall x \in \mathcal{X}
\end{equation}
其中:

\begin{itemize}
\item $\mathbf{a} = \nabla f(x^*)$,即目标函数 $f$在最优点 $x^*$处的梯度;
\item $\mathcal{X}$为优化问题的可行域。
\end{itemize}
\end{definition}

\subsubsection{支撑超平面的核心意义:}

\begin{itemize}
\item 几何层面:该超平面与凸集 $\mathcal{X}$在 $x^*$处“相切”,且凸集 $\mathcal{X}$完全位于超平面的一侧;
\item 优化层面:超平面的法向量(即梯度 $\nabla f(x^*)$)指向目标函数的上升方向,因此不存在从 $x^*$出发、指向可行域内部且能使目标函数值下降的方向;
\item 理论关联:该条件与KKT条件(Karush-Kuhn-Tucker条件)完全对应,本质是“梯度与约束法向一致”的几何体现。
\end{itemize}

\begin{quote}
\textit{KKT条件后面会讲到}
\end{quote}

\subsection{凸组合与最优性路径}

在凸优化问题中,全局最优点 $x^*$常可解释为\textbf{多个可行极值点的凸组合(convex combination)},具体表现为:  

以资源分配问题为例:若存在 $k$种可行的资源配置方案,每种方案对应可行域 $\mathcal{X}$内的一个点 $x_i$($i=1,2,\dots,k$),则最终的全局最优解 $x^*$一定位于这些点的\textbf{凸包(convex hull)}内。  

凸包:
\begin{equation}
\text{conv}\{x_1, x_2, \dots, x_k\} = \left\{ \sum_{i=1}^k \theta_i x_i \mid \theta_i \geq 0, \sum_{i=1}^k \theta_i = 1 \right\}
\end{equation}

\subsubsection{关键结论:}

凸包是包含所有点 $x_1, x_2, \dots, x_k$的最小凸集,全局最优解 $x^*$位于凸包内,体现了凸优化最优解的“折中性质”——最终解是多个局部可行方案的合理权衡,而非极端方案。

\begin{quote}
\textit{几何示例:若两种可行方案对应平面上的两个点,其凸包为两点间的线段,全局最优解即为线段上使目标函数最小的点。}
\end{quote}


\section{凸优化的转化以及示例}

\subsection{优化问题的等价变换以及实例}

\begin{definition}[等价变换条件]
等价变换需满足三个条件:\textbf{最优值相同、最优解可相互恢复、可行域一一对应}。
\end{definition}

\subsubsection{1. SVM 软间隔原问题 → Hinge 损失形式(最典型等价转化)}

\begin{itemize}
\item \textbf{原问题(含松弛变量)}:软间隔 SVM 原问题为  
\begin{equation}
\min _{w, b, \xi} \frac{1}{2}\| w\| ^{2}+C \sum_{i} \xi_{i}, \quad \text{s.t.} \quad y_{i}\left(w^{T} x_{i}+b\right) \geq 1-\xi_{i}, \xi_{i} \geq 0
\end{equation}  
该问题含变量 $w, b, \xi$,约束涉及松弛变量 $\xi_i$,求解需同时处理三类变量。

\item \textbf{等价转化:Hinge 损失形式}:通过 “松弛变量消去”的等价思路,将原问题转化为仅含 $w$ 的无约束问题:  
\begin{equation}
\min _{w} \frac{1}{2}\| w\| ^{2}+C \sum_{i} \max \left(0,1-y_{i} w^{T} x_{i}\right)
\end{equation}  
其中 $\max(0,1-y_i w^T x_i)$ 为 Hinge 损失,本质是用“损失项”隐式替代松弛变量 $\xi_i$($\xi_i \geq \max(0,1-y_i w^T x_i)$,最小化目标时二者取等)。

\item \textbf{等价性验证}:  

  \begin{itemize}
  \item 最优值相同:原问题通过 $\xi_i$ 控制“间隔违反程度”,Hinge 损失直接量化该程度,最小化目标的核心逻辑一致;  
  \item 最优解可恢复:从 Hinge 损失的最优解 $w^*$,可反向计算 $b^*$(如通过支持向量满足 $y_i(w^{*T}x_i + b^*) = 1$),松弛变量 $\xi_i^* = \max(0,1-y_i w^{*T}x_i)$;  
  \item 可行域一一对应:原问题的可行域($\xi_i \geq 0$ 且 $y_i(w^T x_i + b) \geq 1-\xi_i$)与 Hinge 损失的“隐含可行域”(无显式约束,但损失项确保等价约束)完全对应 。
  \end{itemize}

\item \textbf{实例价值}:转化后无需处理松弛变量 $\xi$,将带约束问题简化为无约束凸优化,可直接用梯度法求解,同时保持 SVM“最大间隔分类”的核心逻辑。
\end{itemize}
\subsubsection{2. 逻辑回归:似然最大化 → 负对数似然最小化(单调变换等价)}

\begin{itemize}
\item \textbf{原问题(非凸形式)}:逻辑回归的核心是“最大化样本似然概率”,似然函数为 $\prod_{i=1}^m \sigma(y_i w^T x_i)$($\sigma(\cdot)$ 为 Sigmoid 函数),该函数是乘积形式,非凸且求解困难。

\item \textbf{等价转化:凸形式}:利用对数函数的单调性($\log(\cdot)$ 单调递增,最大化 $f$ 等价于最大化 $\log f$),再通过“取负”将最大化问题转化为最小化问题,最终目标函数为:  
\end{itemize}
\begin{equation}
\min_w -\sum_{i=1}^m \log\left(\sigma(y_i w^T x_i)\right)
\end{equation}  
该“负对数似然函数为凸函数”。

\begin{itemize}
\item \textbf{等价性验证}:由于 $\log(\cdot)$ 和“取负”均为单调变换,原问题的最优解 $w^*$ 与转化后问题的最优解完全一致,最优值仅相差常数倍(对数与负号的影响),满足等价变换的三个条件 。

\item \textbf{实例价值}:将非凸的乘积似然转化为凸的加法函数,可通过梯度法、牛顿法高效求解,且保障解为全局最优。
\end{itemize}

\subsubsection{3. 最小二乘回归:隐含约束 → 无约束凸问题(可行域等价)}

- \textbf{原问题(隐含约束)}:最小二乘的目标是 
\begin{equation}
\min_w \|Xw - y\|^2
\end{equation} 
其“可行域”本质是“所有使误差有定义的 $w$”(即全空间 $\mathbb{R}^n$),无显式约束。
\begin{itemize}
\item \textbf{等价转化:无约束凸问题}:无需引入额外变量,直接利用凸函数判定条件——目标函数的 Hessian 矩阵为 $2X^T X \succeq 0$(半正定),因此是凸函数。此时问题等价于“在凸集(全空间)上最小化凸函数”,完全符合凸优化的定义。

\item \textbf{实例价值}:通过“可行域等价”避免复杂约束处理,直接用闭式解($w^* = (X^T X)^{-1} X^T y$)或梯度法求解,是机器学习中最易实现的凸优化实例之一。
\end{itemize}
\subsection{非凸到凸的重构思路在实例中的延伸(解决“非凸建模难题”)}

四类非凸到凸的重构技术:\textbf{函数松弛、对偶化、变量替换、线性化},这些思路在实例中被灵活应用,让原本非凸的模型具备凸优化的“全局最优性”。

\subsubsection{1. 函数松弛:Hinge 损失替代 0-1 损失(SVM 中的非凸近似)}

SVM 的 Hinge 损失形式,本质是“函数松弛”技术的落地——用凸上界替代非凸项:

\begin{itemize}
\item \textbf{非凸痛点}:分类任务的理想损失是“0-1 损失”(正确分类损失为 0,错误分类损失为 1),但 0-1 损失是阶梯函数,非凸且不可微,无法用于凸优化建模。
\item \textbf{凸松弛方案}:根据“函数松弛”思路,用 Hinge 损失($\max(0,1-y_i w^T x_i)$)作为 0-1 损失的\textbf{凸上界}——对所有 $w$,均有 $\max(0,1-y_i w^T x_i) \geq \mathbb{I}(y_i w^T x_i < 1)$($\mathbb{I}$ 为指示函数,即 0-1 损失) 。
\item \textbf{实例应用}:SVM 选择 Hinge 损失作为目标项,既保留“惩罚间隔违反样本”的核心逻辑,又通过“凸松弛”让目标函数成为凸函数 。此时问题转化为凸优化,保障全局最优,避免 0-1 损失导致的“局部最优陷阱”。
\end{itemize}

\subsubsection{2. 对偶化:Lasso 回归的对偶问题(处理不可微凸项)}

Lasso 回归的求解,依赖 “对偶化”技术——通过拉格朗日对偶将原问题转化为更易求解的凸问题:

\begin{itemize}
\item \textbf{原问题痛点}:Lasso 回归的目标函数为 $\min_w \|Xw - y\|^2 + \lambda\|w\|_1$,其中 $\|w\|_1$ 是凸函数但不可微(在 $w_i=0$ 处无梯度),直接用梯度法求解困难 。

\item \textbf{对偶化方案}:根据 “对偶化”思路,构造原问题的拉格朗日对偶问题,给出 Lasso 对偶问题为:  


\begin{equation}
\max _{\alpha} \sum_{i} \alpha_{i}-\frac{1}{2} \sum_{i, j} \alpha_{i} \alpha_{j} y_{i} y_{j}\left(x_{i}^{T} x_{j}\right), \quad \text{s.t.} \quad 0 \leq \alpha_{i} \leq C, \sum_{i} \alpha_{i} y_{i}=0
\end{equation}  
该对偶问题是\textbf{凸二次规划}(目标函数为二次函数,Hessian 半正定;约束均为线性,可行域为凸集) 。


\item \textbf{实例价值}:对偶问题虽与原问题变量不同(从 $w$ 变为 $\alpha$),但满足 Slater 条件(对偶间隙为 0),最优值与原问题相同,且对偶问题可通过成熟的二次规划算法求解(无需处理不可微性)。求解后再通过 $w^* = \sum_{i} \alpha_i^* y_i x_i$ 恢复原问题最优解,同时保留 Lasso“稀疏性”的核心特性 。
\end{itemize}

\subsubsection{3. 线性化:投资组合优化的非线性约束处理(Markowitz 模型延伸)}

\begin{itemize}
\item \textbf{基础模型(凸二次规划)}:标准 Markowitz 模型为 
\begin{equation}
\min_w \frac{1}{2}w^T \Sigma w - \mu^T w, \quad \text{s.t.} \quad 1^T w = 1, \, w_i \geq 0
\end{equation} 
由于 $\Sigma$(协方差矩阵)半正定,目标函数凸,可行域凸,属于凸二次规划 。
\item \textbf{非凸扩展痛点}:实际投资中常存在非线性约束(如交易成本为 $w_i^2$,或最小持仓比例为非线性函数),这些约束会破坏可行域的凸性,导致模型非凸。
\item \textbf{线性化方案}:根据“线性化”思路,对非线性约束进行“局部线性近似”或“分段线性逼近(PWL)”——例如将交易成本 $w_i^2$ 在可行域内分段用线性函数近似,每个分段内约束为线性,整体可行域仍为凸集 。
\item \textbf{实例价值}:线性化后模型仍保持“目标函数凸+可行域凸”的特性,保障全局最优,解决了“实际投资约束下模型不可解”的问题,同时让最优解仍落在“有效前沿”上 。
\end{itemize}
\subsection{转化逻辑的核心价值}

\textbf{所有实例均通过“等价变换”或“非凸重构”,最终满足凸优化的定义(目标函数凸+可行域凸),进而利用 “局部最优即全局最优”的定理,实现“高效求解”与“结果可靠”的双重目标}。具体可总结为三类转化路径:

\begin{itemize}
\item \textbf{复杂凸 → 简洁凸}:如 SVM 原问题→Hinge 损失(消去松弛变量)、最小二乘→无约束问题(简化可行域);
\item \textbf{非凸 → 凸近似}:如 0-1 损失→Hinge 损失(函数松弛)、非线性约束→分段线性约束(线性化);
\item \textbf{难求解凸 → 易求解凸}:如 Lasso 原问题→对偶问题(处理不可微性)。
\end{itemize}